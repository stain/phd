
\section{Main Contributions}
\label{intro:contributions}

The contributions from this PhD include:

\begin{itemize}
    \item An evaluation of FAIR Digital Objects and Linked Data, considering them from a developer perspective as distributed object systems.
    \item A Research Object implementation based on familiar Web technologies, adapted and extended by numerous research projects and software developers.
    \item A profile to capture provenance of computational workflow runs using this implementation, implemented by at least six workflow management systems.
\end{itemize}

These contributions have not evolved in isolation, but in co-development with multiple international collaborations (see Appendix \vref{ch11:acknowledgements}).


\section{Thesis Overview}
\label{intro:overview}

Chapter \vref{chapter:background} gives the background of the concepts \emph{FAIR Digital Object} (FDO) and \emph{Linked Data}, including a brief history of the \emph{Semantic Web}, followed by a critical analysis of these technologies and their use. 

Chapter \vref{chapter:fdo} targets RQ1 and contributes a framework-based evaluation of Linked Data and FDO as possible architectures for implementing a distributed object system for the purpose of FAIR data publishing. The discussion in this chapter considers how the two approaches can benefit from each other's strengths. 

Chapter \vref{chapter:ro-crate} addresses RQ2 by introducing the contribution of \emph{RO-Crate}---a pragmatic data packaging mechanism using Linked Data standards to implement FDO and be extensible for domain-specific metadata.  

Chapter \vref{chapter:workflows} considers RQ3 by exploring the relationship between Computational Workflows and FAIR practices using RO-Crate and FDO, with use cases from molecular dynamics and specimen digitization. The contribution of the \emph{Workflow Run Crate profiles} is presented as an interoperable way to capture and publish workflow execution provenance. 

Chapter \vref{chapter:conclusions} summarises and discusses the contributions from this thesis, reflects on later third-party developments and concludes by evaluating the research questions.

