In order to investigate \textbf{RQ3} (\vpageref*{rq3}), and considering an important part of the FAIR principles is \emph{Reuse} and \emph{provenance}, this chapter examines in closer details how FAIR Digital Objects and RO-Crate can be used with \textbf{Computational Workflows}. 

Section \vref{ch6:making-canonical-workflow-building-blocks-interoperable-across-workflow-languages} proposes that tools in computational workflows, when wrapped as interoperable building blocks, can be considered as FAIR Digital Objects, with a use case from biomolecular simulation.

Sections \vref{ch8:the-specimen-data-refinery} and \vref{ch7:incrementally-building-fair-digital-objects-with-specimen-data-refinery-workflows} explore how FDOs and Research Objects can be constructed incrementally using computational workflows, with a use case from specimen digitization in natural history collections.

Section \vref{ch54:wrroc} presents a profile of RO-Crate to capture workflow execution provenance, with incremental granularity levels and six workflow engine implementations. Use cases include machine learning-aided tumour detection and compatibility with PROV approaches. 

Supplementary materials that may assist readers of this chapter provide further details on FAIR Computational Workflows \cite{Goble 2020}, \footurl{https://s11.no/2021/phd/workflow-collaboratory/}{WorkflowHub} \cite{Goble 2021}, \footurl{https://s11.no/2022/phd/methods-included/}{Common Workflow Language} \cite{Crusoe 2022} and \footurl{https://s11.no/2022/phd/10-simple-rules-for-workflow-tools/}{making a software tool workflow ready} \cite{ch6-37}. 

On the aspects of workflow provenance, recommended reading in supplementary materials covers CWLProv \cite{Khan 2019}, \footurl{https://s11.no/2022/phd/galaxy-ro-crate/}{RO-Crate in Galaxy} \cite{De Geest 2022} and Common Provenance Model \cite{Wittner 2020,Wittner 2023}.
