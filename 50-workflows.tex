As an important part of the FAIR principles is \emph{Reuse} and \emph{provenance}, in this chapter I examine in closer details how FAIR Digital Objects and RO-Crate can used with \textbf{Computational Workflows}. 
Section \vref{ch6:making-canonical-workflow-building-blocks-interoperable-across-workflow-languages} explores building blocks.

Supplementary materials that may assist readers of this chapter provides further details on FAIR Computational Workflows \cite{Goble 2020}, \footurl{https://s11.no/2021/phd/workflow-collaboratory/}{WorkflowHub} \cite{Goble 2021}, \footurl{https://s11.no/2022/phd/methods-included/}{Common Workflow Language} \cite{Crusoe 2022} and \footurl{https://s11.no/2022/phd/10-simple-rules-for-workflow-tools/}{making a software tool workflow ready} \cite{ch6-37}. 
On the aspects of workflow provenance, supplementary materials cover CWLProv \cite{ch5-68}, \footurl{https://s11.no/2022/phd/galaxy-ro-crate/}{RO-Crate in Galaxy} \cite{De Geest 2022}, Common Provenance Model \cite{Wittner 2020,Wittner 2023} and \footurl{https://s11.no/2023/phd/workflow-run-crate/}{Workflow Run Crate} \cite{workflow-run-crate}.
