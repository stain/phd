

\newacronym[description={
        A set of principles for sharing of research data and metadata using machine-readable formats \cite{Wilkinson 2016}. See section \ref{ch10:fair-principles}
}]{FAIR}{FAIR}{Findable, Accessible, Interoperable, Reusable}

\newacronym[description={
        Guidelines and specifications for machine-actionable and self-described digital resources, with an emphasis on identifiers, types, attributes and operations. See section \ref{ch3:fdo}
}]{FDO}{FDO}{FAIR Digital Object}

\newacronym[description={
        Grouping of digital scholarly outputs with relationships, semantic descriptions and executable code \cite{Bechhofer 2013}. See section \ref{intro:rq2}
}]{RO}{RO}{Research Object}

\newacronym[description={
        Method to package Research Objects with metad40-cratesata in an embedded JSON-LD file \cite{Soiland-Reyes 2022a} (section \ref{40-crates}).
}]{RO-Crate}{RO-Crate}{Research Object Crate}

\newacronym[description={
        EU initiative to foster Open Science by integrating research infrastructures, interoperability standards, core services (e.g. persistent identifiers, authentication) and best practices \cite{Ayris 2016} 
}]{EOSC}{EOSC}{European Open Science Cloud}

\newacronym[description={
        A structured general purpose data format derived from the JavaScript programming language, commonly preferred by Web APIs. A JSON document consists of key-based objects (dictionaries), arrays, strings and numbers \cite{Bray 2017}
}]{JSON}{JSON}{JavaScript Object Notation}

\newacronym[description={
        A JSON-based serialization of Linked Data \cite{Sporny 2020}. Typically includes a \texttt{@context} which defines the mapping to RDF. Examples shown in Figure \ref{ch3:fig:jsonld} and listing \ref{ch5:lis1}
}]{JSON-LD}{JSON-LD}{JSON Linked Data}


\newacronym[description={
        Conceptual model for knowledge graph representation in the form of triples that relate Web resources
        \cite{Schreiber 2014}, realised through multiple serialisation formats
}]{RDF}{RDF}{Resource Description Framework}

\newacronym[description={
        Set of practices for publishing and relating data on the Web using RDF technologies \cite{Bizer 2009}, see also section \ref{ch3:ld-web}
}]{LD}{LD}{Linked Data}

\newacronym[description={
        Globally unique string (e.g. \texttt{http://example.com/}) to identify a resource that can be retrieved (located) using its defined protocol (e.g. \texttt{http}). 
}]{URL}{URL}{Uniform Resource Locator}

\newacronym[description={
        Globally unique identifier string (e.g. \texttt{http://example.com/} or \texttt{urn:isbn:0451450523}). The \emph{scheme} (\textt{http}, \texttt{urn:isbn}) classifies the identifier. URIs are superset of URLs, as the scheme is not required to have an associated network protocol \cite{Berners-Lee 2005}
}]{URI}{URI}{Uniform Resource Identifier}

\newacronym[description={
        Globally unique identifier string, equivalent to URIs, but permitting all internatinal Unicode characters witout \texttt{\%} escaping.
}]{IRI}{IRI}{International Resource Identifier}

\newacronym[description={
        Network protocol used for retrieving Web pages and invoking Web APIs
        \cite{Fielding 1999}. The secure version of the protocol \texttt{https} adds 
}]{http}{HTTP}{Hypertext Transfer Protocol}

\newacronym[description={
        Architectural style for Web applications, where the stateless nature of Web is exploited by navigating the application state through Web resources, facilitating hypermedia formats \cite{Fielding 2000}. 
        \emph{RESTful} Web Services are the dominant form of Web APIs, although the hypermedia aspect is frequently neglected
}]{REST}{REST}{Representational State Transfer}


\newacronym[description={
        An identifier string that is globally unique, resolvable (typically through a resolver or registry) and with an organisational promise of persistence, suitable for inclusion in long-term archives and data publications
        \cite{McMurry 2017}
}]{PID}{PID}{Persistent Identifier}


%
\newacronym[description={
RDFS is..
}]{RDFS}{RDFS}{RDFS ...}
%
\newacronym[description={
OWL is..
}]{OWL}{OWL}{OWL ...}
%
\newacronym[description={
DOIP is..
}]{DOIP}{DOIP}{DOIP ...}
%
\newacronym[description={
Handle is..
}]{Handle}{Handle}{Handle ...}
%
\newacronym[description={
TLS is..
}]{TLS}{TLS}{TLS ...}
%
\newacronym[description={
DRM is..
}]{DRM}{DRM}{DRM ...}
%
\newacronym[description={
DCAT is..
}]{DCAT}{DCAT}{DCAT ...}
%
\newacronym[description={
        A set of recommendations for interoperability across services and data in the European Open Science Cloud \cite{Kurowski 2021,Åkerström 2024} 
}]{EOSC-IF}{EOSC-IF}{EOSC Interoperability Framework}
%
\newacronym[description={
SKOS is..
}]{SKOS}{SKOS}{SKOS ...}
%
\newacronym[description={
ROR is..
}]{ROR}{ROR}{ROR ...}
%
\newacronym[description={
DNS is..
}]{DNS}{DNS}{DNS ...}
%
\newacronym[description={
API is..
}]{API}{API}{API ...}
%
\newacronym[description={
CRUD is..
}]{CRUD}{CRUD}{CRUD ...}
%
\newacronym[description={
HTML is..
}]{HTML}{HTML}{HTML ...}

\newglossaryentry{signposting}{
        name={Signposting},
        description={A method to provide machines with navigational link relations \cite{Nottingham 2017} between Web resources, by providing HTTP \texttt{Link:} headers, HTML \texttt{<link>} elements or a linkset \cite{Wilde 2020} \cite{Van de Sompel 2022}
}}



%\newacronym[description={}]{ABC}{ABC}{A B C}