\section{Updating Linked Data practices for FAIR Digital Object
principles}
\label{ch2:updating-linked-data-practices-for-fair-digital-object-principles}

Realization of FAIR Digital Objects has a great potential if combined with the mature technology stack of Linked Data and knowledge graphs.

Here I will briefly discuss how FDO principles can be achieved using existing standards that have powered the Web for the last 30 years. Using this mature approach can accelerate uptake of FDO by scholars and existing research infrastructures.

I will also reflect on how the Linked Data community can adapt to better welcome approaches like FDO.

\subsubsection{Background}
\label{ch2:background}

The \emph{FAIR principles} \cite{Wilkinson 2016} are
fundamental for data discovery, sharing, consumption and reuse; however
their broad interpretation and many ways to implement can lead to
inconsistencies and incompatibility
\cite{Jacobsen 2020}.

The European Open Science Cloud (\href{https://www.eosc.eu/}{EOSC}) has
been instrumental in maturing and encouraging FAIR practices across a
wide range of research areas. Linked Data in the form of
\href{https://www.w3.org/TR/rdf11-primer/}{RDF} (Resource Description
Framework) is the common way to implement machine-readability in FAIR,
however the principles do not prescribe RDF or any particular technology
\cite{Mons 2017}.

\paragraph{FAIR Digital Object}
\label{ch2:fair-digital-object}

\textbf{FAIR Digital Object} (FDO)
\cite{Schultes 2019}
has been proposed to improve researcher's access to digital objects
through formalising their metadata, types, identifiers and exposing
their computational operations, making them actionable FAIR objects
rather than passive data sources.

FDO is a set of principles \cite{bonino2019}, implementable in multiple ways. Current realisations mostly
use \emph{Digital Object Interface Protocol} (DOIPv2)
\cite{DONA 2018}, with the
main implementation
\href{https://www.cordra.org/documentation/api/doip.html}{CORDRA}. We
can consider DOIPv2 as a simplified combination of object-oriented
(CORBA, SOAP) and document-based (HTTP, FTP) approaches.

More recently, the \href{https://fairdo.org/}{FDO Forum} has prepared
detailed
\href{https://drive.google.com/drive/u/0/folders/1-SbZk7enOqjy2Rf57PMB-CQW7qMOUXgO}{recommendations}\footnote{See section \vref{ch3:next-step-fdo}},
currently open for comments, including a
\href{https://docs.google.com/document/d/10ESWe-m0ex7fIW0ZYOYeVg6gAcIaBKyO3W-Vg8tXzdw/edit\#}{DOIP
endorsement} \cite{fdo-DOIPEndorsement} and updated
\href{https://docs.google.com/document/d/1-4_yGRrIcgdMIwaFvHyUt6lxDdfzGpqLUCEihE0vJ-g/edit\#heading=h.gjdgxs}{FDO
requirements} \cite{fdo-RequirementSpec}. These point out \textbf{Linked Data} as another possible
technology stack, which is the focus of this work.

\paragraph{Linked Data}
\label{ch2:linked-data}

\href{https://www.w3.org/standards/semanticweb/data}{Linked Data
standards} (LD), based on the Web architecture, are commonplace in
sciences like bioinformatics, chemistry and medical informatics -- in
particular to publish Open Data as machine-readable resources. LD has
become ubiquitous on the general Web, the
\href{https://schema.org/}{schema.org} vocabulary is used by over 10
million sites for indexing by search engines --
\href{https://w3techs.com/technologies/details/da-jsonld}{43\% of all
websites} use \href{https://json-ld.org/}{JSON-LD}.

Although LD practices align to FAIR \cite{Hasnain 2018},
they do not fully encompass active aspects of FDOs. The HTTP protocol is
used heavily for applications (e.g.~mobile apps and cloud services),
with REST APIs of customised \href{https://json-schema.org/}{JSON
structures}. Approaches that merge the LD and REST worlds include
\href{https://www.w3.org/TR/ldp/}{Linked Data Platform} (LDP),
\href{https://www.hydra-cg.com/}{Hydra} and
\href{https://www.w3.org/TR/webpayments-http-messages/}{Web Payments}.


\subsubsection{Meeting FDO principles using Linked Data
standards}\label{ch2:meeting-fdo-principles-using-linked-data-standards}

Considering the potential of FDOs when combined with the mature
technology stack of LD, here we briefly discuss how FDO principles can
be achieved using existing standards. The general principles (G1--G9)
apply well: Open standards with HTTP being stable for 30 years, JSON-LD
is widely used, FAIR practitioners mainly use RDF, and a clear
abstraction between the RDF model with stable bindings available in
multiple serialisations.

However, when considering the specific principles (FDOF1--FDOF12) we
find that additional constraints and best practices need to be
established -- arbitrary LD resources cannot be assumed to follow FDO
principles. This is equivalent to how existing use of DOIP is not
FDO-compliant without additional constraints.

Namely, persistent identifiers (PIDs) \cite{McMurry 2017}
(FDOF1) are common in LD world (e.g.~using \url{http://purl.org/} or
\url{https://w3id.org/}), however they don't always have a declared type
(FDOF2), or the PID may not even appear in the metadata. URL-based PIDs
are resolvable (FDOF3), typically over HTTP using redirections and
content-negotiation. One great advantage of RDF is that all attributes
are defined semantic artefacts with PIDs (FDOF4), and attributes can be
reused across vocabularies.

While CRUD operations (FDOF6) are supported by native HTTP operations
(GET/PUT/POST/DELETE) as in \href{https://www.w3.org/TR/ldp/}{LDP},
there is little consistency on how to define operation interfaces in LD
(FDOF5). Existing REST approaches like
\href{https://swagger.io/specification/}{OpenAPI} \cite{OpenAPISpecificationV3} and
URI templates \cite{rfc6570} are mature and
good candidates, and should be related to defined types to support
machine-actionable composition (FDOF7). HTTP error code \emph{410 Gone}
is used in tombstone pages for removed resources (FDOF12), although more
frequent is \emph{404 Not Found}.

Metadata is resolved to HTTP documents with their own URIs, but these
frequently don't have their own PID (FDOF8).
\href{https://w3c.github.io/rdf-star/}{RDF-Star} and nanopublications
\cite{Kuhn 2021} give ways
to identify and trace provenance of individual assertions.

Different metadata levels (FDOF9) are frequently developed for LD
vocabularies across different communities (FDOF10), such as
\href{http://hl7.org/fhir/}{FHIR} for health data,
\href{https://bioschemas.org/}{Bioschemas} for bioinformatics and
\textgreater1000 more specific
\href{https://bioportal.bioontology.org/ontologies}{bioontologies}.
Increased declaration and navigation of \emph{profiles} is therefore
essential for machine-actionability and consistent consumption across
FAIR endpoints.

Several standards exist for rich collections (FDOF11),
e.g.~\href{https://www.openarchives.org/ore/}{OAI-ORE},
\href{https://www.w3.org/TR/vocab-dcat-3/}{DCAT},
\href{https://www.researchobject.org/ro-crate/}{RO-Crate},
\href{https://www.w3.org/TR/ldp/}{LDP}. These are used and extended
heterogeneously across the Web, but consistent machine-actionable FDOs
will need specific choices of core standards and vocabularies. Another
challenge is when multiple PIDs refer to ``almost the same'' concept in
different collections -- significant effort have created manual and
automated semantic mappings \cite{Baker 2013,de Mello 2022}.

Currently the FDO Forum has suggested the use of LDP as a possible
alternative for implementing FAIR Digital Objects \cite{bonino2021}, which
proposes a novel approach of content-negotiation with custom media
types.

\subsubsection{Discussion}\label{ch2:discussion}

The Linked Data stack provides a set of specifications, tools and
guidelines in order to help the FDO principles become a reality. This
mature approach can accelerate uptake of FDO by scholars and existing
research infrastructures such as the European Open Science Cloud (EOSC).

However, the amount of standards and existing metadata vocabularies
poses a potential threat for adoption and interoperability. Yet, the
challenges for agreeing on usage profiles apply equally to DOIP as LD
approaches.

We have worked with different scientific communities to define 
RO-Crate \cite{Soiland-Reyes 2022}, a
lightweight method to package research outputs along with their
metadata. While RO-Crate's use of schema.org shows just one possible
metadata model, it's powerful enough to be able to express FDOs, and
familiar to web developers.

We have also used FAIR Signposting \cite{Van de Sompel 2022} with HTTP
\texttt{Link:} headers as a way to support navigation to the individual
core properties of an FDO (PID, type, metadata, licence, bytestream)
that does not require heuristics of content-negotiation and is agnostic
to particular metadata vocabularies and serialisations.

We believe that by adopting Linked Data principles, we can accelerate
FDO today -- and even start building practical ways to assist scientists
in efficiently answering topical questions based on knowledge graphs.

