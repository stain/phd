\section{Evaluating FAIR Digital Object and Linked Data as distributed object systems}\label{ch3:evaluating-fdo-ld}

FAIR Digital Object (FDO) is an emerging concept that is highlighted by European Open Science Cloud (EOSC) as a potential candidate for building an ecosystem of machine-actionable research outputs. In this work we systematically evaluate FDO and its implementations as a global distributed object system, by using five different conceptual frameworks that cover interoperability, middleware, FAIR principles, EOSC requirements and FDO guidelines themselves.

We compare the FDO approach with established Linked Data practices and the existing Web architecture, and provide a brief history of the Semantic Web\footnote{
  In this thesis moved to section \vref{ch3:background} as background information on Linked Data and FDO.} while discussing why these technologies may have been difficult to adopt for FDO purposes. We conclude with recommendations for both Linked Data and FDO communities to further their adaptation and alignment.

\subsection{Introduction}\label{ch3:introduction}

The FAIR principles \cite{Wilkinson 2016} encourage sharing of scientific data with machine-readable metadata and the use of interoperable formats, and are being adapted by a wide range of research infrastructures. They have been recognised by the research community and policy makers as a goal to strive for \cite{h2020fair2016}. In particular, the European Open Science Cloud \footurl{https://www.eosc.eu/}{(EOSC)} has promoted adaptation of FAIR data sharing of data resources across electronic research infrastructures \cite{Mons 2017}. The EOSC Interoperability Framework \cite{eosc-interop-framework} puts particular emphasis on how interoperability can be achieved technically, semantically, organisationally, and legally -- laying out a vision of how data, publication, software and services can work together to form an ecosystem of digital objects that are extensively described. Such descriptions for interoperability connect a range of information -- from protocols and presentations, to hardware designs and scientific workflows, including extensive metadata of the information itself.

Specifically, the EOSC Interoperability framework highlights the emerging FAIR Digital Object (FDO) concept \cite{Schultes 2019} as a possible foundation for building a semantically interoperable ecosystem to fully realise the FAIR principles beyond individual repositories and infrastructures. The FDO approach has great potential, as it proposes strong requirements for identifiers, types, access and formalises interactive operations on objects.

In other discourse, Linked Data \cite{Bizer 2009} has been seen as an established set of principles based on Semantic Web technologies that can achieve the vision of the FAIR principles \cite{boninodasilvasantosFAIRDataPoints2016a,Hasnain 2018}. Yet regular researchers and developers of emerging platforms for computation and data management are reluctant to adapt such a ``FAIR Linked Data'' approach fully \cite{verborghSemanticWebIdentity2020a}, opting instead for custom in-house models and JSON-derived formats from RESTful Web services \cite{merono-penuelaConclusionFutureChallenges2021a,neumannAnalysisPublicREST2021a}. While such focus on simplicity allows rapid development and highly specialised services, it raises wider concerns about interoperability \cite{turcoaneLinkedDataJSONLD2014a,wilkinsonWorkflowsWhenParts2022b}.

One challenge that may, perhaps counter-intuitively, steer developers towards a not-invented-here mentality \cite{stefiDevelopersMakeUnbiased2015,stefiDevelopReuseTwo2015a} when exposing their data on the Web is the heterogeneity and apparent complexity of Semantic Web approaches themselves \cite{merono-penuelaWebDataApis2021b}.

These approaches -- FDO and Linked Data -- thus, form two of the major avenues for allowing developers and the wider research community to achieve the goal of FAIR data. Given their importance, in this article we compare FAIR Digital Objects with Linked Data and the Web architecture in the context of the discourse around FAIR data.

Concretely, the contribution of this paper is a {\bf systematic comparison between FDO and Linked Data using 5 different conceptual frameworks} that capture different perspectives on interoperability and readiness for implementation.

In chapter \vref{ch3:background} we gave a background primer on FDO and Linked Data to provide a foundation for this work. The rest of this article is organised as follows:  In the Method section \vref{ch3:method}, we introduce the conceptual frameworks we use for comparison. Subsequently, in the Results section \vref{ch3:results}, we systematically step through the outcomes of applying these frameworks to both FDO and Linked Data. For each framework, we derive key observations. We end in section \vref{ch3:discussion} with a discussion of these results and their implications for both approaches and conclude.


\subsection{Method}\label{ch3:method}

\subsubsection{Comparing FDO and existing approaches}\label{ch3:comparing}

Our main motivation for this article is to investigate how FAIR Digital Objects may differ from the learnt experiences of Linked Data and the Web. We also aim to reflect back from FDO's motivation of machine-actionability to consider the Web as a distributed computational system.

To better understand the relationship between the FDO framework and other existing approaches, we use the following for analysis:

\begin{enumerate}
\tightlist
\item
  An Interoperability Framework and Distributed Platform for Fast Data Applications \cite{delgadoInteroperabilityFrameworkDistributed2016a}, which proposes quality measurements for comparing how frameworks support interoperability, particularly from a service architectural view.
\item
  The FAIR Digital Object guidelines \cite{bonino2019}, validated against its current implementations for completeness.
\item
  A Comparison Framework for Middleware Infrastructures \cite{zarrasComparisonFrameworkMiddleware2004a}, which suggest dimensions like openness, performance and transparency, mainly focused on remote computational methods.
\item
  Cross-checks against RDA's FAIR Data Maturity Model \cite{bahimFAIRDataMaturity2020a} to find how the FAIR principles are achieved in FDO, in particular considering access, sharing and openness.
\item
  EOSC Interoperability Framework \cite{eosc-interop-framework} which gives recommendations for technical, semantic, organisational and legal interoperability, particularly from a metadata perspective.
\end{enumerate}

Conceptual framework 1, 3, 5 considers more general views of interoperability between systems, whereas frameworks 2 and 4 are developed specifically for addressing FAIR principles. 

The reason for this wide-ranged comparison is to exercise the different dimensions that together form FAIR Digital Objects: Data, Metadata, Service, Access, Operations, Computation.
We have left out further comparisons on type systems, persistent identifiers and social aspects as principles and practices within these dimensions are still taking form within the FDO community (as detailed in section \vpageref{ch3:next-step-fdo}).

Some of these frameworks invite a comparison on a conceptual level, while others relate better to implementations and current practices. For these conceptual comparisons we consider FAIR Digital Objects and the Web broadly. For implementations we contrast the main FDO realisation using the DOIPv2 protocol \cite{DONA 2018} against Linked Data as implemented in general practice\footnote{For further background on FDO implemented with Linked Data see \cite{bonino2021,10.3897/rio.8.e94501}}.

For all our comparisons, our process was to perform a mapping between the relevant specifications and/or implementation and the given conceptual model through detailed reading of the defining documents. We aim in all cases for traceability between the given specification and our mapping such that readers can validate our analysis.

\subsection{Results}\label{ch3:results}

\subsubsection{Considering FDO/Web as interoperability framework for Fast Data}\label{ch3:interoperability-compare}


The Interoperability Framework for Fast Data Applications \cite{delgadoInteroperabilityFrameworkDistributed2016a} categorises interoperability between applications along 6 strands, covering different architectural levels: from \emph{symbiotic} (agreement to cooperate) and \emph{pragmatic} (ability to choreograph processes), through \emph{semantic} (common understanding) and \emph{syntactic} (common message formats), to low-level \emph{connective} (transport-level) and \emph{environmental} (deployment practices).

We have chosen to investigate using this framework as it covers the higher levels of the OSI Model \cite{stallingsHandbookComputercommunicationsStandards1990} better with regards to automated machine-to-machine interaction (and thus interoperability), which is a crucial aspect of the FAIR principles. In Table \ref{ch3:fdo-web-interoperability-framework} we use the interoperability framework to compare the current FAIR Digital Object approach against the Web and its Linked Data practices.

\renewcommand*{\arraystretch}{1.4}
\begin{longtable}[]{@{}
  >{\raggedright\arraybackslash}p{(\columnwidth - 4\tabcolsep) * \real{0.1642}}
  >{\arraybackslash}p{(\columnwidth - 4\tabcolsep) * \real{0.4179}}
  >{\arraybackslash}p{(\columnwidth - 4\tabcolsep) * \real{0.4179}}@{}}
\caption[Considering FDO and Web according to Interoperability Framework for Fast Data]{Considering FDO and Web according to the quality levels of the Interoperability Framework for Fast Data \cite{delgadoInteroperabilityFrameworkDistributed2016a}.
\label{ch3:fdo-web-interoperability-framework}}\tabularnewline
\toprule
Quality & 
FDO with DOIP & 
Web with Linked Data \\
\midrule
\endfirsthead
\toprule
Quality &
FDO w/ DOIP & 
Web w/ Linked Data \\
\midrule
\endhead
\textbf{Symbiotic}: \emph{to what extent multiple applications can agree to interact, align, collaborate or cooperate}
  & The purpose of FDO is to enable federated machine actionable digital objects for scholarly purposes, in practice this also requires agreement of compatibility between FDO types. FDO encourages research communities to develop common type registries to be shared across instances. In current DOIP practice, each service have their own types, attributes and operations. The wider symbiosis is consistent use of PIDs.
  & The Web is loosely coupled and encourages collaboration and linking by URL. In practice, REST APIs \cite{fieldingArchitecturalStylesDesign2000a} end up being mandated centrally by dominant (often commercial) providers \cite{fieldingReflectionsRESTArchitectural2017a}, and the clients are required to use each API as-is with special code per service. Use of Linked Data enables common tooling and semantic mapping across differences. \\
\textbf{Pragmatic}: \emph{using interaction contracts so processes can be choreographed in workflows}
  & FDO types and operations enable detailed choreography (Canonical Workflows \cite{cwfr}). Attributes\footnote{
    DOIP's predefined attributes, types and operations have Handle-like identifiers with prefixes \texttt{0.TYPE} and \texttt{0.DOIP}, these are however not registered in the Handle system.
  } \texttt{0.TYPE/DOIPOperation} has lightweight definition of operation, \texttt{0.DOIP/Request} or \texttt{0.DOIP/Response} may give FDO Type or any other kind of ``specifics'' (incl.~human-readable docs). Semantics/purpose of operations not formalised (similar operations can be grouped with \texttt{0.DOIP/OperationReference}).
  & ``Follow your nose'' crawler navigation, which may lead to frequent dead ends. Operational composition, typically within a single API provider, documented by OpenAPI 3 \cite{OpenAPISpecificationV3}, schema.org Actions \cite{SchemaOrgActions}, WSDL/SOAP \cite{w3-wsdl20-primer}, but frequently just as human-readable developer documentation with examples. \\
\textbf{Semantic}: \emph{ensuring consistent understanding of messages, interoperability of rules, knowledge and ontologies}
  & FDO semantic enable navigation and typing. Every FDO has a type. Types maintained in FDO Type registries, which may add additional semantics, e.g.~the ePIC \footurl{https://hdl.handle.net/21.11104/c1a0ec5ad347427f25d6}{PID-InfoType for Model}. No single type semantic, Type FDOs can link to existing vocabularies \& ontologies. JSON-LD used within some FDO objects (e.g.~DISSCO Digital Specimen, NIST Material Science schema) \cite{wittenburgFAIRDigitalObject2022b}
  & Lightweight HTTP semantics for authenticity/navigation. Semantic Type not commonly expressed on PID/header level, may be declared within Linked Data metadata. Semantic of type implied by Linked Data formats (e.g.~OWL2, RDFS), although choice of type system may not be explicit. \\
\textbf{Syntactic}: \emph{serialising messages for digital exchange, structure representation}
  & DOIP serialise FDOs as JSON, metadata commonly use JSON, typed with JSON Schema. Multiple byte stream attachments of any media type.
  & Textual HTTP headers (including any signposting), single byte stream of any media type, e.g.~Linked Data formats (JSON-LD, Turtle, RDF/XML) or embedded in document (HTML with RDFa, JSON-LD or Microdata). XML was previously the main syntax used by APIs, JSON is now dominant. \\
\textbf{Connective}: \emph{transferring messages to another application, e.g.~wrapping in other protocols}
  & DOIP \cite{DONA 2018} is transport-independent, commonly TLS TCP/IP port 9000, DOIP over HTTP \cite{DOIPAPIHTTPa}
  & HTTP/1.1 TCP/IP port 80 \cite{rfc2616}; HTTP/1.1+TLS, TCP/IP 443 \cite{rfc2818}; HTTP/2, as HTTP/1* but binary \cite{rfc7540}; HTTP/3, like HTTP/2+TLS but UDP \cite{rfc9114} \\
\textbf{Environmental}: \emph{how applications are deployed and affected by its environment, portability}
  & Main DOIP implementation is \footurl{https://www.cordra.org/}{\emph{Cordra}}, which can be single-instance or \footurl{https://www.cordra.org/documentation/configuration/distributed-deployment.html}{distributed}. Cordra \footurl{https://www.cordra.org/documentation/configuration/storage-backends.html}{storage backends} include file system, S3, MongoDB (itself scalable). Unique DOIP protocol can be hard to add to existing Web application frameworks, although proxy services have been developed (e.g.~B2SHARE adapter).
  & HTTP services widely deployed in a myriad of ways, ranging from single instance servers, horizontally \& vertically scaled application servers, to multi-cloud Content-Delivery Networks (CDN). Current scalable cloud technologies for Web hosting may not support HTTP features previously seen as important for Semantic Web, e.g.~content negotiation and semantic HTTP status codes. \\
\bottomrule
\end{longtable}

\paragraph{Observations}

Based on the analysis shown in Table \ref{ch3:fdo-web-interoperability-framework}, we draw the following conclusions:

The Web has already showed us how one can compose workflows of hetereogeneous Web Services \cite{wolstencroftTavernaWorkflowSuite2013d}. However, this is mostly done via developer or human interaction \cite{lamprechtPerspectivesAutomatedComposition2021b}. Similiarly, FDO does not enable automatic composition because operation semantics are not well defined. There is a question as to whether the extensive documentation and broad developer usage that is available for Web APIs could potentially be utilised for FDO.

A difference between Web technologies and FDO is the stringency of the requirements for both syntax and semantics. Whereas the Web allows many different syntactic formats (e.g.~from HTML to XML, PDFs), FDO realised with DOIP requires JSON. On the semantic front, FDO mandates that every object have a well-defined type and structured form. This is clearly not the case on the Web.

In terms of connectivity and the deployment of applications, the Web has a plethora of software, services, and protocols that are widely deployed. These have shown interoperability. The Web standards bodies (e.g.~IETF and W3C) follow the OpenStand principles \cite{ModernStandardsParadigm} to embrace openness, transparency, and broad consensus. In contrast, FDO has a small number of implementations and corresponding protocols, although with a growing community, as evidenced at the first international FDO conference \cite{looFirstInternationalConference2022}. This is not to say that it is not worth developing further Handle+DOIP implementations in the future, but we note that the current FDO functionality can easily be implemented using Web technologies, even as DOIP-over-HTTP \cite{DOIPAPIHTTPa}.

It is also a question as to whether a highly constrained protocol revolving around persistent identifiers is in fact necessary. For example, DOIs are mostly resolved on the web using HTTP redirects with the common \texttt{https://doi.org/} prefix, hiding their Handle nature as an implementation detail \cite{DOIHandbookResolution}.


\subsubsection{Mapping of Metamodel concepts}\label{ch3:mapping-of-metamodel-concepts}

The Interoperability Framework for Fast Data also provides a brief \emph{metamodel} which we use in Table \vref{ch3:metamodel-concepts} to map and examplify corresponding concepts in FDO's DOIP realization and the Web using HTTP semantics \cite{rfc9110}.

From this mapping\footnote{An equivalent SKOS mapping \cite{w3-skos-primer} is provided as part of the \href{https://w3id.org/ro/doi/10.5281/zenodo.8075229}{RO-Crate for this article} \cite{soilandreyes2023}.} we can identify the conceptual similarities between DOIP and HTTP, often with common terminology. Notable are that neither DOIP or HTTP have strong support for transactions (explored further in section \vref{ch3:middleware}), as well that HTTP has poor direct support for processes, as the Web is primarily stateless by design \cite{fieldingArchitecturalStylesDesign2000a}.

\begin{table}[htp]
  \centering
  \caption[Mapping the Interoperability Framework Metamodel concepts to FDO and Web]{Mapping the Metamodel concepts from the Interoperability Framework for Fast Data \cite{delgadoInteroperabilityFrameworkDistributed2016a} to equivalent concepts for FDO and Web.
  \label{ch3:metamodel-concepts}}\tabularnewline
   \begin{tabular}{ m{5em}  m{15em} m{15em} } 
   \hline
  Metamodel concept & 
  FDO/DOIP concept & 
  Web/HTTP concept \\ 
   \hline
  Resource	  & FDO/DO	            & Resource \\
  Service	    & DOIP service	      & Server/endpoint \\
  Transaction	& (not supported)	    & Conditional requests, \texttt{409\ Conflict} \\
  Process	    & Extended operations	& (primarily stateless), \texttt{100\ Continue}, \texttt{202\ Accepted} \\
  Operation	  & DOIP Operation	    & Method, query parameters \\
  Request	    & DOIP Request	      & Request \\
  Response	  & DOIP Response	      & Response \\
  Message	    & Segment, \texttt{requestId} 
                                    & Message, Representation \\
  Channel	    & DOIP Transport protocol (e.g.~TCP/IP, TLS). JSWS signatures.
                                    & TCP/IP, TLS, UDP \\
  Protocol  	& DOIP 2.0, ++	      & HTTP/1.1, HTTP/2, HTTP/3 \\
  Link	      & PID/Handle	        & URL \\ 
   \hline
   \end{tabular}\end{table}
  

\subsubsection{Assessing FDO implementations}\label{ch3:doip-fdo-compare}

The FAIR Digital Object guidelines \cite{bonino2019} sets out recommendations for FDO implementations. Note that the proposed update to FDO specification \cite{fdo-RequirementSpec} clarifies these definitions with equivalent identifiers\footnote{Newer \cite{fdo-RequirementSpec} renames \texttt{FDOF*} to \texttt{FDOR*} but follows same ordering.} and relates them to further FDO requirements such as FDO Data Type Registries.

In Table \vref{ch3:fdo-checks} we evaluate completeness of the guidelines in two current realizations: 1) DOIPv2 \cite{DONA 2018} and 2) Linked Data Platform \cite{w3-ldp}, as proposed by \cite{bonino2021}.
We provide our analysis of each realisation with respect to the FDO Guideline and also provide suggestions for that realisation to meet the given guideline

\begin{landscape}
  \begin{small}
  \begin{longtable}[]{@{}
    >{\centering\arraybackslash}p{(\columnwidth - 8\tabcolsep) * \real{0.1}}
    >{\raggedleft\arraybackslash}p{(\columnwidth - 8\tabcolsep) * \real{0.25}}
    >{\raggedright\arraybackslash}p{(\columnwidth - 8\tabcolsep) * \real{0.2}}
    >{\raggedleft\arraybackslash}p{(\columnwidth - 8\tabcolsep) * \real{0.25}}
    >{\raggedright\arraybackslash}p{(\columnwidth - 8\tabcolsep) * \real{0.2}}@{}}
    \caption[Checking FDO guidelines against its implementations]{Checking FDO guidelines \cite{bonino2019,fdo-RequirementSpec} against its current implementations as DOIP \cite{DONA 2018} and Linked Data Platform (LDP) \cite{bonino2021}, with suggestions for required additions.
  \label{ch3:fdo-checks}}\tabularnewline
  \toprule
  \textbf{FDO Guideline} & 
  DOIP 2.0 & 
  FDO suggestions & 
  Linked Data Platform & 
  LDP suggestion \\
  \midrule
  \endfirsthead
  \toprule
  \textbf{FDO} & 
  DOIP 2.0 & 
  FDO suggestions & 
  Linked Data Platform & 
  LDP suggestion \\
  \midrule
  \endhead
G1: \emph{invest for many decades}
  & Handle system stable for 20 years, DOIP 2.0 since 2017.
  & Ensure FDO types will not be protocol-bound as DOIP might be updated/replaced
  & HTTP stable for 30 years, Semantic Web for 20 years. \texttt{http://} URIs mostly replaced by \texttt{https://}.
  & Keep flexibility of RDF serialisation formats which may change more frequently \\
G2: \emph{trustworthiness}
  & DOI/Handle trusted by all major academic publishers and data repositories. DOIP relatively unknown, in effect only one implementation.
  & Further promote DOIP and justify its benefits. Build tutorials and OSI open source implementations. Standardise DOIP-over-HTTP alternative.
  & JSON-LD used by half of all websites \cite{UsageStatisticsJSONLD}, however previous bad experiences with Semantic Web may deter adopters
  & Ensure simplicity for end developers, rather than semantic overengineering. Example-driven documentation. \\
G3: \emph{follows FAIR principles}
  & See Table \vref{ch3:fair-data-maturity-model}
  & Ensure all FAIR principles are covered, build complete examples.
  & Touched briefly, see Table \vref{ch3:fair-data-maturity-model}
  & Add explicit expression to show each FAIR principle fulfilled. \\
G4: \emph{machine actionability}
  & CRUD and extension operations dynamically listed (see Table \vref{ch3:fdo-web-middleware})
  & Specify which operations should work for a given type, to reduce need for dynamic lookup. Specify input/output expectations formally (e.g.~JSON Schema).
  & HTTP CRUD operations, Open API (see Table \vref{ch3:fdo-web-middleware})
  & Document operations so client can make subsequent HTTP calls. \\
G5: \emph{abstraction principle}
  & Handle PIDs as abstraction base. DOIP operations can use any transport protocol.
  & Document transport protocols as FDOs, recommend which transport to use.
  & URI as abstraction base. Does not specify PID requirements.
  & Give stronger deployment recommendations. \\
G6: \emph{stable binding between entities}
  & Machine-navigation through PIDs and operations specified per type. Unclear when metadata field is a PID or plain text.
  & Make datatype of fields explicit to support navigation.
  & Machine-navigation through URIs via properties and types. Unclear when URI should be followed or is just identifier, but always distinct from plain text.
  & \\
G7: \emph{encapsulation}
  & Operations discovered at runtime (\texttt{0.DOIP/Op.ListOperations}).
  & Allow method discovery by type FDOs in advance, see \cite{fdo-TypingFDOs}.
  & HTTP methods discovered at runtime (\texttt{OPTIONS}), indempotent methods attempted directly. Unsupported methods reported using LDP constraints to human-readable text.
  & Declare supported methods in advance, e.g.~OpenAPI \cite{OpenAPISpecificationV3} \\
G8: \emph{technology independence}
  & In theory independent, in reality depends on single implementations of Handle system and DOIP
  & Encourage open source DOIP testbeds and lighter reference implementations
  & Multiple HTTP implementations, multiple LDP implementations. No FDOF implementations.
  & Develop demonstrator of FDOF usage based on existing LDP server. \\
G9: \emph{standard compliance}
  & Handle \cite{rfc3650}, DOIP \cite{DONA 2018}. FDO requirements not standardised yet.
  & Formalise standard process of FDO requirements \cite{fdo-DocProcessStd}
  & HTTP, LDP. However FDOF is not yet standardised.
  & Formalise FDOF from FDOF-SEM working group. \\
FDOF1: \emph{PID as basis}
  & Extensive use of Handle system.
  & Clarify how local testing handles can be used during development, how ``temporary'' FDOs should evolve \cite{fdo-PIDProfileAttributes}. Register \texttt{0.DOIP/*} and \texttt{0.FDO/*} as actual PIDs.
  & HTTP URLs as basis for identifiers, but they are frequently not persistent.
  & Add strong guidance for PID services like w3id and persistence policies \cite{McMurry 2017}. \\
FDOF2: \emph{PID record w/ type}
  & Unspecified how to resolve from Handle to DOIP Service (!), in practice \texttt{10320/loc}, \texttt{0.TYPE/DOIPService}, \texttt{URL}, \texttt{URL\_REPLICA}
  & Document requirements for PID Record
  & w3id/purl PIDs redirect without giving any metadata. Datacite DOIs content-negotiate to give registered metadata.
  & Add FAIR Signposting \cite{Van de Sompel 2022} at PID provider for minimal PID record \\
FDOF3: \emph{PID resolvable to bytestream \& metadata}
  & Byte stream resolvable (\texttt{0.DOIP/Retrieve}), \texttt{includeElementData} option can retrieve bytestream or full object structure. No method/attribute defined for separate metadata, only directly in PID Record. Unclear meaning of multiple items and bytestream chunks.
  & Clarify expectations for multiple items. Recommend chunks to not be used.
  & URIs resolvable by default. Multiple ways to resolve metadata, unclear preference.
  & Add FAIR Signposting and preference order. \\
FDOF4: \emph{Additional attributes}
  & Freetext attribute keys. Attributes should be defined for FDO type (?).
  & Require that attribute keys should be PIDs (or have predefined mapping to PIDs). Explicitly allow attributes not already defined in type.
  & All attributes individually identified. Any Linked Data attributes can be used by URI or with mapped prefix.
  & Clarify type expectations of required/recommended/optional attributes. \\
FDOF5: \emph{Interface: operation by PID}
  & Extended operations use PID, but ``pid-like'' DOIP operations/types are not registered as handles.
  & Register \texttt{0.DOIP/*} and \texttt{0.FDO/*} as PIDs. Clarify that operations can be mapped to protocol directly.
  & CRUD operations used directly in HTTP (e.g.~\texttt{PUT}). Unclear how to provide PID for additional operations.
  & Specify how additional operations should be called over HTTP. \\
FDOF6: \emph{CRUD operations + extensions}
  & \texttt{0.DOIP/Op.Create}, \texttt{Op.Retrieve}, \texttt{Op.Update}, \texttt{Op.Delete} but also \texttt{0.DOIP/Op.Search}.
  & Document
  & \texttt{PUT}, \texttt{GET}, \texttt{POST}, \texttt{DELETE}, \texttt{PATCH}, \texttt{HEAD} -- extension operations (e.g.~WebDAV \texttt{COPY}) not used, resource patterns \cite{martinekuanWebAPIDesign} are used instead.
  & Document how operation resources can be discovered from an LPD container. Document search API. \\
FDOF7: \emph{FDOF Types related to operations}
  & Not yet formalised, by DOIP discoverable on a given FDO rather than type. PR-TypingFDOs leaves this open.
  & Add explicit relation between type and operations
  & \texttt{OPTIONS} per LDP Resource, but not by type. Common types (\texttt{ldp:Resource}, \texttt{ldp:Container}) indicate LDP support, but are not required.
  & Always make LDP types explicit in FDO profile. \\
FDOF8: \emph{Metadata as FDO, semantic assertions}
  & DOIP includes all metadata in PID Record. Separate Metadata FDO need custom property.
  & Specify a \texttt{0.FDO/metadata} or similar to point to Metadata FDOs.
  & Assertions are always with semantics, using RDF vocabularies. Unspecified how to find additional metadata resources, \texttt{rdfs:seeAlso} is common.
  & Use FAIR Signposting \texttt{describedby} link relation to additional metadata PIDs \\
FDOF9: \emph{Different metadata levels}
  & Defines open-ended ``Response Attributes'' without namespaces, but mandated as ``None'' for all CRUD operations. Metadata would need to be bundled within custom FDO types or attributes. Unclear how levels are separated within single FDO representation (need FDOF8?).
  & Declare which metadata are expected within response attribute or within FDO object. Require PIDs for custom attributes. Define how alternate metadata levels can be represented separately.
  & Undefined how to handle multiple metadata granularities or domains, alternative LDP containers can present different views on same stored objects.
  & Define how to navigate to alternate views and their semantic implications, e.g.~\texttt{owl:sameAs} \\
FDOF10: \emph{Metadata schemas by community}
  & Metadata schemas are in practice managed on single CORDA server as local types, using JSON Schema.
  & Require types to be FDOs with registered PIDs, implement shared types.
  & Plethora of existing RDF vocabularies/ontologies managed by larger communities, e.g.~\footurl{https://obofoundry.org/}{OBO Foundry} \cite{smithOBOFoundryCoordinated2007a}
  & Rather document better how individual ad-hoc schemas can be started for prototypes. \\
FDOF11: \emph{FDO collections w/ semantic relations}
  & Collection type undefined by DOIP. Informal use of \texttt{HAS\_PARTS} Handle attribute (e.g. \cite{DataInformationView}).
  &
  & LDP Containers required by specification, also user-created (eg. \texttt{BasicContainer}).
  & Clarify relation to other collections like DCAT 3 \cite{w3-vocab-dcat-3}, \footurl{https://schema.org/Dataset}{Schema.org Dataset}, OAI-ORE \cite{ORESpecificationAbstract} \\
FDOF12: \emph{Deleted FDO preserve PID w/ tombstone}
  & Tombstone for deleted resource undefined by DOIP. \texttt{0.DOIP/Status.104} status code does not distinguish ``Not Found'' or ``Gone''
  & Formalise tombstone requirements with new FDO type
  & \texttt{410\ Gone} recommended, but \texttt{404\ Not\ Found} common. No requirement for tombstone serialisation
  & Formalise tombstone requirements and serialisation \\
\bottomrule
\end{longtable}
\end{small}
\end{landscape}

A key observation from this is that simply using DOIP does not achieve many of the FDO guidelines. Rather the guidelines set out how a protocol like DOIPs should be used to achieve FAIR Digital Object goals. The DOIP Endorsement \cite{fdo-DOIPEndorsement} sets out that to comply, DOIP must be used according to the set of FDO requirement documents (details in section \vref{ch3:next-step-fdo}), and notes \emph{Achieving FDO compliance requires more than DOIP and full compliance is thus left to system designers}. Likewise, a Linked Data approach will need to follow the same requirements to comply as an FDO implementation.

\paragraph{Observations}

\begin{itemize}
  \item
    G1 and G2 call for stability and trustworthiness. While the foundations of both DOIP and Linked Data approaches are now well established -- the FDO requirements and in particular, how they can be implemented, are still taking shape and subject to change.
  \item
    Machine actionability (G4, G6) is a core feature of both FDOs and Linked Data. Conceptually they differ in the which way types and operations are discovered, with FDO seemingly more rigorous. In practice, however, we see that DOIP also relies on dynamic discovery of operations and that operation expectations for types (FDOF7) have not yet been defined.
  \item
    FDO proposes that types can have additional operations beyond CRUD (FDOF5, FDOF6), while Linked Data mainly achieves this with RESTful patterns using CRUD on additional resources, e.g.~\texttt{order/152/items}. These are mainly stylistics but affect the architectural view -- FDOs have more of an object-oriented approach.
  \item
    FDO puts strong emphasis on the use of PIDs (FDOF1, FDOF2, FDOF3, FDOF5), but in current practice DOIP use local types, local extended operations (FDOF5) and attributes (FDOF4) that are not bound to any global namespace.
  \item
    Linked Data have a strong emphasis on semantics (FDOF8), and metadata schemas developed by community agreements (FDOF10). FDO types need to be made reusable across servers.
  \item
    While FDO recommends nested metadata FDOs (FDOF8, FDOF9), in practice this is not found (or linked with custom keys), particularly due to lack of namespaces and the favouring of local types rather than type/property re-use. Linked Data frequently have multiple representations, but often not sufficiently linked (link relation \texttt{alternate} \cite{rfc8288}) or related (\texttt{prov:specializationOf} from \cite{w3-prov-o}).
  \item
    FDO collections are not yet defined for DOIP, while Linked Data seemingly have too many alternatives. LDP has specific native support for containers.
  \item
    Tombstones for deleted resources are not well supported, nor specified, for either approach, although the continued availability of metadata when data is removed is a requirement for FAIR principles (see RDA-A2-01M in Table \vref{ch3:RDA-A2-01M}).
  \item
    DOIP supports multiple chunks of data for an object (FDOF3), while Linked Data can support content-negotiation. In either case it can be unclear to clients what is the meaning or equivalence of any additional chunks.
  \end{itemize}

\subsubsection{Comparing FDO and Web as middleware infrastructures}\label{ch3:middleware}

In this section, we take the perspective that FDO principles are in effect proposing a global infrastructure of machine-actionable digital objects. As such we can consider implementations of FDO as \textbf{middleware infrastructures} for programmatic usage, and can evaluate them based on expectations for client and server developers.

We argue that the Web, with its now ubiquitous use of REST API \cite{fieldingArchitecturalStylesDesign2000a}, can be compared as a similar global middleware. Note that while early moves for developing Semantic Web Services \cite{fenselSemanticWebServices2011} attempted to merge the Web Service and RDF aspects, we are here considering mainly the current programmatic Web and its mostly light-weight use of 3 out of possible \emph{5 stars Linked Data} \cite{OpenData}.

For this purpose, we here utillise the Comparison Framework for Middleware Infrastructures \cite{zarrasComparisonFrameworkMiddleware2004a} that formalise multiple dimensions of openness, scalability, transparency, as well as characteristics known from Object-oriented programming such as modularity, encapsulation and inheritance.

\begin{landscape}
  \begin{small}
  \begin{longtable}[]{@{}
    >{\raggedright\arraybackslash}p{(\columnwidth - 4\tabcolsep) * \real{0.1642}}
    >{\raggedright\arraybackslash}p{(\columnwidth - 4\tabcolsep) * \real{0.4179}}
    >{\raggedright\arraybackslash}p{(\columnwidth - 4\tabcolsep) * \real{0.4179}}@{}}
    \caption[Comparing FAIR Digital Object and Web technologies as middleware infrastructures]{Comparing FAIR Digital Object (with the DOIP 2.0 protocol \cite{DONA 2018}) and Web technologies (using Linked Data) as middleware infrastructures \cite{zarrasComparisonFrameworkMiddleware2004a}
  \label{ch3:fdo-web-middleware}}\tabularnewline
  \toprule
  Quality & 
  FDO with DOIP & 
  Web with Linked Data \\
  \midrule
  \endfirsthead
  \toprule
  Quality & 
  FDO w/ DOIP & 
  Web w/ Linked Data \\
  \midrule
  \endhead
  \textbf{Openness}: \emph{framework enable extension of applications}
    & FDOs can be cross-linked using PIDs, pointing to multiple FDO endpoints. Custom DOIP operations can be exposed, although it is unclear if these can be outside the FDO server. PID minting requires Handle.net prefix subscription, or use of services like \footurl{https://datacite.org/}{Datacite}, \footurl{https://eudat.eu/services/userdoc/b2handle}{B2Handle}.
    & The Web is inherently open and made by cross-linked URLs. Participation requires DNS domain purchase (many free alternatives also exists). PID minting can be free using PURL/ARK services, or can use DOI/Handle with HTTP redirects. \\
  \textbf{Scalability}: \emph{application should be effective at many different scales}
    & No defined methods for caching or mirroring, although this could be handled by backend, depending on exposed FDO operations (e.g.~Cordra can scale to multiple backend nodes)
    & Cache control headers reduce repeated transfer and assist explicit and transparent proxies for speed-up. HTTP \texttt{GET} can be scaled to world-population-wide with Content-Delivery Networks (CDNs), while write-access scalability is typically manage by backend. \\
  \textbf{Performance}: \emph{efficient and predictable execution}
    & DOIP has been shown moderately scalable to 100 millions of objects, create operation at 900 requests/second. DOIP protocol is reusable for many operations, multiple requests may be answered out of order (by \texttt{requestId}). Multiple connections possible. Setup is typically through TCP and TLS which adds latency.
    & HTTP traffic is about 10\% of global Internet traffic, excluding video and social networks \cite{sandvineGlobalInternetPhenomena}. HTTP 1 connections are serial and reusable, and concurrent connections is common. HTTP/2 adds asynchronous responses and multiplexed streams \cite{rfc7540} but still has TCP+TLS startup costs. For reduced latency, HTTP/3 \cite{rfc9114} use QUIC \cite{rfc9000} rather than TCP, already adapted heavily (30\% of EMEA traffic) of which Instagram \& Facebook video is the majority of traffic \cite{joras2020}. \\
  \textbf{Distribution transparency}: \emph{application perceived as a consistent whole rather than independent elements.}
    & Each FDO is accessed separately along with its components (typically from the same endpoint). FDOs should provide the mandatory kernel metadata fields. FDOs of the same declared type typically share additional attributes (although that schema may not be declared). DOIP does not enforce metadata typing constraints, this need to be established as FDO conventions.
    & Each URL accessed separately. Common HTTP headers provide basic metadata, although it is often not reliable. A multitude of schemas and serializations for metadata exists, conventions might be implied by a declared profile or certain media types. Metadata is not always machine findable, may need pre-agreed API URI Templates \cite{rfc6570}, content-negotiation \cite{ContentNegotiationHTTP} or FAIR Signposting \cite{Van de Sompel 2022}. \\
  \textbf{Access transparency}: \emph{local/remote elements accessed similarly}
    & FDOs should be accessed through PID indirection, this means difficult to make private test setup. Commonly a fixed DOIP server is used directly, which permits local non-PID identifiers.
    & Global HTTP protocol frequently used locally and behind firewalls, but at risk of non-global URIs (e.g.~\texttt{http://localhost/object/1}) and SSL issues (e.g.~self-signed certificates, local CAs) \\
  \textbf{Location transparency}: \emph{elements accessed without knowledge of physical location}
    & FDOs always accessed through PIDs. Multiple locations possible in Handle system, can expose geo-info.
    & PIDs and URL redirects. DNS aliases and IP routing can hide location. Geo-localised servers common for large cloud deployments. \\
  \textbf{Concurrency transparency}: \emph{concurrent processing without interference}
    & No explicit concurrency measures. FDO kernel metadata can include checksum and date.
    & HTTP operations are classified as being stateless/idempotent or not (e.g.~\texttt{PUT} changes state, but can be repeated on failure), although these constraints are occassionally violated by Web applications. Cache control, \texttt{ETag} (e.g. checksum) and modification date in HTTP headers allows detection of concurrent changes on a single resource. \\
  \textbf{Failure transparency}: \emph{service provisioning resilient to failures}
    & DOIP status codes, e.g.~\texttt{0.DOIP/Status.104}, additional codes can be added as custom attributes
    & HTTP \footurl{https://datatracker.ietf.org/doc/html/rfc7231\#section-6.5}{status codes} e.g.~\texttt{404\ Not\ Found}, specific meaning of standard codes can be \footurl{https://swagger.io/docs/specification/describing-responses/}{documented in Open API}. Custom codes uncommon. \\
  \textbf{Migration transparency}: \emph{allow relocating elements without interfering application}
    & Update of PID record URLs, indirection through \texttt{0.TYPE/DOIPServiceInfo} (not always used consistently). No redirection from DOIP service.
    & HTTP \texttt{30x} status codes provide temporary or permanent redirections, commonly used for PURLs but also by endpoints. \\
  \textbf{Persistence transparency}: \emph{conceal deactivation/reactivation of elements from their users}
    & FDO requires use of PIDs for object persistence, including a tombstone response for deleted objects. There is no guarantee that an FDO is immutable or will even stay the same type (note: Cordra extends DOIP with \footurl{https://www.cordra.org/documentation/design/object-versioning.html}{version tracking}).
    & URLs are not required to persist, although encouraged \cite{berners-lee-cool-uris}. Persistence requires convention to use PIDs/PURLs and HTTP \texttt{410\ Gone}. An URL may change its content, change in type may sometimes force new URLs if exposing extensions like \texttt{.json}. Memento \cite{rfc7089} expose versioned snapshots. WebDAV \texttt{VERSION-CONTROL} method \cite{rfc3253} (used by SVN). \\
  \textbf{Transaction transparency}: \emph{coordinate execution of atomic/isolated transactions}
    & No transaction capabilities declared by FDO or DOIP. Internal synchronisation possible in backend for Extended operations.
    & Limited transaction capabilities (e.g.~\texttt{If-Unmodified-Since}) on same resource. WebDAV \footurl{https://datatracker.ietf.org/doc/html/rfc4918\#section-6}{locking mechanisms} \cite{rfc4918} with \texttt{LOCK} and \texttt{UNLOCK} methods. \\
  \textbf{Modularity}: \emph{application as collection of connected/distributed elements}
    & FDOs are inheritedly modular using global PID spaces and their cross-references. In practice, FDOs of a given type are exposed through a single server shared within a particular community/institution.
    & The Web is inheritently modular in that distributed objects are cross-referenced within a global URI space. In practice, an API's set of resources will be exposed through a single HTTP service, but modularity enables fine-grained scalability in backend. \\
  \textbf{Encapsulation}: \emph{separate interface from implementation. Specify interface as contract, multiple implementations possible}
    & Indirection by PID gives separation. FDO principles are protocol independent, although it may be unclear which protocol to use for which FDO (although \texttt{0.DOIP/Transport} can be specified after already contacting DOIP). Cordra supports \footurl{https://www.cordra.org/documentation/api/doip.html}{native DOIP}, \footurl{https://www.cordra.org/documentation/api/doip-api-for-http-clients.html}{DOIP over HTTP} and \footurl{https://www.cordra.org/documentation/api/rest-api.html}{Cordra REST API}
    & HTTP/1.1 semantics can seemlessly upgrade to HTTP/2 and HTTP/3. \texttt{http} vs \texttt{https} URIs exposes encryption detail\protect\footnote{The \texttt{http} protocol (port 80) can in theory also upgrade \cite{rfc2817} to TLS encryption, as commonly used by \emph{Internet Printing Protocol} (\url{https://www.rfc-editor.org/rfc/rfc8010.html\#section-8.2}) for \texttt{ipp} URIs, but on the Web, best practice is explicit \texttt{https} (port 443) URLs to ensure following links stay secure.}. Implementation details may leak into URIs (e.g.~\texttt{search.aspx}), countered by deliberate design of URI patterns \cite{berners-lee-cool-uris} and PIDs via Persistent URLs (PURL). \\
  \textbf{Inheritance}: \emph{Deriving specialised interface from another type}
    & DOIP types nested with parents, implying shared FDO structures (unclear if operations are inherited). FDO establishes need for multiple Data Type Registries (e.g.~managed by a community for a particular domain). Semantics of type system currently undefined for FDO and DOIP, syntactic types can also piggyback of FDO type's schema (e.g.~\footurl{https://www.cordra.org/documentation/design/schemas.html\#schema-references}{Cordra \texttt{\$ref}} use of \footurl{https://json-schema.org/draft/2020-12/json-schema-core.html\#references}{JSON Schema references} \cite{Draftbhuttonjsonschema})
    & Syntactically Media Type with multiple suffixes \cite{Draftietfmediamansuffixes00MediaTypes} (mainly used with \texttt{+json}), declaration of subtypes as profiles (RFC6906) \cite{rfc6906}. In metadata, semantic type systems (RDFS \cite{w3-rdf-schema}), OWL2 \cite{w3-owl2-overview}, SKOS \cite{w3-skos-primer}). OpenAPI 3 \cite{OpenAPISpecificationV3} \footurl{https://spec.openapis.org/oas/v3.1.0\#composition-and-inheritance-polymorphism}{inheritance and Polymorphism}. XML \texttt{xsd:schemaLocation} or \texttt{xsd:type} \cite{w3-xmlschema11}, JSON \texttt{\$schema} \cite{Draftbhuttonjsonschema}), JSON-LD \texttt{@context} \cite{w3-json-ld}. Large number of domain-specific and general ontologies define semantic types, but finding and selecting remains a challenge. \\
  \textbf{Signal interfaces}: \emph{asynchronous handling of messages}
    & DOIP 2.0 is synchronous, in FDO async operations undefined. Could be handled as custom jobs/futures FDOs
    & HTTP/2 \footurl{https://datatracker.ietf.org/doc/html/rfc7540\#section-5}{multiplexed streams} \cite{rfc7540}, Web Sockets \cite{WebSocketsStandard}, Linked Data Notifications \cite{w3-ldn}, AtomPub \cite{rfc5023}, SWORD \cite{SWORDSpecification}, Micropub \cite{w3-micropub}, more typically ad-hoc jobs/futures REST resources \\
  \textbf{Operation interfaces}: \emph{defining operations possible on an instance, interface of request/response messages}
    & CRUD predefined in DOIP, custom operations through \texttt{0.DOIP/Op.ListOperations} (can be FDOs of type \texttt{0.TYPE/DOIPOperation}, more typically local identifiers like \texttt{"getProvenance"})
    & CRUD predefined in \footurl{https://datatracker.ietf.org/doc/html/rfc7231\#section-4.3}{HTTP methods} \cite{rfc7231}, \footurl{https://www.iana.org/assignments/http-methods/http-methods.xhtml}{(extended by registration)}, URI Templates \cite{rfc6570}, \footurl{https://spec.openapis.org/oas/v3.1.0.html\#operation-object}{OpenAPI operations} \cite{OpenAPISpecificationV3}, HATEOAS\footnote{HATEOAS: Hypermedia as the Engine of Application State \cite{fieldingArchitecturalStylesDesign2000a}, an important element of the REST architectural style.} incl.~Hydra \cite{HydraW3CCommunity}, schema.org Actions \cite{SchemaOrgActions}, JSON HAL \cite{Draftkellyjsonhal08} \& Link headers (RFC8288) \cite{rfc8288} \\
  \textbf{Stream interfaces}: \emph{operations that can handle continuous information streams}
    & Undefined in FDO. DOIP can support multiple byte stream elements (need custom FDO type to determine stream semantics)
    & HTTP 1.1 \cite{rfc7230} \footurl{https://datatracker.ietf.org/doc/html/rfc7230\#section-4.1}{chunked transfer}, HLS (RFC8216) \cite{rfc8216}, MPEG-DASH \cite{iso23009} \\
  \bottomrule
  \end{longtable}
  \end{small}
  \end{landscape}
  
\paragraph{Observations}

Based on the analysis in Table \vref{ch3:fdo-web-middleware}, we make the following observations:

\begin{itemize}
  \item
    With respect to the aspect of \emph{Performance}, it is interesting to note that while the first version of DOIP \cite{DigitalObjectInterface} supported multiplexed channels similar to HTTP/2 (allowing concurrent transfer of several digital objects). Multiplexing was removed for the much simplified DOIP 2.0 \cite{DONA 2018}. Unlike DOIP 1.0, DOIP 2.0 will require a DO response to be sent back completely, as a series of segments (which again can be split the bytes of each binary \emph{element} into sized \emph{chunks}), before transmission of another DO response can start on the transport channel. It is unclear what is the purpose of splitting a binary into chunks on a channel which no longer can be multiplexed and the only property of a chunk is its size\footnote{Although it is possible with \texttt{0.DOIP/Op.Retrieve} to request only particular individual elements of an DO (e.g.~one file), unlike HTTP's \texttt{Range} request, it is not possible to select individual chunks of an element's bytestream.}.
  \item
    HTTP has strong support for scalability and caching, but this mostly assumes read-operations from static resources. FDO has no view on immutability or validity of retrieved objects, but this should be taken into consideration to support large-scale usage.
  \item
    HTTP optimisations for performance (e.g.~HTTP/2, multiplexing) are largely used for commercial media distribution (e.g.~Netflix), and not commonly used by providers of FAIR data
  \item
    Cloud deployment of Web applications give many middleware benefits (Scalability, Distribution, Access transparancy, Location transparancy) -- it is unclear how DOIP as a custom protocol would perform in a cloud setting as most of this infrastructure assumes HTTP as the protocol.
  \item
    Programmatically the Web is rather unstructured as middleware, as there are many implementation choices. Usually it is undeclared what to expect for a given URI/service, and programmers follow documented examples for a particular service rather than automated programmatic exploration across providers. This mean one can consider the Web as an ecosystem of smaller middlewares with commonalities.
  \item
    Many providers of FAIR Linked Data also provide programmatic REST API endpoints, e.g.~\footurl{https://www.uniprot.org/help/programmatic_access}{UNIPROT}, \footurl{https://chembl.gitbook.io/chembl-interface-documentation/web-services}{ChEMBL}, but keeping the FAIR aspects such as retrieving metadata in such a scenario may require combining different services using multiple formats and identifier conventions.
\end{itemize}


\subsubsection{Assessing FDO against FAIR}\label{ch3:fair-compare}

In addition to having ``FAIR'' in its name, the FAIR Digital Object guidelines \cite{fdo-RequirementSpec} also include \emph{G3: FDOs must offer compliance with the FAIR principles through measurable indicators of FAIRness}.

Here we evaluate to what extent the FDO guidelines and its implementation with DOIP and Linked Data Platform \cite{bonino2021} comply with the FAIR principles \cite{Wilkinson 2016}. Here we have used the RDA's FAIR Data Maturity Model \cite{groupFAIRDataMaturity2020} as it has decomposed the FAIR principles to a structured list of FAIR indicators \cite{bahimFAIRDataMaturity2020a}, importantly considering \emph{Data} and \emph{Metadata} separately. In our interpretation for Table \vref{ch3:fair-data-maturity-model} we have for simplicity chosen to interpret ``data'' in FDOs as the associated bytestream of arbitrary formats, with remaining JSON or RDF structures always considered as metadata.

\begin{landscape}
  \begin{small}
  \begin{longtable}[]{@{}
    >{\raggedright\arraybackslash}p{(\columnwidth - 10\tabcolsep) * \real{0.1}}
    >{\raggedright\arraybackslash}p{(\columnwidth - 10\tabcolsep) * \real{0.2}}
    >{\centering\arraybackslash}p{(\columnwidth - 10\tabcolsep) * \real{0.1}}
    >{\centering\arraybackslash}p{(\columnwidth - 10\tabcolsep) * \real{0.2}}
    >{\centering\arraybackslash}p{(\columnwidth - 10\tabcolsep) * \real{0.2}}
    >{\centering\arraybackslash}p{(\columnwidth - 10\tabcolsep) * \real{0.2}}@{}}
    \caption[Assessing RDA's FAIR Data Maturity Model against the FDO guidelines]{Assessing RDA's FAIR Data Maturity Model \cite{groupFAIRDataMaturity2020,bahimFAIRDataMaturity2020a} (first 2 columns) against the FDO guidelines \cite{bonino2019}, FDO implemented with the protocol DOIPv2 \cite{DONA 2018}, Linked Data Platform (LDP) \cite{bonino2021} and examples from Linked Data practices in general. (--- indicates \emph{Unspecified}, may be possible with additional conventions)
  \label{ch3:fair-data-maturity-model}}\tabularnewline
  \toprule
  FAIR ID &
  Indicator &
  FDO guidelines &
  FDO/DOIP &
  FDO/LDP &
  Linked Data examples \\
  \midrule
  \endfirsthead
  \toprule
  FAIR ID &
  Indicator &
  FDO &
  FDO/DOIP &
  FDO/LDP &
  LD examples \\
  \midrule
  \endhead
RDA-F1-01M
  & Metadata is identified by a persistent identifier
  & FDOF4
  & Optional \emph{Metadata FDO} w/separate PID
  & Content-negotiation to URL, not required to be PID
  & Metadata typically don't have own PID \\
RDA-F1-01D
  & Data is identified by a persistent identifier
  & FDOF1
  & PIDs required (FDOF1). Handle, DOI.
  & FDOF-IR (Identifier Record). PID can be any URI
  & ``Cool'' URIs \cite{berners-lee-cool-uris}, PURL services incl.~\texttt{purl.org}, \texttt{w3id.org} \\
RDA-F1-02M
  & Metadata is identified by a globally unique identifier
  & FDOR4 FDOF8
  & Optional \emph{Metadata FDO}, unspecified how to indicate
  & Content-negotiation to URL
  & Not required, content-negotiation can redirect to URL or \texttt{Content-Location}. FAIR Signposting. \\
RDA-F1-02D
  & Data is identified by a globally unique identifier
  & FDOF1
  & All FDOs have PIDs (FDOR1), DOIP uses Handle system
  & FDOF-IR (Identifier Record)
  & Always accessed by URL \\
RDA-F2-01M
  & Rich metadata is provided to allow discovery
  & FDOF2 FDOF4 FDOF8 FDOF9
  & FDO has key-value metadata. Unclear how to link to additional metadata.
  & FDOF-IR links to multiple metadata records
  & RDF-based metadata by content negotiation or FAIR Signposting. Embedded in landing page (RDFa). \\
RDA-F3-01M
  & Metadata includes the identifier for the data
  & ---
  & \texttt{id} and \texttt{type} are required metadata elements PIDs, also implicit as requests must use PID
  & PID only required in FDOF-IR record.
  & PID inclusion typical, but often inconsistent (e.g.~\texttt{www.example.com} vs \texttt{example.com}) or missing (use of \texttt{\textless{}\textgreater{}} as \emph{this} subject) \\
RDA-F4-01M
  & Metadata is offered in such a way that it can be harvested and indexed
  & FDOF10
  & No, registries not required (except Data Type Registries). Handle registry only searchable by PID.
  & ---
  & Not specified, several registries/catalogues for vocabularies/types (e.g. \cite{NCBOBioPortal}). Indexing by search engines if exposing HTML w/schema.org. \\
RDA-A1-01M
  & Metadata contains information to enable the user to get access to the data
  & FDOF3 FDOF6
  & Directly by DOIP, but not included in FDO metadata. \texttt{handle.net} HTTP resolution may redirect to landing page
  & Any property can point to URIs, but unclear if it is data
  & Common with clickable ``follow your nose'' URLs \\
RDA-A1-02M
  & Metadata can be accessed manually (i.e.~with human intervention)
  & ---
  & (Cordra HTML landing page from \texttt{handle.net} URIs)
  & Optional content-negotiation, e.g.~by Apache Marmotta, OpenLink Virtuoso
  & HTTP content-negotiation to HTML is common \\
RDA-A1-02D
  & Data can be accessed manually (i.e.~with human intervention)
  & ---
  & (Cordra HTML landing page from \texttt{handle.net} URIs)
  & Optional content-negotiation
  & Direct download, HTML landing pages common for DOIs \\
RDA-A1-03M
  & Metadata identifier resolves to a metadata record
  & FDOF8+FDOF2
  & ---
  & ---
  & \texttt{Content-Location} or HTTP redirection may indicate metadata URI \\
RDA-A1-03D
  & Data identifier resolves to a digital object
  & FDOF2
  & Required, but frequently not directly resolvable
  & Recommended, but any URI acceptable
  & Resolvable HTTP/HTTPS URIs are most common, now infrequent URNs are not directly resolvable \\
RDA-A1-04M
  & Metadata is accessed through standardised protocol
  & G9 FDOF3
  & Retrievable from PID (FDOF3). Informal DOIP standard maintained by DONA Foundation
  & LDP standard maintained by W3C, HTTP standards maintained by IETF, FDO components resolved by informal proposals (custom vocabulary, extra HTTP methods) or HTTP content negotiation)
  & Formal HTTP standards maintained by IETF, HTTP content negotiation, informal FAIR Signposting \\
RDA-A1-04D
  & Data is accessible through standardised protocol
  & G9
  & (see above)
  & HTTP \cite{rfc9110}
  & HTTP/HTTPS, FTP (now less common), GridFTP \cite{allcockGlobusStripedGridFTP} (for large data), ARK \cite{ARKIdentifierScheme} \\
RDA-A1-05D
  & Data can be accessed automatically (i.e.~by a computer program)
  & G4 FDOF3 FDOF6
  & Required, but few client libraries
  & HTTP \texttt{GET}, content-negotiation for \texttt{fdof/object}
  & Ubiquitous, hundreds of HTTP libraries \\
RDA-A1.1-01M
  & Metadata is accessible through a free access protocol    
  & G1 G8 G9
  & Partially realised: Handle system is open\footnote{
        The \texttt{Handle.net} system was previously covered by software patent US6135646A(\url{https://patents.google.com/patent/US6135646A/en}) which expired in 2013 (\url{https://circleid.com/posts/20161025_selling_dona_snake_oil_at_the_itu\#11461})} 
    protocol \cite{rfc3652}. One server implementation \cite{HandleNetRegistry}, free\footnote{
        The \footurl{http://www.handle.net/HNRj/HNR-9-License.pdf}{Handle.net public license} is not OSI-approved \cite{LicensesStandardsOpen}  as an open source license -- it includes usage restrictions and requires Service Agreements. It is not a DOIP requirement to host a local Handle instance, e.g.~EOSC provides the \emph{B2HANDLE} service for acquiring Handle prefixes (\url{https://sp.eudat.eu/catalog/resources/fc6b2d30-09cd-4c25-b71a-7bc6de77910c}).}. 
    One DOIPv2 implementation \footurl{https://www.cordra.org/}{(Cordra)}: free under BSD-like license (not recognised as Open Source).    
  & LDP is open W3C recommendation \cite{w3-ldp}. \footurl{https://www.w3.org/wiki/LDP_Implementations}{Multiple LDP implementations}.    
  & DNS, HTTP, TLS, RDF standards are open, free and universal, large number of Open Source clients and \footurl{https://en.wikipedia.org/wiki/Comparison_of_web_server_software}{servers}. \\
RDA-A1.1-01D
  & Data is accessible through a free access protocol
  & G9
  & (see above)
  & URI, DNS, HTTP, TLS
  & URI, DNS, HTTP, TLS. Non-free DRM may be used (e.g.~subscription video streaming) \\
RDA-A1.2-01D
  & Data is accessible through an access protocol that supports authentication and authorisation
  & (FDOR9)
  & TLS certificates, \texttt{authentication} field (details unspecified)
  & Implied
  & HTTP authentication, TLS certificates \\
RDA-A2-01M\label{ch3:RDA-A2-01M}
  & Metadata is guaranteed to remain available after data is no longer available
  & FDOF12
  & ---
  & Unspecified, however FDOF-IR links to separate metadata records
  & --- \\
RDA-I1-01M
  & Metadata uses knowledge representation expressed in standardised format
  & FDOF8
  & Required, but not currently defined
  & ---
  & Always implied by use of RDF syntaxes. \\
RDA-I1-01D
  & Data uses knowledge representation expressed in standardised format
  & ---
  & ---
  & ---
  & Common (e.g.~HDF5, JSON, XML), yet common scientific data formats frequently not standardised \\
RDA-I1-02M
  & Metadata uses machine-understandable knowledge representation
  & FDOF8
  & Required
  & Optional RDF metadata with any vocabulary
  & Always implied by use of RDF syntaxes. \\
RDA-I1-02D
  & Data uses machine-understandable knowledge representation
  & G4 G7 FDOR2
  & No requirements on binary data formats
  & Only indirectly, \footurl{https://www.w3.org/TR/ldp/\#dfn-linked-data-platform-basic-container}{LDP Basic Container} reference only information resources
  & Common, specially for scientific data formats \\
RDA-I2-01M
  & Metadata uses FAIR-compliant vocabularies
  & G3 FDOF10
  & Informally required
  & Unspecified, implied by use of RDF?
  & FAIR practices for LD vocabularies increasingly common, sometimes inconsistent (e.g.~PURLs that don't resolve) or incomplete (e.g.~unknown license) \\
RDA-I2-01D
  & Data uses FAIR-compliant vocabularies
  & ---
  & ---
  & ---
  & Uncommon, except for some XML and RDF-embedding formats, e.g.~Extensible Metadata Platform (XMP) \cite{iso16684} \\
RDA-I3-01M
  & Metadata includes references to other metadata
  & FDOR8
  & Implied (attributes to PIDs), currently unspecified if given attribute is value or reference
  & ---
  & By definition (Linked Data reference existing URIs \cite{DataW3C}), \texttt{rdfs:seeAlso}, FAIR signposting \cite{Van de Sompel 2022} \texttt{describedby} \\
RDA-I3-01D
  & Data includes references to other data
  & G6 FDOR3 FDOR11
  & ---
  & ---
  & URL hyperlinks common in several formats (HTML, PDF, JSON, XML). \\
RDA-I3-02M
  & Metadata includes references to other data
  & G6 FDOR3 FDOR8
  & Implied from custom FDO type's attribute
  & LDP Direct Container members can be any resources
  & URI objects are frequently data references, may be indirect via PID \\
RDA-I3-02D
  & Data includes qualified references to other data
  & FDOR3 FDOR11
  & Only indirectly through FDO metadata
  & Indirectly through LDP membership
  & Uncommon: Link relations, FAIR Signposting \\
RDA-I3-03M
  & Metadata includes qualified references to other metadata
  & (FDOR3)
  & Qualification by attribute keys defined per FDO Type
  & \footurl{https://www.w3.org/TR/ldp/\#dfn-linked-data-platform-direct-container}{LDP Direct Container}
  & Qualifications by property, PROV bundles \cite{w3-prov-links}, \footurl{https://schema.org/Role}{schema.org/Role} \\
RDA-I3-04M
  & Metadata include qualified references to other data
  & (FDOR3)
  & Qualification by attribute keys defined per FDO type
  & \footurl{https://www.w3.org/TR/ldp/\#dfn-linked-data-platform-indirect-container}{LDP Indirect Container}
  & Qualifications by property, n-ary indirection (schema.org Role \cite{hollandIntroducingRole2014}, \texttt{prov:specializationOf} \cite{w3-prov-o}, OAI-ORE Proxy \cite{ORESpecificationAbstract}) \\
RDA-R1-01M
  & Plurality of accurate and relevant attributes are provided to allow reuse
  & FDOF4
  & Required. Kernel metadata attributes desired \cite{fdo-KernelAttributes} but not assigned PIDs yet.
  & Unspecified. Multiple metadata records can allow multiple semantic profiles.
  & Large number of general and domain-specific vocabularies can make it hard to find relevant attributes. Rough consensus on kernel metadata: schema.org \cite{schema.org}, Dublin Core Terms \cite{DCMIMetadataTerms}, DCAT \cite{DCAT2 2020}, FOAF \cite{FOAFVocabularySpecification} \\
RDA-R1.1-01M
  & Metadata includes information about the licence under which the data can be reused
  & ---
  & \texttt{licenseConditions} URL/PID in kernel metadata \cite{fdo-KernelAttributes}
  & ---
  & Dublin Core Terms \texttt{dct:license} frequently recommended, frequently not required, e.g.~\footurl{https://www.w3.org/TR/vocab-dcat-2/\#Property:distribution_license}{by DCAT 2} \cite{DCAT2 2020} \\
RDA-R1.1-02M
  & Metadata refers to a standard reuse licence
  & ---
  & ---
  & ---
  & \footurl{https://spdx.org/licenses/}{SPDX} and \footurl{https://creativecommons.org/}{Creative Commons} URIs common, identifiers often inconsistent \\
RDA-R1.1-03M
  & Metadata refers to a machine-understandable reuse licence
  & ---
  & ---
  & ---
  & \footurl{https://spdx.dev/resources/use/\#documents}{SPDX documents} uncommon \\
RDA-R1.2-01M
  & Metadata includes provenance information according to community-specific standards
  & FDOR9 FDOR10
  & Unspecified (some Cordra types add \texttt{getProvenance} methods). PID Kernel attributes? 
  & ----
  & Unspecified W3C PROV-O, PAV \\
RDA-R1.2-02M
  & Metadata includes provenance information according to a cross-community language
  & FDOR9 FDOR8
  & ---
  & ---
  & W3C PROV-O \cite{w3-prov-o}, PAV \cite{ciccaresePAVOntologyProvenance2013e}, Dublin Core Terms \cite{DCMIMetadataTerms} \\
RDA-R1.3-01M
  & Metadata complies with a community standard
  & FDOR10 FROR8
  & (Emerging, e.g.~DiSSCo Digital Specimen \cite{Hardisty 2022})
  & ---
  & Common, e.g.~DCAT 2 \cite{DCAT2 2020}, BioSchemas \cite{Gray 2017} \\
RDA-R1.3-01D
  & Data complies with a community standard
  & (FDOR3)
  & ---
  & ---
  & Common, HTTP use registered IANA \footurl{https://www.iana.org/assignments/media-types/media-types.xhtml}{media types}, additional scientific file formats frequently not standardised or identified \\
RDA-R1.3-02M
  & Metadata is expressed in compliance with a machine-understandable community standard
  & FDOF4 FDOF10
  & Recommended
  & ---
  & Common practice for ontologies, specially in bioinformatics, e.g.~BioPortal \cite{NCBOBioPortal}, Darwin Core \cite{wieczorekDarwinCoreEvolving2012} \\
RDA-R1.3-02D
  & Data is expressed in compliance with a machine-understandable community standard
  & (FDOR2)
  & No, FDO is typed but data can be any bytestream
  & ---
  & Occassionally, (e.g.~\footurl{https://github.com/The-Sequence-Ontology/Specifications/blob/master/gff3.md}{GFF3}, \footurl{https://fits.gsfc.nasa.gov/fits_standard.html}{FITS}, \footurl{https://www.loc.gov/preservation/digital/formats/fdd/fdd000280.shtml}{ESRI}) \\
\bottomrule
\end{longtable}
\end{small}
\end{landscape}

\paragraph{Observations}

\begin{itemize}
\item
  Linked Data in general is strong on metadata indicators, but LDP approach is weak as it has little concrete metadata guidance.
\item
  FDO/DOIP are stronger on identifier indicators, while Linked Data approach for identifiers relies on best practices. 
\item
  Indicators on standard protocols (RDA-A1-04M, RDA-A1-04D, RDA-A1.1-01M, RDA-A1.1-01D) favour LDP's mature standards (HTTP, URI) -- the DOIPv2 specification \cite{DONA 2018} has currently only a couple of implementations and is expressed informally. The underlying Handle system for PIDs is arguably mature and commonly used by researchers (this article alone references about 80 DOIs), however DOIs are more commonly accessed as HTTP redirects through resolvers like \url{https://doi.org/} and \url{http://hdl.handle.net/} rather than the Handle protocol.
\item
  RDA-A1-02M and RDA-A1-02D highlights access by manual intervention, which is common for http/https URIs, but also using above PID resolvers for DOIP implementation \footurl{https://www.cordra.org/}{Cordra} (e.g.~\url{https://hdl.handle.net/21.14100/90ec1c7b-6f5e-4e12-9137-0cedd16d1bce}), yet neither LDP, FDO nor DOIP specifications recommends human-readable representations to be provided
\item
  Neither DOIP nor LDP require license to be expressed (RDA-R1.1-01M, RDA-R1.1-02M, RDA-R1.1-03M), yet this is crucial for re-use and machine actionability of FAIR data and metadata to be legal
\item
  Machine-understandable types, provenance and data/metadata standards (RDA-R1.1-03M RDA-R1.3-02M, RDA-R1.3-02M, RDA-R1.3-02D) are important for machine actionability, but are currently unspecified for FDOs. \cite{fdo-ImplAttributesTypesProfiles} explores possible machine-readable FDO types, however the type systems themselves have not yet been formalised. Linked Data on the other side have too many semantic and syntactic type systems, making it difficult to write consistent clients.
\item
  Indicators for FAIR data are weak for either approach, as too much reliance is put on metadata. For instance in Linked Data, given a URL of a CSV file, what is its persistant identifier or license information? Signposting \cite{vandesompel2015} can improve findability of metadata using HTTP Link relations, which enable an FDO-like overlay for any HTTP resource. In DOIP, responses for bytestreams can include the data identifier: if that is a PID (not enforced by DOIP), its metadata is accessible.
\item
  Resolving FDOs via Handle PIDs to the corresponding DOIP server is currently undefined by FDO and DOIP specifications. \texttt{0.TYPE/DOIPServiceInfo} lookup is only possible once DOIP server is known.
\end{itemize}


\subsubsection{EOSC Interoperability Framework}\label{ch3:eosc-interoperability-framework}

The European Open Science Cloud (EOSC) is a large EU initiative to promote Open Science by implementing a joint research infrastructure by federating existing and new services and focusing on interoperability, accessability, best practices as well as technical infrastructure \cite{10.2777/940154}. The EOSC Interoperability Framework \cite{eosc-interop-framework} details the principles for creating a common way to achieve interoperability between all digital aspects of research activities in EOSC, including data, protocols and software. The recommendations are realised through 4 layers, Technical (e.g. protocols), Semantic (e.g. metadata models), Organisational (e.g. recommendations) and Legal (e.g. agreements), with a particular aim to address the FAIR interoperability principles and building on the concept of FAIR Digital Objects. 

As covered in our introduction in section \vref{ch3:introduction}, EOSC proposes FAIR Digital Objects as a way to improve interoperability, for instance invoked by scientific workflows, carried by metadata frameworks and semantic artefacts. Therefore, we here find it important to summarize how FDO and Linked Data can help satisfy the EOSC requirements.

In Table \vref{ch3:eosctable} we review the EOSC Interoperability Framework (EOSC IF) recommendations, and evaluate to what extent they are addressed by the principles of FDO and Linked Data or their common implementations.

%\begin{landscape}
  \begin{longtable}[]{@{}
    >{\raggedright\arraybackslash}p{(\columnwidth - 6\tabcolsep) * \real{0.15}}
    >{\raggedright\arraybackslash}p{(\columnwidth - 6\tabcolsep) * \real{0.30}}
    >{\raggedright\arraybackslash}p{(\columnwidth - 6\tabcolsep) * \real{0.30}}
    >{\raggedright\arraybackslash}p{(\columnwidth - 6\tabcolsep) * \real{0.30}}@{}}
  \caption[Assessing EOSC Interoperability Framework, against FDO \& Linked Data]{Assessing EOSC Interoperability Framework \cite[section 3.6]{eosc-interop-framework} against the FDO guidelines \cite{bonino2019} and Linked Data practices.}
  \label{ch3:eosctable}\tabularnewline
  \hline
  \toprule
  Layer &
  Recommendation &
  FDO &
  Linked Data \\
  \midrule
  \endfirsthead
  \hline
  \toprule
  Layer &
  Recommendation &
  FDO &
  Linked Data \\
  \midrule
  \endhead
  Technical      & Open Specification 
    & FDO specifications are semi-open, process gradually more transparent 
    & Open and transparent standard processes through W3C \& IETF \\
  Technical      & Common security \& privacy framework 
    & Unspecified 
    & TLS for encryption, multiple approaches for single-sign-on (e.g.~ORCID, Life Science Login). Privacy largely unspecified. \\
  Technical      & Easy SLAs for service providers 
    & Unspecified 
    & None \\
  Technical      & Access data in different formats 
    & None formalised, custom operations or relations 
    & Content-negotiation, \texttt{rel=alternate} relations \\
  Technical      & Coarse-grained/fine-grained search tools 
    & Freetext \texttt{0.DOIP/Op.Search} on local DOIP, no federation 
    & Coarse-grained e.g.~\footurl{https://datasetsearch.research.google.com/}{Google Dataset Search}, fine-grained (e.g.~federated SPARQL) require detailed vocabulary/metadata insight \\
  Technical      & Clear PID policy 
    & Strong FDO requirements, tends towards Handle system. 
    & Not required, different communities set policies \\
  Semantic       & Clear definitions for concepts/metadata/schemas 
    & Required by FDO requirements, but not yet formalised 
    & Ontologies, SKOS, OWL \\
  Semantic       & Semantic artefacts w/ open licenses 
    & All artefacts are PIDs, license not yet required by kernel metadata
    & Open License is best practice for ontology publishing \\
  Semantic       & Documentation for each semantic artefact 
    & No direct rendering from FDO, no requirement for human-readable description 
    & Ontology rendering, content-negotiation \\
  Semantic       & Repositories of artefacts 
    & Required, but not formalised 
    & Bioontologies, otherwise not usually federated \\
  Semantic       & Repositories w/ clear governance 
    & Recommended 
    & Largely self-governed repositories, if well-established may have clear governance. \\
  Semantic       & Minimal metadata model for federated discovery 
    & Kernel metadata \cite{fdo-KernelAttributes} based on RDA recommendations \cite{weigelRDARecommendationPID2018}.
    & DCAT, schema.org, Dublin Core \\
  Semantic       & Crosswalks from minimal metadata model 
    & FDO Typing recommends referencing existing type definitions, but not as separate crosswalks 
    & Multiple crosswalks for common metadata models, but frequently not in semantic format \\
  Semantic       & Extensibility options for diciplinary metadata 
    & Communities encouraged to establish own types 
    & Extensible by design, domain-specific metadata may be at different granularity \\
  Semantic       & Clear protocols/building blocks for federation/harvesting of artefact catalogues 
    & Collection types not yet defined 
    & SWORD, OAI-PMH \\
  Organisational  & Interoperability-focused rules of participation recommendations 
    & Recommended 
    & Implied only by some communities, tendency to specialise \\
  Organisational  & Usage recommendations of standardised data formats 
    & None 
    & None -- but common for metadata (e.g.~JSON-LD) \\
  Organisational  & Usage recommendations of vocabularies 
    & Recommended by community 
    & Common (see \footurl{https://rdmkit.elixir-europe.org/metadata_management}{RDMKit}) \\
  Organisational  & Usage recommendations of metadata 
    & Recommended by community 
    & RO-Crate, Gray 2017 \\
  Organisational  & Management of permanent organization names/functions 
    & Handle owner, but unclear contact. Contact info in DOIP service provider 
    & ROR. DCAT contacts. \\
  Legal          & Standardised human and machine-readable licenses 
    & None 
    & \footurl{https://spdx.org/licenses/}{SPDX}, but not that frequently used \\
  Legal          & Permissive licenses for metadata (CC0, CC-BY-4.0) 
    & Undefined 
    & Both CC0, CC-BY-4.0 common, e.g.~in DCAT \\
  Legal          & Different licenses for different parts 
    & Each part as separate FDO can have separate license 
    & DCAT, RO-Crate, Named graphs for splitting metadata \\
  Legal          & Mark expired/inexistent copyright 
    & Undefined 
    & Unclear, semantics assume copyright valid \\
  Legal          & Mark orphaned data 
    & Tombstone for deleted data, but no owner of DOIP server means FDO disappears 
    & Frequently data and endpoint has no known maintainer, archiving in common repositories becoming common \\
  Legal          & List recommended licenses 
    & Undefined 
    & Best practice recommendations \\
  Legal          & Track license evolution for dataset 
    & Undefined 
    & Versioning with PAV/PROV/DCAT \\
  Legal          & Policy/guidance for patent/trade secrets violation 
    & Undefined 
    & Undefined, legal owner may be specified. \footurl{https://www.w3.org/TR/2018/REC-odrl-vocab-20180215/}{ODRL} can express policies \\
  Legal          & GDPR compliance for personal data 
    & Undefined 
    & Undefined \\
  Legal          & Restrict access/use if legally required 
    & By transport protocol (undefined by FDO/DOIP) 
    & Diverging approaches, typically landing pages w/ auth\&auth or click-thru \\
  Legal          & Harmonised terms-of-use 
    & Undefined 
    & Undefined \\
  Legal          & Alignment between EOSC and national legislation 
    & Not applicable 
    & Not applicable \\
  \bottomrule
  \end{longtable}
  %\end{landscape}

\paragraph{Observations}


Firstly, we observe that the EOSC IF recommendations are at a high level, mainly affecting governance and practices by communities. This \emph{Organizational} level is also highlighted by the FDO recommendations, for instance the FDO Typing \cite{fdo-TypingFDOs} propose a governance structure to recognize community-endorsed services. While these community aspects are not mandated by Linked Data practices, best practices have become established for aspects like ontology development \cite{10.1186/s13326-021-00240-6}. EOSC IF's \emph{Technical} layer is likewise at a architecturally high level, such as service-level agreements, but also highlight PID policies which is strongly required by FDO, while Linked Data communities choose PID practices separately. The recommendations for the \emph{Semantic} layer, are largely already implemented by Linked Data practices, yet for FDO mostly consist of encouragements. For instance \emph{clear definitions of semantic concepts} is required by FDO guidelines, but how to technically define them has not been formalised by FDO specifications. 

The \emph{Legal} layer of interoperability is perhaps the one most emphasised by EOSC, by enabling collaboration across organizational barriers to joinly build a research infrastructure, but this is an area that both FDO and Linked Data are relatively weak in directly supporting. The EOSC IF recommendations in this layer are largely related to governance practices and metadata, for instance licensing, privacy and usage policies; yet these are also essential for cross-institutional and cross-repository access of FAIR objects. 

Likewise, search and indexing is important FAIR aspect for Findability, but is poorly supported globally by FDO and Linked Data. Efforts such as Open Research Knowledge Graph (ORKG) \cite{10.1007/978-3-030-30760-8_31}, DataCite's PID Graph \cite{10.5438/jwvf-8a66} and Google Knowledge Graph \cite{singhal2012} have improved programmatic findability to some degree, however not significantly for domain-specific semantic artefacts, currently scattered across multiple semantic catalogues \cite{10.48550/arXiv.2305.06746}.  There is a strong role for organizations like EOSC to provide such broader registries, moving beyond scholarly output metadata federations. The EOSC Marketplace\footnote{\url{https://marketplace.eosc-portal.eu/}} has for instance recently been expanded to include training material, software and data sources.


\subsection{Discussion}\label{ch3:discussion}

We have evaluated the FAIR Digital Object concept using multiple frameworks, and contrasted FDO against existing experiences from Linked Data on the Web. In this section we discuss the implications of this evaluation, and propose how these two approaches can be better combined.

\subsubsection{Framework evaluation}\label{ch3:frameworkevaluation}

Having considered FDO and the Web architecture as interoperability frameworks (\vref{ch3:interoperability-compare}), we observe that neither are magic bullets, but each bring different aspects of interoperability. The Web comes with a large degree of flexibility and openness, however this means interoperability can suffer as services have different APIs and data models, although with common patterns. This is also true for Linked Data on the Web, with many overlapping ontologies and frequent inconsistencies in resolution mechanisms; although somewhat alleviated in recent years by schema.org becoming common metadata model for semantic markup inline in Web pages. The Web is based on a common HTTP protocol which has remained stable architecturally throughout its 32 years of largely backwards-compatible evolution. FDO on the other side sets down multiple rigid rules for identifiers, types, methods etc. that are advanterous for interoperability and predictability for FAIR consumption. Yet there is a large degree of freedom in how the FDO rules can be implemented by a given community, for instance there is no common metadata model or identifier resolution mechanism, and DOIP is just one possible transport method for FDOs, which itself does not enforce these rules. 

When evaluating FDO implementations against the FDO guidelines (\vref{ch3:doip-fdo-compare}) we see that several technical pieces and community practices still need to be developed and further defined, for instance the FDO type system, how to declare FDO actions, how to resolve persistent identifiers, or how to know which pattern of FDO composition is used. Achieving fully interoperable FAIR Digital Objects would require further convergence on implementation practices, and it is not given that this needs to diverge from the established Web architecture.  It is not clear from FDO guidelines if moving from HTTP/DNS to DOIP/Handle as a way to expose distributed digital objects will benefit FAIR practitioners, when both approaches require additional equably implementable restrictions and conventions, such as using persistent identifiers or pre-defining an object's type. 

Considering this, by comparing FDO and Web as middleware (\vref{ch3:middleware}) we saw that programmatic access to digital objects, a core promise of FDO, is not particularly improved by the use of the protocol DOIP as compared to HTTP, e.g. lack of concurrency and transparency. Recent updates to HTTP have added many features needed for large-scale usage such as video streaming services (e.g. caching, multiplexing, cloud deployments), and having the option to transparently apply these also to FDOs seems like a strong incentive. Many programmatic features for distributed objects are, however, missing or needing custom extensions in both aspects, such as transactions, asynchronous operations and streaming.

By assessing FDO against the FAIR principles (\vref{ch3:fair-compare}) we found that both FDO implementations are underspecified in several aspects (licences, provenance, data references, data vocabularies, metadata persistence). While there are implementations of each of these in general Linked Data examples, there is no single set of implementation guides that fully realizes the FAIR principles. \emph{FAIRification} efforts like the FAIR Cookbook \cite{faircookbook} and FAIR Implementation Profiles \cite{FIP} are bringing existing practices together, but there remains a potential role for FDO in giving a coherent set of implementation practices that can practically achieve FAIR. Significant effort, also within EOSC, is now moving towards FAIR metrics \cite{Devaraju_2021}, which in practice need to make additional assumptions on how FAIR principles are implemented, but these are not always formalised \cite{10.5281/zenodo.7463421} nor can they be taken to be universally correct \cite{10.5281/zenodo.7848102}. Given that most of the existing FAIR guides and assessment tools are focused on Web and Linked Data, it would be reasonable for FDO to then provide a profile of such implementation choices that can achieve best of both worlds.

EOSC has been largely supportive of FDO, FAIR and related services. By contrasting the EOSC Interoperability Framework (\vref{ch3:eosc-interoperability-framework}) with FDO, we found that there are important dimensions that are not solved at a technical level, but through organization collaboration, legal requirements and building community practices. FDO recommendations highlight community aspects, but at the same time the largest FAIR communities in many science domains are already producing and consuming Linked Data. Just as the Linked Data community has a challenge in convincing more research fields to use Semantic Web technologies, FDO currently need to build many new communities in areas that have shown interest in that approach (e.g. material science).  It may be advantageous for both these effort to be aligned and jointly promoted under the EOSC umbrella. 



\subsubsection{What does FDO mean for Linked Data?}\label{ch3:what-does-it-mean-for-linked-data}

The FAIR Digital Object approach raises many important points for Linked Data practictioners.
At first glance, the explicit requirements of FDOs may seem to be easy to furfill by different parts of the Semantic Web Cake \cite[slide 10]{SemanticWebXML2000}, as has previously been proposed \cite{10.3897/rio.8.e94501}.
However, this deeper investigation, based on multiple frameworks, highlights that the openness and variability of how Linked Data is deployed can make it difficult to achieve the FDO goals without significant effort.

While RDF and Linked Data have been suggested as prime candidates for making FAIR data, we argue that when different developers have too many degrees of freedom (such as serialization formats, vocabularies, identifiers, navigation), interoperability is hampered -- this makes it hard for machines to reliably consume multiple FAIR resources across repositories and data providers. 
Indeed, this may be one reason why the initial FDO effort steered away from Linked Data approaches, but now seems in a danger of opening the many same degrees of freedom within FDO.

We therefore identify the need for a new explicit FDO profile of Linked Data that sets pragmatic constraints and stronger recommendations for consistent and developer-friendly deployment of digital objects. 
Such a combination of efforts could utillise both the benefits of mature Semantic Web technologies (e.g.~federated knowledge graph queries and rich validation) and data management practices that follow FDO guidance in order to grow an ecosystem of machine-actionable objects. 
It is beyond the scope of this work to detail such a profile, but we suggest the following potential key aspects:

\begin{itemize}
  \item Use HTTP(S) as protocol.
  \item Use URIs as identifiers, with persistent identifier promises.
  \item Provide consistent identifier resolution that does not require heuristics.
  \item Common core metadata model.
  \item References are always URIs, and should be persistent identifiers.
  \item Types, attributes and actions are self-defined by their identifier.
  \item Use Web approaches directly where possible, rather than wrap in a new model.
\end{itemize}

The FAIR and Linked Data communities likewise need to recognize the need for simpler, more pragmatic approaches that make it easier for FAIR practitioners to adapt the technologies with "just enough" semantics. 
%We have previously proposed the combination of RO-Crate \cite{Soiland-Reyes 2022} and Signposting \cite{Van de Sompel 2022} as a mean to implement FDO \cite{10.3897/rio.8.e93937} over HTTP using a common Linked Data metadata model. 

%However it may be sufficient to use HTTP-based FAIR Signposting alone to achieve the above list, if one considers only a small metadata model, and rather reference from the signposting which metadata resources are additionally available. 
%This will allow any Linked Data resource to gradually participate in the FDO ecosystem, with minimal effort and non-intrusive implementation changes. FDO implementations like Cordra typically already use HTTP APIs that align with DOIP \cite{DOIPAPIHTTPa}, these can be augmented with Signposting headers without necessarily moving to a Linked Data metadata model. 


\subsection{Conclusion}\label{ch3:conclusion}

In this work, we have considered FAIR Digital Objects (FDO) as a potential distributed object system for FAIR data and compared it with established Web approaches focusing on Linked Data. We have described the background of the Semantic Web and FAIR Digital Objects, and evaluated both using multiple conceptual frameworks.

We find that both FDO and Linked Data approaches can significantly benefit from each-other and should be aligned further. Namely, Linked Data proponents need to make their technologies more approachable, agreeing on predictable and consistent implementations of FAIR principles. 

The FDO recommendations show that FAIR thinking in this regard need to move beyond data publishing and into machine actionability across digital objects, and with broader community consensus. 
As flexibility for extensions is a necessary ingredient alongside rigidity for core concepts, the FDO community likewise need to settle on directly implementable specifications rather than just guidelines, and avoid making similar mistakes learnt by the early Semantic Web adopters. 

By implementing the goals of FAIR Digital Objects with the mature technology stack developed for Linked Data, EOSC research infrastructures and researchers in general can create and use FAIR machine-actionable research outputs for decades to come.

