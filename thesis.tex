%%%%%%%%%%%%%%%%%% USAGE INSTRUCTIONS %%%%%%%%%%%%%%%%%%
% - Compile using LuaLaTeX and biber, unless there is a particular reason not to. Do not use the older LaTex/PDFLaTeX or BibTeX (the fonts won't work correctly.)
% - Font and the report 'year' must be specified when all \documentclass or the template won't work correctly. (There's no error checking/default cases!)
% - Options for fonts are: calibri, times, palatino, garamond, arial, tahoma, verdana, trebuchet. Note however that not all these are installed on Overleaf, you need to install to your project 
% - Options for 'year' are: first, second, thesis.
% - If not a thesis, should probably remove the COVID-19 impact statement page
% - As many further packages as wanted can be loaded. Below are just an example set. Note that template itself loads a number of packages, including hyperref.
% - References are handed using biblatex.
% - Don't need to include a \uomdeclarations unless this is a thesis
% - Note there is more documentation in the header of uom_these_casson.cls file if you need more help
% - Link to the presentation of theses policy: http://www.regulations.manchester.ac.uk/pgr-presentation-theses/



%%%%%%%%%%%%%%%%%% META DATA SETUP %%%%%%%%%%%%%%%%%%
% This is where the document title and author are set. Other details for the title page are set later
\begin{filecontents*}{\jobname.xmpdata}
  \Title{A data reduction algorithm incorporating a low power continuous wavelet transform for use in wearable electroencephalography systems} 
  \Author{Alexander J. Casson}
  \Language{en-GB}
  \Copyrighted{False}
  % More meta-data fielda can be added here if wanted, see https://ctan.org/pkg/pdfx?lang=en for fields
\end{filecontents*}



%%%%%%%%%%%%%%%%%% DOCUMENT SETUP %%%%%%%%%%%%%%%%%%
\documentclass[palatino,thesis]{uom_thesis_casson} % See above for font options Year can be: first, second or thesis.



%%%%%%%%%%%%%%%%%% PACKAGES AND COMMANDS %%%%%%%%%%%%%%%%%%

% Packages - some useful examples
\usepackage{graphicx,psfrag,color} % for postscript graphics files
  \graphicspath{ {./images/} }
\usepackage{amsmath}               % assumes amsmath package installed
  \allowdisplaybreaks[1]           % allow eqnarrays to break across pages
\usepackage{amssymb}               % assumes amsmath package installed 
\usepackage{url}                   % format hyperlinks correctly
\usepackage{rotating}              % allow portrait figures and tables
\usepackage{multirow}              % allows merging of rows in tables
%\usepackage{lscape}                % allows pages to be typeset in landscape mode
\usepackage{tabularx}              % allows fixed width tables
\usepackage{verbatim}              % enhanced version of built-in verbatim environment
\usepackage{footnote}              % allows more control over footnote environments
\usepackage{float}                 % allows H option on floats to force here placement
\usepackage{booktabs}              % improve table line spacing
\usepackage[base]{babel}           % required for lisum package
\usepackage{lipsum}                % for adding dummy text here
%\usepackage{subcaption}            % for multiple sub-figures in a single float
\usepackage{siunitx}               % add SI units
% Add your packages here
% for table resizing
\usepackage{booktabs}
\usepackage{multirow}
% All needed for longtable to work
\usepackage{longtable,booktabs,array,calc}
%\usepackage{tmultirow}

% so rotated pages rotate in PDF
\usepackage{pdflscape}

% Nicer cross-references
\usepackage{varioref}

% Custom commands
\newcommand{\degree}{\ensuremath{^\circ}}
\newcommand{\sus}[1]{$^{\mbox{\scriptsize #1}}$} % superscript in text (e.g. 1st can be 1\sus{st})
\newcommand{\sub}[1]{$_{\mbox{\scriptsize #1}}$} % subscript in text
\newcommand{\chap}[1]{Chapter~\ref{#1}}
\newcommand{\sect}[1]{Section~\ref{#1}}
\newcommand{\fig}[1]{Fig.~\ref{#1}}
\newcommand{\tab}[1]{Table~\ref{#1}}
\newcommand{\equ}[1]{(\ref{#1})}
\newcommand{\appx}[1]{Appendix~\ref{#1}}
% Add your commands here
\def\tightlist{}
% pandoc stuff
\usepackage{color}
\usepackage{fancyvrb}
\newcommand{\VerbBar}{|}
\newcommand{\VERB}{\Verb[commandchars=\\\{\}]}
\DefineVerbatimEnvironment{Highlighting}{Verbatim}{commandchars=\\\{\}}
% Add ',fontsize=\small' for more characters per line
\newenvironment{Shaded}{}{}
\newcommand{\AlertTok}[1]{\textcolor[rgb]{1.00,0.00,0.00}{\textbf{#1}}}
\newcommand{\AnnotationTok}[1]{\textcolor[rgb]{0.38,0.63,0.69}{\textbf{\textit{#1}}}}
\newcommand{\AttributeTok}[1]{\textcolor[rgb]{0.49,0.56,0.16}{#1}}
\newcommand{\BaseNTok}[1]{\textcolor[rgb]{0.25,0.63,0.44}{#1}}
\newcommand{\BuiltInTok}[1]{#1}
\newcommand{\CharTok}[1]{\textcolor[rgb]{0.25,0.44,0.63}{#1}}
\newcommand{\CommentTok}[1]{\textcolor[rgb]{0.38,0.63,0.69}{\textit{#1}}}
\newcommand{\CommentVarTok}[1]{\textcolor[rgb]{0.38,0.63,0.69}{\textbf{\textit{#1}}}}
\newcommand{\ConstantTok}[1]{\textcolor[rgb]{0.53,0.00,0.00}{#1}}
\newcommand{\ControlFlowTok}[1]{\textcolor[rgb]{0.00,0.44,0.13}{\textbf{#1}}}
\newcommand{\DataTypeTok}[1]{\textcolor[rgb]{0.56,0.13,0.00}{#1}}
\newcommand{\DecValTok}[1]{\textcolor[rgb]{0.25,0.63,0.44}{#1}}
\newcommand{\DocumentationTok}[1]{\textcolor[rgb]{0.73,0.13,0.13}{\textit{#1}}}
\newcommand{\ErrorTok}[1]{\textcolor[rgb]{1.00,0.00,0.00}{\textbf{#1}}}
\newcommand{\ExtensionTok}[1]{#1}
\newcommand{\FloatTok}[1]{\textcolor[rgb]{0.25,0.63,0.44}{#1}}
\newcommand{\FunctionTok}[1]{\textcolor[rgb]{0.02,0.16,0.49}{#1}}
\newcommand{\ImportTok}[1]{#1}
\newcommand{\InformationTok}[1]{\textcolor[rgb]{0.38,0.63,0.69}{\textbf{\textit{#1}}}}
\newcommand{\KeywordTok}[1]{\textcolor[rgb]{0.00,0.44,0.13}{\textbf{#1}}}
\newcommand{\NormalTok}[1]{#1}
\newcommand{\OperatorTok}[1]{\textcolor[rgb]{0.40,0.40,0.40}{#1}}
\newcommand{\OtherTok}[1]{\textcolor[rgb]{0.00,0.44,0.13}{#1}}
\newcommand{\PreprocessorTok}[1]{\textcolor[rgb]{0.74,0.48,0.00}{#1}}
\newcommand{\RegionMarkerTok}[1]{#1}
\newcommand{\SpecialCharTok}[1]{\textcolor[rgb]{0.25,0.44,0.63}{#1}}
\newcommand{\SpecialStringTok}[1]{\textcolor[rgb]{0.73,0.40,0.53}{#1}}
\newcommand{\StringTok}[1]{\textcolor[rgb]{0.25,0.44,0.63}{#1}}
\newcommand{\VariableTok}[1]{\textcolor[rgb]{0.10,0.09,0.49}{#1}}
\newcommand{\VerbatimStringTok}[1]{\textcolor[rgb]{0.25,0.44,0.63}{#1}}
\newcommand{\WarningTok}[1]{\textcolor[rgb]{0.38,0.63,0.69}{\textbf{\textit{#1}}}}
%%


%%%%%%%%%%%%%%%%%% REFERENCES SETUP %%%%%%%%%%%%%%%%%%

% Setup your references here. Change the reference style here if wanted
%%\usepackage[backend=biber,seconds=true,alldates=iso,labeldate=year,style=authoryear,backref=true,hyperref=auto]{biblatex}
% Note backref=true adds a page number (and hyperlink) to each reference so you can easily go back from the references to the main document. You may prefer backref=false if you need to stick strictly to a given reference style


% Fixes which can't be applied in the .cls file
%%\DefineBibliographyStrings{english}{backrefpage = {cited on p\adddot},  backrefpages = {cited on pp\adddot}}
%%  \renewcommand*{\bibfont}{\large}


% Add more .bib files here if wanted
%\addbibresource{references.bib}



%%%%%%%%%%%%%%%%%% START DOCUMENT %%%%%%%%%%%%%%%%%%
\begin{document}


%%%%%%%%%%%%%%%%%% TITLE PAGE %%%%%%%%%%%%%%%%%%

% Title and author are automatically taken from the document meta-data defined above
\makeatletter
\title{\xmp@Title}
\author{\xmp@Author}
\makeatother

% Set the below yourself
\faculty{Science}                  % "Faculty of" is added automatically
\department{Informatics Institute} % regulations allow School, Division, or Department to be put here
\submitdate{2023}                                  % regulations ask only for the year, not month
\wordcount{1000}		                           % use \wordcount{} to set the count, \thewordcount to print in the text
\maketitle



%%%%%%%%%%%%%%%%%% LISTS OF CONTENT %%%%%%%%%%%%%%%%%%

% Probably don't need all of these unless final thesis
\uomtoc % contents 
% \uomlof % figures
% \uomlot % tables
%%\begin{uomlop} % list of publications.
%%  % Can use biblatex to \printbibliography[heading=none] to populate automatically, or can add a custom list via the tocloft package, but probably easier to just type in unless you have lots!
%%  Publications go here.
%%\end{uomlop}
%%\begin{uomterms}
%%  Enter terms and abbreviations as table or similar
%%  % Add list of terms and abbreviations by hand if wanted. Is no formal requirement to have one. There are lots of packages to help with this if you don't want to do by hand
%% \end{uomterms}



%%%%%%%%%%%%%%%%%% ABSTRACT %%%%%%%%%%%%%%%%%%
\begin{abstract} % put abstract here. Limit is 1 page.
  This is abstract text. 
  
  \lipsum[1-2]
\end{abstract}%
\clearpage



%%%%%%%%%%%%%%%%%% LAY ABSTRACT %%%%%%%%%%%%%%%%%%
\begin{uomlay} % put lay abstract here. Limit is 1 page. Not compulsory
  This is lay abstract text. 
  
  \lipsum[1-2]
\end{uomlay}



%%%%%%%%%%%%%%%%%% DECLARATIONS %%%%%%%%%%%%%%%%%%
%\uomdeclarations % Don't need unless final thesis. No options are needed. Having this command will add the required declarations



%%%%%%%%%%%%%%%%%% LIST OF THESIS REVISIONS %%%%%%%%%%%%%%%%%%
%\begin{uomlotr} % Only required for resubmitted theses
%Put list of revisions here. Only required for resubmitted theses. Delete if not needed
%\end{uomlotr} 



%%%%%%%%%%%%%%%%%% ACKNOWLEDGEMENTS %%%%%%%%%%%%%%%%%%
%\begin{uomacknowledgements} % probably don't need unless final thesis
%Acknowledgements go here.
%\end{uomacknowledgements}



%%%%%%%%%%%%%%%%%% ABOUT THE AUTHOR %%%%%%%%%%%%%%%%%%
%\begin{uomauthor} % Optional
%If desired, a brief statement for External Examiners giving the candidate’s degree(s) and research experience, even if the latter consists only of the work done for this thesis.
%\end{uomauthor} 



%%%%%%%%%%%%%%%%%% INTRO %%%%%%%%%%%%%%%%%%


\chapter{Introduction}


\hypertarget{this-thesis-chapter-will-introduce-the-purpose-of-this-phd-study-and-give-an-overview-of-the-thesis}{%
\subsection{(This thesis chapter will introduce the purpose of this PhD
study and give an overview of the
thesis)}\label{this-thesis-chapter-will-introduce-the-purpose-of-this-phd-study-and-give-an-overview-of-the-thesis}}

%%%%%%%%%%%%%%%%%% MAIN CHAPTERS %%%%%%%%%%%%%%%%%%


\chapter{FAIR Digital Objects and Linked Data}
\section{Updating Linked Data practices for FAIR Digital Object
principles}
\label{ch2:updating-linked-data-practices-for-fair-digital-object-principles}

Realization of FAIR Digital Objects has a great potential if combined with the mature technology stack of Linked Data and knowledge graphs.

Here I will briefly discuss how FDO principles can be achieved using existing standards that have powered the Web for the last 30 years. Using this mature approach can accelerate uptake of FDO by scholars and existing research infrastructures.

I will also reflect on how the Linked Data community can adapt to better welcome approaches like FDO.

\subsubsection{Background}
\label{ch2:background}

The \emph{FAIR principles} \cite{Wilkinson 2016} are
fundamental for data discovery, sharing, consumption and reuse; however
their broad interpretation and many ways to implement can lead to
inconsistencies and incompatibility
\cite{Jacobsen 2020}.

The European Open Science Cloud (\href{https://www.eosc.eu/}{EOSC}) has
been instrumental in maturing and encouraging FAIR practices across a
wide range of research areas. Linked Data in the form of
\href{https://www.w3.org/TR/rdf11-primer/}{RDF} (Resource Description
Framework) is the common way to implement machine-readability in FAIR,
however the principles do not prescribe RDF or any particular technology
\cite{Mons 2017}.

\paragraph{FAIR Digital Object}
\label{ch2:fair-digital-object}

\textbf{FAIR Digital Object} (FDO)
\cite{Schultes 2019}
has been proposed to improve researcher's access to digital objects
through formalising their metadata, types, identifiers and exposing
their computational operations, making them actionable FAIR objects
rather than passive data sources.

FDO is a set of principles \cite{bonino2019}, implementable in multiple ways. Current realisations mostly
use \emph{Digital Object Interface Protocol} (DOIPv2)
\cite{DONA 2018}, with the
main implementation
\href{https://www.cordra.org/documentation/api/doip.html}{CORDRA}. We
can consider DOIPv2 as a simplified combination of object-oriented
(CORBA, SOAP) and document-based (HTTP, FTP) approaches.

More recently, the \href{https://fairdo.org/}{FDO Forum} has prepared
detailed
\href{https://drive.google.com/drive/u/0/folders/1-SbZk7enOqjy2Rf57PMB-CQW7qMOUXgO}{recommendations}\footnote{See section \vref{ch3:next-step-fdo}},
currently open for comments, including a
\href{https://docs.google.com/document/d/10ESWe-m0ex7fIW0ZYOYeVg6gAcIaBKyO3W-Vg8tXzdw/edit\#}{DOIP
endorsement} and updated
\href{https://docs.google.com/document/d/1-4_yGRrIcgdMIwaFvHyUt6lxDdfzGpqLUCEihE0vJ-g/edit\#heading=h.gjdgxs}{FDO
requirements}. These point out \textbf{Linked Data} as another possible
technology stack, which is the focus of this work.

\paragraph{Linked Data}
\label{ch2:linked-data}

\href{https://www.w3.org/standards/semanticweb/data}{Linked Data
standards} (LD), based on the Web architecture, are commonplace in
sciences like bioinformatics, chemistry and medical informatics -- in
particular to publish Open Data as machine-readable resources. LD has
become ubiquitous on the general Web, the
\href{https://schema.org/}{schema.org} vocabulary is used by over 10
million sites for indexing by search engines --
\href{https://w3techs.com/technologies/details/da-jsonld}{43\% of all
websites} use \href{https://json-ld.org/}{JSON-LD}.

Although LD practices align to FAIR
{[}\href{https://doi.org/10.1007/978-3-319-98192-5_60}{Hasnain 2018}{]},
they do not fully encompass active aspects of FDOs. The HTTP protocol is
used heavily for applications (e.g.~mobile apps and cloud services),
with REST APIs of customised \href{https://json-schema.org/}{JSON
structures}. Approaches that merge the LD and REST worlds include
\href{https://www.w3.org/TR/ldp/}{Linked Data Platform} (LDP),
\href{https://www.hydra-cg.com/}{Hydra} and
\href{https://www.w3.org/TR/webpayments-http-messages/}{Web Payments}.


\subsubsection{Meeting FDO principles using Linked Data
standards}\label{ch2:meeting-fdo-principles-using-linked-data-standards}

Considering the potential of FDOs when combined with the mature
technology stack of LD, here we briefly discuss how FDO principles can
be achieved using existing standards. The general principles (G1--G9)
apply well: Open standards with HTTP being stable for 30 years, JSON-LD
is widely used, FAIR practitioners mainly use RDF, and a clear
abstraction between the RDF model with stable bindings available in
multiple serialisations.

However, when considering the specific principles (FDOF1--FDOF12) we
find that additional constraints and best practices need to be
established -- arbitrary LD resources cannot be assumed to follow FDO
principles. This is equivalent to how existing use of DOIP is not
FDO-compliant without additional constraints.

Namely, persistent identifiers (PIDs)
{[}\href{https://doi.org/10.1371/journal.pbio.2001414}{McMurry 2017}{]}
(FDOF1) are common in LD world (e.g.~using \url{http://purl.org/} or
\url{https://w3id.org/}), however they don't always have a declared type
(FDOF2), or the PID may not even appear in the metadata. URL-based PIDs
are resolvable (FDOF3), typically over HTTP using redirections and
content-negotiation. One great advantage of RDF is that all attributes
are defined semantic artefacts with PIDs (FDOF4), and attributes can be
reused across vocabularies.

While CRUD operations (FDOF6) are supported by native HTTP operations
(GET/PUT/POST/DELETE) as in \href{https://www.w3.org/TR/ldp/}{LDP},
there is little consistency on how to define operation interfaces in LD
(FDOF5). Existing REST approaches like
\href{https://swagger.io/specification/}{OpenAPI} and
\href{https://doi.org/10.17487/RFC6570}{URI templates} are mature and
good candidates, and should be related to defined types to support
machine-actionable composition (FDOF7). HTTP error code \emph{410 Gone}
is used in tombstone pages for removed resources (FDOF12), although more
frequent is \emph{404 Not Found}.

Metadata is resolved to HTTP documents with their own URIs, but these
frequently don't have their own PID (FDOF8).
\href{https://w3c.github.io/rdf-star/}{RDF-Star} and nanopublications
{[}\href{https://doi.org/10.7717/peerj-cs.387}{Kuhn 2021}{]} give ways
to identify and trace provenance of individual assertions.

Different metadata levels (FDOF9) are frequently developed for LD
vocabularies across different communities (FDOF10), such as
\href{http://hl7.org/fhir/}{FHIR} for health data,
\href{https://bioschemas.org/}{Bioschemas} for bioinformatics and
\textgreater1000 more specific
\href{https://bioportal.bioontology.org/ontologies}{bioontologies}.
Increased declaration and navigation of \emph{profiles} is therefore
essential for machine-actionability and consistent consumption across
FAIR endpoints.

Several standards exist for rich collections (FDOF11),
e.g.~\href{https://www.openarchives.org/ore/}{OAI-ORE},
\href{https://www.w3.org/TR/vocab-dcat-3/}{DCAT},
\href{https://www.researchobject.org/ro-crate/}{RO-Crate},
\href{https://www.w3.org/TR/ldp/}{LDP}. These are used and extended
heterogeneously across the Web, but consistent machine-actionable FDOs
will need specific choices of core standards and vocabularies. Another
challenge is when multiple PIDs refer to ``almost the same'' concept in
different collections -- significant effort have created manual and
automated semantic mappings
{[}\href{https://doi.org/10.1016/j.websem.2013.05.001}{Baker 2013},
\href{https://doi.org/10.1007/s12553-022-00639-w}{de Mello 2022}{]}).

Currently the FDO Forum has suggested the use of LDP as a possible
alternative for implementing FAIR Digital Objects \cite{bonino2021}, which
proposes a novel approach of content-negotiation with custom media
types.

\subsubsection{Discussion}\label{ch2>discussion}

The Linked Data stack provides a set of specifications, tools and
guidelines in order to help the FDO principles become a reality. This
mature approach can accelerate uptake of FDO by scholars and existing
research infrastructures such as the European Open Science Cloud (EOSC).

However, the amount of standards and existing metadata vocabularies
poses a potential threat for adoption and interoperability. Yet, the
challenges for agreeing on usage profiles apply equally to DOIP as LD
approaches.

We have worked with different scientific communities to define RO-Crate
{[}\href{https://doi.org/10.3233/DS-210053}{Soiland-Reyes 2022}{]}, a
lightweight method to package research outputs along with their
metadata. While RO-Crate's use of schema.org shows just one possible
metadata model, it's powerful enough to be able to express FDOs, and
familiar to web developers.

We have also used FAIR Signposting
{[}\href{https://signposting.org/FAIR/}{Van de Sompel 2022}{]} with HTTP
\texttt{Link:} headers as a way to support navigation to the individual
core properties of an FDO (PID, type, metadata, licence, bytestream)
that does not require heuristics of content-negotiation and is agnostic
to particular metadata vocabularies and serialisations.

We believe that by adopting Linked Data principles, we can accelerate
FDO today -- and even start building practical ways to assist scientists
in efficiently answering topical questions based on knowledge graphs.


\section{Evaluating FAIR Digital Object and Linked Data as distributed object systems}\label{ch3:evaluating-fdo-ld}

FAIR Digital Object (FDO) is an emerging concept that is highlighted by European Open Science Cloud (EOSC) as a potential candidate for building a ecosystem of machine-actionable research outputs. In this work we systematically evaluate FDO and its implementations as a global distributed object system, by using five different conceptual frameworks that cover interoperability, middleware, FAIR principles, EOSC requirements and FDO guidelines themself.

We compare the FDO approach with established Linked Data practices and the existing Web architecture, and provide a brief history of the Semantic Web while discussing why these technologies may have been difficult to adopt for FDO purposes. We conclude with recommendations for both Linked Data and FDO communities to further their adaptation and alignment.

\subsection{Introduction}\label{ch3:introduction}

The FAIR principles \cite{Wilkinson 2016} encourage sharing of scientific data with machine-readable metadata and the use of interoperable formats, and are being adapted by a wide range of research infrastructures. They have been widely recognised by the research community and policy makers as a goal to strive for. In particular, the European Open Science Cloud (\href{https://www.eosc.eu/}{EOSC}) has promoted adaptation of FAIR data sharing of data resources across electronic research infrastructures \cite{Mons 2017}. The EOSC Interoperability Framework \cite{eosc-interop-framework} puts particular emphasis on how interoperability can be achieved technically, semantically, organisationally, and legally -- laying out a vision of how data, publication, software and services can work together to form an ecosystem of rich digital objects.

Specifically, the EOSC Interoperability framework highlights the emerging FAIR Digital Object (FDO) concept \cite{schultesFAIRPrinciplesDigital2019a} as a possible foundation for building a semantically interoperable ecosystem to fully realise the FAIR principles beyond individual repositories and infrastructures. The FDO approach has great potential, as it proposes strong requirements for identifiers, types, access and formalises interactive operations on objects.

In other discourse, Linked Data \cite{Bizer 2009} has been seen as an established set of principles based on Semantic Web technologies that can achieve the vision of the FAIR principles \cite{boninodasilvasantosFAIRDataPoints2016a,Hasnain 2018}. Yet regular researchers and developers of emerging platforms for computation and data management are reluctant to adapt such a FAIR Linked Data approach fully \cite{verborghSemanticWebIdentity2020a}, opting instead for custom in-house models and JSON-derived formats from RESTful Web services \cite{merono-penuelaConclusionFutureChallenges2021a,neumannAnalysisPublicREST2021a}. While such focus on simplicity gives rapid development and highly specialised services, it raises wider concerns on interoperability \cite{turcoaneLinkedDataJSONLD2014a,wilkinsonWorkflowsWhenParts2022b}.

One challenge that may, perhaps counter-intuitively, steer developers towards a not-invented-here mentality \cite{stefiDevelopersMakeUnbiased2015,stefiDevelopReuseTwo2015a} when exposing their data on the Web is the heterogeneity and apparent complexity of Semantic Web approaches themselves \cite{merono-penuelaWebDataApis2021b}.

These approaches, thus, form two of the major avenues for allowing developers and the wider research community to achieve the goal of FAIR data. Given their importance, in this article, we aim to examine the relationships between FAIR and FAIR Digital Objects, contrasted with Linked Data and the Web in general.

Concretely, the contribution of this paper is a systematic comparison between FDO and Linked Data using 5 different conceptual frameworks that capture different perspectives on interoperability and readiness for implementation.

\subsection{Background and related work}\label{ch3:background}

In the following, we discuss the related work with respect to FAIR Digital Objects and Linked Data. We do so by looking through the lens of development of these technologies over time, including future directions.

The concept of \textbf{FAIR Digital Objects} \cite{schultesFAIRPrinciplesDigital2019a} has been introduced as way to expose research data as active objects that conform to the FAIR principles \cite{Wilkinson 2016}. This builds on the \emph{Digital Object} (DO) concept \cite{kahnFrameworkDistributedDigital2006b}, first introduced in 1995 \cite{kahnFrameworkDistributedDigital1995a} as a system of \emph{repositories} containing \emph{digital objects} identified by \emph{handles} and described by \emph{metadata} which may have references to other handles. DO was the inspiration for the ITU X.1255 framework \cite{x1255FrameworkDiscovery} which introduced an abstract \emph{Digital Entity Interface Protocol} for managing such objects programmatically, first realised by the Digital Object Interface Protocol (DOIP) \cite{DigitalObjectInterface}.

In brief, the structure of a FAIR Digital Object (FDO) is to, given a \emph{persistent identifier} (PID) such as a DOI, resolve to a \emph{PID Record} that gives the object a \emph{type} along with a mechanism to retrieve its \emph{bit sequences}, \emph{metadata} and references to further programmatic \emph{operations}. The type of an FDO (itself an FDO) defines attributes to semantically describe and relate such FDOs to other concepts (typically other FDOs referenced by PIDs). The premise of systematically building an ecosystem of such digital objects is to give researchers a way to organise complex digital entities, associated with identifiers, metadata, and supporting automated processing \cite{wittenburgDigitalObjectsDrivers2019a}.

Recently, FDOs have been recognised by the European Open Science Cloud (\href{https://eosc.eu/}{EOSC}) as a suggested part of its Interoperability Framework \cite{eosc-interop-framework}, in particular for deploying active and interoperable FAIR resources that are \emph{machine actionable}. Development of the FDO concept continued within Research Data Alliance (\href{https://www.rd-alliance.org/}{RDA}) groups and EOSC projects like \href{https://www.go-fair.org/}{GO-FAIR}, concluding with a set of guidelines for implementing FDO \cite{bonino2019}. The \href{https://fairdo.org/}{FAIR Digital Objects Forum} has since taken over the maturing of FDO through focused working groups which have currently drafted several more detailed specification documents (see \emph{Next steps for FDO} \vpageref{ch3:next-step-fdo}).

\subsubsection{FDO approaches}\label{fdo-approaches}

FDO is an evolving concept. A set of FDO Demonstrators \cite{wittenburgFAIRDigitalObject2022b} highlight how current adapters are approaching implementations of FDO from different angles:

\begin{itemize}
\tightlist
\item
  Building on the Digital Object concept, using the simplified DOIP v2.0 \cite{DONA 2018} specification, which detail how to exchange JSON objects through a text-based protocol\footnote{For a brief introduction to DOIP 2.0, see \cite{DOIPExamplesCordraa}} (usually TCP/IP over TLS). The main DOIP operations are retrieving, creating and updating digital objects. These are mostly realised using the reference implementation \href{https://cordra.org/}{Cordra}. FDO types are registered in the local Cordra instance, where they are specified using JSON Schema \cite{Draftbhuttonjsonschema} and PIDs are assigned using the Handle system. Several type registries have been established.
\item
  Following the traditional Linked Data approach, but using the DOIP protocol, e.g.~using JSON-LD and schema.org within DOIP (NIST for material science).
\item
  Approaching the FDO principles from existing Linked Data practices on the Web (e.g.~WorkflowHub use of RO-Crate and schema.org).
\end{itemize}

From this it becomes apparent that there is a potentially large overlap between the goals and approaches of FAIR Digital Objects and Linked Data, which we'll cover \vpageref{ch3:ld}.


\subsubsection{Next steps for FDO}\label{ch3:next-step-fdo}

The FAIR Digital Object Forum \cite{FAIRDigitalObjects} working groups have prepared detailed requirement documents \cite{fdo-Specs} setting out the path for realising FDOs, named \emph{FDO Recommendations}. As of 2023-02-02, most of these documents are open for public review, while some are still in draft stages for internal review. As these documents clarify the future aims and focus of FAIR Digital Objects \cite{fdo-Roadmap}, we provide their brief summaries below:

\textbf{FAIR Digital Object Overview and Specifications} \cite{fdo-Overview} is a comprehensive overview of FAIR Digital Object specifications listed below. It serves as a primer that introduces FDO concepts and the remaining documents. It is accompanied by an FDO Glossary \cite{fdo-Glossary}.

The \textbf{FDO Forum Document Standards} \cite{fdo-DocProcessStd} documents the recommendation process within the forum, starting at \emph{Working Draft} (WD) status within the closed working group and later within the open forum, then \emph{Proposed Recommendation} (PR) published for public review, finalised as \emph{FDO Forum Recommendation} (REC) following any revisions. In addition, the forum may choose to \emph{endorse} existing third-party notes and specifications.

The \textbf{FDO Requirement Specifications} \cite{fdo-RequirementSpec} is an update of \cite{bonino2019} as the foundational definition of FDO. This sets the criteria for classifying an digital entity as a FAIR Digital Object, allowing for multiple implementations. The requirements shown in Table \vref{ch3:fdo-checks} are largely equivalent, but in this specification clarified with references to other FDO documents.

The \textbf{Machine actionability} \cite{fdo-MachineActionDef} sets out to define what is meant by \emph{machine actionability} for FDOs. \emph{Machine readable} is defined as elements of bit-sequences defined by structural specification, \emph{machine interpretable} elements that can be identified and related with semantic artefacts, while \emph{machine actionable} are elements with a type with operations in a symbolic grammar. The document largely describes requirements for resolving an FDO to metadata, and how types should be related to possible operations.

\textbf{Configuration Types} \cite{fdo-ConfigurationTypes} classifies different granularities for organising FDOs in terms of PIDs, PID Records, Metadata and bit sequences, e.g.~as a single FDO or several daisy-chained FDOs. Different patterns used by current DOIP deployments are considered, as well as FAIR Signposting \cite{Van de Sompel 2022}

\textbf{PID Profiles \& Attributes} \cite{fdo-PIDProfileAttributes} specifies that PIDs must be formally associated with a \emph{PID Profile}, a separate FDO that defines attributes required and recommended by FDOs following said profile. This forms the \emph{kernel attributes}, building on recommendations from RDA's \emph{PID Information Types} working group \cite{weigelRDARecommendationPID2018}. This document makes a clear distinction between a minimal set of attributes needed for PID resolution and FDO navigation, which needs to be part of the \emph{PID Record} \cite{islam_2023}, compared with a richer set of more specific attributes as part of the \emph{metadata} for an FDO, possibly represented as a separate FDO.

\textbf{Kernel Attributes \& Metadata} \cite{fdo-KernelAttributes} elaborates on categories of FDO Mandatory, FDO Optional and Community Attributes, recommending kernel attributes like \texttt{dateCreated}, \texttt{ScientificDomain}, \texttt{PersistencePolicy}, \texttt{digitalObjectMutability}, etc. This document expands on RDA Recommendation on PID Kernel Information \cite{weigelRDARecommendationPID2018}. It is worth noting that both documents are relatively abstract and do not establish PIDs or namespaces for the kernel attributes.

\textbf{Granularity, Versioning, Mutability} \cite{fdo-Granularity} considers how granularity decisions for forming FDOs must be agreed by different communities depending on their pragmatic usage requirements. The affect on versioning, mutability and changes to PIDs are considered, based on use cases and existing PID practices.

\textbf{DOIP Endorsement Request} \cite{fdo-DOIPEndorsement} is an endorsement of the DOIP v2.0 \cite{DONA 2018} specification as a potential FDO implementation, as it has been applied by several institutions \cite{wittenburgFAIRDigitalObject2022b}. The document proposes that DOIP shall be assessed for completeness against FDO -- in this initial draft this is justified as \emph{``we can state that DOIP is compliant with the FDO specification documents in process''} (the documents listed above).

\textbf{Upload of FDO} \cite{fdo-FDO-Upload} illustrates the operations for uploading an FDO to a repository, what checks it should do (for instance conformance with the PID Profile, if PIDs resolve). ResourceSync \cite{ResourceSyncFrameworkSpecification} is suggested as one type of service to list FDOs. This document highlights potential practices by repositories and their clients, without adding any particular requirements.

\textbf{Typing FAIR Digital Objects} \cite{fdo-TypingFDOs} defines what \emph{type} means for FDOs, primarily to enable machine actionability and to define an FDO's purpose. This document lays out requirements for how \emph{FDO Types} should themselves be specified as FDOs, and how an \emph{FDO Type Framework} allows organising and locating types. Operations applicable to an FDO is not predefined for a type, however operations naturally will require certain FDO types to work. How to define such FDO operations is not specified.

\textbf{Implementation of Attributes, Types, Profiles and Registries} \cite{fdo-ImplAttributesTypesProfiles} details how to establish FDO registries for types and FDO profiles, with their association with PID systems. This document suggest policies and governance structures, together with guidelines for implementations, but without mandating any explicit technology choices. Differences in use of attributes are examplified using FDO PIDs for scientific instruments, and the proto-FDO approach of \href{https://de.dariah.eu/}{DARIAH-DE} \cite{schwardmannTwoExamplesHow2022}.

It is worth pointing out at that, except for the DOIP endorsement, all of these documents are conceptual, in the sense that they permit any technical implementation of FDO, if used according to the recommendations. 
%See bibliography \vref*{ch3:fdo-bibliography} for the citation per document above.


\subsubsection{From the Semantic Web to Linked Data}\label{ch3:ld}

In order to describe \emph{Linked Data} as it is used today, we'll start with an (opinionated) description of the evolution of its foundation, the \emph{Semantic Web}.

\subsubsection*{A brief history of the Semantic Web}\label{ch3:semweb}

The \textbf{Semantic Web} was developed as a vision by Tim Berners-Lee \cite{berners-leeWeavingWebOriginal1999}, at a time that the Web had already become widely established for information exchange, being a global set of hypermedia documents which are cross-related using universal links in the form of URLs. The foundations of the Web (e.g.~URLs, HTTP, SSL/TLS, HTML, CSS, ECMAScript/JavaScript, media types) were standardised by \href{https://www.w3.org/standards/}{W3C}, \href{https://www.ecma-international.org/}{Ecma}, \href{https://www.ietf.org/standards/}{IETF} and later \href{https://whatwg.org/}{WHATWG}. The goal of Semantic Web was to further develop the machine-readable aspects of the Web, in particular adding \emph{meaning} (or semantics) to not just the link relations, but also to the \emph{resources} that the URLs identified, and for machines thus being able to meaningfully navigate across such resources, e.g.~to answer a particular query.

Through W3C, the Semantic Web was realised with the Resource Description Framework (RDF) \cite{w3-rdf11-primer} that used \emph{triples} of subject-predicate-object statements, with its initial serialisation format \cite{w3-rdf-syntax99} being RDF/XML (XML was at the time seen as a natural data-focused evolution from the document-centric SGML and HTML).

While triple-based knowledge representations were not new \cite{stanczykProcessModellingInformation1987}, the main innovation of RDF was the use of global identifiers in the form of URIs\footnote{URIs \cite{rfc3986} are generalised forms of URLs that include locator-less identifiers such as ISBN book numbers (URNs). The distinction between locator-full and locator-less identifiers have weakened in recent years \cite{InfoURIRegistry}, for instance DOI identifiers now are commonly expressed with the prefix \texttt{https://doi.org/} rather than as URNs with \texttt{info:doi:} given that the URL/URN gap has been bridged by HTTP resolvers and the use of Persistent Identifiers (PIDs) \cite{jutyIdentifiersOrgMIRIAM2011}. RDF 1.1 formats use Unicode to support IRIs \cite{Duerst 2005}, which extends URIs to include international characters and domain names.} as the primary identifier of the \emph{subject} (what the statement is about), \emph{predicate} (relation/attribute of the subject) and \emph{object} (what is pointed to). By using URIs not just for documents\footnote{URIs can also identify \emph{non-information resources} for any kind of physical object (e.g.~people), such identifiers can resolve with \texttt{303\ See\ Other} redirections to a corresponding \emph{information resources} \cite{sauermannCoolURIsSemantic2011}.}, the Semantic Web builds a self-described system of types and properties, where the meaning of a relation can be resolved by following its hyperlink to the definition within a \emph{vocabulary}. By applying these principles as well to any kind of resource that could be described at a URL, this then forms a global distributed Semantic Web.

The early days of the Semantic Web saw fairly lightweight approaches with the establishment of vocabularies such as FOAF (to describe people and their affiliations) and Dublin Core (for bibliographic data). Vocabularies themselves were formalised using RDFS or simply as human-readable HTML web pages defining each term. The main approach of this \emph{Web of Data} was that a URI identified a \emph{resource} (e.g.~an author) with a HTML \emph{representation} for human readers, along with a RDF representation for machine-readable data of the same resource. By using \emph{content negotiation} in HTTP\footnote{\url{https://developer.mozilla.org/en-US/docs/Web/HTTP/Content_negotiation}}, the same identifier could be used in both views, avoiding \texttt{index.html} vs \texttt{index.rdf} exposure in the URLs. The concept of \emph{namespaces} gave a way to give a group of RDF resources with the same URI base from a Semantic Web-aware service a common \emph{prefix}, avoiding repeated long URLs.

The mid-2000s saw large academic interest and growth of the Semantic Web, with the development of more formal representation system for ontologies, such as OWL \cite{w3-owl2-overview}, allowing complex class hierarchies and logic inference rules following \emph{open world} paradigm (e.g.~a \emph{ex:Parent} is equivalent to a subclass of \emph{foaf:Person} which must \emph{ex:hasChild} at least one \emph{foaf:Person}, then if we know \emph{:Alice a ex:Parent} we can infer \emph{:Alice ex:hasChild {[}a foaf:Person{]}} even if we don't know who that child is). More human-readable syntaxes of RDF such as Turtle (shown in this paragraph) evolved at this time, and conferences such as \href{https://iswc2022.semanticweb.org/}{ISWC} \cite{horrocksSemanticWebISWC2002} gained traction, with a large interest in knowledge representation and logic systems based on Semantic Web technologies evolving at the same time.

Established Semantic Web services and standards include SPARQL \cite{w3-sparql11-overview} (pattern-based triple queries), \href{https://www.w3.org/TR/rdf11-concepts/\#section-dataset}{named graphs} \cite{w3-rdf11-concepts} (triples expanded to \emph{quads} to indicate statement source or represent conflicting views), triple/quad stores (graph databases such as OpenLink Virtuoso, GraphDB, 4Store), mature RDF libraries (including Redland RDF, Apache Jena, Eclipse RDF4J, RDFLib, RDF.rb, rdflib.js), and numerous graph visualisation (many of which struggle with usability for more than 20 nodes).

The creation of RDF-based knowledge graphs grew particularly in fields like bioinformatics, e.g.~for describing genomes and proteins \cite{gobleStateNationData2008c,williamsOpenPHACTSSemantic2012c}. In theory, the use of RDF by the life sciences would enable interoperability between the many data repositories and support combined views of the many aspects of bio-entities -- however in practice most institutions ended up making their own ontologies and identifiers, for what to the untrained eye would mean roughly the same. One can argue that the toll of adding the semantic logic system of rich ontologies meant that small, but fundamental, differences in opinion (e.g.~\emph{should a gene identifier signify just the particular DNA sequence letters, or those letters as they appear in a particular position on a human chromosome?}) lead to large differences in representational granularity, and thus needed different identifiers.

Facing these challenges, thanks to the use of universal identifiers in the form of URIs, \emph{mappings} could retrospectively be developed not just between resources, but also across vocabularies. Such mappings can be expressed themselves using lightweight and flexible RDF vocabularies such as SKOS \cite{w3-skos-primer} (e.g.~\texttt{dct:title\ skos:closeMatch\ schema:name} to indicate near equivalence of two properties). Automated ontology mappings have identified large potential overlaps (e.g.~372 definitions of \texttt{Person}) \cite{huHowMatchableAre2011a}.

The move towards \emph{Open Science} data sharing practices did from the late 2000s encourage knowledge providers to distribute collections of RDF descriptions as downloadable \emph{datasets} \footnote{\emph{Datasets} that distribute RDF graphs should not be confused with \href{https://www.w3.org/TR/rdf11-concepts/\#section-dataset}{RDF Datasets} used for partitioning \emph{named graphs}.}, so that their clients can avoid thousands of HTTP requests for individual resources. This enabled local processing, mapping and data integration across datasets (e.g.~Open PHACTS \cite{grothAPIcentricLinkedData2014b}), rather than relying on the providers' RDF and SPARQL endpoints (which could become overloaded when handling many concurrent, complex queries).

With these trends, an emerging problem was that adopters of the Semantic Web primarily utillised it as a set of graph technologies, with little consideration to existing Web resources. This meant that links stayed mainly within a single information system, with little URI reuse even with large term overlaps \cite{kamdarSystematicAnalysisTerm2017a}. Just like \emph{link rot} affect regular Web pages and their citations from scholarly communication \cite{kleinScholarlyContextNot2014a}, for a majority of described RDF resources in the \href{https://lod-cloud.net/}{Linked Open Data} (LOD) Cloud's gathering of more than thousand datasets, unfortunately do not actually link to (still) downloadable (\emph{dereferenceable}) Linked Data \cite{polleresMoreDecentralizedVision2020a}. Another challenge facing potential adopters is the plethora of choices, not just to navigate, understand and select to reuse the many possible vocabularies and ontologies \cite{carrieroLandscapeOntologyReuse2020a}, but also technological choices on RDF serialisation (at least \href{https://www.w3.org/TR/rdf11-primer/\#section-graph-syntax}{7 formats}), type system (RDFS \cite{w3-rdf-schema}, OWL \cite{w3-owl2-overview}, OBO \cite{tirmiziMappingOBOOWL2011a}, SKOS \cite{w3-skos-primer}), hash vs slash, HTTP status codes and PID redirection strategies \cite{sauermannCoolURIsSemantic2011}.

\subsubsection{Linked Data: Rebuilding the Web of Data}\label{ch3:ld-web}

The \textbf{Linked Data} concept \cite{Bizer 2009} was kickstarted as a set of best practices \cite{LinkedDataDesign} to bring the Web aspect back into focus. Crucially to Linked Data is the \emph{reuse of existing URIs}, rather than making new identifiers. This means a loosening of the semantic restrictions previously applied, and an emphasis on building navigable data resources, rather than elaborate graph representations.

Vocabularies like \href{https://schema.org/}{schema.org} evolved not long after, intended for lightweight semantic markup of existing Web pages, primarily to improve search engines' understanding of types and embedded data. In addition to several such embedded \emph{microformats} \cite{OpenGraphProtocol,w3-rdfa-primer,HTMLStandard} or we find JSON-LD \cite{w3-json-ld} as a Web-focused RDF serialisation that aims for improved programmatic generation and consumption, including from Web applications. JSON-LD is as of 2023-05-18 used\footnote{Presumably this large uptake of JSON-LD is mainly for the purpose of Search Engine Optimisation (SEO), with typically small amounts of metadata which may not constitute Linked Data as introduced above, however this deployment nevertheless constitute machine-actionable structured data.} by 45\% of the top 10 million websites \cite{UsageStatisticsJSONLD}.

Recently there has been a renewed emphasis to improve the \emph{Developer Experience} \cite{DesigningLinkedData2018} for consumption of Linked Data, for instance RDF Shapes -- expressed in SHACL \cite{w3-shacl} or ShEx \cite{ShapeExpressionsShEx} -- can be used to validate RDF Data \cite{gayoValidatingRDFData2017a,thorntonUsingShapeExpressions2019a} before consuming it programmatically, or reshaping data to fit other models. While a varied set of tools for Linked Data consumptions have been identified, most of them still require developers to gain significant knowledge of the underlying Semantic Web technologies, which hampers adaption by non-LD experts \cite{klimekSurveyToolsLinked2019a}, which then tend to prefer non-semantic two-dimensional formats such as CSV files.

A valid concern is that the Semantic Web research community has still not fully embraced the Web, and that the ``final 20\%'' engineering effort is frequently overlooked in favour of chasing new trends such as Big Data and AI, rather than making powerful Linked Data technologies available to the wider groups of Web developers \cite{verborghSemanticWebIdentity2020a}. One bridging gap here by the Linked Data movement has been ``linked data by stealth'' approaches such as structured data entry spreadsheets powered by ontologies \cite{wolstencroftRightFieldEmbeddingOntology2011b}, the use of Linked Data as part of REST Web APIs \cite{pageRESTLinkedData2011}, and as shown by the big uptake by publishers to annotate the Web using schema.org \cite{bernsteinNewLookSemantic2016a}, with vocabulary use patterns documented by copy-pastable JSON-LD examples, rather than by formalised ontologies or developer requirements to understand the full Semantic Web stack.



\subsection{Method}\label{ch3:method}

\subsubsection{Comparing FDO and existing approaches}\label{ch3:comparing}

Our main motivation for this article is to investigate how the promises of FAIR Digital Objects may differ from the learnt experiences of Linked Data and the Web. We also reflect back from FDO's motivation of machine-actionability to consider the Web as a distributed computational system.

To better understand the relationship between the FDO framework and other exisiting approaches, we use the following for analysis:

\begin{enumerate}
\tightlist
\item
  An Interoperability Framework and Distributed Platform for Fast Data Applications \cite{delgadoInteroperabilityFrameworkDistributed2016a}, which proposes quality measurements for comparing how frameworks support interoperability, particularly from a service architectural view.
\item
  The FAIR Digital Object guidelines \cite{bonino2019}, validated against its current implementations for completeness.
\item
  A Comparison Framework for Middleware Infrastructures \cite{zarrasComparisonFrameworkMiddleware2004a}, which suggest dimensions like openness, performance and transparency, mainly focused on remote computational methods.
\item
  Cross-checks against RDA's FAIR Data Maturity Model \cite{bahimFAIRDataMaturity2020a} to find how the FAIR principles are achieved in FDO, in particular considering access, sharing and openness.
\item
  EOSC Interoperability Framework \cite{eosc-interop-framework} which gives recommendations for technical, semantic, organisational and legal interoperability, particularly from a metadata perspective.
\end{enumerate}

The reason for this wide-ranged comparison is to exercise the different dimensions that together form FAIR Digital Objects: Data, Metadata, Service, Access, Operations, Computation.
We have left out further comparisons on type systems, persistent identifiers and social aspects as principles and practices within these dimensions are still taking form within the FDO community (as detailed \vpageref*{ch3:next-step-fdo}).

Some of these frameworks invite a comparison on a conceptual level, while others relate better to implementations and current practices. For these we consider FAIR Digital Objects and the Web conceptually, and for implementations we contrast between the main FDO realisation using the DOIPv2 protocol \cite{DONA 2018} against Linked Data in general practice.


\subsubsection{Considering FDO/Web as interoperability framework for Fast Data}\label{ch3:interoperability-compare}


The Interoperability Framework for Fast Data Applications \cite{delgadoInteroperabilityFrameworkDistributed2016a} categorises interoperability between applications along 6 strands, covering different architectural levels: from \emph{symbiotic} (agreement to cooperate) and \emph{pragmatic} (ability to choreograph processes), through \emph{semantic} (common understanding) and \emph{syntactic} (common message formats), to low-level \emph{connective} (transport-level) and \emph{environmental} (deployment practices).

We have chosen to investigate using this framework as it covers the higher levels of the OSI Model \cite{stallingsHandbookComputercommunicationsStandards1990} better with regards to automated machine-to-machine interaction (and thus interoperability), which is a crucial aspect of the FAIR principles. In Table \vref{ch3:fdo-web-interoperability-framework} we use the interoperability framework to compare the current FAIR Digital Object approach against the Web and its Linked Data practices.

\renewcommand*{\arraystretch}{1.4}
\begin{longtable}[]{@{}
  >{\raggedright\arraybackslash}p{(\columnwidth - 4\tabcolsep) * \real{0.1642}}
  >{\arraybackslash}p{(\columnwidth - 4\tabcolsep) * \real{0.4179}}
  >{\arraybackslash}p{(\columnwidth - 4\tabcolsep) * \real{0.4179}}@{}}
\caption{Considering FDO and Web according to the quality levels of the Interoperability Framework for Fast Data \cite{delgadoInteroperabilityFrameworkDistributed2016a}.
\label{ch3:fdo-web-interoperability-framework}}\tabularnewline
\toprule
\emph{Quality} & 
FDO w/ DOIP & 
Web w/ Linked Data \\
\midrule
\endfirsthead
\toprule
\emph{Quality} & 
FDO w/ DOIP & 
Web w/ Linked Data \\
\midrule
\endhead
\textbf{Symbiotic}: \emph{to what extent multiple applications can agree to interact, align, collaborate or cooperate}
  & The purpose of FDO is to enable federated machine actionable digital objects for scholarly purposes, in practice this also requires agreement of compatibility between FDO types. FDO encourages research communities to develop common type registries to be shared across instances. In current DOIP practice, each service have their own types, attributes and operations. The wider symbiosis is consistent use of PIDs.
  & The Web is loosely coupled and encourages collaboration and linking by URL. In practice, REST APIs \cite{fieldingArchitecturalStylesDesign2000a} end up being mandated centrally by dominant (often commercial) providers \cite{fieldingReflectionsRESTArchitectural2017a}, and the clients are required to use each API as-is with special code per service. Use of Linked Data enables common tooling and semantic mapping across differences. \\
\textbf{Pragmatic}: \emph{using interaction contracts so processes can be choreographed in workflows}
  & FDO types and operations enable detailed choreography (Canonical Workflow Frameworks for Research \cite{cwfr}). \texttt{0.TYPE/DOIPOperation} has lightweight definition of operation, \texttt{0.DOIP/Request} or \texttt{0.DOIP/Response} may give FDO Type or any other kind of ``specifics'' (incl.~human readable docs). Semantics/purpose of operations not formalised (similar operations can be grouped with \texttt{0.DOIP/OperationReference}).
  & ``Follow your nose'' crawler navigation, which may lead to frequent dead ends. Operational composition, typically within a single API provider, documented by OpenAPI 3 \cite{OpenAPISpecificationV3}, schema.org Actions \cite{SchemaOrgActions}, WSDL/SOAP \cite{w3-wsdl20-primer}, but frequently just as human-readable developer documentation/examples. \\
\textbf{Semantic}: \emph{ensuring consistent understanding of messages, interoperability of rules, knowledge and ontologies}
  & FDO semantic enable navigation and typing. Every FDO has a type. Types maintained in FDO Type registries, which may add additional semantics, e.g.~the ePIC \href{https://hdl.handle.net/21.11104/c1a0ec5ad347427f25d6}{PID-InfoType for Model}. No single type semantic, Type FDOs can link to existing vocabularies \& ontologies. JSON-LD used within some FDO objects (e.g.~DISSCO Digital Specimen, NIST Material Science schema) \cite{wittenburgFAIRDigitalObject2022b}
  & Lightweight HTTP semantics for authenticity/navigation. Semantic Type not commonly expressed on PID/header level, may be declared within Linked Data metadata. Semantic of type implied by Linked Data formats (e.g.~OWL2, RDFS), although choice of type system may not be explicit. \\
\textbf{Syntactic}: \emph{serialising messages for digital exchange, structure representation}
  & DOIP serialise FDOs as JSON, metadata commonly use JSON, typed with JSON Schema. Multiple byte stream attachments of any media type.
  & Textual HTTP headers (including any signposting), single byte stream of any media type, e.g.~Linked Data formats (JSON-LD, Turtle, RDF/XML) or embedded in document (HTML with RDFa, JSON-LD or Microdata). XML was previously the main syntax used by APIs, JSON is now dominant. \\
\textbf{Connective}: \emph{transferring messages to another application, e.g.~wrapping in other protocols}
  & DOIP \cite{DONA 2018} is transport-independent, commonly TLS TCP/IP port 9000), \href{https://www.cordra.org/documentation/api/doip-api-for-http-clients.html}{DOIP over HTTP}
  & HTTP/1.1 (TCP/IP port 80), HTTP/1.1+TLS (TCP/IP 443), HTTP/2 (as HTTP/1* but binary), HTTP/3 (like HTTP/2+TLS but UDP) \\
\textbf{Environmental}: \emph{how applications are deployed and affected by its environment, portability}
  & Main DOIP implementation is \href{https://www.cordra.org/}{Cordra}, which can be single-instance or \href{https://www.cordra.org/documentation/configuration/distributed-deployment.html}{distributed}. Cordra \href{https://www.cordra.org/documentation/configuration/storage-backends.html}{storage backends} include file system, S3, MongoDB (itself scalable). Unique DOIP protocol can be hard to add to existing Web application frameworks, although proxy services have been developed (e.g.~B2SHARE adapter).
  & HTTP services widely deployed in a myriad of ways, ranging from single instance servers, horizontally \& vertically scaled application servers, to (for static content) multi-cloud Content-Delivery Networks (CDN). Current scalable cloud technologies for Web hosting may not support HTTP features previously seen as important for Semantic Web, e.g.~content negotiation and semantic HTTP status codes. \\
\bottomrule
\end{longtable}


Based on the analysis shown in Table \ref{ch3:fdo-web-interoperability-framework}, we draw the following conclusions:

The Web has already showed us how one can compose workflows of hetereogeneous Web Services \cite{wolstencroftTavernaWorkflowSuite2013d}. However, this is mostly done via developer or human interaction \cite{lamprechtPerspectivesAutomatedComposition2021b}. Similiarly, FDO does not enable automatic composition because operation semantics are not well defined. There is a question as to whether the extebsuve documentation and broad developer usage that is available for Web APIs could potentially be utillised for FDO.

A difference between Web technologies and FDO is the stringency of the requirements for both syntax and semantics. Whereas the Web allows many different syntactic formats (e.g.~from HTML to XML, PDFs), FDO realised with DOIP requires JSON. On the semantic front, FDO mandates that every object have a well-defined type and structured form. This is clearly not the case on the Web.

In terms of connectivity and the deployment of applications, the Web has a plethora of software, services, and protocols that are widely deployed. These have shown interoperability. The Web standards bodies (e.g.~IETF and W3C) follow the OpenStand principles \cite{ModernStandardsParadigm} to embrace openness, transparency, and broad consensus. In contrast, FDO has a small number of implementations and corresponding protocols, although with a growing community, as evidenced at the first international FDO conference \cite{looFirstInternationalConference2022}. This is not to say that it is not worth developing further Handle+DOIP implementations in the future, but we note that the current FDO functionality can easily be implemented using Web technologies, even as DOIP-over-HTTP \cite{DOIPAPIHTTPa}.

It's also a question as to whether a highly constrained protocol revolving around persistent identifiers is in fact necessary. For example, DOIs are mostly resolved on the web \cite{DOIResolutionDocumentation} using HTTP redirects with the common \texttt{https://doi.org/} prefix, hiding their Handle nature as an implementation detail \cite{DOIHandbookResolution}.


\subsubsection{Mapping of Metamodel concepts}\label{mapping-of-metamodel-concepts}

The Interoperability Framework for Fast Data also provides a brief \emph{metamodel} which we use in Table \vref{ch3:metamodel-concepts} to map and examplify corresponding concepts in FDO's DOIP realization and the Web using HTTP semantics \cite{rfc9110}.

From this mapping we can identify the conceptual similarities between DOIP and HTTP, often with common terminology. Notable are that neither DOIP or HTTP have strong support for transactions (explored further \vpageref{ch3:middleware}), as well that HTTP has poor direct support for processes, as the Web is primarily stateless by design.

\begin{table}[h!]
  \centering
  \caption{Mapping the Metamodel concepts from the Interoperability Framework for Fast Data \cite{delgadoInteroperabilityFrameworkDistributed2016a} to equivalent concepts for FDO and Web.
  \label{ch3:metamodel-concepts}}\tabularnewline
   \begin{tabular}{ m{5em}  m{15em} m{15em} } 
   \hline
  Metamodel concept & 
  FDO/DOIP concept & 
  Web/HTTP concept \\ 
   \hline
  Resource	  & FDO/DO	            & Resource \\
  Service	    & DOIP service	      & Server/endpoint \\
  Transaction	& (not supported)	    & Conditional requests, \texttt{409\ Conflict} \\
  Process	    & Extended operations	& (primarily stateless), \texttt{100\ Continue}, \texttt{202\ Accepted} \\
  Operation	  & DOIP Operation	    & Method, query parameters \\
  Request	    & DOIP Request	      & Request \\
  Response	  & DOIP Response	      & Response \\
  Message	    & Segment, \texttt{requestId} 
                                    & Message, Representation \\
  Channel	    & DOIP Transport protocol (e.g.~TCP/IP, TLS). JSWS signatures.
                                    & TCP/IP, TLS, UDP \\
  Protocol  	& DOIP 2.0, ++	      & HTTP/1.1, HTTP/2, HTTP/3 \\
  Link	      & PID/Handle	        & URL \\ 
   \hline
   \end{tabular}\end{table}
  

\subsubsection{Assessing FDO implementations}\label{ch3:doip-fdo-compare}

The FAIR Digital Object guidelines \cite{bonino2019} sets out recommendations for FDO implementations. In Table \vref{ch3:fdo-checks} we evaluate the two current implementations, using DOIPv2 \cite{DONA 2018} and using Linked Data Platform \cite{w3-ldp}, as proposed by \cite{bonino2021}.

\begin{landscape}
  \begin{small}
  \begin{longtable}[]{@{}
    >{\centering\arraybackslash}p{(\columnwidth - 8\tabcolsep) * \real{0.1}}
    >{\raggedleft\arraybackslash}p{(\columnwidth - 8\tabcolsep) * \real{0.25}}
    >{\raggedright\arraybackslash}p{(\columnwidth - 8\tabcolsep) * \real{0.2}}
    >{\raggedleft\arraybackslash}p{(\columnwidth - 8\tabcolsep) * \real{0.25}}
    >{\raggedright\arraybackslash}p{(\columnwidth - 8\tabcolsep) * \real{0.2}}@{}}
    \caption[Checking FDO guidelines against its implementations]{Checking FDO guidelines \cite{bonino2019,fdo-RequirementSpec} against its current implementations as DOIP \cite{DONA 2018} and Linked Data Platform (LDP) \cite{bonino2021}, with suggestions for required additions.
  \label{ch3:fdo-checks}}\tabularnewline
  \toprule
  \textbf{FDO Guideline} & 
  DOIP 2.0 & 
  FDO suggestions & 
  Linked Data Platform & 
  LDP suggestion \\
  \midrule
  \endfirsthead
  \toprule
  \textbf{FDO Guideline} & 
  DOIP 2.0 & 
  FDO suggestions & 
  Linked Data Platform & 
  LDP suggestion \\
  \midrule
  \endhead
G1: \emph{invest for many decades}
  & Handle system stable for 20 years, DOIP 2.0 since 2017.
  & Ensure FDO types will not be protocol-bound as DOIP might be updated/replaced
  & HTTP stable for 30 years, Semantic Web for 20 years. \texttt{http://} URIs replaced by \texttt{https://}.
  & Keep flexibility of RDF serialisation formats which may change more frequently \\
G2: \emph{trustworthiness}
  & DOI/Handle trusted by all major academic publishers and data repositories. DOIP relatively unknown, in effect only one implementation.
  & Further promote DOIP and justify its benefits. Build tutorials and OSI open source implementations. Standardise DOIP-over-HTTP alternative.
  & JSON-LD used by half of all websites \cite{UsageStatisticsJSONLD}, however previous bad experiences with Semantic Web may deter adopters
  & Ensure simplicity for end developers, rather than semantic overengineering. Example-driven documentation. \\
G3: \emph{follows FAIR principles}
  & See Table \vref{ch3:fair-data-maturity-model}
  & Ensure all FAIR principles are covered, build complete examples.
  & Touched briefly, see Table \vref{ch3:fair-data-maturity-model}
  & Add explicit expression to show each FAIR pcinciple fulfilled. \\
G4: \emph{machine actionability}
  & CRUD and extension operations dynamically listed (see Table \vref{ch3:fdo-web-middleware})
  & Specify which operations should work for a given type, to reduce need for dynamic lookup. Specify input/output expectations formally (e.g.~JSON Schema).
  & HTTP CRUD operations, Open API (see Table \vref{ch3:fdo-web-middleware})
  & Document operations so client can make subsequent HTTP calls. \\
G5: \emph{abstraction principle}
  & Handle PIDs as abstraction base. DOIP operations can use any transport protocol.
  & Document transport protocols as FDOs, recommend which transport to use.
  & URI as abstraction base. Does not specify PID requirements.
  & Give stronger deployment recommendations. \\
G6: \emph{stable binding between entities}
  & Machine-navigation through PIDs and operations specified per type. Unclear when metadata field is a PID or plain text.
  & Make datatype of fields explicit to support navigation.
  & Machine-navigation through URIs via properties and types. Unclear when URI should be followed or is just identifier, but always distinct from plain text.
  & \\
G7: \emph{encapsulation}
  & Operations discovered at runtime (\texttt{0.DOIP/Op.ListOperations}).
  & Allow method discovery by type FDOs in advance (see PR-TypingFDOs-2.0-20220608).
  & HTTP methods discovered at runtime (\texttt{OPTIONS}), indempotent methods attempted directly. Unsupported methods reported using LDP constraints to human-readable text.
  & Declare supported methods in advance, e.g.~OpenAPI \cite{OpenAPISpecificationV3} \\
G8: \emph{technology independence}
  & In theory independent, in reality depends on single implementations of Handle system and DOIP
  & Encourage open source DOIP testbeds and lighter reference implementations
  & Multiple HTTP implementations, multiple LDP implementations. No FDOF implementations.
  & Develop demonstrator of FDOF usage based on existing LDP server. \\
G9: \emph{standard compliance}
  & Handle \cite{rfc3650}, DOIP \cite{DONA 2018}. FDO requirements not standardised yet.
  & Formalise standard process of FDO requirements \cite{fdo-DocProcessStd}
  & HTTP, LDP. FDOF not yet standardised
  & Formalise FDOF from FDOF-SEM working group \\
FDOF1: \emph{PID as basis}
  & Extensive use of Handle system.
  & Clarify how local testing handles can be used during development, how ``temporary'' FDOs should evolve \cite{fdo-PIDProfileAttributes}. Register \texttt{0.DOIP/*} and \texttt{0.FDO/*} as PIDs.
  & HTTP URLs as basis for identifiers, but they are frequently not persistent.
  & Add strong guidance for PID services like w3id and persistence policies. \\
FDOF2: \emph{PID record w/ type}
  & Unspecified how to resolve from Handle to DOIP Service (!), in practice \texttt{10320/loc}, \texttt{0.TYPE/DOIPService}, \texttt{URL}, \texttt{URL\_REPLICA}
  & Document requirements for PID Record
  & w3id/purl PIDs redirect without giving any metadata. Datacite DOIs content-negotiate to give registered metadata.
  & Add FAIR Signposting at PID provider for minimal PID record \\
FDOF3: \emph{PID resolvable to bytestream \& metadata}
  & Byte stream resolvable (\texttt{0.DOIP/Retrieve}), \texttt{includeElementData} option can retrieve bytestream or full object structure. No method/attribute defined for separate metadata, only directly in PID Record. Unclear meaning of multiple items and bytestream chunks.
  & Clarify expectations for multiple items. Recommend chunks to not be used.
  & URIs resolvable by default. Multiple ways to resolve metadata, unclear preference.
  & Add FAIR Signposting and preference order. \\
FDOF4: \emph{Additional attributes}
  & Freetext attribute keys. Attributes should be defined for FDO type (?).
  & Require that attribute keys should be PIDs (or have predefined mapping to PIDs). Explicitly allow attributes not already defined in type.
  & All attributes individually identified. Any Linked Data attributes can be used by URI or with mapped prefix.
  & Clarify type expectations of required/recommended/optional attributes. \\
FDOF5: \emph{Interface: operation by PID}
  & Extended operations use PID, but ``pid-like'' DOIP operations/types are not registered as handles.
  & Register \texttt{0.DOIP/*} and \texttt{0.FDO/*} as PIDs. Clarify that operations can be mapped to protocol directly.
  & CRUD operations used directly in HTTP (e.g.~\texttt{PUT}). Unclear how to provide PID for additional operations.
  & Specify how additional operations should be called over HTTP. \\
FDOF6: \emph{CRUD operations + extensions}
  & \texttt{0.DOIP/Op.Create}, \texttt{Op.Retrieve}, \texttt{Op.Update}, \texttt{Op.Delete} but also \texttt{0.DOIP/Op.Search}.
  & Document
  & \texttt{PUT}, \texttt{GET}, \texttt{POST}, \texttt{DELETE}, \texttt{PATCH}, \texttt{HEAD} -- extension operations (e.g.~WebDAV \texttt{COPY}) not used, resource patterns \cite{martinekuanWebAPIDesign} are used instead.
  & Document how operation resources can be discovered from an LPD container. Document search API. \\
FDOF7: \emph{FDOF Types related to operations}
  & Not yet formalised, by DOIP discoverable on a given FDO rather than type. PR-TypingFDOs leaves this open.
  & Add explicit relation between type and operations
  & \texttt{OPTIONS} per LDP Resource, but not by type. Common types (\texttt{ldp:Resource}, \texttt{ldp:Container}) indicate LDP support, but are not required.
  & Always make LDP types explicit in FDO profile. \\
FDOF8: \emph{Metadata as FDO, semantic assertions}
  & DOIP includes all metadata in PID Record. Separate Metadata FDO need custom property.
  & Specify a \texttt{0.FDO/metadata} or similar to point to Metadata FDOs.
  & Assertions are always with semantics, using RDF vocabularies. Unspecified how to find additional metadata resources, \texttt{rdfs:seeAlso} is common.
  & Use FAIR Signposting \texttt{describedby} link relation to additional metadata PIDs \\
FDOF9: \emph{Different metadata levels}
  & Defines open-ended ``Response Attributes'' without namespaces, but mandated as ``None'' for all CRUD operations. Metadata would need to be bundled within custom FDO types or attributes. Unclear how levels are separated within single FDO representation (need FDOF8?).
  & Declare which metadata are expected within response attribute or within FDO object. Require PIDs for custom attributes. Define how alternate metadata levels can be represented separately.
  & Undefined how to handle multiple metadata granularities or domains, alternative LDP containers can present different views on same stored objects.
  & Define how to navigate to alternate views and their semantic implications, e.g.~\texttt{owl:sameAs} \\
FDOF10: \emph{Metadata schemas by community}
  & Metadata schemas are in practice managed on single CORDA server as local types, using JSON Schema.
  & Require types to be FDOs with registered PIDs, implement shared types.
  & Plethora of existing RDF vocabularies/ontologies managed by larger communities, e.g.~\href{https://obofoundry.org/}{OBO Foundry} \cite{smithOBOFoundryCoordinated2007a}
  & Rather document better how individual ad-hoc schemas can be started for prototypes. \\
FDOF11: \emph{FDO collections w/ semantic relations}
  & Collection type undefined by DOIP. Informal use of \texttt{HAS\_PARTS} Handle attribute (e.g. \cite{DataInformationView}).
  &
  & LDP Containers required by specification, also user-created (eg. \texttt{BasicContainer}).
  & Clarify relation to other collections like DCAT 3 \cite{w3-vocab-dcat-3}, \href{https://schema.org/Dataset}{Schema.org Dataset}, OAI-ORE \cite{ORESpecificationAbstract} \\
FDOF12: \emph{Deleted FDO preserve PID w/ tombstone}
  & Tombstone for deleted resource undefined by DOIP. \texttt{0.DOIP/Status.104} status code does not distinguish ``Not Found'' or ``Gone''
  & Formalise tombstone requirements with new FDO type
  & \texttt{410\ Gone} recommended, but \texttt{404\ Not\ Found} common. No requirement for tombstone serialisation
  & Formalise tombstone requirements and serialisation \\
\bottomrule
\end{longtable}
\end{small}
\end{landscape}

Note that the draft update to FDO specification \cite{fdo-RequirementSpec} clarifies these definitions with equivalent identifiers\footnote{Newer \cite{fdo-RequirementSpec} renames \texttt{FDOF*} to \texttt{FDOR*} but follows same ordering.} and relates them to further FDO requirements such as FDO Data Type Registries.

A key observation from this is that simply using DOIP does not achieve many of the FDO guidelines. Rather the guidelines set out how a protocol like DOIPs should be used to achieve FAIR Digital Object goals. The DOIP Endorsement \cite{fdo-DOIPEndorsement} sets out that to comply, DOIP must be used according to the set of FDO requirement documents (details \vpageref{ch3:next-step-fdo}), and notes \emph{Achieving FDO compliance requires more than DOIP and full compliance is thus left to system designers}. Likewise, a Linked Data approach will need to follow the same requirements to comply as an FDO implementation.

From our evaluation, we can observe:

\begin{itemize}
  \item
    G1 and G2 call for stability and trustworthiness. While the foundations of both DOIP and Linked Data approaches are now well established -- the FDO requirements and in particular how they can be implemented are still taking shape and subject to change.
  \item
    Machine actionability (G4, G6) is a core feature of both FDOs and Linked Data. Conceptually they differ in the which way types and operations are discovered, with FDO seemingly more rigorous. In practice, however, we see that DOIP also relies on dynamic discovery of operations and that operation expectations for types (FDOF7) have not yet been defined.
  \item
    FDO proposes that types can have additional operations beyond CRUD (FDOF5, FDOF6), while Linked Data mainly achieves this with RESTful patterns using CRUD on additional resources, e.g.~\texttt{order/152/items}. These are mainly stylistics but affect the architectural view -- FDOs have more of an object-oriented approach.
  \item
    FDO puts strong emphasis on the use of PIDs (FDOF1, FDOF2, FDOF3, FDOF5), but in current practice DOIP use local types, local extended operations (FDOF5) and attributes (FDOF4) that are not bound to any global namespace.
  \item
    Linked Data have a strong emphasis on semantics (FDOF8), and metadata schemas developed by community agreements (FDOF10). FDO types need to be made reusable across servers.
  \item
    While FDO recommends nested metadata FDOs (FDOF8, FDOF9), in practice this is not found (or linked with custom keys), particularly due to lack of namespaces and the favouring of local types rather than type/property re-use. Linked Data frequently have multiple representations, but often not sufficiently linked, perhaps \texttt{prov:specializationOf} \cite{w3-prov-o}
  \item
    FDO collections are not yet defined for DOIP, while Linked Data seemingly have too many alternatives, LDP has specific native support for containers.
  \item
    Tombstones for deleted resources are not well supported, nor specified, for either approach, although the continued availability of metadata when data is removed is a requirement for FAIR principles (see RDA-A2-01M in Table \vref{RDA-A2-01M}).
  \item
    DOIP supports multiple chunks of data for an object (FDOF3), while Linked Data can support content-negotiation. In either case it can be unclear to clients what is the meaning or equivalence of any additional chunks.
  \end{itemize}

\subsubsection{Comparing FDO and Web as middleware infrastructures}\label{ch3:middleware}

In this section we take the perspective that FDO principles are in effect proposing a global infrastructure of machine-actionable digital objects. As such we can consider implementations of FDO as \textbf{middleware infrastructures} for programmatic usage, and can evaluate them based on expectations for client and server developers.

We argue that the Web, with its now ubiquitous use of REST API \cite{fieldingArchitecturalStylesDesign2000a}, can be compared as a similar global middleware. Note that while early moves for developing Semantic Web Services \cite{fenselSemanticWebServices2011} attempted to merge the Web Service and RDF aspects, we are here considering mainly the current programmatic Web and its mostly light-weight use of 3 out of possible \emph{5 stars Linked Data} \cite{OpenData}.

For this purpose, we here utillise the Comparison Framework for Middleware Infrastructures \cite{zarrasComparisonFrameworkMiddleware2004a} that formalise multiple dimensions of openness, scalability, transparency, as well as characteristics known from Object-oriented programming such as modularity, encapsulation and inheritance.

\begin{landscape}
  \begin{small}
  \begin{longtable}[]{@{}
    >{\raggedright\arraybackslash}p{(\columnwidth - 4\tabcolsep) * \real{0.1642}}
    >{\raggedright\arraybackslash}p{(\columnwidth - 4\tabcolsep) * \real{0.4179}}
    >{\raggedright\arraybackslash}p{(\columnwidth - 4\tabcolsep) * \real{0.4179}}@{}}
    \caption[Comparing FAIR Digital Object and Web technologies as middleware infrastructures]{Comparing FAIR Digital Object (with the DOIP 2.0 protocol \cite{DONA 2018}) and Web technologies (using Linked Data) as middleware infrastructures \cite{zarrasComparisonFrameworkMiddleware2004a}
  \label{ch3:fdo-web-middleware}}\tabularnewline
  \toprule
  \emph{Quality} & 
  FDO w/ DOIP & 
  Web w/ Linked Data \\
  \midrule
  \endfirsthead
  \toprule
  \emph{Quality} & 
  FDO w/ DOIP & 
  Web w/ Linked Data \\
  \midrule
  \endhead
  \textbf{Openness}: \emph{framework enable extension of applications}
    & FDOs can be cross-linked using PIDs, pointing to multiple FDO endpoints. Custom DOIP operations can be exposed, although it is unclear if these can be outside the FDO server. PID minting requires Handle.net prefix subscription, or use of services like \href{https://datacite.org/}{Datacite}, \href{https://eudat.eu/services/userdoc/b2handle}{B2Handle}.
    & The Web is inherently open and made by cross-linked URLs. Participation requires DNS domain purchase (many free alternatives also exists). PID minting can be free using PURL/ARK services, or can use DOI/Handle with HTTP redirects. \\
  \textbf{Scalability}: \emph{application should be effective at many different scales}
    & No defined methods for caching or mirroring, although this could be handled by backend, depending on exposed FDO operations (e.g.~Cordra can scale to multiple backend nodes)
    & Cache control headers reduce repeated transfer and assist explicit and transparent proxies for speed-up. HTTP \texttt{GET} can be scaled to world-population-wide with Content-Delivery Networks (CDNs), while write-access scalability is typically manage by backend. \\
  \textbf{Performance}: \emph{efficient and predictable execution}
    & DOIP has been shown moderately scalable to 100 millions of objects, create operation at 900 requests/second . DOIP protocol is reusable for many operations, multiple requests may be answered out of order (by \texttt{requestId}). Multiple connections possible. Setup is typically through TCP and TLS which adds latency.
    & HTTP traffic is about 10\% of global Internet traffic, excluding video and social networks \cite{sandvineGlobalInternetPhenomena}. HTTP 1 connections are serial and reusable, and concurrent connections is common. HTTP/2 adds asynchronous responses and multiplexed streams \cite{rfc7540} but still has TCP+TLS startup costs. For reduced latency, HTTP/3 \cite{rfc9114} use QUIC \cite{rfc9000} rather than TCP, already adapted heavily (30\% of EMEA traffic) of which Instagram \& Facebook video is the majority of traffic \cite{joras2020}. \\
  \textbf{Distribution transparency}: \emph{application perceived as a consistent whole rather than independent elements.}
    & Each FDO is accessed separately along with its components (typically from the same endpoint). FDOs should provide the mandatory kernel metadata fields. FDOs of the same declared type typically share additional attributes (although that schema may not be declared). DOIP does not enforce metadata typing constraints, this need to be established as FDO conventions.
    & Each URL accessed separately. Common HTTP headers provide basic metadata, although it is often not reliable. A multitude of schemas and serializations for metadata exists, conventions might be implied by a declared profile or certain media types. Metadata is not always machine findable, may need pre-agreed API URI Templates \cite{rfc6570}, content-negotiation \cite{ContentNegotiationHTTP} or FAIR Signposting \cite{Van de Sompel 2022}. \\
  \textbf{Access transparency}: \emph{local/remote elements accessed similarly}
    & FDOs should be accessed through PID indirection, this means difficult to make private test setup. Commonly a fixed DOIP server is used directly, which permits local non-PID identifiers.
    & Global HTTP protocol frequently used locally and behind firewalls, but at risk of non-global URIs (e.g.~\texttt{http://localhost/object/1}) and SSL issues (e.g.~self-signed certificates, local CAs) \\
  \textbf{Location transparency}: \emph{elements accessed without knowledge of physical location}
    & FDOs always accessed through PIDs. Multiple locations possible in Handle system, can expose geo-info.
    & PIDs and URL redirects. DNS aliases and IP routing can hide location. Geo-localised servers common for large cloud deployments. \\
  \textbf{Concurrency transparency}: \emph{concurrent processing without interference}
    & No explicit concurrency measures. FDO kernel metadata can include checksum and date.
    & HTTP operations are classified as being stateless/idempotent or not (e.g.~\texttt{PUT} changes state, but can be repeated on failure), although these constraints are occassionally violated by Web applications. Cache control, \texttt{ETag} (e.g. checksum) and modification date in HTTP headers allows detection of concurrent changes on a single resource. \\
  \textbf{Failure transparency}: \emph{service provisioning resilient to failures}
    & DOIP status codes, e.g.~\texttt{0.DOIP/Status.104}, additional codes can be added as custom attributes
    & HTTP \href{https://datatracker.ietf.org/doc/html/rfc7231\#section-6.5}{status codes} e.g.~\texttt{404\ Not\ Found}, specific meaning of standard codes can be \href{https://swagger.io/docs/specification/describing-responses/}{documented in Open API}. Custom codes uncommon. \\
  \textbf{Migration transparency}: \emph{allow relocating elements without interfering application}
    & Update of PID record URLs, indirection through \texttt{0.TYPE/DOIPServiceInfo} (not always used consistently). No redirection from DOIP service.
    & HTTP \texttt{30x} status codes provide temporary or permanent redirections, commonly used for PURLs but also by endpoints. \\
  \textbf{Persistence transparency}: \emph{conceal deactivation/reactivation of elements from their users}
    & FDO requires use of PIDs for object persistence, including a tombstone response for deleted objects. There is no guarantee that an FDO is immutable or will even stay the same type (note: CORDRA extends DOIP with \href{https://www.cordra.org/documentation/design/object-versioning.html}{version tracking}).
    & URLs are not required to persist, although encouraged \cite{berners-lee-cool-uris}. Persistence requires convention to use PIDs/PURLs and HTTP \texttt{410\ Gone}. An URL may change its content, change in type may sometimes force new URLs if exposing extensions like \texttt{.json}. Memento \cite{rfc7089} expose versioned snapshots. WebDAV \texttt{VERSION-CONTROL} method \cite{rfc3253} (used by SVN). \\
  \textbf{Transaction transparency}: \emph{coordinate execution of atomic/isolated transactions}
    & No transaction capabilities declared by FDO or DOIP. Internal synchronisation possible in backend for Extended operations.
    & Limited transaction capabilities (e.g.~\texttt{If-Unmodified-Since}) on same resource. WebDAV \href{https://datatracker.ietf.org/doc/html/rfc4918\#section-6}{locking mechanisms} \cite{rfc4918} with \texttt{LOCK} and \texttt{UNLOCK} methods. \\
  \textbf{Modularity}: \emph{application as collection of connected/distributed elements}
    & FDOs are inheritedly modular using global PID spaces and their cross-references. In practice, FDOs of a given type are exposed through a single server shared within a particular community/institution.
    & The Web is inheritently modular in that distributed objects are cross-referenced within a global URI space. In practice, an API's set of resources will be exposed through a single HTTP service, but modularity enables fine-grained scalability in backend. \\
  \textbf{Encapsulation}: \emph{separate interface from implementation. Specify interface as contract, multiple implementations possible}
    & Indirection by PID gives separation. FDO principles are protocol independent, although it may be unclear which protocol to use for which FDO (although \texttt{0.DOIP/Transport} can be specified after already contacting DOIP). Cordra supports \href{https://www.cordra.org/documentation/api/doip.html}{native DOIP}, \href{https://www.cordra.org/documentation/api/doip-api-for-http-clients.html}{DOIP over HTTP} and \href{https://www.cordra.org/documentation/api/rest-api.html}{Cordra REST API})
    & HTTP/1.1 semantics can seemlessly upgrade to HTTP/2 and HTTP/3. \texttt{http} vs \texttt{https} URIs exposes encryption detail\footnote{The \texttt{http} protocol (port 80) can in theory also upgrade \cite{rfc2817} to TLS encryption, as commonly used by \href{https://www.rfc-editor.org/rfc/rfc8010.html\#section-8.2}{Internet Printing Protocol} for \texttt{ipp} URIs, but on the Web, best practice is explicit \texttt{https} (port 443) URLs to ensure following links stay secure.}. Implementation details may leak into URIs (e.g.~\texttt{search.aspx}), countered by deliberate design of URI patterns \cite{berners-lee-cool-uris}) and PIDs via Persistent URLs (PURL). \\
  \textbf{Inheritance}: \emph{Deriving specialised interface from another type}
    & DOIP types nested with parents, implying shared FDO structures (unclear if operations are inherited). FDO establishes need for multiple Data Type Registries (e.g.~managed by a community for a particular domain). Semantics of type system currently undefined for FDO and DOIP, syntactic types can also piggyback of FDO type's schema (e.g.~\href{(https://www.cordra.org/documentation/design/schemas.html\#schema-references)}{CORDRA \texttt{\$ref}} use of \href{https://json-schema.org/draft/2020-12/json-schema-core.html\#references}{JSON Schema references} \cite{Draftbhuttonjsonschema})
    & Syntactically Media Type with multiple suffixes \cite{Draftietfmediamansuffixes00MediaTypes} (mainly used with \texttt{+json}), declaration of subtypes as profiles (RFC6906) \cite{rfc6906}. In metadata, semantic type systems (RDFS \cite{w3-rdf-schema}), OWL2 \cite{w3-owl2-overview}, SKOS \cite{w3-skos-primer}). OpenAPI 3 \cite{OpenAPISpecificationV3} \href{https://spec.openapis.org/oas/v3.1.0\#composition-and-inheritance-polymorphism}{inheritance and Polymorphism}. XML \texttt{xsd:schemaLocation} or \texttt{xsd:type} \cite{w3-xmlschema11}, JSON \texttt{\$schema} \cite{Draftbhuttonjsonschema}), JSON-LD \texttt{@context} \cite{w3-json-ld}. Large number of domain-specific and general ontologies define semantic types, but finding and selecting remains a challenge. \\
  \textbf{Signal interfaces}: \emph{asynchronous handling of messages}
    & DOIP 2.0 is synchronous, in FDO async operations undefined. Could be handled as custom jobs/futures FDOs
    & HTTP/2 \href{https://datatracker.ietf.org/doc/html/rfc7540\#section-5}{multiplexed streams} \cite{rfc7540}, Web Sockets \cite{WebSocketsStandard}, Linked Data Notifications \cite{w3-ldn}, AtomPub \cite{rfc5023}, SWORD \cite{SWORDSpecification}, Micropub \cite{w3-micropub}, more typically ad-hoc jobs/futures REST resources \\
  \textbf{Operation interfaces}: \emph{defining operations possible on an instance, interface of request/response messages}
    & CRUD predefined in DOIP, custom operations through \texttt{0.DOIP/Op.ListOperations} (can be FDOs of type \texttt{0.TYPE/DOIPOperation}, more typically local identifiers like \texttt{"getProvenance"})
    & CRUD predefined in \href{https://datatracker.ietf.org/doc/html/rfc7231\#section-4.3}{HTTP methods} \cite{rfc7231}, (\href{https://www.iana.org/assignments/http-methods/http-methods.xhtml}{extended by registration}), URI Templates \cite{rfc6570}, \href{https://spec.openapis.org/oas/v3.1.0.html\#operation-object}{OpenAPI operations} \cite{OpenAPISpecificationV3}, HATEOAS\footnote{HATEOAS: Hypermedia as the Engine of Application State \cite{fieldingArchitecturalStylesDesign2000a}, an important element of the REST architectural style.} incl.~Hydra \cite{HydraW3CCommunity}, schema.org Actions \cite{SchemaOrgActions}, JSON HAL \cite{Draftkellyjsonhal08} \& Link headers (RFC8288) \cite{rfc8288} \\
  \textbf{Stream interfaces}: \emph{operations that can handle continuous information streams}
    & Undefined in FDO. DOIP can support multiple byte stream elements (need custom FDO type to determine stream semantics)
    & HTTP 1.1 \cite{rfc7230} \href{https://datatracker.ietf.org/doc/html/rfc7230\#section-4.1}{chunked transfer}, HLS (RFC8216) \cite{rfc8216}, MPEG-DASH \cite{iso23009} \\
  \bottomrule
  \end{longtable}
  \end{small}
  \end{landscape}

Based on the analysis in Table \vref{ch3:fdo-web-middleware}, we make the following observations:

\begin{itemize}
  \item
    With respect to the aspect of \emph{Performance}, it is interesting to note that while the first version of DOIP \cite{DigitalObjectInterface} supported multiplexed channels similar to HTTP/2 (allowing concurrent transfer of several digital objects). Multiplexing was removed for the much simplified DOIP 2.0 \cite{DONA 2018}. Unlike DOIP 1.0, DOIP 2.0 will require a DO response to be sent back completely, as a series of segments (which again can be split the bytes of each binary \emph{element} into sized \emph{chunks}), before transmission of another DO response can start on the transport channel. It is unclear what is the purpose of splitting a binary into chunks on a channel which no longer can be multiplexed and the only property of a chunk is its size\footnote{Although it is possible with \texttt{0.DOIP/Op.Retrieve} to request only particular individual elements of an DO (e.g.~one file), unlike HTTP's \texttt{Range} request, it is not possible to select individual chunks of an element's bytestream.}.
  \item
    HTTP has strong support for scalability and caching, but this mostly assumes read-operations from static resources. FDO has no view on immutability or validity of retrieved objects, but this should be taken into consideration to support large-scale usage.
  \item
    HTTP optimisations for performance (e.g.~HTTP/2, multiplexing) is largely used for commercial media distribution (e.g.~Netflix), and not commonly used by providers of FAIR data
  \item
    Cloud deployment of Web applications give many middleware benefits (Scalability, Distribution, Access transparancy, Location transparancy) -- it is unclear how DOIP as a custom protocol would perform in a cloud setting as most of this infrastructure assumes HTTP as the protocol.
  \item
    Programmatically the Web is rather unstructured as middleware, as there are many implementation choices. Usually it is undeclared what to expect for a given URI/service, and programmers follow documented examples for a particular service rather than automated programmatic exploration across providers. This mean one can consider the Web as an ecosystem of smaller middlewares with commonalities.
  \item
    Many providers of FAIR Linked Data also provide programmatic REST API endpoints, e.g.~\href{https://www.uniprot.org/help/programmatic_access}{UNIPROT}, \href{https://chembl.gitbook.io/chembl-interface-documentation/web-services}{ChEMBL}, but keeping the FAIR aspects such as retrieving metadata in such a scenario may require combining different services using multiple formats and identifier conventions.
\end{itemize}


\subsubsection{Assessing FDO against FAIR}\label{ch3:fair-compare}

In addition to having ``FAIR'' in its name, the FAIR Digital Object guidelines \cite{fdo-RequirementSpec} also include \emph{G3: FDOs must offer compliance with the FAIR principles through measurable indicators of FAIRness}.

Here we evaluate to what extent the FDO guidelines and its implementation with DOIP and Linked Data Platform \cite{bonino2021} comply with the FAIR principles \cite{Wilkinson 2016}. Here we've used the RDA's FAIR Data Maturity Model \cite{groupFAIRDataMaturity2020} as it has decomposed the FAIR principles to a structured list of FAIR indicators \cite{bahimFAIRDataMaturity2020a}, importantly considering \emph{Data} and \emph{Metadata} separately. In our interpretation for Table \vref{ch3:fair-data-maturity-model} we have for simplicity chosen to interpret ``data'' in FDOs as the associated bytestream of arbitrary formats, with remaining JSON or RDF structures always considered as metadata.

\begin{landscape}
  \begin{small}
  \begin{longtable}[]{@{}
    >{\raggedright\arraybackslash}p{(\columnwidth - 10\tabcolsep) * \real{0.1}}
    >{\raggedright\arraybackslash}p{(\columnwidth - 10\tabcolsep) * \real{0.2}}
    >{\centering\arraybackslash}p{(\columnwidth - 10\tabcolsep) * \real{0.1}}
    >{\centering\arraybackslash}p{(\columnwidth - 10\tabcolsep) * \real{0.2}}
    >{\centering\arraybackslash}p{(\columnwidth - 10\tabcolsep) * \real{0.2}}
    >{\centering\arraybackslash}p{(\columnwidth - 10\tabcolsep) * \real{0.2}}@{}}
    \caption[Assessing RDA's FAIR Data Maturity Model against the FDO guidelines]{Assessing RDA's FAIR Data Maturity Model \cite{groupFAIRDataMaturity2020,bahimFAIRDataMaturity2020a} (first 2 columns) against the FDO guidelines \cite{bonino2019}, FDO implemented with the protocol DOIPv2 \cite{DONA 2018}, Linked Data Platform (LDP) \cite{bonino2021} and examples from Linked Data practices in general. (--- indicates \emph{Unspecified}, may be possible with additional conventions)
  \label{ch3:fair-data-maturity-model}}\tabularnewline
  \toprule
  FAIR ID &
  Indicator &
  FDO guidelines &
  FDO/DOIP &
  FDO/LDP &
  Linked Data examples \\
  \midrule
  \endfirsthead
  \toprule
  FAIR ID &
  Indicator &
  FDO guidelines &
  FDO/DOIP &
  FDO/LDP &
  Linked Data examples \\
  \midrule
  \endhead
RDA-F1-01M
  & Metadata is identified by a persistent identifier
  & FDOF4
  & Optional \emph{Metadata FDO} w/separate PID
  & Content-negotiation to URL, not required to be PID
  & Metadata typically don't have own PID \\
RDA-F1-01D
  & Data is identified by a persistent identifier
  & FDOF1
  & PIDs required (FDOF1). Handle, DOI.
  & FDOF-IR (Identifier Record). PID can be any URI
  & ``Cool'' URIs \cite{berners-lee-cool-uris}, PURL services incl.~\texttt{purl.org}, \texttt{w3id.org} \\
RDA-F1-02M
  & Metadata is identified by a globally unique identifier
  & FDOR4 FDOF8
  & Optional \emph{Metadata FDO}, unspecified how to indicate
  & Content-negotiation to URL
  & Not required, content-negotiation can redirect to URL or \texttt{Content-Location}. FAIR Signposting. \\
RDA-F1-02D
  & Data is identified by a globally unique identifier
  & FDOF1
  & All FDOs have PIDs (FDOR1), DOIP uses Handle system
  & FDOF-IR (Identifier Record)
  & Always accessed by URL \\
RDA-F2-01M
  & Rich metadata is provided to allow discovery
  & FDOF2 FDOF4 FDOF8 FDOF9
  & FDO has key-value metadata. Unclear how to link to additional metadata.
  & FDOF-IR links to multiple metadata records
  & RDF-based metadata by content negotiation or FAIR Signposting. Embedded in landing page (RDFa). \\
RDA-F3-01M
  & Metadata includes the identifier for the data
  & ---
  & \texttt{id} and \texttt{type} are required metadata elements PIDs, also implicit as requests must use PID
  & PID only required in FDOF-IR record.
  & PID inclusion typical, but often inconsistent (e.g.~\texttt{www.example.com} vs \texttt{example.com}) or missing (use of \texttt{\textless{}\textgreater{}} as \emph{this} subject) \\
RDA-F4-01M
  & Metadata is offered in such a way that it can be harvested and indexed
  & FDOF10
  & No, registries not required (except Data Type Registries). Handle registry only searchable by PID.
  & Not specified
  & Not specified, several registries/catalogues for vocabularies/types (e.g. \cite{NCBOBioPortal}). Indexing by search engines if exposing HTML w/schema.org. \\
RDA-A1-01M
  & Metadata contains information to enable the user to get access to the data
  & FDOF3 FDOF6
  & Directly by DOIP, but not included in FDO metadata. \texttt{handle.net} HTTP resolution may redirect to landing page
  & Any property can point to URIs, but unclear if it is data
  & Common with clickable ``follow your nose'' URLs \\
RDA-A1-02M
  & Metadata can be accessed manually (i.e.~with human intervention)
  & ---
  & (Cordra HTML landing page from \texttt{handle.net} URIs)
  & Optional content-negotiation, e.g.~by Apache Marmotta, OpenLink Virtuoso
  & HTTP content-negotiation to HTML is common \\
RDA-A1-02D
  & Data can be accessed manually (i.e.~with human intervention)
  & ---
  & (Cordra HTML landing page from \texttt{handle.net} URIs)
  & Optional content-negotiation
  & Direct download, HTML landing pages common for DOIs \\
RDA-A1-03M
  & Metadata identifier resolves to a metadata record
  & FDOF8+FDOF2
  & ---
  & ---
  & \texttt{Content-Location} or HTTP redirection may indicate metadata URI \\
RDA-A1-03D
  & Data identifier resolves to a digital object
  & FDOF2
  & Required, but frequently not directly resolvable
  & Recommended, but any URI acceptable
  & Resolvable HTTP/HTTPS URIs are most common, now infrequent URNs are not directly resolvable \\
RDA-A1-04M
  & Metadata is accessed through standardised protocol
  & G9 FDOF3
  & Retrievable from PID (FDOF3). Informal DOIP standard maintained by DONA Foundation
  & LDP standard maintained by W3C, HTTP standards maintained by IETF, FDO components resolved by informal proposals (custom vocabulary, extra HTTP methods) or HTTP content negotiation)
  & Formal HTTP standards maintained by IETF, HTTP content negotiation, informal FAIR Signposting \\
RDA-A1-04D
  & Data is accessible through standardised protocol
  & G9
  & (see above)
  & HTTP \cite{rfc9110}
  & HTTP/HTTPS, FTP (now less common), GridFTP \cite{allcockGlobusStripedGridFTP} (for large data), ARK \cite{ARKIdentifierScheme} \\
RDA-A1-05D
  & Data can be accessed automatically (i.e.~by a computer program)
  & G4 FDOF3 FDOF6
  & Required, but few client libraries
  & & Ubiquitous, hundreds of HTTP libraries \\
RDA-A1.1-01M
  & Metadata is accessible through a free access protocol    
  & G1 G8 G9
  & Partially realised: Handle system is open\footnote{
        The \texttt{Handle.net} system was previously covered by software patent \href{https://patents.google.com/patent/US6135646A/en}{US6135646A} which \href{https://circleid.com/posts/20161025_selling_dona_snake_oil_at_the_itu\#11461}{expired} in 2013.} 
    protocol \cite{rfc3652}. One server implementation \cite{HandleNetRegistry}, free\footnote{
        The \href{http://www.handle.net/HNRj/HNR-9-License.pdf}{Handle.net public license} is not OSI-approved \cite{LicensesStandardsOpen}  as an open source license -- it includes usage restrictions and requires Service Agreements. It is not a DOIP requirement to host a local Handle instance, e.g.~EOSC provides the \href{https://sp.eudat.eu/catalog/resources/fc6b2d30-09cd-4c25-b71a-7bc6de77910c}{B2HANDLE} service for acquiring Handle prefixes.}. 
    One DOIPv2 implementation (\href{https://www.cordra.org/}{CORDRA}): free under BSD-like license (not recognised as Open Source).    
  & LDP is open W3C recommendation \cite{w3-ldp}. \href{https://www.w3.org/wiki/LDP_Implementations}{Multiple LDP implementations}.    
  & DNS, HTTP, TLS, RDF standards are open, free and universal, large number of Open Source clients and \href{https://en.wikipedia.org/wiki/Comparison_of_web_server_software}{servers}. \\
RDA-A1.1-01D
  & Data is accessible through a free access protocol
  & G9
  & (see above)
  & URI, DNS, HTTP, TLS
  & URI, DNS, HTTP, TLS. Non-free DRM may be used (e.g.~subscription video streaming) \\
RDA-A1.2-01D
  & Data is accessible through an access protocol that supports authentication and authorisation
  & (FDOR9)
  & TLS certificates, \texttt{authentication} field (details unspecified)
  & Implied
  & HTTP authentication, TLS certificates \\
RDA-A2-01M\label{RDA-A2-01M}
  & Metadata is guaranteed to remain available after data is no longer available
  & FDOF12
  & ---
  & Unspecified, however FDOF-IR links to separate metadata records
  & --- \\
RDA-I1-01M
  & Metadata uses knowledge representation expressed in standardised format
  & FDOF8
  & Required, but not currently defined
  & ---
  & Always implied by use of RDF syntaxes. \\
RDA-I1-01D
  & Data uses knowledge representation expressed in standardised format
  & ---
  & ---
  & ---
  & Common (e.g.~HDF5, JSON, XML), yet common scientific data formats frequently not standardised \\
RDA-I1-02M
  & Metadata uses machine-understandable knowledge representation
  & FDOF8
  & Required
  & Optional RDF metadata with any vocabulary
  & Always implied by use of RDF syntaxes. \\
RDA-I1-02D
  & Data uses machine-understandable knowledge representation
  & G4 G7 FDOR2
  & No requirements on binary data formats
  & Only indirectly, \href{https://www.w3.org/TR/ldp/\#dfn-linked-data-platform-basic-container}{LDP Basic Container} reference only information resources
  & Common, specially for scientific data formats \\
RDA-I2-01M
  & Metadata uses FAIR-compliant vocabularies
  & G3 FDOF10
  & Informally required
  & Unspecified, implied by use of RDF?
  & FAIR practices for LD vocabularies increasingly common, sometimes inconsistent (e.g.~PURLs that don't resolve) or incomplete (e.g.~unknown license) \\
RDA-I2-01D
  & Data uses FAIR-compliant vocabularies
  & ---
  & ---
  & ---
  & Uncommon, except for some XML and RDF-embedding formats, e.g.~Extensible Metadata Platform (XMP) \cite{iso16684} \\
RDA-I3-01M
  & Metadata includes references to other metadata
  & FDOR8
  & Implied (attributes to PIDs), currently unspecified if given attribute is value or reference
  & ---
  & By definition (Linked Data reference existing URIs \cite{DataW3C}), \texttt{rdfs:seeAlso}, FAIR signposting \cite{Van de Sompel 2022} \texttt{describedby} \\
RDA-I3-01D
  & Data includes references to other data
  & G6 FDOR3 FDOR11
  & ---
  & ---
  & URL hyperlinks common in several formats (HTML, PDF, JSON, XML). \\
RDA-I3-02M
  & Metadata includes references to other data
  & G6 FDOR3 FDOR8
  & Implied from custom FDO type's attribute
  & LDP Direct Container members can be any resources
  & URI objects are frequently data references, may be indirect via PID \\
RDA-I3-02D
  & Data includes qualified references to other data
  & FDOR3 FDOR11
  & Only indirectly through FDO metadata
  & Indirectly through LDP membership
  & Uncommon: Link relations, FAIR Signposting \\
RDA-I3-03M
  & Metadata includes qualified references to other metadata
  & (FDOR3)
  & Qualification by attribute keys defined per FDO Type
  & \href{https://www.w3.org/TR/ldp/\#dfn-linked-data-platform-direct-container}{LDP Direct Container}
  & Qualifications by property, PROV bundles \cite{w3-prov-links}, \href{https://schema.org/Role}{schema.org/Role} \\
RDA-I3-04M
  & Metadata include qualified references to other data
  & (FDOR3)
  & Qualification by attribute keys defined per FDO type
  & \href{https://www.w3.org/TR/ldp/\#dfn-linked-data-platform-indirect-container}{LDP Indirect Container}
  & Qualifications by property, n-ary indirection (schema.org Role \cite{hollandIntroducingRole2014}, \texttt{prov:specializationOf} \cite{w3-prov-o}, OAI-ORE Proxy \cite{ORESpecificationAbstract}) \\
RDA-R1-01M
  & Plurality of accurate and relevant attributes are provided to allow reuse
  & FDOF4
  & Required. Kernel metadata attributes desired \cite{fdo-KernelAttributes} but not assigned PIDs yet.
  & Unspecified. Multiple metadata records can allow multiple semantic profiles.
  & Large number of general and domain-specific vocabularies can make it hard to find relevant attributes. Rough consensus on kernel metadata: schema.org \cite{schema.org}, Dublin Core Terms \cite{DCMIMetadataTerms}, DCAT \cite{DCAT2 2020}, FOAF \cite{FOAFVocabularySpecification} \\
RDA-R1.1-01M
  & Metadata includes information about the licence under which the data can be reused
  & ---
  & \texttt{licenseConditions} URL/PID in kernel metadata \cite{fdo-KernelAttributes}
  & ---
  & Dublin Core Terms \texttt{dct:license} frequently recommended, frequently not required, e.g.~\href{https://www.w3.org/TR/vocab-dcat-2/\#Property:distribution_license}{by DCAT 2} \cite{DCAT2 2020} \\
RDA-R1.1-02M
  & Metadata refers to a standard reuse licence
  & ---
  & ---
  & ---
  & \href{https://spdx.org/licenses/}{SPDX} and \href{https://creativecommons.org/}{Creative Commons} URIs common, identifiers often inconsistent \\
RDA-R1.1-03M
  & Metadata refers to a machine-understandable reuse licence
  & ---
  & ---
  & ---
  & \href{https://spdx.dev/resources/use/\#documents}{SPDX documents} uncommon \\
RDA-R1.2-01M
  & Metadata includes provenance information according to community-specific standards
  & FDOR9 FDOR10
  & Unspecified (some CORDRA types add getProvenance methods). PID Kernel attributes? Unspecified W3C PROV-O, PAV
  & & \\
RDA-R1.2-02M
  & Metadata includes provenance information according to a cross-community language
  & FDOR9 FDOR8
  & ---
  & ---
  & W3C PROV-O \cite{w3-prov-o}, PAV \cite{ciccaresePAVOntologyProvenance2013e}, Dublin Core Terms \cite{DCMIMetadataTerms} \\
RDA-R1.3-01M
  & Metadata complies with a community standard
  & FDOR10 FROR8
  & (Emerging, e.g.~DiSSCo Digital Specimen \cite{Hardisty 2022})
  & ---
  & Common, e.g.~DCAT 2 \cite{DCAT2 2020}, BioSchemas \cite{bioschema-salad} \\
RDA-R1.3-01D
  & Data complies with a community standard
  & (FDOR3)
  & ---
  & ---
  & Common, HTTP use registered IANA \href{https://www.iana.org/assignments/media-types/media-types.xhtml}{media types}, additional scientific file formats frequently not standardised or identified \\
RDA-R1.3-02M
  & Metadata is expressed in compliance with a machine-understandable community standard
  & FDOF4 FDOF10
  & Recommended
  & ---
  & Common practice for ontologies, specially in bioinformatics, e.g.~BioPortal \cite{NCBOBioPortal}, Darwin Core \cite{wieczorekDarwinCoreEvolving2012} \\
RDA-R1.3-02D
  & Data is expressed in compliance with a machine-understandable community standard
  & (FDOR2)
  & No, FDO is typed but data can be any bytestream
  & ---
  & Occassionally, (e.g.~\href{https://github.com/The-Sequence-Ontology/Specifications/blob/master/gff3.md}{GFF3}, \href{https://fits.gsfc.nasa.gov/fits_standard.html}{FITS}, \href{https://www.loc.gov/preservation/digital/formats/fdd/fdd000280.shtml}{ESRI}) \\
\bottomrule
\end{longtable}
\end{small}
\end{landscape}


From this evaluation we observe:

\begin{itemize}
\item
  Linked Data in general is strong on metadata indicators, but LDP approach is weak as it has little concrete metadata guidance.
\item
  FDO/DOIP are stronger on identifier indicators, while Linked Data approach for identifiers relies on best practices. 
\item
  Indicators on standard protocols (RDA-A1-04M, RDA-A1-04D, RDA-A1.1-01M, RDA-A1.1-01D) favour LDP's mature standards (HTTP, URI) -- the DOIPv2 specification \cite{DONA 2018} has currently only a couple of implementations and is expressed informally. The underlying Handle system for PIDs is arguably mature and commonly used by researchers (this article alone references about 80 DOIs), however DOIs are more commonly accessed as HTTP redirects through resolvers like \url{https://doi.org/} and \url{http://hdl.handle.net/} rather than the Handle protocol.
\item
  RDA-A1-02M and RDA-A1-02D highlights access by manual intervention, which is common for http/https URIs, but also using above PID resolvers for DOIP implementation \href{https://www.cordra.org/}{CORDRA} (e.g.~\url{https://hdl.handle.net/21.14100/90ec1c7b-6f5e-4e12-9137-0cedd16d1bce}), yet neither LDP, FDO nor DOIP specifications recommends human-readable representations to be provided
\item
  Neither DOIP nor LDP require license to be expressed (RDA-R1.1-01M, RDA-R1.1-02M, RDA-R1.1-03M), yet this is crucial for re-use and machine actionability of FAIR data and metadata to be legal
\item
  Machine-understandable types, provenance and data/metadata standards (RDA-R1.1-03M RDA-R1.3-02M, RDA-R1.3-02M, RDA-R1.3-02D) are important for machine actionability, but are currently unspecified for FDOs. \cite{fdo-ImplAttributesTypesProfiles} explores possible machine-readable FDO types, however the type systems themselves have not yet been formalised. Linked Data on the other side have too many semantic and syntactic type systems, making it difficult to write consistent clients.
\item
  Indicators for FAIR data are weak for either approach, as too much reliance is put on metadata. For instance in Linked Data, given a URL of a CSV file, what is its persistant identifier or license information? FAIR Signposting \cite{Van de Sompel 2022} can improve findability of metadata using HTTP Link relations, which enable an FDO-like overlay for any HTTP resource. In DOIP, responses for bytestreams can include the data identifier: if that is a PID (not enforced by DOIP), its metadata is accessible.
\item
  Resolving FDOs via Handle PIDs to the corresponding DOIP server is currently undefined by FDO and DOIP specifications. \texttt{0.TYPE/DOIPServiceInfo} lookup is only possible once DOIP server is known.
\end{itemize}


\subsection{EOSC Interoperability Framework}\label{eosc-interoperability-framework}

The European Open Science Cloud (EOSC) is a large EU initiative to promote Open Science by implementing a joint research infrastructure by federating existing and new services and focusing on interoperability, accessability, best practices as well as technical infrastructure \cite{10.2777/940154}. The EOSC Interoperability Framework \cite{eosc-interop-framework} details the principles for creating a common way to achieve interoperability between all digital aspects of research activities in EOSC, including data, protocols and software. The recommendations are realized through 4 layers, Technical (e.g. protocols), Semantic (e.g. metadata models), Organisational (e.g. recommendations) and Legal (e.g. agreements), with a particular aim to address the FAIR interoperability principles and building on the concept of FAIR Digital Objects. 

In Table \vref{ch3:eosc} we review the EOSC Interoperability Framework (EOSC IF) recommendations, and evaluate to what extent they are addressed by the principles of FDO and Linked Data or their common implementations.

%\begin{landscape}
  \begin{longtable}[]{@{}
    >{\raggedright\arraybackslash}p{(\columnwidth - 6\tabcolsep) * \real{0.15}}
    >{\raggedright\arraybackslash}p{(\columnwidth - 6\tabcolsep) * \real{0.30}}
    >{\raggedright\arraybackslash}p{(\columnwidth - 6\tabcolsep) * \real{0.30}}
    >{\raggedright\arraybackslash}p{(\columnwidth - 6\tabcolsep) * \real{0.30}}@{}}
  \caption[Assessing EOSC Interoperability Framework, against FDO \& Linked Data]{Assessing EOSC Interoperability Framework \cite[section 3.6]{eosc-interop-framework} against the FDO guidelines \cite{bonino2019} and Linked Data practices.}
  \label{ch3:eosc}\tabularnewline
  \toprule
  Layer &
  Recommendation &
  FDO &
  Linked Data \\
  \midrule
  \endhead
  Technical      & Open Specification 
    & FDO specifications are semi-open, process gradually more transparent 
    & Open and transparent standard processes through W3C \& IETF \\
  Technical      & Common security \& privacy framework 
    & Unspecified 
    & TLS for encryption, multiple approaches for single-sign-on (e.g.~ORCID, Life Science Login). Privacy largely unspecified. \\
  Technical      & Easy SLAs for service providers 
    & Unspecified 
    & None \\
  Technical      & Access data in different formats 
    & None formalised, custom operations or relations 
    & Content-negotiation, \texttt{rel=alternate} relations \\
  Technical      & Coarse-grained/fine-grained search tools 
    & Freetext \texttt{0.DOIP/Op.Search} on local DOIP, no federation 
    & Coarse-grained e.g.~\href{https://datasetsearch.research.google.com/}{Google Dataset Search}, fine-grained (e.g.~federated SPARQL) require detailed vocabulary/metadata insight \\
  Technical      & Clear PID policy 
    & Strong FDO requirements, tends towards Handle system. 
    & Not required, different communities set policies \\
  Semantic       & Clear definitions for concepts/metadata/schemas 
    & Required by FDO requirements, but not yet formalised 
    & Ontologies, SKOS, OWL \\
  Semantic       & Semantic artefacts w/ open licenses 
    & All artefacts are PIDs, license not yet required by kernel metadata
    & Open License is best practice for ontology publishing \\
  Semantic       & Documentation for each semantic artefact 
    & No direct rendering from FDO, no requirement for human-readable description 
    & Ontology rendering, content-negotiation \\
  Semantic       & Repositories of artefacts 
    & Required, but not formalised 
    & Bioontologies, otherwise not usually federated \\
  Semantic       & Repositories w/ clear governance 
    & Recommended 
    & Largely self-governed repositories, if well-established may have clear governance. \\
  Semantic       & Minimal metadata model for federated discovery 
    & Kernel metadata \cite{fdo-KernelAttributes} based on RDA recommendations \cite{weigelRDARecommendationPID2018}.
    & DCAT, schema.org, Dublin Core \\
  Semantic       & Crosswalks from minimal metadata model 
    & FDO Typing recommends referencing existing type definitions, but not as separate crosswalks 
    & Multiple crosswalks for common metadata models, but frequently not in semantic format \\
  Semantic       & Extensibility options for diciplinary metadata 
    & Communities encouraged to establish own types 
    & Extensible by design, domain-specific metadata may be at different granularity \\
  Semantic       & Clear protocols/building blocks for federation/harvesting of artefact catalogues 
    & Collection types not yet defined 
    & SWORD, OAI-PMH \\
  Organisational  & Interoperability-focused rules of participation recommendations 
    & Recommended 
    & Implied only by some communities, tendency to specialise \\
  Organisational  & Usage recommendations of standardised data formats 
    & None 
    & None -- but common for metadata (e.g.~JSON-LD) \\
  Organisational  & Usage recommendations of vocabularies 
    & Recommended by community 
    & Common (see \href{https://rdmkit.elixir-europe.org/metadata_management}{RDMKit}) \\
  Organisational  & Usage recommendations of metadata 
    & Recommended by community 
    & RO-Crate, bioschema-salad \\
  Organisational  & Management of permanent organization names/functions 
    & Handle owner, but unclear contact. Contact info in DOIP service provider 
    & ROR. DCAT contacts. \\
  Legal          & Standardised human and machine-readable licenses 
    & None 
    & \href{https://spdx.org/licenses/}{SPDX}, but not that frequently used \\
  Legal          & Permissive licenses for metadata (CC0, CC-BY-4.0) 
    & Undefined 
    & Both CC0, CC-BY-4.0 common, e.g.~in DCAT \\
  Legal          & Different licenses for different parts 
    & Each part as separate FDO can have separate license 
    & DCAT, RO-Crate, Named graphs for splitting metadata \\
  Legal          & Mark expired/inexistent copyright 
    & Undefined 
    & Unclear, semantics assume copyright valid \\
  Legal          & Mark orphaned data 
    & Tombstone for deleted data, but no owner of DOIP server means FDO disappears 
    & Frequently data and endpoint has no known maintainer, archiving in common repositories becoming common \\
  Legal          & List recommended licenses 
    & Undefined 
    & Best practice recommendations \\
  Legal          & Track license evolution for dataset 
    & Undefined 
    & Versioning with PAV/PROV/DCAT \\
  Legal          & Policy/guidance for patent/trade secrets violation 
    & Undefined 
    & Undefined, legal owner may be specified \\
  Legal          & GDPR compliance for personal data 
    & Undefined 
    & Undefined \\
  Legal          & Restrict access/use if legally required 
    & By transport protocol (undefined by FDO/DOIP) 
    & Diverging approaches, typically landing pages w/ auth\&auth or click-thru \\
  Legal          & Harmonised terms-of-use 
    & Undefined 
    & Undefined \\
  Legal          & Alignment between EOSC and national legislation 
    & Not applicable 
    & Not applicable \\
  \bottomrule
  \end{longtable}
  %\end{landscape}
 

Firstly, we observe that the EOSC IF recommendations are at a high level, mainly affecting governance and practices by communities. This \emph{Organizational} level is also highlighted by the FDO recommendations, for instance the FDO Typing \cite{fdo-TypingFDOs} propose a governance structure to recognize community-endorsed services. While these community aspects are not mandated by Linked Data practices, best practices have become established for aspects like ontology development \cite{10.1186/s13326-021-00240-6}. EOSC IF's technical layer is likewise at a architecturally high level, such as service-level agreements, but also highlight PID policies which is strongly required by FDO, while Linked Data communities choose PID practices separately. The recommendations for the Semantic layer, is largely already implemented by Linked Data practices, yet for FDO mostly consist of encouragements. For instance \emph{clear definitions of semantic concepts} is required by FDO guidelines, but how to technically define them has not been formalised by FDO specifications. 

The Legal layer of interoperability is perhaps the one most emphasised by EOSC, by enabling collaboration across organizational barriers to joinly build a research infrastructure, but this is an area that both FDO and Linked Data are relatively weak in directly supporting. The EOSC IF recommendations in this layer are still largely related to governance practices and metadata, for instance licensing, privacy and usage policies; yet these are essential for cross-institutional and cross-repository access of FAIR objects. 

Likewise, search and indexing is important FAIR aspect for Findability, but is poorly supported globally by FDO and Linked Data. Efforts such as Open Research Knowledge Graph (ORKG) \cite{10.1007/978-3-030-30760-8_31}, DataCite's PID Graph \cite{10.5438/jwvf-8a66} and Google Knowledge Graph \cite{singhal2012} have improved programmatic findability to some degree, however not significantly for domain-specific semantic artefacts, currently scattered across multiple semantic catalogues \cite{10.48550/arXiv.2305.06746}.  There is a strong role for organizations like EOSC to provide such broader registries, moving beyond scholarly output metadata federations. The EOSC Marketplace\footnote{\url{https://marketplace.eosc-portal.eu/}} has for instance recently been expanded to include training material, software and data sources.


\subsection{Discussion}\label{ch3:discussion}

We have evaluated the FAIR Digital Object concept using multiple frameworks, and contrasted FDO against existing experiences from Linked Data on the Web. In this section we discuss the implications of this evaluation, and propose how these two approaches can be better combined.

\subsection{Framework evaluation}

Having considered FDO and the Web architecture as interoperability frameworks (\vref*{ch3:interoperability-compare}), we observe that neither are magic bullets, but each bring different aspects of interoperability. The Web comes with a large degree of flexibility and openness, however this means interoperability can suffer as services have different APIs and data models, although with common patterns. This is also true for Linked Data on the Web, with many overlapping ontologies and frequent inconsistencies in resolution mechanisms; although somewhat alleviated in recent years by schema.org becoming common metadata model for semantic markup inline in Web pages. The Web is based on a common HTTP protocol which has remained stable architecturally throughout its 32 years of largely backwards-compatible evolution. FDO on the other side sets down multiple rigid rules for identifiers, types, methods etc. that are advanterous for interoperability and predictability for FAIR consumption. Yet there is a large degree of freedom in how the FDO rules can be implemented by a given community, for instance there is no common metadata model or identifier resolution mechanism, and DOIP is just one possible transport method for FDOs, which itself does not enforce these rules. 

When evaluating FDO implementations against the FDO guidelines (\vref*{ch3:doip-fdo-compare}) we see that several technical pieces and community practices still need to be developed and further defined, for instance the FDO type system, how to declare FDO actions, how to resolve persistent identifiers, or how to know which pattern of FDO composition is used. Achieving fully interoperable FAIR digital objects would require further convergence on implementation practices, and it is not given that his need to diverge from the established Web architecture.  It is not clear from FDO guidelines if moving from HTTP/DNS to DOIP/Handle as a way to expose distributed digital objects will benefit FAIR practitioners, when both approaches require additional restrictions, equably implementable, such as using persistent identifiers or pre-defining an object's type. 

Considering this, by comparing FDO and Web as middleware (\vref*{ch3:middleware}) we saw that programmatic access to digital objects, a core promise of FDO, is not particularly improved by the use of the protocol DOIP as compared to HTTP, e.g. lack of concurrency transparancy. Recent updates to HTTP have added many features needed for large-scale usage such as video streaming services (e.g. caching, multiplexing, cloud deployments), and having the option to transparantly apply these also to FDOs seems like a strong incentive. Many programmatic features are however missing or needing custom extensions in both aspects, such as transactions, asynchronous operations and streaming.

By assessing FDO against the FAIR principles (\vref*{ch3:fair-compare}) we found that both FDO implementations are underspecified in several aspects (licences, provenance, data references, data vocabularies, metadata persistence). While there are implementations of each of these in general Linked Data examples, there is no single set of implementation guides that fully realizes the FAIR principles. \emph{FAIRification} efforts like the FAIR Cookbook \cite{faircookbook} and FAIR Implementation Profiles \cite{FIP} are bringing existing practices together, but there remains a potential role for FDO in giving a coherent set of implementation practices that can practically achieve FAIR. Significant effort, also within EOSC, is now moving towards FAIR metrics \cite{Devaraju_2021}, which in practice need to make additional assumptions on how FAIR principles are implemented, but these are not always formalized \cite{10.5281/zenodo.7463421} nor can they be taken to be universally correct \cite{10.5281/zenodo.7848102}. Given that most of the existing FAIR guides and assessment tools are focused on Web and Linked Data, it would be reasonable for FDO to then provide a profile of such implementation choices that can achieve best of both worlds.

EOSC has been largely supportive of FDO, FAIR and related services. By contrasting the EOSC Interoperability Framework (\vref*{ch3:eosc-interoperability-framework}) with FDO, we found that there are important dimensions that are not solved at a technical level, but through organization collaboration, legal requirements and building community practices. FDO recommendations highlight community aspects, but at the same time the largest FAIR communities in many science domains are already producing and consuming Linked Data. Just as the Linked Data community has a challenge in convincing more research fields to use Semantic Web technologies, FDO currently need to build many new communities in areas that have shown interest in that approach (e.g. material science).  It may be advantegous for both these effort to be aligned and jointly promoted under the EOSC umbrella. 



\subsection{What does FDO mean for Linked Data?}\label{ch3:what-does-it-mean-for-linked-data}

The FAIR Digital Object approach raises many important points for Linked Data practictioners.
At first glance, the explicit requirements of FDOs may seem to be easy to furfill by different parts of the Semantic Web Cake \cite[][slide 10]{SemanticWebXML2000}, as we have previously proposed \cite{10.3897/rio.8.e94501}.
However, this deeper investigation, based on multiple frameworks, highlights that the openness and variability of how Linked Data is deployed can make it difficult to achieve the FDO goals without significant effort.

While RDF and Linked Data have been suggested as prime candidates for making FAIR data, we argue that when different developers have too many degrees of freedom (such as serialization formats, vocabularies, identifiers, navigation), interoperability is hampered -- this makes it hard for machines to reliably consume multiple FAIR resources across repositories and data providers. 
Indeed, this may be one reason why the initial FDO effort steered away from Linked Data approaches, but now seems in a danger of opening the many same degrees of freedom within FDO.

We therefore identify the need for a new explicit FDO profile of Linked Data that sets pragmatic constraints and stronger recommendations for consistent and developer-friendly deployment of digital objects. 
Such a combination of efforts will utillise both the benefits of mature Semantic Web technologies (e.g.~federated knowledge graph queries and rich validation) and data management practices that follow FDO guidance in order to grow a rigid (yet flexible) ecosystem of machine-actionable scholarly objects. 
It is beyond the scope of this work to detail such a profile, but its main priorities could be:

\begin{itemize}
  \item Use HTTP(S) as protocol
  \item Use URIs as identifiers, with persistent identifier promises
  \item Provide consistent identifier resolution that does not require heuristics
  \item Common core metadata model
  \item References are always URIs, and should be persistent identifiers
  \item Types, attributes and actions are self-defined by their identifier
\end{itemize}

The FAIR and Linked Data communities likewise need to recognize the need for simpler, more pragmatic approaches that make it easier for FAIR practitioners to adapt the technologies with "just enough" semantics. 
We have previously proposed the combination of RO-Crate \cite{Soiland-Reyes 2022} and Signposting \cite{Van de Sompel 2022} as a mean to implement FDO \cite{10.3897/rio.8.e93937} over HTTP using a common Linked Data metadata model. 

However it may be sufficient to use HTTP-based FAIR Signposting alone to achieve the above list, if one considers only a small metadata model, and rather reference from the signposting which metadata resources are additionally available. 
This will allow any Linked Data resource to gradually participate in the FDO ecosystem, with minimal effort and non-intrusive implementation changes. FDO implementations like Cordra typically already use HTTP APIs that align with DOIP \cite{DOIPAPIHTTPa}, these can be augmented with Signposting headers without necessarily moving to a Linked Data metadata model. 


\subsection{Conclusion}\label{conclusion}

In this work we have considered FAIR Digital Objects (FDO) as a potential distributed object system and compared FDO with established Web approaches focusing on Linked Data. We have described the background of the Semantic Web and FAIR Digital Objects, and evaluated both using multiple conceptual frameworks.

We find that both FDO and Linked Data approaches can significantly benefit from each-other and should be aligned further. Namely Linked Data proponents need to make their technologies more approachable, agreeing on predictable and consistent implementations of FAIR principles. 

The FDO recommendations show that FAIR thinking in this regard need to move beyond data publishing and into machine actionability across digital objects, and with broader community consensus. 
As flexibility for extensions is a necessary ingredient alongside rigidity for core concepts, the FDO community likewise need to settle on directly implementable specifications rather than just guidelines, and avoid making similar mistakes as the early Semantic Web adopters. 

By implementing the goals of FAIR Digital Objects with the mature technology stack developed for Linked Data, EOSC research infrastructures and researchers in general can create and use FAIR machine-actionable research outputs for decades to come.



\chapter{RO-Crate}
\include{chapter04}
%\section{Packaging research artefacts with
RO-Crate}\label{packaging-research-artefacts-with-ro-crate}

An increasing number of researchers support reproducibility by including
pointers to and descriptions of datasets, software and methods in their
publications. However, scientific articles may be ambiguous, incomplete
and difficult to process by automated systems. In this paper we
introduce RO-Crate, an open, community-driven, and lightweight approach
to packaging research artefacts along with their metadata in a machine
readable manner. RO-Crate is based on Schema.org annotations in JSON-LD,
aiming to establish best practices to formally describe metadata in an
accessible and practical way for their use in a wide variety of
situations.

An RO-Crate is a structured archive of all the items that contributed to
a research outcome, including their identifiers, provenance, relations
and annotations. As a general purpose packaging approach for data and
their metadata, RO-Crate is used across multiple areas, including
bioinformatics, digital humanities and regulatory sciences. By applying
``just enough'' Linked Data standards, RO-Crate simplifies the process
of making research outputs FAIR while also enhancing research
reproducibility.



\hypertarget{introduction}{%
\subsection{Introduction}\label{introduction}}

The move towards Open Science has increased the need and demand for the
publication of artefacts of the research process
{[}\href{http://ptsefton.com/2021/04/07/rdmpic/}{Sefton 2021}{]}. This is
particularly apparent in domains that rely on computational experiments;
for example, the publication of software, datasets and records of the
dependencies that such experiments rely on
{[}\href{https://stodden.net/papers/ERCM2016-STODDEN.pdf}{Stodden 2016}{]}.

It is often argued that the publication of these assets, and
specifically software
{[}\href{https://doi.org/10.3233/DS-190026}{80}{]}, workflows
{[}\href{https://doi.org/10.1162/dint_a_00033}{55}{]} and data, should
follow the FAIR principles
{[}\href{https://doi.org/10.1038/sdata.2016.18}{Wilkinson 2016}{]}; namely, that
they are Findable, Accessible, Interoperable and Reusable. These
principles are agnostic to the \emph{implementation} strategy needed to
comply with them. Hence, there has been an increasing amount of work in
the development of platforms and specifications that aim to fulfil these
goals {[}\href{https://identifiers.org/isbn/9781315351148}{91}{]}.

Important examples include data publication with rich metadata
(e.g.~Zenodo {[}\href{https://doi.org/10.3897/biss.3.37080}{40}{]}),
domain-specific data deposition (e.g.~PDB
{[}\href{https://doi.org/10.1093/nar/gkl971}{Berman 2007}{]}) and following
practices for reproducible research software
{[}\href{https://doi.org/10.1371/journal.pcbi.1003285}{101}{]} (e.g.~use
of containers). While these platforms are useful, experience has shown
that it is important to put greater emphasis on the interconnection of
the multiple artefacts that make up the research process
{[}\href{https://doi.org/10.1016/j.ijhcs.2020.102562}{71}{]}.

The notion of \textbf{Research Objects}
{[}\href{https://www.research.manchester.ac.uk/portal/en/publications/why-linked-data-is-not-enough-for-scientists(479e591e-b295-4478-b0c7-a145c19dcd45).html}{12}{]}
(RO) was introduced to address this connectivity, providing semantically
rich \emph{aggregations} of (potentially distributed) resources with a
layer of structure over a research study; this is then to be delivered
in a \emph{machine-readable format}.

A Research Object combines the ability to bundle multiple types of
artefacts together, such as spreadsheets, code, examples, and figures.
The RO is augmented with annotations and relationships that describe the
artefacts' \emph{context} (e.g.~a CSV being used by a script, a figure
being a result of a workflow).

This notion of ROs provides a compelling vision as an approach for
implementing FAIR data. However, existing Research Object
implementations require a large technology stack
{[}\href{https://doi.org/10.1016/j.websem.2015.01.003}{Belhajjame 2015}{]}, are
typically tailored to a particular platform and are also not easily
usable by end-users.

To address this gap, a new community came together
{[}\href{https://doi.org/10.5281/zenodo.3250687}{23}{]} to develop
\textbf{RO-Crate} --- an \emph{approach to package and aggregate
research artefacts with their metadata and relationships}. The aim of
this paper is to introduce RO-Crate and assess it as a strategy for
making multiple types of research artefacts FAIR. Specifically, the
contributions of this paper are as follows:

\begin{enumerate}
\def\labelenumi{\arabic{enumi}.}
\tightlist
\item
  An introduction to RO-Crate, its purpose and context;
\item
  A guide to the RO-Crate community and tooling;
\item
  Examples of RO-Crate usage, demonstrating its value as connective
  tissue for different artefacts from different communities.
\end{enumerate}

The rest of this paper is organised as follows. We first describe
RO-Crate through its development methodology that formed the RO-Crate
concept, showing its foundations in Linked Data and emerging principles.
We then define RO-Crate technically, before we introduce the community
and tooling. We move to analyse RO-Crate with respect to usage in a
diverse set of domains. Finally, we present related work and conclude
with some remarks including RO-Crate highlights and future work. The
appendix adds a formal definition of RO-Crate using First-Order logic.

\hypertarget{rocrate}{%
\subsection{RO-Crate}\label{rocrate}}

RO-Crate aims to provide an approach to packaging research artefacts
with their metadata that can be easily adopted. To illustrate this, let
us imagine a research paper reporting on the sequence analysis of
proteins obtained from an experiment on mice. The sequence output files,
sequence analysis code, resulting data and reports summarising
statistical measures are all important and inter-related research
artefacts, and consequently would ideally all be co-located in a
directory and accompanied with their corresponding metadata. In reality,
some of the artefacts (e.g.~data or software) will be recorded as
external reference to repositories that are not necessarily following
the FAIR principles. This conceptual directory, along with the
relationships between its constituent digital artefacts, is what the
RO-Crate model aims to represent, linking together all the elements of
an experiment that are required for the experiment's reproducibility and
reusability.

The question then arises as to how the directory with all this material
should be packaged in a manner that is accessible and usable by others.
This means programmatically and automatically accessible by machines and
human readable. A de facto approach to sharing collections of resources
is through compressed archives (e.g.~a zip file). This solves the
problem of ``packaging'', but it does not guarantee downstream access to
all artefacts in a programmatic fashion, nor describe the role of each
file in that particular research. Both features, the ability to
automatically access and reason about an object, are crucial and lead to
the need for explicit metadata about the contents of the folder,
describing each and linking them together.

Examples of metadata descriptions across a
\href{https://rdamsc.bath.ac.uk/scheme-index}{wide range of domains}
abound within the literature, both in research data management
{[}\href{https://doi.org/10.1007/s10209-016-0475-y}{Amorim 2016}{]}
{[}\href{https://dcpapers.dublincore.org/pubs/article/view/3714}{46}{]}
{[}\href{https://doi.org/10.2777/620649}{75}{]} and within
\href{https://www.loc.gov/librarians/standards}{library and information
systems} {[}\href{https://identifiers.org/isbn/9781563081910}{24}{]}
{[}\href{https://doi.org/10.1515/9783598441844}{127}{]}. However, many
of these approaches require knowledge of metadata schemas, particular
annotation systems, or the use of complex software stacks. Indeed,
particularly within research, these requirements have led to a lack of
adoption and growing frustration with current tooling and specifications
{[}\href{https://cameronneylon.net/blog/as-a-researcher-im-a-bit-bloody-fed-up-with-data-management/}{94}{]}
{[}\href{https://doi.org/10.1007/s00267-014-0258-2}{119}{]}
{[}\href{https://doi.org/10.1038/s41597-020-0524-5}{102}{]}.

RO-Crate seeks to address this complexity by:

\begin{enumerate}
\def\labelenumi{\arabic{enumi}.}
\tightlist
\item
  being conceptually simple and easy to understand for developers;
\item
  providing strong, easy tooling for integration into community
  projects;
\item
  providing a strong and opinionated guide regarding current best
  practices;
\item
  adopting de-facto standards that are widely used on the Web.
\end{enumerate}

In the following sections we demonstrate how the RO-Crate specification
and ecosystem achieve these goals.

\hypertarget{methodology}{%
\subsubsection{Development Methodology}\label{methodology}}

It is a good question as to what base level we assume for `conceptually
simple'. We take simplicity to apply at two levels: for the
\emph{developers} who produce the platforms and for the \emph{data
practitioners} and users of those platforms.

For our development methodology we followed the mantra of working
closely with a small group to really get a deep understanding of
requirements and ensure rapid feedback loops. We created a pool of early
adopter projects from a range of disciplines and groups, primarily
addressing developers of platforms. Thus the base level for simplicity
was \textbf{developer friendliness}.

We assumed a developer familiar with making Web applications with JSON
data (who would then learn how to make \emph{RO-Crate JSON-LD}), which
informed core design choices for our JSON-level documentation approach
and RO-Crate serialization (section on
\protect\hyperlink{implementation}{implementation}). Our group of early
adopters, growing as the community evolved, drove the RO-Crate
requirements and provided feedback through our multiple communication
channels including bi-monthly meetings, which we describe in section on
\protect\hyperlink{community}{community} along with the established
norms.

Addressing the simplicity of understanding and engaging with RO-Crate by
data practitioners is through the platforms, for example with
interactive tools (section \protect\hyperlink{tooling}{RO-Crate
tooling}) like
\href{https://arkisto-platform.github.io/describo/}{Describo}
\href{https://arkisto-platform.github.io/describo/}{{[}78{]}} and
Jupyter notebooks
{[}\href{https://doi.org/10.3233/978-1-61499-649-1-87}{70}{]}, and by
close discussions with domain scientists on how to appropriately capture
what they determine to be relevant metadata. This ultimately requires a
new type of awareness and learning material separate from developer
specifications, focusing on the simplicity of extensibility to serve the
user needs, along with user-driven development of new RO-Crate Profiles
specific for their needs (section on \protect\hyperlink{inuse}{in use}).

\hypertarget{conceptual}{%
\subsubsection{Conceptual Definition}\label{conceptual}}

A key premise of RO-Crate is the existence of a wide variety of
resources on the Web that can help describe research. As such, RO-Crate
relies on the Linked Data principles
{[}\href{https://doi.org/10.2200/S00334ED1V01Y201102WBE001}{63}{]}.
\protect\hyperlink{fig:conceptual}{Figure 1} shows the main conceptual
elements involved in an RO-Crate: The RO-Crate Metadata File (top)
describes the Research Object using structured metadata including
external references, coupled with the contained artefacts (bottom)
bundled and described by the RO-Crate.

The conceptual notion of a \emph{Research Object}
{[}\href{https://www.research.manchester.ac.uk/portal/en/publications/why-linked-data-is-not-enough-for-scientists(479e591e-b295-4478-b0c7-a145c19dcd45).html}{12}{]}
is thus realised with the RO-Crate model and serialised using Linked
Data constructs within the RO-Crate metadata file.

\{\{\textless{} figure src=``ro-crate-overview.svg''
link=``ro-crate-overview.svg'' id=``fig:conceptual'' width=``100\%''
title=``Conceptual RO-Crate Overview'' caption=``A \emph{Persistent
Identifier} (PID)
{[}\href{https://doi.org/10.1371/journal.pbio.2001414}{86}{]} points to
a \emph{Research Object} (RO), which may be archived using different
packaging approaches like BagIt
{[}\href{https://doi.org/10.17487/rfc8493}{74}{]}, OCFL
\href{https://ocfl.io/1.0/spec/}{{[}96{]}}, git or ZIP. The RO is
described within a \emph{RO-Crate Metadata File}, providing identifiers
for \emph{authors} using ORCID, \emph{organisations} using Research
Organization Registry (ROR)
{[}\href{https://doi.org/10.6087/kcse.192}{79}{]} and licences such as
Creative Commons using SPDX identifiers. The \emph{RO-Crate content} is
further described with additional metadata following a Linked Data
approach. Data can be embedded files and directories, as well as links
to external Web resources, PIDs and nested RO-Crates.'' \textgreater\}\}

\hypertarget{linkeddata}{%
\paragraph{Linked Data as a foundation}\label{linkeddata}}

The \textbf{Linked Data} principles
{[}\href{https://doi.org/10.4018/978-1-60960-593-3.ch008}{Bizer 2011}{]} (use of
IRIs\footnote{To execute the wrapped tool, a containerized workflow
  engine would need \emph{nested containers} which are not generally
  recommended for security reasons. It is possible to work around this
  limitation using \href{https://sylabs.io/singularity/}{Singularity} or
  \href{https://docs.bioexcel.eu/cwl-best-practice-guide/devpractice/containers/conda.html}{Conda}.}
to identify resources (i.e.~artefacts), resolvable via HTTP, enriched
with metadata and linked to each other) are core to RO-Crate; therefore
IRIs are used to identify an RO-Crate, its constituent parts and
metadata descriptions, and the properties and classes used in the
metadata.

RO-Crates are \emph{self-described} and follow the Linked Data
principles to describe all of their resources in both human and machine
readable manner. Hence, resources are identified using global
identifiers (absolute IRIs) where possible; and relationships between
two resources are defined with links.

The foundation of Linked Data and shared vocabularies also means that
multiple RO-Crates and other Linked Data resources can be indexed,
combined, queried, validated or transformed using existing Semantic Web
technologies such as
\href{https://www.w3.org/TR/sparql11-overview}{SPARQL},
\href{https://www.w3.org/TR/shacl/}{SHACL} and well established
\emph{knowledge graph} triple stores like
\href{https://jena.apache.org/}{Apache Jena} and
\href{https://www.ontotext.com/products/graphdb/}{OntoText GraphDB}.

The possibilities of consuming\footnote{Some consideration is needed in
  processing of RO-Crates as knowledge graphs, e.g.~establishing
  absolute IRIs for files inside a ZIP archive, detailed
  \href{https://www.researchobject.org/ro-crate/1.1/appendix/relative-uris.html}{in
  the RO-Crate specification}} RO-Crate metadata with such powerful
tools gives another strong reason for using Linked Data as a foundation.
This use of mature Web\footnote{Note that an RO-Crate is not required to
  be published on the Web, see section on
  \protect\hyperlink{selfdescribed}{self-described}.} technologies also
means its developers and consumers are not restricted to the Research
Object aspects that have already been specified by the RO-Crate
community, but can extend and integrate RO-Crate in multiple
standardised ways.

\hypertarget{selfdescribed}{%
\paragraph{RO-Crate is a self-described container}\label{selfdescribed}}

An
\href{https://www.researchobject.org/ro-crate/1.1/structure.html\#ro-crate-metadata-file-ro-crate-metadatajson}{RO-Crate
is defined} as a self-described \textbf{Root Data Entity} that describes
and contains \emph{data entities}, which are further described by
referencing \emph{contextual entities}. A \textbf{data entity} is either
a \emph{file} (i.e.~a byte sequence stored on disk somewhere) or a
\emph{directory} (i.e.~set of named files and other directories). A file
does not need to be stored inside the RO-Crate root, it can be
referenced via a PID/IRI. A \textbf{contextual entity} exists outside
the information system (e.g.~a Person, a workflow language) and is
stored solely by its metadata. The representation of a \emph{data
entity} as a byte sequence makes it possible to store a variety of
research artefacts including not only data but also, for instance,
software and text.

The Root Data Entity is a directory, the \emph{RO-Crate Root},
identified by the presence of the \textbf{RO-Crate Metadata File}
\texttt{ro-crate-metadata.json} (top of
\protect\hyperlink{fig:conceptual}{Figure 1}. This file fdescribes the
RO-Crate using Linked Data, its content and related metadata using
Linked Data in JSON-LD format
\href{https://www.w3.org/TR/2014/REC-json-ld-20140116/}{{[}112{]}}. This
is a W3C standard RDF serialisation that has become popular; it is easy
to read by humans while also offering some advantages for data exchange
on the Internet. JSON-LD, a subset of the widely supported and
well-known JSON format, has tooling available for many
\href{https://json-ld.org/\#developers}{programming languages}.

The minimal
\href{https://www.researchobject.org/ro-crate/1.1/root-data-entity.html\#direct-properties-of-the-root-data-entity}{requirements
for the root data entity metadata} are \texttt{name},
\texttt{description} and \texttt{datePublished}, as well as a contextual
entity identifying its \texttt{license} --- additional metadata are
commonly added to entities depending on the purpose of the particular
RO-Crate.

RO-Crates can be stored, transferred or published in multiple ways,
e.g.~BagIt {[}\href{https://doi.org/10.17487/rfc8493}{74}{]}, Oxford
Common File Layout \href{https://ocfl.io/1.0/spec/}{{[}96{]}} (OCFL),
downloadable ZIP archives in Zenodo or through dedicated online
repositories, as well as published directly on the Web, e.g.~using
\href{https://pages.github.com/}{GitHub Pages}. Combined with Linked
Data identifiers, this caters for a diverse set of storage and access
requirements across different scientific domains, from metagenomics
workflows producing hundreds of gigabytes of genome data to cultural
heritage records with access restrictions for personally identifiable
data. Specific \emph{RO-Crate profiles} (section on
\protect\hyperlink{profiles}{extensibility}) may constrain serialization
and publication expectations, and require additional contextual types
and properties.

\hypertarget{contextualentities}{%
\paragraph{Data Entities are described using Contextual
Entities}\label{contextualentities}}

RO-Crate distinguishes between
\href{https://www.researchobject.org/ro-crate/1.1/contextual-entities.html\#contextual-vs-data-entities}{data
and contextual entities} in a similar way to HTTP terminology's early
attempt to separate \emph{information} (data) and \emph{non-information}
(contextual) resources
\href{https://www.w3.org/2001/tag/doc/httpRange-14/2007-08-31/HttpRange-14.html}{{[}120{]}}.
Data entities are usually files and directories located by relative IRI
references within the RO-Crate Root, but they can also be Web resources
or restricted data identified with absolute IRIs, including
\emph{Persistent Identifiers} (PIDs)
{[}\href{https://doi.org/10.1371/journal.pbio.2001414}{86}{]}.

As both types of entities are identified by IRIs, their distinction is
allowed to be blurry; data entities can be located anywhere and be
complex, while contextual entities can have a Web presence beyond their
description inside the RO-Crate. For instance
\texttt{https://orcid.org/0000-0002-1825-0097} is primarily an
identifier for a person, but secondarily it is also a Web page and a way
to refer to their academic work.

A particular IRI may appear as a contextual entity in one RO-Crate and
as a data entity in another; the distinction lies in the fact that data
entities can be considered to be \emph{contained} or captured by that
RO-Crate (\emph{RO Content} in \protect\hyperlink{fig:conceptual}{Figure
1}, while contextual entities mainly \emph{explain} an RO-Crate or its
content (although this distinction is not a formal requirement).

In RO-Crate, a referenced contextual entity (e.g.~a person identified by
ORCID) should always be described within the RO-Crate Metadata File with
at least a \emph{type} and \emph{name}, even where their PID might
resolve to further Linked Data. This is so that clients are not required
to follow every link for presentation purposes, for instance HTML
rendering. Similarly any imported
\href{https://www.researchobject.org/ro-crate/1.1/appendix/jsonld.html\#extending-ro-crate}{extension
terms} would themselves also have a human-readable description in the
case where their PID does not directly resolve to human-readable
documentation.

\protect\hyperlink{fig:uml}{Figure 2} shows a simplified UML class
diagram of RO-Crate, highlighting the different types of data entities
and contextual entities that can be aggregated and related. While an
RO-Crate would usually contain one or more data entities
(\texttt{hasPart}), it may also be a pure aggregation of contextual
entities (\texttt{mentions}).

\{\{\textless{} figure src=``ro-crate-uml.svg''
link=``ro-crate-uml.svg'' id=``fig:uml'' width=``100\%''
title=``Simplified UML class diagram of RO-Crate'' caption=``The
\emph{RO-Crate Metadata File} conforms to a version of the
specification; and contains a JSON-LD graph
\href{https://www.w3.org/TR/2014/REC-json-ld-20140116/}{{[}112{]}} that
describes the entities that make up the RO-Crate. The \emph{RO-Crate
Root Data Entity} represent the Research Object as a dataset. The
RO-Crate aggregates \emph{data entities} (\texttt{hasPart}) which are
further described using \emph{contextual entities} (which may include
aggregated and non-aggregated data entities). Multiple types and
relations from Schema.org allow annotations to be more specific,
including figures, nested datasets, computational workflows, people,
organisations, instruments and places. Contextual entities not otherwise
cross-referenced from other entities' properties (\emph{describes}) can
be grouped under the root entity (\texttt{mentions}).'' \textgreater\}\}

\hypertarget{recommendedpractices}{%
\paragraph{Guide through Recommended
Practices}\label{recommendedpractices}}

RO-Crate as a specification aims to build a set of recommended practices
on how to practically apply existing standards in a common way to
describe research outputs and their provenance, without having to learn
each of the underlying technologies in detail.

As such, the \href{https://w3id.org/ro/crate/1.1}{RO-Crate 1.1}
specification {[}\href{https://doi.org/10.5281/zenodo.4541002}{106}{]}
can be seen as an opinionated and example-driven guide to writing
\href{https://schema.org/}{Schema.org}
{[}\href{https://doi.org/10.1145/2857274.2857276}{62}{]} metadata as
JSON-LD
\href{https://www.w3.org/TR/2014/REC-json-ld-20140116/}{{[}112{]}} (see
section on \protect\hyperlink{implementation}{implementation}, which
leaves it open for implementers to include additional metadata using
other Schema.org types and properties, or even additional Linked Data
vocabularies/ontologies or their own ad-hoc terms.

However the primary purpose of the RO-Crate specification is to assist
developers in leveraging Linked Data principles for the focused purpose
of describing Research Objects in a structured language, while reducing
the steep learning curve otherwise associated with Semantic Web
adaptation, like development of ontologies, identifiers, namespaces, and
RDF serialization choices.

\hypertarget{simplicity}{%
\paragraph{Ensuring Simplicity}\label{simplicity}}

One aim of RO-Crate is to be conceptually simple. This simplicity has
been repeatedly checked and confirmed through an informal community
review process. For instance, in the discussion on supporting
\href{https://github.com/ResearchObject/ro-crate/issues/71}{ad-hoc
vocabularies} in RO-Crate, the community explored potential Linked Data
solutions. The conventional wisdom in
\href{https://www.w3.org/TR/swbp-vocab-pub/}{RDF best practices} is to
establish a vocabulary with a new IRI namespace, formalised using
\href{http://www.w3.org/TR/2014/REC-rdf-schema-20140225/}{RDF Schema} or
\href{http://www.w3.org/TR/2012/REC-owl2-overview-20121211/}{OWL}
ontologies. However, this may seem an excessive learning curve for
non-experts in semantic knowledge representation, and the RO-Crate
community instead agreed on a dual lightweight approach: (i)
\href{https://www.researchobject.org/ro-crate/1.1/appendix/jsonld.html\#adding-new-or-ad-hoc-vocabulary-terms}{Document}
how projects with their own Web-presence can make a pure HTML-based
vocabulary, and (ii) provide a community-wide PID namespace under
\texttt{https://w3id.org/ro/terms} that redirect to simple CSV files
\href{https://github.com/ResearchObject/ro-terms}{maintained in GitHub}.

To further verify this idea of simplicity, we have formalised the
RO-Crate definition (see Appendix on
\protect\hyperlink{formaldefinition}{Formal Definition}). An important
result of this exercise is that the underlying data structure of
RO-Crate, although conceptually a graph, is represented as a
depth-limited tree. This formalisation also emphasises the
\emph{boundedness} of the structure; namely, the fact that elements are
specifically identified as being either semantically \emph{contained} by
the RO-Crate as \emph{Data Entities} (\texttt{hasPart}) or mainly
referenced (\texttt{mentions}) and typed as \emph{external} to the
Research Object as \emph{Contextual Entities}. It is worth pointing out
that this semantic containment can extend beyond the physical
containment of files residing within the RO-Crate Root directory on a
given storage system, as the RO-Crate data entities may include any data
resource globally identifiable using IRIs.

\hypertarget{profiles}{%
\paragraph{Extensibility and RO-Crate profiles}\label{profiles}}

The RO-Crate specification provides a core set of conventions to
describe research outputs using types and properties applicable across
scientific domains. However we have found that domain-specific use of
RO-Crate will, implicitly or explicitly, form a specialised
\textbf{profile} of RO-Crate; i.e., \emph{a set of conventions, types
and properties that are minimally required and one can expect to be
present in that subset of RO-Crates}. For instance, RO-Crates used for
exchange of workflows will have to contain a data entity of type
\texttt{ComputationalWorkflow}, or cultural heritage records should have
a \texttt{contentLocation}.

Making such profiles explicit allow further reliable programmatic
consumption and generation of RO-Crates beyond the core types defined in
the RO-Crate specification. Following the RO-Crate mantra of
\emph{guidance over strictness}, profiles are mainly \emph{duck-typing}
rather than strict syntactic or semantic types, but may also have
corresponding machine-readable schemas at multiple levels (file formats,
JSON, RDF shapes, RDFS/OWL semantics).

The next version of the RO-Crate specification 1.2 will define a
\href{https://www.researchobject.org/ro-crate/1.2-DRAFT/profiles}{formalization}
for publishing and declaring conformance to RO-Crate profiles. Such a
profile is primarily a human-readable document of before-mentioned
expectations and conventions, but may also define a machine-readable
profile as a \textbf{Profile Crate}: Another RO-Crate that describe the
profile and in addition can list schemas for validation, compatible
software, applicable repositories, serialization/packaging formats,
extension vocabularies, custom JSON-LD contexts and examples (see for
example the
\href{https://w3id.org/workflowhub/workflow-ro-crate/}{Workflow RO-Crate
profile}).

In addition, there are sometimes existing domain-specific metadata
formats, but they are either not RDF-based (and thus time-consuming to
construct terms for in JSON-LD) or are at a different granularity level
that might become overwhelming if represented directly in the RO-Crate
Metadata file (e.g.~W3C PROV bundle detailing every step execution of a
workflow run
{[}\href{https://doi.org/10.1093/gigascience/giz095}{68}{]}). RO-Crate
allows such \emph{alternative metadata files} to co-exist, and be
described as data entities with references to the standards and
vocabularies they conform to. This simplifies further programmatic
consumption even where no filename or file extension conventions have
emerged for those metadata formats.

Section on \protect\hyperlink{inuse}{in use} examines the observed
specializations of RO-Crate use in several domains and their emerging
profiles.

\hypertarget{implementation}{%
\subsubsection{Technical implementation of the RO-Crate
model}\label{implementation}}

The RO-Crate conceptual model has been realised using JSON-LD and
Schema.org in a prescriptive form as discussed in section on
\protect\hyperlink{conceptual}{conceptual definition}. These technical
choices were made to cater for simplicity from a developer perspective
(as introduced in section on
\protect\hyperlink{methodology}{methodology}).

\href{https://json-ld.org/}{JSON-LD}
\href{https://www.w3.org/TR/2014/REC-json-ld-20140116/}{{[}112{]}}
provides a way to express Linked Data as a JSON structure, where a
\emph{context} provides mapping to RDF properties and classes. While
JSON-LD cannot map arbitrary JSON structures to RDF, we found that it
does lower the barrier compared to other RDF syntaxes, as the JSON
syntax nowadays is a common and popular format for data exchange on the
Web.

However, JSON-LD alone has too many degrees of freedom and hidden
complexities for software developers to reliably produce and consume
without specialised expertise or large RDF software frameworks. A large
part of the RO-Crate specification is therefore dedicated to describing
the acceptable subset of JSON structures.

\hypertarget{jsonld}{%
\paragraph{RO-Crate JSON-LD}\label{jsonld}}

RO-Crate
\href{https://www.researchobject.org/ro-crate/1.1/appendix/jsonld.html}{mandates}
the use of flattened, compacted JSON-LD in the RO-Crate Metadata file
\texttt{ro-crate-metadata.json}\footnote{The avid reader may spot that
  the RO-Crate Metadata file use the extension \texttt{.json} instead of
  \texttt{.jsonld}, this is to emphasise the developer expectations as a
  JSON format, while the file's JSON-LD nature is secondary. See
  \href{https://github.com/ResearchObject/ro-crate/issues/82}{ResearchObject/ro-crate\#82}.}
where a single \texttt{@graph} array contains all the data and
contextual entities in a flat list. An example can be seen in the
JSON-LD snippet in Listing 1 below, describing a simple RO-Crate
containing data entities described using contextual entities:

\begin{Shaded}
\begin{Highlighting}[]
\FunctionTok{\{} \DataTypeTok{"@context"}\FunctionTok{:} \StringTok{"https://w3id.org/ro/crate/1.1/context"}\FunctionTok{,}
  \DataTypeTok{"@graph"}\FunctionTok{:} \OtherTok{[}
      \FunctionTok{\{} \DataTypeTok{"@id"}\FunctionTok{:} \StringTok{"ro{-}crate{-}metadata.json"}\FunctionTok{,}      
        \DataTypeTok{"@type"}\FunctionTok{:} \StringTok{"CreativeWork"}\FunctionTok{,}
        \DataTypeTok{"conformsTo"}\FunctionTok{:} \FunctionTok{\{}\DataTypeTok{"@id"}\FunctionTok{:} \StringTok{"https://w3id.org/ro/crate/1.1"}\FunctionTok{\},}
        \DataTypeTok{"about"}\FunctionTok{:} \FunctionTok{\{}\DataTypeTok{"@id"}\FunctionTok{:} \StringTok{"./"}\FunctionTok{\}}
      \FunctionTok{\}}\OtherTok{,}
      \FunctionTok{\{} \DataTypeTok{"@id"}\FunctionTok{:} \StringTok{"./"}\FunctionTok{,}
        \DataTypeTok{"@type"}\FunctionTok{:} \StringTok{"Dataset"}\FunctionTok{,}
        \DataTypeTok{"name"}\FunctionTok{:} \StringTok{"A simplified RO{-}Crate"}\FunctionTok{,}
        \DataTypeTok{"author"}\FunctionTok{:} \FunctionTok{\{}\DataTypeTok{"@id"}\FunctionTok{:} \StringTok{"\#alice"}\FunctionTok{\},}
        \DataTypeTok{"license"}\FunctionTok{:} \FunctionTok{\{}\DataTypeTok{"@id"}\FunctionTok{:} \StringTok{"https://spdx.org/licenses/CC{-}BY{-}4.0"}\FunctionTok{\},}
        \DataTypeTok{"datePublished"}\FunctionTok{:} \StringTok{"2021{-}11{-}02T16:04:43Z"}\FunctionTok{,}
        \DataTypeTok{"hasPart"}\FunctionTok{:} \OtherTok{[}
          \FunctionTok{\{}\DataTypeTok{"@id"}\FunctionTok{:} \StringTok{"survey{-}responses{-}2019.csv"}\FunctionTok{\}}\OtherTok{,}
          \FunctionTok{\{}\DataTypeTok{"@id"}\FunctionTok{:} \StringTok{"https://example.com/pics/5707039334816454031\_o.jpg"}\FunctionTok{\}}
        \OtherTok{]}
      \FunctionTok{\}}\OtherTok{,}
      \FunctionTok{\{} \DataTypeTok{"@id"}\FunctionTok{:} \StringTok{"survey{-}responses{-}2019.csv"}\FunctionTok{,}
        \DataTypeTok{"@type"}\FunctionTok{:} \StringTok{"File"}\FunctionTok{,}
        \DataTypeTok{"about"}\FunctionTok{:} \FunctionTok{\{}\DataTypeTok{"@id"}\FunctionTok{:} \StringTok{"https://example.com/pics/5707039334816454031\_o.jpg"}\FunctionTok{\},}
        \DataTypeTok{"author"}\FunctionTok{:} \FunctionTok{\{}\DataTypeTok{"@id"}\FunctionTok{:} \StringTok{"\#alice"}\FunctionTok{\}}
      \FunctionTok{\}}\OtherTok{,}
      \FunctionTok{\{} \DataTypeTok{"@id"}\FunctionTok{:} \StringTok{"https://example.com/pics/5707039334816454031\_o.jpg"}\FunctionTok{,}
        \DataTypeTok{"@type"}\FunctionTok{:} \OtherTok{[}\StringTok{"File"}\OtherTok{,} \StringTok{"ImageObject"}\OtherTok{]}\FunctionTok{,}
        \DataTypeTok{"contentLocation"}\FunctionTok{:} \FunctionTok{\{}\DataTypeTok{"@id"}\FunctionTok{:} \StringTok{"http://sws.geonames.org/8152662/"}\FunctionTok{\},}
        \DataTypeTok{"author"}\FunctionTok{:} \FunctionTok{\{}\DataTypeTok{"@id"}\FunctionTok{:} \StringTok{"https://orcid.org/0000{-}0002{-}1825{-}0097"}\FunctionTok{\}}
      \FunctionTok{\}}\OtherTok{,}
      \FunctionTok{\{} \DataTypeTok{"@id"}\FunctionTok{:} \StringTok{"\#alice"}\FunctionTok{,}
        \DataTypeTok{"@type"}\FunctionTok{:} \StringTok{"Person"}\FunctionTok{,}
        \DataTypeTok{"name"}\FunctionTok{:} \StringTok{"Alice"}
      \FunctionTok{\}}\OtherTok{,}
      \FunctionTok{\{} \DataTypeTok{"@id"}\FunctionTok{:} \StringTok{"https://orcid.org/0000{-}0002{-}1825{-}0097"}\FunctionTok{,}
        \DataTypeTok{"@type"}\FunctionTok{:} \StringTok{"Person"}\FunctionTok{,}
        \DataTypeTok{"name"}\FunctionTok{:} \StringTok{"Josiah Carberry"}
      \FunctionTok{\}}\OtherTok{,}
      \FunctionTok{\{} \DataTypeTok{"@id"}\FunctionTok{:} \StringTok{"http://sws.geonames.org/8152662/"}\FunctionTok{,}
        \DataTypeTok{"@type"}\FunctionTok{:} \StringTok{"Place"}\FunctionTok{,}
        \DataTypeTok{"name"}\FunctionTok{:} \StringTok{"Catalina Park"}
      \FunctionTok{\}}\OtherTok{,}
      \FunctionTok{\{} \DataTypeTok{"@id"}\FunctionTok{:} \StringTok{"https://spdx.org/licenses/CC{-}BY{-}4.0"}\FunctionTok{,}
        \DataTypeTok{"@type"}\FunctionTok{:} \StringTok{"CreativeWork"}\FunctionTok{,}
        \DataTypeTok{"name"}\FunctionTok{:} \StringTok{"Creative Commons Attribution 4.0"}
      \FunctionTok{\}}
  \OtherTok{]}
\FunctionTok{\}}
\end{Highlighting}
\end{Shaded}

\textbf{Listing 1}: Simplified\footnote{Recommended properties for types
  shown in Listing 1 also include \texttt{affiliation},
  \texttt{citation}, \texttt{contactPoint}, \texttt{description},
  \texttt{encodingFormat}, \texttt{funder}, \texttt{geo},
  \texttt{identifier}, \texttt{keywords}, \texttt{publisher}; these
  properties and corresponding contextual entities are excluded here for
  brevity. See
  \href{https://www.researchobject.org/2021-packaging-research-artefacts-with-ro-crate/listing1/}{complete
  example}.} RO-Crate metadata file showing the flattened compacted
JSON-LD \texttt{@graph} array containing the data entities and
contextual entities, cross-referenced using \texttt{@id}. The
\texttt{ro-crate-metadata.json} entity self-declares conformance with
the RO-Crate specification using a versioned persistent identifier,
further RO-Crate descriptions are on the root data entity \texttt{./} or
any of the referenced data or contextual entities. This is exemplified
by the data entity \texttt{ImageObject} referencing contextual entities
for \texttt{contentLocation} and \texttt{author} that differs from that
of the overall RO-Crate. In this crate, \texttt{about} of the CSV data
entity reference the \texttt{ImageObject}, which then take the roles of
both a data entity and contextual entity. While \texttt{Person} entities
ideally are identified with ORCID PIDs as for Josiah, \texttt{\#alice}
is here in contrast an RO-Crate local identifier, highlighting the
pragmatic ``just enough'' Linked Data approach. \normalsize

In this flattened profile of JSON-LD, each \texttt{\{entity\}} are
directly under \texttt{@graph} and represents the RDF triples with a
common \emph{subject} (\texttt{@id}), mapped \emph{properties} like
\texttt{hasPart}, and \emph{objects} --- as either literal
\texttt{"string"} values, referenced \texttt{\{objects\}} (which
properties are listed in its own entity), or a JSON \texttt{{[}list{]}}
of these. If processed as JSON-LD, this forms an RDF graph by matching
the \texttt{@id} IRIs and applying the \texttt{@context} mapping to
Schema.org terms. \normalsize

\hypertarget{flattened-json-ld}{%
\paragraph{Flattened JSON-LD}\label{flattened-json-ld}}

When JSON-LD 1.0
\href{https://www.w3.org/TR/2014/REC-json-ld-20140116/}{{[}112{]}} was
proposed, one of the motivations was to seamlessly apply an RDF nature
on top of regular JSON as frequently used by Web APIs. JSON objects in
APIs are frequently nested with objects at multiple levels, and the
perhaps most common form of JSON-LD is the
\href{https://json-ld.org/spec/REC/json-ld/20140116/\#compacted-document-form}{compacted
form} which follows this expectation
(\href{https://www.w3.org/TR/2020/REC-json-ld11-20200716/}{JSON-LD 1.1}
further expands these capabilities, e.g.~allowing nested
\texttt{@context} definitions).

While this feature of JSON-LD can be seen as a way to ``hide'' its RDF
nature, we found that the use of nested trees (e.g.~a \texttt{Person}
entity appearing as \texttt{author} of a \texttt{File} which nests under
a \texttt{Dataset} with \texttt{hasPart}) counter-intuitively forces
consumers to consider the JSON-LD as an RDF Graph, since an identified
\texttt{Person} entity can appear at multiple and repeated points of the
tree (e.g.~author of multiple files), necessitating node merging or
duplication, which can become complicated as this approach also invites
the use of \emph{blank nodes} (entities missing \texttt{@id}).

By comparison, a single flat \texttt{@graph} array approach, as required
by RO-Crate, means that applications can choose to process and edit each
entity as pure JSON by a simple lookup based on \texttt{@id}. At the
same time, lifting all entities to the same level reflects the Research
Object principles
{[}\href{https://www.research.manchester.ac.uk/portal/en/publications/why-linked-data-is-not-enough-for-scientists(479e591e-b295-4478-b0c7-a145c19dcd45).html}{12}{]}
in that describing the context and provenance is just as important as
describing the data, and the requirement of \texttt{@id} of every entity
forces RO-Crate generators to consciously
\href{https://www.researchobject.org/ro-crate/1.1/appendix/jsonld.html\#describing-entities-in-json-ld}{consider
existing IRIs and identifiers}.

\hypertarget{json-ld-context}{%
\paragraph{JSON-LD context}\label{json-ld-context}}

In JSON-LD, the \texttt{@context} is a reference to another JSON-LD
document that provides mapping from JSON keys to Linked Data term IRIs,
and can enable various JSON-LD directives to cater for customised JSON
structures for translating to RDF.

RO-Crate reuses vocabulary terms and IRIs from Schema.org, but provides
its own versioned \href{https://w3id.org/ro/crate/1.1/context}{JSON-LD
context}, which has a flat list with the mapping from JSON-LD keys to
their IRI equivalents (e.g.~key \texttt{"author"} maps to the
\url{http://schema.org/author} property).

The rationale behind this decision is to support JSON-based RO-Crate
applications that are largely unaware of JSON-LD, that still may want to
process the \texttt{@context} to find or add Linked Data definitions of
otherwise unknown properties and types. Not reusing the official
Schema.org context means RO-Crate is also able to map in additional
vocabularies where needed, namely the \emph{Portland Common Data Model}
(PCDM) \href{https://github.com/duraspace/pcdm/wiki}{{[}31{]}} for
repositories and Bioschemas
\href{https://iswc2017.semanticweb.org/paper-579/}{{[}58{]}} for
describing computational workflows. RO-Crate profiles may
\href{https://www.researchobject.org/ro-crate/1.1/appendix/jsonld.html\#extending-ro-crate}{extend}
the \texttt{@context} to re-use additional domain-specific ontologies.

Similarly, while the Schema.org context
\href{https://schema.org/version/13.0/schemaorg-current-http.jsonld}{currently}
have \texttt{"@type":\ "@id"} annotations for implicit object
properties, RO-Crate JSON-LD distinguishes explicitly between references
to other entities (\texttt{\{"@id":\ "\#alice"\}}) and string values
(\texttt{"Alice"}) --- meaning RO-Crate applications can find references
for corresponding entities and IRIs without parsing the
\texttt{@context} to understand a particular property. Notably this is
exploited by the \emph{ro-crate-html-js}
\href{https://www.npmjs.com/package/ro-crate-html-js}{{[}95{]}} tool to
provide reliable HTML rendering for otherwise unknown properties and
types.

\hypertarget{community}{%
\subsubsection{RO-Crate Community}\label{community}}

The RO-Crate conceptual model, implementation and best practices are
developed by a growing community of researchers, developers and
publishers. RO-Crate's community is a key aspect of its effectiveness in
making research artefacts FAIR. Fundamentally, the community provides
the overall context of the implementation and model and ensures its
interoperability.

The RO-Crate community consists of:

\begin{enumerate}
\def\labelenumi{\arabic{enumi}.}
\tightlist
\item
  a diverse set of people representing a variety of stakeholders;
\item
  a set of collective norms;
\item
  an open platform that facilitates communication (GitHub, Google Docs,
  monthly teleconferences).
\end{enumerate}

\hypertarget{people}{%
\paragraph{People}\label{people}}

The initial concept of RO-Crate was formed at the first Workshop on
Research Objects
(\href{https://www.researchobject.org/ro2018/}{RO2018}), held as part of
the IEEE conference on eScience. This workshop followed up on
considerations made at a
\href{https://rd-alliance.org/approaches-research-data-packaging-rda-11th-plenary-bof-meeting}{Research
Data Alliance (RDA) meeting on Research Data Packaging} that found
similar goals across multiple data packaging efforts
{[}\href{https://doi.org/10.5281/zenodo.3250687}{23}{]}: simplicity,
structured metadata and the use of JSON-LD.

An important outcome of discussions that took place at RO2018 was the
conclusion that the original Wf4Ever Research Object ontologies
{[}\href{https://doi.org/10.1016/j.websem.2015.01.003}{Belhajjame 2015}{]}, in
principle sufficient for packaging research artefacts with rich
descriptions, were, in practice, considered inaccessible for regular
programmers (e.g., Web developers) and in danger of being
incomprehensible for domain scientists due to their reliance on Semantic
Web technologies and other ontologies.

DataCrate {[}\href{https://doi.org/10.5281/zenodo.1445817}{103}{]} was
presented at RO2018 as a promising lightweight alternative approach, and
an agreement was made by a group of volunteers to attempt building what
was initially called \emph{``RO Lite''} as a combination of DataCrate's
implementation and Research Object's principles.

This group, originally made up of library and Semantic Web experts, has
subsequently grown to include domain scientists, developers, publishers
and more. This perspective of multiple views led to the specification
being used in a variety of domains, from bioinformatics and regulatory
submissions to humanities and cultural heritage preservation.

The RO-Crate community is strongly engaged with the European-wide
biology/bioinformatics collaborative e-Infrastructure ELIXIR
{[}\href{https://doi.org/10.1016/j.tibtech.2012.02.002}{34}{]}, along
with \href{https://eosc.eu/}{European Open Science Cloud} (EOSC)
projects including \href{https://www.eosc-life.eu/}{EOSC-Life},
\href{https://fairplus-project.eu/}{FAIRplus},
\href{https://cs3mesh4eosc.eu/}{CS3MESH4EOSC} and
\href{https://by-covid.eu/}{BY-COVID}. RO-Crate has also established
collaborations with Bioschemas
\href{https://iswc2017.semanticweb.org/paper-579/}{{[}58{]}}, GA4GH
{[}\href{https://doi.org/10.1016/j.xgen.2021.100029}{99}{]}, OpenAIRE
\href{https://doi.org/10.5860/crln.76.6.9326}{{[}100{]}} and multiple
H2020 projects.

A key set of stakeholders are developers: the RO-Crate community has
made a point of attracting developers who can implement the
specifications but, importantly, keeps ``developer user experience'' in
mind. This means that the specifications are straightforward to
implement and thus do not require expertise in technologies that are not
widely deployed.

This notion of catering to ``developer user experience'' is an example
of the set of norms that have developed and now define the community.

\hypertarget{norms}{%
\paragraph{Norms}\label{norms}}

The RO-Crate community is driven by informal conventions and notions
that are prevalent but not neccessarily written down. Here, we distil
what we as authors believe are the critical set of norms that have
facilitated the development of RO-Crate and contributed to the ability
for RO-Crate research packages to be FAIR. This is not to say that there
are no other norms within the community nor that everyone in the
community holds these uniformly. Instead, what we emphasise is that
these norms are helpful and also shaped by community practices.

\begin{enumerate}
\def\labelenumi{\arabic{enumi}.}
\tightlist
\item
  Simplicity
\item
  Developer friendliness
\item
  Focus on examples and best practices rather than rigorous
  specification
\item
  Reuse ``just enough'' Web standards
\end{enumerate}

A core norm of RO-Crate is that of \textbf{simplicity}, which sets the
scene for how we guide developers to structure metadata with RO-Crate.
We focus mainly on documenting simple approaches to the most common use
cases, such as authors having an affiliation. This norm also influences
our take on \textbf{developer friendliness}; for instance, we are using
the Web-native JSON format, allowing only a few of JSON-LD's flexible
Linked Data features. Moreover, the RO-Crate documentation is largely
built up by \textbf{examples} showcasing \textbf{best practices}, rather
than rigorous specifications. We build on existing \textbf{Web
standards} that themselves are defined rigorously, which we utilise
\emph{``\textbf{just enough}''} in order to benefit from the advantages
of Linked Data (e.g., extensions by namespaced vocabularies), without
imposing too many developer choices or uncertainties (e.g., having to
choose between the many RDF syntaxes).

While the above norms alone could easily lead to the creation of ``yet
another'' JSON format, we keep the goal of \textbf{FAIR
interoperability} of the captured metadata, and therefore follow closely
FAIR best practices and current developments such as data citations,
PIDs, open repositories and recommendations for sharing research outputs
and software.

\hypertarget{open-platforms}{%
\paragraph{Open Platforms}\label{open-platforms}}

The critical infrastructure that enables the community around RO-Crate
is the use of open development platforms. This underpins the importance
of open community access to supporting FAIR. Specifically, it is
difficult to build and consume FAIR research artefacts without being
able to access the specifications, understand how they are developed,
know about any potential implementation issues, and discuss usage to
evolve best practices.

The development of RO-Crate was driven by capturing documentation of
real-life examples and best practices rather than creating a rigorous
specification. At the same time, we agreed to be opinionated on the
syntactic form to reduce the jungle of implementation choices; we wanted
to keep the important aspects of Linked Data to adhere to the FAIR
principles while retaining the option of combining and extending the
structured metadata using the existing Semantic Web stack, not just
build a standalone JSON format.

Further work during 2019 started adapting the DataCrate documentation
through a more collaborative and exploratory \emph{RO Lite} phase,
initially using Google Docs for review and discussion, then moving to
GitHub as a collaboration space for developing what is now the RO-Crate
specification,
\href{https://github.com/researchobject/ro-crate/}{maintained} as
Markdown in GitHub Pages and published through Zenodo.

In addition to the typical Open Source-style development with GitHub
issues and pull requests, the RO-Crate Community have, at time of
writing, two regular monthly calls, a Slack channel and a mailing list
for coordinating the project; also many of its participants collaborate
on RO-Crate at multiple conferences and coding events such as the
\href{https://biohackathon-europe.org/}{ELIXIR BioHackathon}. The
community is jointly developing the RO-Crate specification and Open
Source tools, as well as providing support and considering new use
cases. The
\href{https://www.researchobject.org/ro-crate/community}{RO-Crate
Community} is open for anyone to join, to equally participate under a
code of conduct, and as of October 2021 has more than 50 members (see
Appendix \protect\hyperlink{communitylist}{RO-Crate Community}).

\hypertarget{tooling}{%
\subsection{RO-Crate Tooling}\label{tooling}}

The work of the community has led to the development of a number of
tools for creating and using RO-Crates. Table 1 shows the current set of
implementations. Reviewing this list, one can see support for commonly
used programming languages, including Python, JavaScript, and Ruby.
Additionally, the tools can be integrated into commonly used research
environments, in particular, the command line tool
\emph{ro-crate-html-js}
\href{https://www.npmjs.com/package/ro-crate-html-js}{{[}95{]}} for
creating a human-readable preview of an RO-Crate as a sidecar HTML file.
Furthermore, there are tools that cater to end-users (\emph{Describo}
\href{https://arkisto-platform.github.io/describo/}{{[}78{]}},
\emph{WorkflowHub} \href{https://w3id.org/workflowhub/}{{[}124{]}}), in
order to simplify creating and managing RO-Crate. For example, Describo
was developed to help researchers of the Australian
\href{https://criminalcharacters.com/}{Criminal Characters project} to
annotate historical prisoner records for greater insight into the
history of Australia
{[}\href{https://doi.org/10.1080/14490854.2020.1796500}{97}{]}.

While the development of these tools is promising, our analysis of their
maturity status shows that the majority of them are in the Beta stage.
This is partly due to the fact that the RO-Crate specification itself
only recently reached 1.0 status, in November 2019
{[}\href{https://doi.org/10.5281/zenodo.3541888}{105}{]}. Now that there
is a fixed point of reference: With version 1.1 (October 2020)
{[}\href{https://doi.org/10.5281/zenodo.4031327}{107}{]} RO-Crate has
stabilised based on feedback from application development, and now we
are seeing a further increase in the maturity of these tools, along with
the creation of new ones.

Given the stage of the specification, these tools have been primarily
targeting developers, essentially providing them with the core libraries
for working with RO-Crate. Another target has been that of research data
managers who need to manage and curate large amounts of data.

\begin{longtable}[]{@{}
  >{\raggedright\arraybackslash}p{(\columnwidth - 8\tabcolsep) * \real{0.15}}
  >{\raggedright\arraybackslash}p{(\columnwidth - 8\tabcolsep) * \real{0.11}}
  >{\raggedright\arraybackslash}p{(\columnwidth - 8\tabcolsep) * \real{0.34}}
  >{\raggedright\arraybackslash}p{(\columnwidth - 8\tabcolsep) * \real{0.09}}
  >{\raggedright\arraybackslash}p{(\columnwidth - 8\tabcolsep) * \real{0.30}}@{}}
\toprule
Tool Name & Targets & Language /Platform & Status & Brief Description \\
\midrule
\endhead
Describo \href{https://arkisto-platform.github.io/describo/}{{[}78{]}} &
Research Data Managers & NodeJS (Desktop) & RC & Interactive desktop
application to create, update and export RO-Crates for different
profiles \\
Describo Online
\href{https://arkisto-platform.github.io/describo-online/}{{[}77{]}} &
Platform developers & NodeJS (Web) & Alpha & Web-based application to
create RO-Crates using cloud storage \\
ro-crate-excel
\href{https://www.npmjs.com/package/ro-crate-excel}{{[}84{]}} & Data
managers & JavaScript & Beta & Command-line tool to create/edit
RO-Crates with spreadsheets \\
ro-crate-html-js
\href{https://www.npmjs.com/package/ro-crate-html-js}{{[}95{]}} &
Developers & JavaScript & Beta & HTML rendering of RO-Crate \\
ro-crate-js
\href{https://github.com/UTS-eResearch/ro-crate-js}{{[}49{]}} & Research
Data Managers & JavaScript & Alpha & Library for creating/manipulating
crates; basic validation code \\
ro-crate-ruby
\href{https://github.com/ResearchObject/ro-crate-ruby}{{[}Bacall 2022b{]}} &
Developers & Ruby & Beta & Ruby library for reading/writing RO-Crate,
with workflow support \\
ro-crate-py \href{https://doi.org/10.5281/zenodo.3956493}{{[}41{]}}) &
Developers & Python & Alpha & Object-oriented Python library for
reading/writing RO-Crate and use by Jupyter Notebook \\
WorkflowHub \href{https://w3id.org/workflowhub/}{{[}124{]}} & Workflow
users & Ruby & Beta & Workflow repository; imports and exports Workflow
RO-Crate \\
Life Monitor \href{https://about.lifemonitor.eu/}{{[}35{]}} & Workflow
developers & Python & Alpha & Workflow testing and monitoring service;
Workflow Testing profile of RO-Crate \\
SCHeMa \href{https://arxiv.org/abs/2103.13138v1}{{[}118{]}} & Workflow
users & PHP & Alpha & Workflow execution using RO-Crate as exchange
mechanism
{[}\href{https://doi.org/10.5281/zenodo.4671709}{10.5281/zenodo.4671709}{]} \\
galaxy2cwl \href{https://github.com/workflowhub-eu/galaxy2cwl}{{[}50{]}}
& Workflow developers & Python & Alpha & Wraps Galaxy workflow as
Workflow RO-Crate \\
Modern PARADISEC \href{https://github.com/CoEDL/modpdsc/}{{[}51{]}} &
Repository managers & Platform & Beta & Cultural Heritage portal based
on OCFL and RO-Crate \\
ONI express
\href{https://arkisto-platform.github.io/tools/portal/}{{[}115{]}} &
Repository managers & Platform & Beta & Platform for publishing data and
documents stored in an OCFL repository via a Web interface \\
ocfl-tools \href{https://github.com/CoEDL/ocfl-tools}{{[}52{]}} &
Developers & JavaScript (CLI) & Beta & Tools for managing RO-Crates in
an OCFL repository \\
RO Composer
\href{https://esciencelab.org.uk/projects/ro-composer/}{{[}Bacall 2019{]}} &
Repository developers & Java & Alpha & REST API for gradually building
ROs for given profile. \\
RDA maDMP Mapper {[}\href{https://doi.org/10.5281/zenodo.3922136}{Arfaoui 2020}{]}
& Data Management Plan users & Python & Beta & Mapping between
machine-actionable data management plans (maDMP) and RO-Crate
{[}\href{https://doi.org/10.4126/frl01-006423291}{87}{]} \\
Ro-Crate\_2\_ma-DMP
{[}\href{https://doi.org/10.5281/zenodo.3903463}{20}{]} & Data
Management Plan users & Python & Beta & Convert between
machine-actionable data management plans (maDMP) and RO-Crate \\
CheckMyCrate
\href{https://github.com/KockataEPich/CheckMyCrate}{{[}Belchev 2021{]}} &
Developers & Python (CLI) & Alpha & Validation according to Workflow
RO-Crate profile \\
RO-Crates-and-Excel
{[}\href{https://doi.org/10.5281/zenodo.5068950}{126}{]} & Data Managers
& Java (CLI) & Alpha & Describe column/data details of spreadsheets as
RO-Crate using DataCube vocabulary \\
\bottomrule
\end{longtable}

Table 1: Applications and libraries implementing RO-Crate, targeting
different types of users across multiple programming languages. Status
is indicative as assessed by this work (Alpha \textless{} Beta
\textless{} Release Candidate (RC) \textless{} Release).

\hypertarget{inuse}{%
\subsection{Profiles of RO-Crate in use}\label{inuse}}

RO-Crate fundamentally forms part of an infrastructure to help build
FAIR research artefacts. In other words, the key question is whether
RO-Crate can be used to share and (re)use research artefacts. Here we
look at three research domains where RO-Crate is being applied:
Bioinformatics, Regulatory Science and Cultural Heritage. In addition,
we note how RO-Crate may have an important role as part of
machine-actionable data management plans and institutional repositories.

From these varied uses of RO-Crate we observe natural differences in
their detail level and the type of entities described by the RO-Crate.
For instance, on submission of an RO-Crate to a workflow repository, it
is reasonable to expect the RO-Crate to contain at least one workflow,
ideally with a declared licence and workflow language. Specific
additional recommendations such as on identifiers is also needed to meet
the emerging requirements of \href{https://fairdo.org/}{FAIR Digital
Objects}.
\href{https://github.com/ResearchObject/ro-crate/issues/153}{Work has
now begun} to formalise these different \emph{profiles} of RO-Crates,
which may impose additional constraints based on the needs of a specific
domain or use case.

\hypertarget{workflows}{%
\subsubsection{Bioinformatics workflows}\label{workflows}}

\href{https://workflowhub.eu/}{WorkflowHub.eu} is a European
cross-domain registry of computational workflows, supported by European
Open Science Cloud projects,
e.g.~\href{https://www.eosc-life.eu/}{EOSC-Life}, and research
infrastructures including the pan-European bioinformatics network
\href{https://elixir-europe.org/}{ELIXIR}
{[}\href{https://doi.org/10.1016/j.tibtech.2012.02.002}{34}{]}. As part
of promoting workflows as reusable tools, WorkflowHub includes
documentation and high-level rendering of the workflow structure
independent of its native workflow definition format. The rationale is
that a domain scientist can browse all relevant workflows for their
domain, before narrowing down their workflow engine requirements. As
such, the WorkflowHub is intended largely as a registry of workflows
already deposited in repositories specific to particular workflow
languages and domains, such as UseGalaxy.eu
{[}\href{https://doi.org/10.1371/journal.ppat.1008643}{Baker 2020}{]} and
Nextflow nf-core
{[}\href{https://doi.org/10.1038/s41587-020-0439-x}{45}{]}.

We here describe three different RO-Crate profiles developed for use
with WorkflowHub.

\hypertarget{profile-for-describing-workflows}{%
\paragraph{Profile for describing
workflows}\label{profile-for-describing-workflows}}

Being cross-domain, WorkflowHub has to cater for many different workflow
systems. Many of these, for instance Nextflow
{[}\href{https://doi.org/10.1038/nbt.3820}{39}{]} and Snakemake
{[}\href{https://doi.org/10.1093/bioinformatics/bts480}{73}{]}, by
virtue of their script-like nature, reference multiple neighbouring
files typically maintained in a GitHub repository. This calls for a data
exchange method that allows keeping related files together. WorkflowHub
has tackled this problem by adopting RO-Crate as the packaging mechanism
{[}\href{https://doi.org/10.5281/zenodo.4705078}{Bietrix 2021}{]}, typing and
annotating the constituent files of a workflow and --- crucially ---
marking up the workflow language, as many workflow engines use common
file extensions like \texttt{*.xml} and \texttt{*.json}. Workflows are
further described with authors, license, diagram previews and a listing
of their inputs and outputs. RO-Crates can thus be used for
interoperable deposition of workflows to WorkflowHub, but are also used
as an archive for downloading workflows, embedding metadata registered
with the WorkflowHub entry and translated workflow files such as
abstract Common Workflow Language (CWL)
{[}\href{https://arxiv.org/abs/2105.07028}{36}{]} definitions and
diagrams {[}\href{https://doi.org/10.5281/zenodo.4605654}{56}{]}.

RO-Crate acts therefore as an interoperability layer between registries,
repositories and users in WorkflowHub. The iterative development between
WorkflowHub developers and the RO-Crate community heavily informed the
creation of the Bioschemas
\href{https://iswc2017.semanticweb.org/paper-579/}{{[}58{]}} profile for
\href{https://bioschemas.org/profiles/ComputationalWorkflow/1.0-RELEASE/}{Computational
Workflows}, which again informed the
\href{https://www.researchobject.org/ro-crate/1.1/workflows.html}{RO-Crate
1.1 specification on workflows} and led to the RO-Crate Python library
\href{https://doi.org/10.5281/zenodo.3956493}{{[}41{]}} and
WorkflowHub's
\href{https://w3id.org/workflowhub/workflow-ro-crate/1.0}{\textbf{Workflow
RO-Crate profile}}, which, in a similar fashion to RO-Crate itself,
recommends which workflow resources and descriptions are required. This
co-development across project boundaries exemplifies the drive for
simplicity and for establishing best practices.

\hypertarget{profile-for-recording-workflow-runs}{%
\paragraph{Profile for recording workflow
runs}\label{profile-for-recording-workflow-runs}}

RO-Crates in WorkflowHub have so far been focused on workflows that are
ready to be run, and development of WorkflowHub is now creating a
\textbf{Workflow Run RO-Crate profile} for the purposes of benchmarking,
testing and executing workflows. As such, RO-Crate serves as a container
of both a \emph{workflow definition} that may be executed and of a
particular \emph{workflow execution with test results}.

This workflow run profile is a continuation of our previous work with
capturing workflow provenance in a Research Object in CWLProv
{[}\href{https://doi.org/10.1093/gigascience/giz095}{68}{]} and
TavernaPROV {[}\href{https://s11.no/2016/provweek-tavernaprov/}{110}{]}.
In both cases, we used the PROV Ontology
\href{https://www.w3.org/TR/2013/REC-prov-o-20130430/}{{[}81{]}},
including details of every task execution with all the intermediate
data, which required significant workflow engine integration.\footnote{CWLProv
  and TavernaProv predate RO-Crate, but use RO-Bundle
  {[}\href{https://w3id.org/bundle/2014-11-05/}{111}{]}, a similar
  Research Object packaging method with JSON-LD metadata.}

Simplifying from the CWLProv approach, the planned Workflow Run RO-Crate
profile will use a high level
\href{https://www.researchobject.org/ro-crate/1.1/provenance.html\#software-used-to-create-files}{Schema.org
provenance} for the input/output boundary of the overall workflow
execution. This \emph{Level 1 workflow provenance}
{[}\href{https://doi.org/10.1093/gigascience/giz095}{68}{]} can be
expressed generally across workflow languages with minimal workflow
engine changes, with the option of more detailed provenance traces as
separate PROV artefacts in the RO-Crate as data entities. In the current
development of \href{https://github.com/DiSSCo/SDR}{Specimen Data
Refinery} {[}\href{https://doi.org/10.3897/rio.6.e57602}{122}{]} these
RO-Crates will document the text recognition workflow runs of digitised
biological specimens, exposed as FAIR Digital Objects
{[}\href{https://doi.org/10.3390/publications8020021}{38}{]}.

WorkflowHub has recently enabled minting of Digital Object Identifiers
(DOIs), a PID commonly used for scholarly artefacts, for registered
workflows, e.g.~\texttt{10.48546/workflowhub.workflow.56.1}
{[}\href{https://doi.org/10.48546/workflowhub.workflow.56.1}{83}{]},
lowering the barrier for citing workflows as computational methods along
with their FAIR metadata -- captured within an RO-Crate. While it is not
an aim for WorkflowHub to be a repository of workflow runs and their
data, RO-Crates of \emph{exemplar workflow runs} serve as useful
workflow documentation, as well as being an exchange mechanism that
preserves FAIR metadata in a diverse workflow execution environment.

\hypertarget{profile-for-testing-workflows}{%
\paragraph{Profile for testing
workflows}\label{profile-for-testing-workflows}}

The value of computational workflows, however, is potentially undermined
by the ``collapse'' over time of the software and services they depend
upon: for instance, software dependencies can change in a
non-backwards-compatible manner, or active maintenance may cease; an
external resource, such as a reference index or a database query
service, could shift to a different URL or modify its access protocol;
or the workflow itself may develop hard-to-find bugs as it is updated.
This \emph{workflow decay} can take a big toll on the workflow's
reusability and on the reproducibility of any processes it evokes
{[}\href{https://www.research.manchester.ac.uk/portal/files/174861334/why_decay.pdf}{125}{]}.

For this reason, WorkflowHub is complemented by a monitoring and testing
service called LifeMonitor
\href{https://about.lifemonitor.eu/}{{[}35{]}}, also supported by
EOSC-Life. LifeMonitor's main goal is to assist in the creation,
periodic execution and monitoring of workflow tests, enabling the early
detection of software collapse in order to minimise its detrimental
effects. The communication of metadata related to workflow testing is
achieved through the adoption of a
\href{https://lifemonitor.eu/workflow_testing_ro_crate}{\textbf{Workflow
Testing RO-Crate profile}} stacked on top of the \emph{Workflow
RO-Crate} profile. This further specialisation of Workflow RO-Crate
allows to specify additional testing-related entities (test suites,
instances, services, etc.), leveraging
\href{https://www.researchobject.org/ro-crate/1.1/appendix/jsonld.html\#extending-ro-crate}{RO-Crate's
extension mechanism} through the addition of terms from custom
namespaces.

In addition to showcasing RO-Crate's extensibility, the testing profile
is an example of the format's flexibility and adaptability to the
different needs of the research community. Though ultimately related to
a computational workflow, in fact, most of the testing-specific entities
are more about describing a protocol for interacting with a monitoring
service than a set of research outputs and its associated metadata.
Indeed, one of LifeMonitor's main functionalities is monitoring and
reporting on test suites running on existing Continuous Integration (CI)
services, which is described in terms of service URLs and job
identifiers in the testing profile. In principle, in this context, data
could disappear altogether, leading to an RO-Crate consisting entirely
of contextual entities. Such an RO-Crate acts more as an exchange format
for communication between services (WorkflowHub and LifeMonitor) than as
an aggregator for research data and metadata, providing a good example
of the format's high versatility.

\hypertarget{regulatorysciences}{%
\subsubsection{Regulatory Sciences}\label{regulatorysciences}}

\href{https://biocomputeobject.org/}{BioCompute Objects} (BCO)
{[}\href{https://doi.org/10.1371/journal.pbio.3000099}{Alterovitz 2018}{]} is a
community-led effort to standardise submissions of computational
workflows to biomedical regulators. For instance, a genomics sequencing
pipeline, as part of a personalised cancer treatment study, can be
submitted to the US Food and Drugs Administration (FDA) for approval.
BCOs are formalised in the standard IEEE 2791-2020
{[}\href{https://www.research.manchester.ac.uk/portal/en/publications/ieee-standard-for-bioinformatics-analyses-generated-by-highthroughput-sequencing-hts-to-facilitate-communication(936de52b-ac53-4f0e-9927-77fd7073e88d).html}{64}{]}
as a combination of \href{https://w3id.org/ieee/ieee-2791-schema/}{JSON
Schemas} that define the structure of JSON metadata files describing
exemplar workflow runs in detail, covering aspects such as the usability
and error domain of the workflow, its runtime requirements, the
reference datasets used and representative output data produced.

BCOs provide a structured view over a particular workflow, informing
regulators about its workings independently of the underlying workflow
definition language. However, BCOs have only limited support for
additional metadata.\footnote{IEEE 2791-2020 do permit user extensions
  in the \emph{extension domain} by referencing additional JSON Schemas.}
For instance, while the BCO itself can indicate authors and
contributors, and in particular regulators and their review decisions,
it cannot describe the provenance of individual data files or workflow
definitions.

As a custom JSON format, BCOs cannot be extended with Linked Data
concepts, except by adding an additional top-level JSON object
formalised in another JSON Schema. A BCO and workflow submitted by
upload to a regulator will also frequently consist of multiple
cross-related files. Crucially, there is no way to tell whether a given
\texttt{*.json} file is a BCO file, except by reading its content and
check for its \texttt{spec\_version}.

We can then consider how a BCO and its referenced artefacts can be
packaged and transferred following FAIR principles.
\href{https://biocompute-objects.github.io/bco-ro-crate/}{\textbf{BCO
RO-Crate}} {[}\href{https://doi.org/10.5281/zenodo.4633732}{109}{]},
part of the BioCompute Object user guides, defines a set of best
practices for wrapping a BCO with a workflow, together with its exemplar
outputs in an RO-Crate, which then provides typing and additional
provenance metadata of the individual files, workflow definition,
referenced data and the BCO metadata itself.

Here the BCO is responsible for describing the \emph{purpose} of a
workflow and its run at an abstraction level suitable for a domain
scientist, while the more open-ended RO-Crate describes the surroundings
of the workflow, classifying and relating its resources and providing
provenance of their existence beyond the BCO. This emerging
\emph{separation of concerns} is shown in
\protect\hyperlink{fig:sep_concerns}{Figure 3}, and highlights how
RO-Crate is used side-by-side of existing standards and tooling, even
where there are apparent partial overlaps.

A similar separation of concerns can be found if considering the
RO-Crate as a set of files, where the \emph{transport-level} metadata,
such as checksum of files, are delegated to separate
\href{https://www.researchobject.org/ro-crate/1.1/appendix/implementation-notes.html\#adding-ro-crate-to-bagit}{BagIt}
manifests, a standard focusing on the preservation challenges of digital
libraries {[}\href{https://doi.org/10.17487/rfc8493}{74}{]}. As such,
RO-Crate metadata files are not required to iterate all the files in
their folder hierarchy, only those that benefit from being described.

Specifically, a BCO description alone is insufficient for reliable
re-execution of a workflow, which would need a compatible workflow
engine depending on the original workflow definition language, so IEEE
2791 recommends using Common Workflow Language (CWL)
{[}\href{https://arxiv.org/abs/2105.07028}{36}{]} for interoperable
pipeline execution. CWL itself relies on tool packaging in software
containers using \href{https://www.docker.com/}{Docker} or
\href{https://docs.conda.io/}{Conda}. Thus, we can consider BCO RO-Crate
as a stack: transport-level manifests of files (BagIt), provenance,
typing and context of those files (RO-Crate), workflow overview and
purpose (BCO), interoperable workflow definition (CWL) and tool
distribution (Docker).

\{\{\textless{} figure src=``ro-crate-bco-sep-of-concerns.svg''
link=``ro-crate-bco-sep-of-concerns.svg'' id=``fig:sep\_concerns''
width=``100\%'' title=``Separation of Concerns in BCO RO-Crate''
caption=``BioCompute Object (IEEE2791) is a JSON file that structurally
explains the purpose and implementation of a computational workflow, for
instance implemented in Common Workflow Language (CWL), that installs
the workflow's software dependencies as Docker containers or BioConda
packages. An example execution of the workflow shows the different kinds
of result outputs, which may be external, using GitHub LFS
\href{https://docs.github.com/en/repositories/working-with-files/managing-large-files}{{[}85{]}}
to support larger data. RO-Crate gathers all these local and external
resources, relating them and giving individual descriptions, for
instance permanent DOI identifiers for reused datasets accessed from
Zenodo, but also adding external identifiers to attribute authors using
ORCID or to identify which licences apply to individual resources. The
RO-Crate and its local files are captured in a BagIt whose checksum
ensures completeness, combined with Big Data Bag
{[}\href{https://www.research.manchester.ac.uk/portal/files/45989205/bagminid.pdf}{25}{]}
features to ``complete'' the bag with large external files such as the
workflow outputs." \textgreater\}\}

\hypertarget{culturalheritage}{%
\subsubsection{Digital Humanities: Cultural
Heritage}\label{culturalheritage}}

The Pacific And Regional Archive for Digital Sources in Endangered
Cultures (\href{https://www.paradisec.org.au/}{PARADISEC})
{[}\href{http://hdl.handle.net/10125/4567}{114}{]} maintains a
repository of more than 500,000 files documenting endangered languages
across more than 16,000 items, collected and digitised over many years
by researchers interviewing and recording native speakers across the
region.

The \href{https://mod.paradisec.org.au/}{Modern PARADISEC demonstrator}
has been
\href{https://arkisto-platform.github.io/case-studies/paradisec/}{proposed}
as an update to the 18 year old infrastructure, to also help long-term
preservation of these artefacts in their digital form. The demonstrator
uses RO-Crate to describe the overall structure and to capture the
metadata of each item. The existing PARADISEC data collection has been
ported and captured as RO-Crates. A Web portal then exposes the
repository and its entries by indexing the RO-Crate metadata files,
presenting a domain-specific view of the items --- the RO-Crate is
``hidden'' and does not change the user interface.

The PARADISEC use case takes advantage of several RO-Crate features and
principles. Firstly, the transcribed metadata are now independent of the
PARADISEC platform and can be archived, preserved and processed in its
own right, using Schema.org as base vocabulary and extended with
PARADISEC-specific terms.

In this approach, RO-Crate is the holder of itemised metadata, stored in
regular files that are organised using
\href{https://ocfl.io/1.0/spec/}{Oxford Common File Layout} (OCFL)
\href{https://ocfl.io/1.0/spec/}{{[}96{]}}, which ensures file integrity
and versioning on a regular shared file system. This lightweight
infrastructure also gives flexibility for future developments and
maintenance. For example a consumer can use Linked Data software such as
a graph database and query the whole corpora using SPARQL triple
patterns across multiple RO-Crates. For long term digital preservation,
beyond the lifetime of PARADISEC portals, a ``last resort'' fallback is
storing the generic RO-Crate HTML preview
\href{https://www.npmjs.com/package/ro-crate-html-js}{{[}95{]}}. Such
human-readable rendering of RO-Crates can be hosted as static files by
any Web server, in line with the approach taken by the Endings
Project.\footnote{The \href{https://endings.uvic.ca/}{Endings Project}
  is a five-year project funded by the Social Sciences and Humanities
  Research Council (SSHRC) that is creating tools, principles, policies
  and recommendations for digital scholarship practitioners to create
  accessible, stable, long-lasting resources in the humanities.}

\hypertarget{dmp}{%
\subsubsection{Machine-actionable Data Management Plans}\label{dmp}}

Machine-actionable Data Management Plans (maDMPs) have been proposed as
an improvement to automate FAIR data management tasks in research
{[}\href{https://doi.org/10.1371/journal.pcbi.1006750}{88}{]}; maDMPs
use PIDs and controlled vocabularies to describe what happens to data
over the research life cycle
{[}\href{https://doi.org/10.1007/978-3-030-45442-5_15}{22}{]}. The
Research Data Alliance's \emph{DMP Common Standard} for maDMPs
{[}\href{https://doi.org/10.15497/rda00039}{121}{]} is one such
formalisation for expressing maDMPs, which can be expressed as Linked
Data using the DMP Common Standard Ontology
{[}\href{https://doi.org/10.4126/frl01-006423289}{21}{]}, a
specialisation of the W3C Data Catalog Vocabulary (DCAT)
\href{https://www.w3.org/TR/2020/REC-vocab-dcat-2-20200204/}{{[}DCAT2{]}}.
RDA maDMPs are usually expressed using regular JSON, conforming to the
DMP JSON Schema.

A mapping has been produced between Research Object Crates and
Machine-actionable Data Management Plans
{[}\href{https://doi.org/10.4126/frl01-006423291}{87}{]}, implemented by
the RO-Crate RDA maDMP Mapper
{[}\href{https://doi.org/10.5281/zenodo.3922136}{7}{]}. A similar
mapping has been implemented by \emph{RO-Crate\_2\_ma-DMP}
{[}\href{https://doi.org/10.5281/zenodo.3903463}{20}{]}. In both cases,
a maDMP can be converted to a RO-Crate, or vice versa. In
{[}\href{https://doi.org/10.4126/frl01-006423291}{87}{]} this
functionality caters for two use cases:

\begin{enumerate}
\def\labelenumi{\arabic{enumi}.}
\tightlist
\item
  Start a skeleton data management plan based on an existing RO-Crate
  dataset, e.g.~an RO-Crate from WorkflowHub.
\item
  Instantiate an RO-Crate based on a data management plan.
\end{enumerate}

An important nuance here is that data management plans are (ideally)
written in \emph{advance} of data production, while RO-Crates are
typically created to describe data \emph{after} it has been generated.
What is significant to note in this approach is the importance of
\textbf{templating} in order to make both tasks automatable and
achievable, and how RO-Crate can fit into earlier stages of the research
life cycle.

\hypertarget{institutionalrepos}{%
\subsubsection{Institutional data repositories -- Harvard Data
Commons}\label{institutionalrepos}}

The concept of a \textbf{Data Commons} for research collaboration was
originally defined as \emph{``cyber-infrastructure that co-locates data,
storage, and computing infrastructure with commonly used tools for
analysing and sharing data to create an interoperable resource for the
research community''}
{[}\href{https://doi.org/10.1109/MCSE.2016.92}{59}{]}. More recently,
Data Commons has been established to mean integration of active
data-intensive research with data management and archival best
practices, along with a supporting computational infrastructure.
Furthermore, the Commons features tools and services, such as
computation clusters and storage for scalability, data repositories for
disseminating and preserving regular, but also large or sensitive
datasets, and other research assets. Multiple initiatives were
undertaken to create Data Commons on national, research, and
institutional levels. For example, the Australian Research Data Commons
(\href{https://ardc.edu.au}{ARDC})
{[}\href{https://doi.org/10.5334/dsj-2019-044}{Barker 2019}{]} is a national
initiative that enables local researchers and industries to access
computing infrastructure, training, and curated datasets for
data-intensive research. NCI's \href{https://gdc.cancer.gov/}{Genomic
Data Commons} (GDC)
{[}\href{https://doi.org/10.1182/blood-2017-03-735654}{65}{]} provides
the cancer research community with access to a vast volume of genomic
and clinical data. Initiatives such as
\href{https://www.rd-alliance.org/groups/global-open-research-commons-ig}{Research
Data Alliance (RDA) Global Open Research Commons} propose standards for
the implementation of Data Commons to prevent them becoming ``data
silos'' and thus, enable interoperability from one Data Commons to
another.

\textbf{Harvard Data Commons}
{[}\href{https://doi.org/10.7557/5.5422}{33}{]} aims to address the
challenges of data access and cross-disciplinary research within a
research institution. It brings together multiple institutional schools,
libraries, computing centres and the
\href{https://dataverse.harvard.edu/}{Harvard Dataverse} data
repository. \href{https://dataverse.org/}{Dataverse}
{[}\href{https://doi.org/10.1045/january2011-crosas}{32}{]} is a free
and open-source software platform to archive, share and cite research
data. The Harvard Dataverse repository is the largest of 70 Dataverse
installations worldwide, containing over 120K datasets with about 1.3M
data files (as of 2021-11-16). Working toward the goal of facilitating
collaboration and data discoverability and management within the
university, Harvard Data Commons has the following primary objectives:

\begin{enumerate}
\def\labelenumi{\arabic{enumi}.}
\tightlist
\item
  the integration of Harvard Research Computing with Harvard Dataverse
  by leveraging Globus endpoints
  {[}\href{https://doi.org/10.1109/MCC.2014.52}{27}{]}; this will allow
  an automatic transfer of large datasets to the repository. In some
  cases, only the metadata will be transferred while the data stays
  stored in remote storage;
\item
  support for advanced research workflows and providing packaging
  options for assets such as code and workflows in the Harvard Dataverse
  repository to enable reproducibility and reuse, and
\item
  interation of repositories supported by Harvard, which include
  \href{https://dash.harvard.edu/}{DASH}, the open access institutional
  repository, the Digital Repository Services (DRS) for preserving
  digital asset collections, and the Harvard Dataverse.
\end{enumerate}

Particularly relevant to this article is the second objective of the
Harvard Data Commons, which aims to support the deposit of research
artefacts to Harvard Dataverse with sufficient information in the
metadata to allow their future reuse (\protect\hyperlink{fig:hdc}{Figure
4}). To support the incorporation of data, code, and other artefacts
from various institutional infrastructures, Harvard Data Commons is
currently working on RO-Crate adaptation. The RO-Crate metadata provides
the necessary structure to make all research artefacts FAIR. The
Dataverse software already has
\href{https://guides.dataverse.org/en/latest/user/appendix.html}{extensive
support for metadata}, including the Data Documentation Initiative
(DDI), Dublin Core, DataCite, and Schema.org. Incorporating RO-Crate,
which has the flexibility to describe a wide range of research
resources, will facilitate their seamless transition from one
infrastructure to the other within the Harvard Data Commons.

Even though the Harvard Data Commons is specific to Harvard University,
the overall vision and the three objectives can be abstracted and
applied to other universities or research organisations. The Commons
will be designed and implemented using standards and commonly-used
approaches to make it interoperable and reusable by others.

\{\{\textless{} figure src=``data-commons-ro-crate-figure-5.svg''
link=``data-commons-ro-crate-figure-5.svg'' width=``100\%''
id=``fig:hdc'' title=``One aspect of Harvard Data Commons''
caption=``Automatic encapsulation and deposit of artefacts from data
management tools used during active research at the Harvard Dataverse
repository.'' \textgreater\}\}

\hypertarget{relatedwork}{%
\subsection{Related Work}\label{relatedwork}}

With the increasing digitisation of research processes, there has been a
significant call for the wider adoption of interoperable sharing of data
and its associated metadata. We refer to
{[}\href{https://doi.org/10.1016/j.patter.2020.100136}{72}{]} for a
comprehensive overview and recommendations, in particular for data;
notably that review highlights the wide variety of metadata and
documentation that the literature prescribes for enabling data reuse.
Likewise, we suggest
{[}\href{https://doi.org/10.1016/j.patter.2021.100322}{82}{]} that
covers the importance of metadata standards in reproducible
computational research.

Here we focus on approaches for bundling research artefacts along with
their metadata. This notion of publishing compound objects for scholarly
communication has a long history behind it
{[}\href{https://doi.org/10.1190/1.1822162}{29}{]}
\href{http://icl.utk.edu/ctwatch/quarterly/articles/2007/08/interoperability-for-the-discovery-use-and-re-use-of-units-of-scholarly-communication/}{{[}117{]}},
but recent approaches have followed three main strands: 1) publishing to
centralised repositories; 2) packaging approaches similar to RO-Crate;
and 3) bundling the computational workflow around a scientific
experiment.

\hypertarget{bundling-and-packaging-digital-research-artefacts}{%
\subsubsection{Bundling and Packaging Digital Research
Artefacts}\label{bundling-and-packaging-digital-research-artefacts}}

Early work making the case for publishing compound scholarly
communication units
\href{http://icl.utk.edu/ctwatch/quarterly/articles/2007/08/interoperability-for-the-discovery-use-and-re-use-of-units-of-scholarly-communication/}{{[}117{]}}
led to the development of the
\href{http://www.openarchives.org/ore/1.0/primer}{Object Re-Use and
Exchange model} (OAI-ORE), providing a structured \textbf{resource map}
of the digital artefacts that together support a scholarly output.

The challenge of describing computational workflows was one of the main
motivations for the early proposal of \emph{Research Objects} (RO)
{[}\href{https://www.research.manchester.ac.uk/portal/en/publications/why-linked-data-is-not-enough-for-scientists(479e591e-b295-4478-b0c7-a145c19dcd45).html}{Bechhofer 2013}{]}
as first-class citizens for sharing and publishing. The RO approach
involves bundling datasets, workflows, scripts and results along with
traditional dissemination materials like journal articles and
presentations, forming a single package. Crucially, these resources are
not just gathered, but also individually typed, described and related to
each other using semantic vocabularies. As pointed out in
{[}\href{https://doi.org/10.1016/j.future.2011.08.004}{Bechhofer 2013}{]} an
open-ended \emph{Linked Data} approach is not sufficient for scholarly
communication: a common data model is also needed in addition to common
and best practices for managing and annotating lifecycle, ownership,
versioning and attributions.

Considering the FAIR principles
{[}\href{https://doi.org/10.1038/sdata.2016.18}{123}{]}, we can say with
hindsight that the initial RO approaches strongly targeted
\emph{Interoperability}, with a particular focus on the reproducibility
of \emph{in-silico experiments} involving computational workflows and
the reuse of existing RDF vocabularies.

The first implementation of Research Objects for sharing workflows in
myExperiment {[}\href{https://doi.org/10.1093/nar/gkq429}{57}{]} was
based on RDF ontologies
\href{http://ceur-ws.org/Vol-523/Newman.pdf}{{[}93{]}}, building on
Dublin Core, FOAF, SIOC, Creative Commons and OAI-ORE to form
myExperiment ontologies for describing social networking, attribution
and credit, annotations, aggregation packs, experiments, view
statistics, contributions, and workflow components
\href{https://web.archive.org/web/20091115080336/http\%3a\%2f\%2frdf.myexperiment.org/ontologies}{{[}92{]}}.

This initially workflow-centric approach was further formalised as the
Wf4Ever Research Object Model
{[}\href{https://doi.org/10.1016/j.websem.2015.01.003}{Belhajjame 2015}{]}, which is
a general-purpose research artefact description framework. This model is
based on existing ontologies (FOAF, Dublin Core Terms, OAI-ORE and
AO/OAC precursors to the W3C Web Annotation Model
\href{https://www.w3.org/TR/2017/REC-annotation-model-20170223/}{{[}28{]}})
and adds specializations for workflow models and executions using W3C
PROV-O \href{https://www.w3.org/TR/2013/REC-prov-o-20130430/}{{[}81{]}}.
The Research Object statements are saved in a \emph{manifest} (the
OAI-ORE \emph{resource map}), with additional annotation resources
containing user-provided details such as title and description.

We now claim that one barrier for wider adoption of the Wf4Eer Research
Object model for general packaging digital research artefacts was
exactly this re-use of multiple existing vocabularies (FAIR principle
I2: \emph{Metadata use vocabularies that follow FAIR principles}), which
in itself is recognized as a challenge
{[}\href{https://doi.org/10.3233/978-1-61499-660-6-9}{67}{]}. Adapters
of the Wf4Ever RO model would have to navigate documentation of multiple
overlapping ontologies, in addition to facing the usual Semantic Web
development choices for RDF serialization formats, identifier minting
and publishing resources on the Web.

Several developments for Research Objects improved on this situation,
such as ROHub used by Earth Sciences
{[}\href{https://arxiv.org/abs/1809.10617}{48}{]}, which provides a
user-interface for making Research Objects, along with Research Object
Bundle {[}\href{https://w3id.org/bundle/2014-11-05/}{111}{]} (RO
Bundle), which is a ZIP-archive embedding data files and a JSON-LD
serialization of the manifest with mappings for a limited set of terms.
RO Bundle was also used for storing detailed workflow run provenance
(TavernaPROV
{[}\href{https://s11.no/2016/provweek-tavernaprov/}{110}{]}).

RO-Bundle evolved to \href{https://w3id.org/ro/bagit}{Research Object
BagIt archives}, a variant of RO Bundle as a BagIt archive
{[}\href{https://doi.org/10.17487/rfc8493}{74}{]}, used by Big Data Bags
{[}\href{https://www.research.manchester.ac.uk/portal/files/45989205/bagminid.pdf}{25}{]},
CWLProv {[}\href{https://doi.org/10.1093/gigascience/giz095}{68}{]} and
WholeTale {[}\href{https://doi.org/10.3233/APC200107}{76}{]}
{[}\href{https://doi.org/10.1109/eScience.2019.00068}{26}{]}.

\hypertarget{fair-digital-objects}{%
\subsubsection{FAIR Digital Objects}\label{fair-digital-objects}}

FAIR Digital Objects (FDO)
{[}\href{https://doi.org/10.3390/publications8020021}{38}{]} have been
proposed as a conceptual framework for making digital resources
available in a Digital Objects (DO) architecture which encourages active
use of the objects and their metadata. In particular, an FDO has five
parts: (i) The FDO \emph{content}, bit sequences stored in an accessible
repository; (ii) a \emph{Persistent Identifier} (PID) such as a DOI that
identifies the FDO and can resolve these same parts; (iii) Associated
rich \emph{metadata}, as separate FDOs; (iv) Type definitions, also
separate FDOs; (v) Associated \emph{operations} for the given types. A
Digital Object typed as a Collection aggregates other DOs by reference.

The Digital Object Interface Protocol
\href{https://www.dona.net/sites/default/files/2018-11/DOIPv2Spec_1.pdf}{{[}47{]}}
can be considered an ``abstract protocol'' of requirements, DOs could be
implemented in multiple ways. One suggested implementation is the
\href{https://fairdigitalobjectframework.org/}{FAIR Digital Object
Framework}, based on HTTP and the Linked Data Principles. While there is
agreement on using PIDs based on DOIs, consensus on how to represent
common metadata, core types and collections as FDOs has not yet been
reached. We argue that RO-Crate can play an important role for FDOs:

\begin{enumerate}
\def\labelenumi{\arabic{enumi}.}
\tightlist
\item
  By providing a predictable and extensible serialisation of structured
  metadata.
\item
  By formalising how to aggregate digital objects as collections (and
  adding their context).
\item
  By providing a natural Metadata FDO in the form of the RO-Crate
  Metadata File.
\item
  By being based on Linked Data and Schema.org vocabulary, meaning that
  PIDs already exist for common types and properties.
\end{enumerate}

At the same time, it is clear that the goal of FDO is broader than that
of RO-Crate; namely, FDOs are active objects with distributed
operations, and add further constraints such as PIDs for every element.
These features improve FAIR features of digital objects and are also
useful for RO-Crate, but they also severely restrict the infrastructure
that needs to be implemented and maintained in order for FDOs to remain
accessible. RO-Crate, on the other hand, is more flexible: it can
minimally be used within any file system structure, or ideally exposed
through a range of Web-based scenarios. A \emph{FAIR profile of
RO-Crate} (e.g.~enforcing PID usage) will fit well within a FAIR Digital
Object ecosystem.

\hypertarget{packaging-workflows}{%
\subsubsection{Packaging Workflows}\label{packaging-workflows}}

The use of computational workflows, typically combining a chain of tools
in an analytical pipeline, has gained prominence in particular in the
life sciences. Workflows might be used primarily to improve
computational scalability, as well as to assist in making computed data
results FAIR {[}\href{https://doi.org/10.1162/dint_a_00033}{55}{]}, for
instance by improving reproducibility
{[}\href{https://doi.org/10.1016/j.future.2017.01.012}{30}{]}, but also
because programmatic data usage help propagate their metadata and
provenance {[}\href{https://doi.org/10.1002/cpe.1228}{69}{]}. At the
same time, workflows raise additional FAIR challenges, since they can be
considered important research artefacts themselves. This viewpoint poses
the problem of capturing and explaining the computational methods of a
pipeline in sufficient machine-readable detail
{[}\href{https://doi.org/10.3233/DS-190026}{80}{]}.

Even when researchers follow current best practices for workflow
reproducibility
{[}\href{https://doi.org/10.1016/j.cels.2018.03.014}{60}{]}
{[}\href{https://doi.org/10.1016/j.future.2017.01.012}{30}{]}, the
communication of computational outcomes through traditional academic
publishing routes effectively adds barriers as authors are forced to
rely on a textual manuscript representations. This hinder
reproducibility and FAIR use of the knowledge previously captured in the
workflow.

As a real-life example, let us look at a metagenomics article
{[}\href{https://doi.org/10.1038/s41586-019-0965-1}{Almeida 2019}{]} that describes
a computational pipeline. Here the authors have gone to extraordinary
efforts to document the individual tools that have been reused,
including their citations, versions, settings, parameters and
combinations. The \emph{Methods} section is two pages in tight
double-columns with twenty four additional references, supported by the
availability of data on an FTP server (60 GB)
\href{http://ftp.ebi.ac.uk/pub/databases/metagenomics/umgs_analyses/}{{[}43{]}}
and of open source code in GitHub
\href{https://github.com/Finn-Lab/MGS-gut}{Finn-Lab/MGS-gut}
\href{https://github.com/Finn-Lab/MGS-gut}{{[}44{]}}, including the
pipeline as shell scripts and associated analysis scripts in R and
Python.

This attention to reporting detail for computational workflows is
unfortunately not yet the norm, and although bioinformatics journals
have strong \emph{data availability} requirements, they frequently do
not require authors to include or cite \emph{software, scripts and
pipelines} used for analysing and producing results
\href{https://twitter.com/soilandreyes/status/1250721245622079488}{{[}108{]}}.
Indeed, in the absence of a specific requirement and an editorial policy
to back it up -- such as eliminating the reference limit -- authors are
effectively discouraged from properly and comprehensively citing
software {[}\href{https://doi.org/10.1038/s41592-019-0350-x}{53}{]}.

However detailed this additional information might be, another
researcher who wants to reuse a particular computational method may
first want to assess if the described tool or workflow is Re-runnable
(executable at all), Repeatable (same results for original inputs on
same platform), Reproducible (same results for original inputs with
different platform or newer tools) and ultimately Reusable (similar
results for different input data), Repurposable (reusing parts of the
method for making a new method) or Replicable (rewriting the workflow
following the method description)
{[}\href{https://doi.org/10.3389/fninf.2017.00069}{Benureau 2017}{]}
\href{http://repscience2016.research-infrastructures.eu/img/CaroleGoble-ReproScience2016v2.pdf}{{[}54{]}}.

Following the textual description alone, researchers would be forced to
jump straight to evaluate ``Replicable'' by rewriting the pipeline from
scratch. This can be expensive and error-prone. They would firstly need
to install all the software dependencies and download reference
datasets. This can be a daunting task, which may have to be repeated
multiple times as workflows typically are developed at small scale on
desktop computers, scaled up to local clusters, and potentially put into
production using cloud instances, each of which will have different
requirements for software installations.

In recent years the situation has been greatly improved by software
packaging and container technologies like Docker and Conda, these
technologies have been increasingly adopted in life sciences
{[}\href{https://doi.org/10.1007/s41019-017-0050-4}{90}{]} thanks to
collaborative efforts such as BioConda
{[}\href{https://doi.org/10.1038/s41592-018-0046-7}{61}{]} and
BioContainers
{[}\href{https://doi.org/10.1093/bioinformatics/btx192}{37}{]}, and
support by Linux distributions (e.g.~Debian Med
{[}\href{https://doi.org/10.1186/1471-2105-11-S12-S5}{89}{]}). As of
November 2021, more than 9,000 software packages are available
\href{https://anaconda.org/bioconda/}{in BioConda alone}, and 10,000
containers \href{https://biocontainers.pro/\#/registry}{in
BioContainers}.

Docker and Conda have been integrated into workflow systems such as
Snakemake
{[}\href{https://doi.org/10.1093/bioinformatics/bts480}{73}{]}, Galaxy
{[}\href{https://doi.org/10.1093/nar/gky379}{Afgan 2018}{]} and Nextflow
{[}\href{https://doi.org/10.1038/nbt.3820}{39}{]}, meaning a downloaded
workflow definition can now be executed on a ``blank'' machine (except
for the workflow engine) with the underlying analytical tools installed
on demand. Even with using containers there is a reproducibility
challenge, for instance
\href{https://www.docker.com/blog/docker-hub-image-retention-policy-delayed-and-subscription-updates/}{Docker
Hub's retention policy will expire container images after six months},
or a lack of recording versions of transitive dependencies of Conda
packages could cause incompatibilities if the packages are subsequently
updated.

These container and package systems only capture small amounts of
metadata\footnote{Docker and Conda can use \emph{build recipes}, a set
  of commands that construct the container image through downloading and
  installing its requirements. However these recipes are effectively
  another piece of software code, which may itself decay and become
  difficult to rerun.}. In particular, they do not capture any of the
semantic relationships between their content. Understanding these
relationships is made harder by the opaque wrapping of arbitrary tools
with unclear functionality, licenses and attributions.

From this we see that computational workflows are themselves complex
digital objects that need to be recorded not just as files, but in the
context of their execution environment, dependencies and analytical
purpose in research -- as well as other metadata (e.g.~version, license,
attribution and identifiers).

It is important to note that having all these computational details in
order to represent them in an RO-Crate is an ideal scenario -- in
practice there will always be gaps of knowledge, and exposing all
provenance details automatically would require improvements to the data
sources, workflow, workflow engine and its dependencies. RO-Crate can be
seen as a flexible annotation mechanism for augmenting automatic
workflow provenance. Additional metadata can be added manually, e.g.~for
sensitive clinical data that cannot be publicly exposed\footnote{FAIR
  principle A2: \emph{Metadata are accessible, even when the data are no
  longer available.}
  {[}\href{https://doi.org/10.1038/sdata.2016.18}{123}{]}}, or to cite
software that lack persistent identifiers. This inline \emph{FAIRifying}
allows researchers to achieve ``just enough FAIR'' to explain their
computational experiments.

\hypertarget{conclusion}{%
\subsection{Conclusion}\label{conclusion}}

RO-Crate has been established as an approach to packaging digital
research artefacts with structured metadata. This approach assists
developers and researchers to produce and consume FAIR archives of their
research.

RO-Crate is formed by a set of best practice recommendations, developed
by an open and broad community. These guidelines show how to use ``just
enough'' standards in a consistent way. The use of structured metadata
with a rich base vocabulary can cover general-purpose contextual
relations, with a Linked Data foundation that ensures extensibility to
domain- and application-specific uses. We can therefore consider an
RO-Crate not just as a structured data archive, but as a multimodal
scholarly knowledge graph that can help ``FAIRify'' and combine metadata
of existing resources.

The adoption of simple Web technologies in the RO-Crate specification
has helped a rapid development of a wide variety of supporting open
source tools and libraries. RO-Crate fits into the larger landscape of
open scholarly communication and FAIR Digital Object infrastructure, and
can be integrated into data repository platforms. RO-Crate can be
applied as a data/metadata exchange mechanism, assist in long-term
archival preservation of metadata and data, or simply used at a small
scale by individual researchers. Thanks to its strong community support,
new and improved profiles and tools are being continuously added to the
RO-Crate landscape, making it easier for adopters to find examples and
support for their own use case.

\hypertarget{strictness-vs-flexibility}{%
\subsubsection{Strictness vs
flexibility}\label{strictness-vs-flexibility}}

There is always a tradeoff between flexibility and strictness
{[}{[}116{]}(https://www.persistent-identifier.nl/urn:nbn:nl:ui:18-14511{]}
when deciding on semantics of metadata models. Strict requirements make
it easier for users and code to consume and populate a model, by
reducing choices and having mandated ``slots'' to fill in. But such
rigidity can also restrict richness and applicability of the model, as
it in turn enforce the initial assumptions about what can be described.

RO-Crate attempts to strike a balance between these tensions, and
provides a common metadata framework that encourages extensions.
However, just like the RO-Crate specification can be thought of as a
\emph{core profile} of Schema.org in JSON-LD, we cannot stress the
importance of also establishing domain-specific RO-Crate profiles and
conventions, as explored in sections
\protect\hyperlink{profiles}{extensibility} and
\protect\hyperlink{inuse}{profiles in use}. Specialization comes
hand-in-hand with the principle of \emph{graceful degradation}; RO-Crate
applications and users are free to choose the semantic detail level they
participate at, as long as they follow the common syntactic
requirements.

\hypertarget{futurework}{%
\subsection{Future Work}\label{futurework}}

The direction of future RO-Crate work is determined by the community
around it as a collaborative effort. We currently plan on further
outreach, building training material (including a comprehensive
entry-level tutorial) and maturing the reference implementation
libraries. We will also collect and build examples of RO-Crate
\emph{consumption}, e.g.~Jupyter Notebooks that query multiple crates
using knowledge graphs. In addition, we are exploring ways to support
some entity types requested by users, e.g.~detailed workflow runs or
container provenance, which do not have a good match in Schema.org. Such
support could be added, for instance, by integrating other vocabularies
or by having separated (but linked) metadata files.

Furthermore, we want to better understand how the community uses
RO-Crate in practice and how it contrasts with other related efforts;
this will help us to improve our specification and tools. By discovering
commonalities in emerging usage (e.g.~additional Schema.org types), the
community helps to reduce divergence that could otherwise occur with
proliferation of further RO-Crate profiles. We plan to gather feedback
via user studies, with the Linked Open Data community or as part of EOSC
Bring-your-own-Data training events.

We operate in an open community where future and potential users of
RO-Crate are actively welcomed to participate and contribute feedback
and requirements. In addition, we are targeting a wider audience through
extensive
\href{https://www.researchobject.org/ro-crate/outreach.html}{outreach
activities} and by initiating new connections. Recent contacts include
American Geophysical Union (AGU) on Data Citation Reliquary
{[}\href{https://doi.org/10.5281/zenodo.4916734}{Agarwal 2021}{]}, National
Institute of Standards and Technology (NIST) on material science, and
\href{https://inveniosoftware.org/products/rdm/}{InvenioRDM} used by the
Zenodo data repository. New Horizon Europe projects adapting RO-Crate
include \href{https://by-covid.org/}{BY-COVID}, which aims to improve
FAIR access to data on COVID-19 and other infectious diseases.

The main addition in the upcoming 1.2 release of the RO-Crate
specifications will be the formalization of
\href{https://www.researchobject.org/ro-crate/1.2-DRAFT/profiles}{profiles}
for different categories of crates. Additional entity types have been
requested by users, e.g.~workflow runs, business workflows, containers
and software packages, tabular data structures; these are not always
matched well with existing Schema.org types but may benefit from other
vocabularies or even separate metadata files, e.g.~from
\href{https://frictionlessdata.io/}{Frictionless Data}. We will be
further aligning and collaborating with related research artefact
description efforts like \href{https://codemeta.github.io/}{CodeMeta}
for software metadata,
\href{https://science-on-schema.org/}{Science-on-Schema.org}
{[}\href{https://doi.org/10.5281/zenodo.4477164}{66}{]} for datasets,
\href{https://fairdo.org/}{FAIR Digital Objects}
{[}\href{https://doi.org/10.3390/publications8020021}{38}{]} and
activities in \href{https://www.eosc.eu/task-force-faq}{EOSC task
forces} including the EOSC Interoperability Framework
{[}\href{https://doi.org/10.2777/620649}{75}{]}.
\include{chapter05-copyedited}
\include{chapter05b}

\chapter{Workflows}
\include{chapter06}
\section{Incrementally building FAIR Digital Objects with Specimen Data
Refinery
workflows}
\label{incrementally-building-fair-digital-objects-with-specimen-data-refinery-workflows}

\emph{Specimen Data Refinery} (SDR) is a developing platform for
automating transcription of specimens from natural history collections
{[}\href{https://doi.org/10.1162/dint_a_00134}{Hardisty 2022}{]} (section \vref{the-specimen-data-refinery}). SDR is
based on computational workflows and digital twins using FAIR Digital
Objects.

We show our recent experiences with building SDR using the Galaxy
workflow system and combining two FDO methodologies with open digital
specimens (openDS) and RO-Crate data packaging. We suggest FDO
improvements for incremental building of digital objects in
computational workflows.

\hypertarget{sdr-workflows}{%
\subsubsection{SDR workflows}\label{sdr-workflows}}

\href{https://sdr.nhm.ac.uk/}{SDR} is realised as the workflow system
Galaxy {[}\href{https://doi.org/10.1093/nar/gky379}{Afgan 2018}{]} with
\href{https://github.com/DiSSCo/SDR}{SDR tools} installed. An Open
Research challenge is that some tools have machine learning models with
a commercial licence. This complicates publishing to
\href{https://toolshed.g2.bx.psu.edu/}{Galaxy toolshed}, however we
created \href{https://www.ansible.com/}{Ansible} scripts to install
equivalent Galaxy servers, including tools and dependencies, accounts
and workflows. SDR workflows are
\href{https://workflowhub.eu/projects/72}{published in WorkflowHub} as
FDOs.

We implemented the use case \emph{De novo digitization} in Galaxy
{[}\href{https://doi.org/10.48546/workflowhub.workflow.373.1}{Brack
2022}{]}. Shown in \protect\hyperlink{fig:workflow}{Figure 1} the
workflow steps exchange openDS JSON
{[}\href{https://doi.org/10.3897/biss.3.37033}{Hardisty 2019}{]}, for
incremental completion of a digital specimen. Initial stages build a
template openDS from a CSV with metadata and image references --
subsequent analysis completes the rest of the JSON with \emph{regions}
of interest, \emph{text} digitised from handwriting, and recognized
\emph{named entities}.

\{\{\textless{} figure src=``figure1.png'' link=``figure1.png''
id=``fig:workflow'' width=``100\%'' title=``FDO propagation in
workflow'' caption=``Draft Galaxy workflow \emph{De Novo digitization}
{[}\href{https://doi.org/10.48546/workflowhub.workflow.373.1}{Brack
2022}{]} shows propagation of partial Open Digital Specimen FDOs between
individual canonical workflow building blocks. First steps process a CSV
file to create the initial openDS, where referenced images are analysed
to detect text lines which are OCRed and then recognized as named
entities. Bands indicate flow of collections of openDS, processed
concurrently by each step. The final step bundles the collection of
openDS FDOs as JSON files in a ZIP archive'' \textgreater\}\}

Galaxy can visualise outputs of each step
(\protect\hyperlink{fig:visualisaton}{Figure 2}), important to make the
FDOs understandable by domain experts and to verify accuracy of SDR.

\{\{\textless{} figure src=``figure2.jpg'' link=``figure2.jpg''
id=``fig:visualisation'' width=``100\%'' title=``Visualising openDS FDO
within Galaxy'' caption=``Showing detected regions of interest
(specimen, labels and scale bar) for a pinned insect.'' \textgreater\}\}

We are adding workflows for partial stages, e.g.~detection of regions
{[}\href{https://doi.org/10.48546/workflowhub.workflow.374.1}{Livermore
2022a}{]} and hand-written text recognition
{[}\href{https://doi.org/10.48546/workflowhub.workflow.375.1}{Livermore
2022b}{]}, which we'll combine with scalability testing and wider
testing by project users. Additional workflows will enhance existing
FDOs and use new tools such as barcode detection of museums' internal
identifiers.

We are now ready to publish digital specimens as FAIR Digital Objects,
with registration into
\href{https://www.dissco.eu/dissco/technical-infrastructure/}{DiSSCO
repositories}, PID assignment and workflow provenance. However, even at
this early stage we have identified several challenges that need to be
addressed.

\hypertarget{fdo-lessons}{%
\subsubsection{FDO lessons}\label{fdo-lessons}}

We highlight the \emph{De novo} use case because this workflow is
exchanging \emph{partial} FDOs -- openDS objects which are not fully
completed and not yet assigned persistent identifiers.
\href{https://github.com/DiSSCo/openDS}{openDS schemas} are still in
development, therefore SDR uses a
more\href{https://github.com/DiSSCo/SDR/blob/main/galaxy-workflow/config/opends-schema.json}{flexible
JSON schema} where only the initial metadata (populated from CSV) are
required. Each step validates the partial FDO before passing it to the
underlying command line tool.

Although workflow steps exchange openDS objects, they cannot be combined
in any order. For instance, \emph{named entity recognition} requires
digitised text in the FDO. We can consider these intermediate steps as
\emph{sub-profiles} of an FDO Type. Unlike hierarchical subclasses,
these FDO profiles are more like
\href{https://en.wikipedia.org/wiki/Duck_typing}{ducktyping}. For
instance a \emph{text detection} step may only require the
\texttt{regions} key, but semantically there is no requirement for say
\texttt{OpenDSWithText} to be a subclass of \texttt{OpenDSWithRegion},
as text also can be transcribed manually without regions.

Similarly, we found that some steps can be executed in parallel, but
this requires merging of partial FDOs. This can be achieved by combining
JSON queries and JSON Schemas, but indicates that it may be more
beneficial to have FDO fragments as separate objects. Adding openDS
fragment steps would however complicate workflows.

Several of our tools process the referenced images, currently https URLs
in openDS. We added a caching layer to avoid repeated image downloading,
coupled with local file-paths wiring in the workflow. A similar
challenge occurs if accessing image data using DOIP, which unlike HTTP,
has no caching mechanisms.

\hypertarget{ro-crate-lessons}{%
\subsubsection{RO-Crate lessons}\label{ro-crate-lessons}}

Galaxy is developing support for importing and
exporting\href{https://www.researchobject.org/workflow-run-crate/}{Workflow
Run Crates}, a profile of RO-Crate
\href{https://doi.org/10.3233/DS-210053}{Soiland-Reyes 2022b} to
captures execution history of a workflow, including its definition and
intermediate data {[}De Geest 2022{]}. SDR is adopting this support to
combine openDS FDOs with workflow provenance, as envisioned by
{[}\href{https://doi.org/10.3897/rio.6.e57602}{Walton 2020}{]}.

Our prototype \emph{de novo} workflow returns results as a ZIP file of
openDS objects. End-users should also get copies of the referenced
images and generated visualisations, along with workflow execution
metadata. We are investigating ways to embed the preliminary Galaxy
workflow history before the final step, so that this result can be an
enriched RO-Crate.

\hypertarget{conclusions-1}{%
\subsubsection{Conclusions}\label{conclusions-1}}

SDR is an example of machine-assisted construction of FDOs, which
highlight the needs for intermediate digital objects that are not yet
FDO compliant. The passing of such ``local FDOs'' is beneficial not just
for efficiency and visual inspection, but also to simplify workflow
composition of canonical workflow building blocks. At the same time we
see that it is insufficient to only pass FDOs as JSON objects, as they
also have references to other data such as images, which should not need
to be re-downloaded.

Further work will investigate the use of RO-Crate as a wrapper of
partial FDOs, but this needs to be coupled with more flexible FDO types
as profiles, in order to restrict ``impossible'' ordering of steps
depending on particular inner FDO fragments. A distinction needs to be
made between open digital specimens that are in ``draft'' state and
those that can be pushed to DiSSCo registries.

We are experimenting with changing the SDR components into Canonical
Workflow Building Blocks
{[}\href{https://doi.org/10.1162/dint_a_00135}{Soiland-Reyes 2022a}{]}
(\vref{making-canonical-workflow-building-blocks-interoperable-across-workflow-languages}) 
using the Common Workflow Language
{[}\href{https://doi.org/10.1145/3486897}{Crusoe 2022}{]}. This gives
flexibility to scalably execute SDR workflows on different compute
backends such as HPC or local cluster, without the additional setup of
Galaxy servers.


\include{chapter08}
  

%%%%%%%%%%%%%%%%%% CONCLUSIONS %%%%%%%%%%%%%%%%%%
\chapter{Conclusions}
(This thesis chapter will summarize this PhD study and give further
considerations)

\hypertarget{deze-samenvatting-zal-op-de-een-of-andere-manier-zelfs-in-het-nederlands-herhaald-worden.}{%
\subsection{(deze samenvatting zal op de een of andere manier zelfs in
het Nederlands herhaald
worden.)}\label{deze-samenvatting-zal-op-de-een-of-andere-manier-zelfs-in-het-nederlands-herhaald-worden.}}


%%%%%%%%%%%%%%%%%%% Extra stuff %%%%%%%%%%%

\chapter{Contributions}
\label{ch10:contributions} 

Here I detail my contributions for each chapter of this thesis, and
list all the other contributors and their affiliations.



\section{Thesis contributions}\label{ch10:my-contributions}

Below are the author contributions to published articles that form part
of this thesis. Contributions are classified primarily according to the
Contributor Roles Taxonomy (CASRAI CrEDiT) \cite{Brand 2015}. See 
also chapter \vref{ch11:acknowledgements} for acknowledgements beyond authorship covered below.

For all chapters except section \ref{ch8:the-specimen-data-refinery}, I am the main author of the corresponding
manuscripts and have contributed to all aspects of the research. See details below:


\subsection{Contributions for \emph{Evaluating FAIR Digital
Object as a distributed object system}}

Section \vref{ch3:evaluating-fdo-ld} was co-authored by:

\begin{description}
\tightlist
\item[Stian Soiland-Reyes]
Conceptualization, Formal Analysis, Funding acquisition, Investigation,
Methodology, Software, Writing -- original draft, Writing -- review and
editing
\item[Carole Goble]
Funding acquisition, Supervision, Writing -- review and editing
\item[Paul Groth]
Conceptualization, Methodology, Supervision, Writing -- original draft, Writing -- review
and editing
\end{description}

I am the main author of the corresponding manuscript and have contributed to all aspects of the research. 


\subsection{Contributions for \emph{Updating
Linked Data practices for FAIR Digital Object principles}}
Section \vref{ch2:updating-linked-data-practices-for-fair-digital-object-principles} was co-authored by:

\begin{description}
\tightlist
\item[Stian Soiland-Reyes]
Conceptualization, Formal Analysis, Funding acquisition, Investigation,
Software, Writing -- original draft, Writing -- review and editing
\item[Leyla Jael Castro]
Writing -- original draft
\item[Daniel Garijo]
Conceptualization, Funding acquisition, Writing -- review and editing
\item[Marc Portier]
Investigation, Writing -- original draft, Writing -- review and editing
\item[Carole Goble:]
Funding acquisition, Supervision
\item[Paul Groth]
Supervision
\end{description}

This work was presented as talk by Stian Soiland-Reyes at First International Conference on FAIR Digital Objects, Leiden, The Netherlands.

\begin{itemize}
\tightlist
\item
  Slides: \url{https://doi.org/10.5281/zenodo.7256428}
\end{itemize}



\subsection{Contributions for \emph{Packaging research artefacts
with RO-Crate}}\label{ch10:packagingrocrate}

Section \vref{ch5:packaging-research-artefacts-with-ro-crate} was co-authored by:

\begin{description}
\tightlist
\item[Stian Soiland-Reyes]
Conceptualization, Data curation, Formal Analysis, Funding acquisition,
Investigation, Methodology, Project administration, Software,
Visualization, Writing -- original draft, Writing -- review \& editing
\item[Peter Sefton]
Conceptualization, Investigation, Methodology, Project administration,
Resources, Software, Writing -- review \& editing
\item[Mercè Crosas]
Writing -- review \& editing
\item[Leyla Jael Castro]
Methodology, Writing -- review \& editing
\item[Frederik Coppens]
Writing -- review \& editing
\item[José M. Fernández]
Methodology, Software, Writing -- review \& editing
\item[Daniel Garijo]
Methodology, Writing -- review \& editing
\item[Björn Grüning]
Writing -- review \& editing
\item[Marco La Rosa]
Software, Methodology, Writing -- review \& editing
\item[Simone Leo]
Software, Methodology, Writing -- review \& editing
\item[Eoghan Ó Carragáin]
Investigation, Methodology, Project administration, Writing -- review \&
editing
\item[Marc Portier]
Methodology, Writing -- review \& editing
\item[Ana Trisovic]
Software, Writing -- review \& editing
\item[RO-Crate Community]
Investigation, Software, Validation, Writing -- review \& editing
\item[Paul Groth]
Methodology, Supervision, Writing -- original draft, Writing -- review
\& editing
\item[Carole Goble]
Conceptualization, Funding acquisition, Methodology, Project
administration, Supervision, Visualization, Writing -- review \& editing
\end{description}

I am the main author of the corresponding manuscript and have contributed to all aspects of the research.  Subsection \vref{ch5:profile-for-testing-workflows} was primarily authored by Simone Leo. Subsection \vref{ch5:profile-for-describing-workflows}, and section \vref{ch5:institutionalrepos} with figure \ref{ch5:fig:hdc} was authored by Mercè Crosas and Ana Trisovic and edited by me.


\subsection{Contributions for \emph{Creating lightweight
FAIR Digital Objects with RO-Crate}}

Section \vref{ch4:lightweight-fdo} was co-authored by:

\begin{description}
\tightlist
\item[Stian Soiland-Reyes]
Conceptualization, Funding acquisition, Project administration,
Software, Writing -- original draft, Writing -- review \& editing
\item[Peter Sefton]
Funding acquisition, Project administration, Software
\item[Leyla Jael Castro]
Writing -- original draft, Writing -- review \& editing
\item[Frederik Coppens]
Funding acquisition, Supervision, Writing -- review \& editing
\item[Daniel Garijo]
Software, Writing -- review and editing
\item[Simone Leo]
Conceptualization, Project administration, Software, Writing -- original
draft
\item[Marc Portier]
Writing -- review \& editing
\item[Paul Groth]
Supervision
\end{description}

I am the main author of the corresponding manuscript and have contributed to all aspects of the research. 


\subsection{Contributions for \emph{Formalizing RO-Crate in First Order Logic}}

Section \vref{ch5:formaldefinition}, published as an appendix in \cite{Soiland-Reyes 2022} (see \vref{ch10:packagingrocrate}).

I am the sole author of the corresponding appendix and have contributed to all aspects of the research. 


\subsection{Contributions for \emph{Making
Canonical Workflow Building Blocks interoperable across workflow
languages}}

Section \vref{ch6:making-canonical-workflow-building-blocks-interoperable-across-workflow-languages} was co-authored by:

\begin{description}
\tightlist
\item[Stian Soiland-Reyes]
Conceptualization, Funding acquisition, Investigation, Methodology,
Project administration, Supervision, Writing -- original draft, Writing
-- review \& editing
\item[Genís Bayarri]
Software, Software Documentation
\item[Pau Andrio]
Methodology, Software, Validation, Software Documentation
\item[Robin Long]
Software, Software Documentation
\item[Douglas Lowe]
Software, Software Documentation
\item[Ania Niewielska]
Methodology, Resources, Software
\item[Adam Hospital]
Methodology, Project administration, Resuorces, Software, Validation,
Visualization, Writing -- original draft, Writing -- review \& editing
\item[Paul Groth]
Methodology, Supervision, Writing -- review \& editing
\end{description}

I am the main author of the corresponding manuscript and have contributed to all aspects of the research. 



\subsection{Contributions for \emph{The Specimen Data
Refinery}}

Section \vref{ch8:the-specimen-data-refinery} was co-authored by:

\begin{description}
\tightlist
\item[Alex Hardisty]
Conceptualization, Investigation, Supervision, Validation, Writing --
original draft, Writing -- review \& editing, Approval.
\item[Paul Brack]
Investigation, Writing -- original draft, Writing -- review \& editing.
\item[Carole Goble]
Conceptualization, Supervision, Writing -- review \& editing
\item[Laurence Livermore]
Conceptualization, Funding acquisition, Investigation, Writing --
original draft, Writing -- review \& editing.
\item[Ben Scott]
Investigation, Writing -- original draft, Writing -- review \& editing.
\item[Quentin Groom]
Funding acquisition, Investigation, Writing -- original draft, Writing
-- review \& editing.
\item[Stuart Owen]
Investigation, Writing -- original draft, Writing -- review \& editing.
\item[Stian Soiland-Reyes]
Investigation, Writing -- original draft, Writing -- review \& editing.
\end{description}

My main contributions are to section \ref{ch8:workflow-management-systems-and-canonical-workflows-for-research}, \ref{ch8:fair-packaging-of-researchworkflow-objects-with-ro-crate}, \ref{ch8:fdo-types},
\ref{ch8:discussion}. In the corresponding research I have contributed to designing, technical advice, insight and supervision.


\subsection{Contributions for \emph{Incrementally
building FAIR Digital Objects with Specimen Data Refinery workflows}}

Section \vref{ch7:incrementally-building-fair-digital-objects-with-specimen-data-refinery-workflows} was co-authored by:

\begin{description}
\tightlist
\item[Oliver Woolland]
Data curation, Resources, Software, Visualization, Writing -- review \&
editing
\item[Paul Brack]
Conceptualization, Software
\item[Stian Soiland-Reyes]
Investigation, Methodology, Supervision, Writing -- original draft,
Writing -- review \& editing
\item[Ben Scott]
Data curation, Software, Validation
\item[Laurence Livermore]
Conceptualization, Data curation, Funding acquisition, Methodology,
Project administration, Resources, Writing -- review \& editing
\end{description}

I am the main author of the corresponding manuscript and have contributed to all aspects of the research. 


This work was presented as a poster by Stian Soiland-Reyes at First International Conference on FAIR Digital Objects, Leiden, The Netherlands.

\begin{itemize}
\tightlist
\item
  Poster: \url{https://doi.org/10.5281/zenodo.7233688}
\end{itemize}


\subsection{Supplementary publications}\label{ch10:supplementary-publications}

I have also contributed as co-author to these articles during the PhD
period, provided as supplements:

\footurl{https://s11.no/2022/phd/10-simple-rules-for-workflow-tools/}{Supplement 1: Ten Simple
Rules for making a software tool workflow-ready} \cite{ch6-37}

\footurl{https://s11.no/2022/phd/galaxy-ro-crate/}{Supplement 2: Enhancing RDM in Galaxy by
integrating RO-Crate} \cite{De Geest 2022}

\footurl{https://s11.no/2021/phd/workflow-collaboratory//}{Supplement 3: Implementing FAIR
Digital Objects in the EOSC-Life Workflow Collaboratory Zenodo white
paper} \cite{Goble 2021}

\footurl{https://s11.no/2022/phd/methods-included/}{Supplement 4: Methods Included:
Standardizing Computational Reuse and Portability with the Common
Workflow Language} \cite{Crusoe 2022}

\footurl{https://s11.no/2021/phd/nanopub/}{Supplement 5: Semantic micro-contributions
with decentralized nanopublication services} \cite{Kuhn 2021}

Supplement 6:
Perspectives on automated composition of workflows in the life sciences \cite{lamprechtPerspectivesAutomatedComposition2021b}

Supplement 7: ISO 23494: Biotechnology - Provenance Information Model for Biological
Specimen and Data \cite{Wittner 2020}

Supplement 8: Towards a
Common Standard for Data and Specimen Provenance in Life Sciences. \cite{Wittner 2023}

Supplement 9: A Community Roadmap for Scientific Workflows Research and Development \cite{ch6-39}

Supplement 10: Unique, Persistent, Resolvable: Identifiers as the Foundation of FAIR \cite{Juty 2020} 
\emph{(Contribution pre-dates UvA affiliation)} 

Supplement 11: FAIR Computational Workflows \cite{Goble 2020} \emph{(Main contribution pre-dates UvA
affiliation)} 

Supplement 12: Sharing
interoperable workflow provenance: A review of best practices and their
practical application in CWLProv \cite{ch5-68} \emph{(Main contribution pre-dates UvA
affiliation)} 

Supplement
13: IEEE Standard for Bioinformatics Analyses Generated by
High-Throughput Sequencing (HTS) to Facilitate Communication: IEEE Std
2791-2020 \cite{ch5-64}

\footurl{https://doi.org/10.37044/osf.io/c724r}{Supplement 14}: BioHackEU22 Project 22: Plant data exchange and standard interoperability

I have been involved in All Aspects of the research for supplements 1, 2,
3, 4, 11, 12.

For Supplement 4 and 12 I am a member of the Common Workflow Language
\footurl{https://www.commonwl.org/governance/}{leadership team}.

For Supplement 13 I am a member of the
\footurl{https://www.biocomputeobject.org/}{BioCompute Object} technical
steering committee and was a member of the IEEE 2791-2020 working
group.

\subsection{Contributor affiliations}\label{ch10:contributor-affiliations}

\begin{description}
\tightlist
\item[Sanne Abeln \url{https://orcid.org/0000-0002-2779-7174}]
Department of Computer Science, VU Amsterdam, Amsterdam, The Netherlands
\item[Peter Amstutz \url{https://orcid.org/0000-0003-3566-7705}]
Curii Corporation, Sommerville, MA, USA
\item[Pau Andrio \url{https://orcid.org/0000-0003-2116-3880}]
The Spanish National Bioinformatics Institute (INB), Barcelona
Supercomputing Center (BSC), Barcelona, Spain
\item[Haris Antonatos]
SciFY, Athens, Greece
\item[Finn Bacall \url{https://orcid.org/0000-0002-0048-3300}]
Department of Computer Science, The University of Manchester,
Manchester, UK
\item[Genís Bayarri \url{https://orcid.org/0000-0003-0513-0288}]
Institute for Research in Biomedicine (IRB Barcelona), The Barcelona
Institute of Science and Technology (BIST), Barcelona, Spain
\item[Paul Brack \url{https://orcid.org/0000-0002-5432-2748}]
Department of Computer Science, The University of Manchester,
Manchester, UK (former)
\item[Salvador Capella-Gutierrez
\url{https://orcid.org/0000-0002-0309-604X}]
Life Sciences Department. Barcelona Supercomputing Center (BSC),
Barcelona, Spain
\item[Eoghan Ó Carragáin \url{https://orcid.org/0000-0001-8131-2150}]
University College Cork, Ireland
\item[John Chilton \url{https://orcid.org/0000-0002-6794-0756},]
Department of Biochemistry and Molecular Biology, Pennsylvania State
University, PA, USA\\
Galaxy Project
\item[Frederik Coppens \url{https://orcid.org/0000-0001-6565-5145}]
Department of Plant Biotechnology and Bioinformatics, Ghent University,
Ghent, Belgium\\
VIB-UGent Center for Plant Systems Biology, Ghent, Belgium
\item[Mercè Crosas \url{https://orcid.org/0000-0003-1304-1939}]
Institute for Quantitative Social Science, Harvard University,
Barcelona Supercomputing Center, Barcelona, ES\\
Secretària de Govern Obert, Catalunya, Barcelona, ES (former)
Cambridge, MA, USA (former)\\
\item[Peter Crowther \url{https://orcid.org/0000-0002-2222-9418}]
Melandra Limited, Stockport, UK
\item[Michael R. Crusoe \url{https://orcid.org/0000-0002-2961-9670}]
Forschungszentrum Jülich, Jülich, Germany\\
Department of Computer Science, VU Amsterdam, Amsterdam, The Netherlands\\
Common Workflow Language project, Software Freedom Conservancy,
Brooklyn, NY, USA
\item[Mathias Dillen \url{https://orcid.org/0000-0002-3973-1252}]
Meise Botanic Garden, Meise, Belgium
\item[Bert Droesbeke \url{https://orcid.org/0000-0003-0522-5674}]
Department of Plant Biotechnology and Bioinformatics, Ghent University,
Ghent, Belgium\\
VIB Center for Plant Systems Biology, Ghent, Belgium
\item[Michel Dumontier \url{https://orcid.org/0000-0003-4727-9435}]
Institute of Data Science, Maastricht University, Maastricht, The
Netherlands
\item[Ignacio Eguinoa \url{https://orcid.org/0000-0002-6190-122X}]
Department of Plant Biotechnology and Bioinformatics, Ghent University,
Ghent, Belgium\\
VIB-UGent Center for Plant Systems Biology, Ghent, Belgium
\item[Vincent Emonet \url{https://orcid.org/0000-0002-1501-1082}]
Institute of Data Science, Maastricht University, Maastricht, The
Netherlands
\item[Philip Ewels \url{https://orcid.org/0000-0003-4101-2502}]
Science for Life Laboratory (SciLifeLab), Department of Biochemistry and
Biophysics, Stockholm University, Stockholm, Sweden
\item[José Mª Fernández \url{https://orcid.org/0000-0002-4806-5140}]
Barcelona Supercomputing Center, Barcelona, Spain
\item[Daniel Garijo \url{https://orcid.org/0000-0003-0454-7145}]
Ontology Engineering Group, Universidad Politécnica de Madrid, Madrid,
Spain
\item[Bogdan Gavrilović \url{https://orcid.org/0000-0003-1550-1716}]
Seven Bridges, Charlestown, MA, USA
\item[Carole Goble \url{https://orcid.org/0000-0003-1219-2137}]
Department of Computer Science, The University of Manchester,
Manchester, UK
\item[Quentin Groom \url{https://orcid.org/0000-0002-0596-5376}]
Meise Botanic Garden, Meise, Belgium
\item[Paul Groth \url{https://orcid.org/0000-0003-0183-6910}]
Informatics Institute, University of Amsterdam, Amsterdam, The
Netherlands
\item[Björn Grüning \url{https://orcid.org/0000-0002-3079-6586}]
Bioinformatics Group, Department of Computer Science,
Albert-Ludwigs-University Freiburg, Freiburg, Germany
\item[Alex Hardisty \url{https://orcid.org/0000-0002-0767-4310}]
School of Computer Science and Informatics, Cardiff University, Cardiff,
UK
\item[Adam Hospital \url{https://orcid.org/0000-0002-8291-8071}]
Institute for Research in Biomedicine (IRB Barcelona), The Barcelona
Institute of Science and Technology (BIST), Barcelona, Spain
\item[Alexandru Iosup \url{https://orcid.org/0000-0001-8030-9398}¹,]
Department of Computer Science, VU Amsterdam, Amsterdam, The Netherlands
\item[Leyla Jael Castro \url{https://orcid.org/0000-0003-3986-0510}]
ZB MED Information Centre for Life Sciences, Cologne, Germany
\item[Tobias Kuhn \url{https://orcid.org/0000-0002-1267-0234}]
Department of Computer Science, VU Amsterdam, Amsterdam, The Netherlands\\
Knowledge Pixels, Zürich, Switzerland
\item[Simone Leo \url{https://orcid.org/0000-0001-8271-5429}]
Center for Advanced Studies, Research, and Development in Sardinia
(CRS4), Pula (CA), Italy
\item[Laurence Livermore \url{https://orcid.org/0000-0002-7341-1842}]
The Natural History Museum, London, UK
\item[Robin Long \url{https://orcid.org/0000-0003-2249-645X}]
Lancaster University, Lancaster, UK\\
Research IT, The University of Manchester, Manchester, UK (former)
\item[Douglas Lowe \url{https://orcid.org/0000-0002-1248-3594}]
Research IT, The University of Manchester, Manchester, UK
\item[Hervé Ménager \url{https://orcid.org/0000-0002-7552-1009}]
Hub de Bioinformatique et Biostatistique, Département Biologie
Computationnelle, Institut Pasteur, Paris, France
\item[Ania Niewielska \url{https://orcid.org/0000-0003-0989-3389}]
European Bioinformatics Institute (EMBL-EBI), Cambridge, UK
\item[Stuart Owen \url{https://orcid.org/0000-0003-2130-0865}]
Department of Computer Science, The University of Manchester,
Manchester, UK
\item[Luca Pireddu \url{https://orcid.org/0000-0002-4663-5613}]
Center for Advanced Studies, Research and Development in Sardinia
(CRS4), Pula, Italy
\item[Marc Portier \url{https://orcid.org/0000-0002-9648-6484}]
Vlaams Instituut voor de Zee (VLIZ), Oostende, Belgium
\item[Laura Rodriguez-Navas \url{https://orcid.org/0000-0003-4929-1219}]
Life Sciences Department. Barcelona Supercomputing Center (BSC),
Barcelona, Spain
\item[Marco La Rosa \url{https://orcid.org/0000-0001-5383-6993}]
PARADISEC, Melbourne, Australia
\item[Ben Scott \url{https://orcid.org/0000-0002-5590-7174}]
The Natural History Museum, London, UK
\item[Peter Sefton \url{https://orcid.org/0000-0002-3545-944X}]
Faculty of Science, University Technology Sydney, Australia

The University of Queensland School of Languages and Cultures, The
University of Queensland, Brisbane, Queensland, Australia
\item[Beatriz Serrano-Solano
\url{https://orcid.org/0000-0002-5862-6132}]
Bioinformatics Group, Albert-Ludwigs-University Freiburg, Freiburg,
Germany
\item[Stian Soiland-Reyes \url{https://orcid.org/0000-0001-9842-9718}]
Department of Computer Science, The University of Manchester,
Manchester, UK\\
Informatics Institute, University of Amsterdam, Amsterdam, The
Netherlands
\item[Ruben Taelman \url{https://orcid.org/0000-0001-5118-256X}]
IDLab, Ghent University, Ghent, Belgium
\item[Nebojša Tijanić \url{https://orcid.org/0000-0001-8316-4067}]
Seven Bridges, Charlestown, MA, USA
\item[Ana Trisovic \url{https://orcid.org/0000-0003-1991-0533}]
Institute for Quantitative Social Science, Harvard University,
Cambridge, MA, USA
\item[Alan R Williams \url{https://orcid.org/0000-0003-3156-2105}] 
Department of Computer Science, The University of Manchester,
Manchester, UK (former)
\item[Oliver Woolland \url{https://orcid.org/0000-0002-4565-9760}]
Research IT, The University of Manchester, Manchester, UK
\end{description}


\section{Community roles}

For section \ref{ch2:updating-linked-data-practices-for-fair-digital-object-principles} and section \ref{ch6:making-canonical-workflow-building-blocks-interoperable-across-workflow-languages} I am a member of FAIR Digital Object Forum \cite{FAIRDigitalObjects} working groups FDO-CWFR, FDO-SEM and have contributions to FDO specifications \cite{fdo-RequirementSpec}, \cite{fdo-Overview}, 

For sections \ref{ch4:lightweight-fdo}, \ref{ch5:packaging-research-artefacts-with-ro-crate},  \ref{ch5:formaldefinition} I co-chair the
\footurl{https://www.researchobject.org/ro-crate/community}{RO-Crate
community}\footnote{see section \vref{ro-crate-community}} together with Peter Sefton. We are the main editors and authors of the RO-Crate specifications \cite{ch5-105,ch5-107,ch5-106}.

For section \ref{ch6:making-canonical-workflow-building-blocks-interoperable-across-workflow-languages} I was deputy work package leader in BioExcel-2, with Adam
Hospital as work package leader.



\section{Software contributions}

During this PhD I have contributed to several software applications and libraries:

\begin{itemize}
  \item  \footurl{https://pypi.org/project/signposting/}{signposting}, link parser library for Python \cite{10.5281/zenodo.7256713}  (main author)
  \item  \footurl{https://w3id.org/a2a-fair-metrics/}{Benchmarks for Apples-to-Apples FAIR Signposting}, main author and maintainer
  \item  \footurl{https://pypi.org/project/rocrate/}{ro-crate-py} \cite{ro-crate-py} (initial author, contributor; main author is Simone Leo)
  \item  \footurl{https://github.com/stain/ro-index-paper}{ro-index-paper} --  early prototype for survey of Research Object usage
  \item  \footurl{https://github.com/ResearchObject/runcrate}{runcrate} https://doi.org/10.5281/zenodo.7764062
  \item  \footurl{https://github.com/marketplace/actions/ro-crate-preview}{ro-crate-preview}, GitHub action Build HTML preview of RO-Crate. Contributed as supervisor, documentation, bug fixes.
  \item  \footurl{https://view.commonwl.org/}{cwlviewer} \cite{cwlviewer}, \footurl{https://github.com/common-workflow-language/cwlviewer/pull/241}{contributed feature}
  \item \footurl{https://github.com/ResearchObject/ro-crate-validator-py}{ro-crate-validator-py}, supervisor
  
  \item ... workflowhub, biobb
\end{itemize}

\section{Standard contributions}
\begin{itemize}
  \item RO-Crate Specification 1.1.3 \cite{ch5-106}, contributing as co-chair of RO-Crate community and editor.
  \item \footurl{https://www.researchobject.org/ro-crate/1.2-DRAFT/}{RO-Crate Specification 1.2} (draft). I am the main editor of this release and have contributed several new sections including \footurl{https://www.researchobject.org/ro-crate/1.2-DRAFT/profiles}{RO-Crate profiles}
  \item IEEE 2791-2020 \cite{ch5-64}, contributing as member of P2791 Working Group. I was responsible for aspects of identifiers and internal review.
  \item \footurl{https://w3id.org/ieee/ieee-2791-schema}{JSON Schema for IEEE 2791}, contributing as member of P2791 Working Group and internal review.
  \item RFC9264 Linkset \cite{RFC9264}, \footurl{https://github.com/dret/I-D/pull/129}{contributed} JSON-LD context and reviewed.
\end{itemize}

\section{Training material contributions}
\begin{itemize}
  \item \url{https://www.researchobject.org/packaging_data_with_ro-crate/}
  \item \url{https://carpentries-incubator.github.io/cwl-novice-tutorial/}
  \item \url{http://docs.bioexcel.eu/cwl-best-practice-guide/}
  \item \url{http://docs.bioexcel.eu/cwl-engine-guide/}
\end{itemize}



\section{Dataset contributions}

\begin{itemize}
  \item \url{https://doi.org/10.5281/zenodo.7676924}
  \item \url{https://doi.org/10.5281/zenodo.3531504}
  \item \url{https://doi.org/10.5281/zenodo.4704867} 
  \item \url{https://doi.org/10.5281/zenodo.5146227}
  \item \url{https://by-covid.github.io/BY-COVID_WP5_T5.2_baseline-use-case/}, contributed as part of BY-COVID project, making initial RO-Crate
  \item \footurl{https://www.researchobject.org/ro-crate/1.1/ro-crate-preview.html}{RO-Crate of RO-Crate specification 1.1} \cite{ch5-106}
  \item \footurl{https://www.researchobject.org/ro-crate/1.2-DRAFT/ro-crate-preview.html}{RO-Crate of RO-Crate specification 1.2-DRAFT}
\end{itemize}

\section{Presentation contributions}
\begin{itemize}

\item ...
  \item https://www.researchobject.org/ro-crate/outreach.html?1921
\end{itemize}

\section{Poster contributions}

\begin{itemize}
  \item \url{https://doi.org/10.5281/zenodo.8004793}
  \item \url{https://doi.org/10.5281/zenodo.8004796}
  \item \url{https://doi.org/10.5281/zenodo.7257146}
  \item \url{https://doi.org/10.5281/zenodo.7245315}
  \item \url{https://doi.org/10.5281/zenodo.7233688}
  \item \url{https://doi.org/10.5281/zenodo.5004842}
  \item \url{https://doi.org/10.5281/zenodo.3343031}
  \item \url{https://doi.org/10.7490/f1000research.1117130.1}
  \item \url{https://doi.org/10.7490/f1000research.1119445.1}
  \item \url{https://doi.org/10.7490/f1000research.1119430.1}
  \item 
\end{itemize}


\chapter{Acknowledgements}
(This thesis chapter will show my contributions for each chapter and
list all the other contributors)

\hypertarget{personal-acknowledgements}{%
\section{Personal acknowledgements}\label{personal-acknowledgements}}

\hypertarget{community-acknowledgements}{%
\section{Community
acknowledgements}\label{community-acknowledgements}}

\section{Funding}

(My funding)

%% Chapter 2

\section{Acknowledgements chapter 2}

\subsection*{Cite as}


\subsection*{Acknowledgements}

We would like to acknowledge the
\href{https://www.researchobject.org/ro-crate/community.html}{RO-Crate
community} and the
\href{https://about.workflowhub.eu/project/acknowledgements/}{WorkflowHub
Club}. Thanks to Rudolf Wittner for valuable comments.

\hypertarget{funding}{%
\subsection*{Funding}\label{funding}}

European Commission Horizon 2020 (EOSC-Life
\href{https://cordis.europa.eu/project/id/824087}{824087}), Horizon
Europe (BY-COVID
\href{https://cordis.europa.eu/project/id/101046203}{101046203},
FAIR-IMPACT
\href{https://cordis.europa.eu/project/id/101057344}{101057344}).

Leyla Jael Castro is supported by a German Research Foundation DFG grant
for NFDI4DataScience.

Daniel Garijo is supported by the Madrid Government (Comunidad de
Madrid-Spain) under the Multiannual Agreement with Universidad
Politécnica de Madrid in the line Support for R\&D projects for Beatriz
Galindo researchers, in the context of the V PRICIT (Regional Programme
of Research and Technological Innovation)


\section{Acknowledgements chapter 3}

\subsection*{Cite as}

...

\subsection*{Acknowledgements}

...

\subsection*{Funding}

This work was funded by the European Union programmes \emph{Horizon 2020} under grant agreement numbers H2020-INFRAEDI-02-2018 823830 (BioExcel-2), H2020-INFRAEOSC-2018-2 824087 (EOSC-Life) and \emph{Horizon Europe} under grant agreement numbers HORIZON-INFRA-2021-EMERGENCY-01 101046203 (BY-COVID), HORIZON-INFRA-2021-EOSC-01 101057388 (EuroScienceGateway), HORIZON-INFRA-2021-TECH-01 101057437 (BioDT); and by UK Research and Innovation (UKRI) under the UK government’s  \emph{Horizon Europe funding guarantee} grant numbers 10038963 (EuroScienceGateway), 10038930 (BioDT).




%% Chapter 4

\section{Acknowledgements Chapter 4}

This chapter was adapted from an abstract, 
presented as poster by Stian Soiland-Reyes at 
First International Conference on FAIR Digital Objects 
(FDO2022) on
2022-08-26/--28 in Leiden, The Netherlands. 
\url{https://www.fdo2022.org/}

\begin{itemize}
\tightlist
\item
  Poster: \url{https://doi.org/10.5281/zenodo.7245315}
\end{itemize}

\subsection*{Cite as}
Stian Soiland-Reyes, Peter Sefton, Leyla Jael Castro, Frederik Coppens,
Daniel Garijo, Simone Leo, Marc Portier, Paul Groth (2022):\\
\textbf{Creating lightweight FAIR Digital Objects with RO-Crate}.\\
1st International Conference on FAIR Digital Objects
(\href{https://www.fdo2022.org/}{FDO 2022}) (poster)\\
\emph{Research Ideas and Outcomes} \textbf{8}:e93937\\
\url{https://doi.org/10.3897/rio.8.e93937}

\hypertarget{acknowledgements-1}{%
\subsection*{Acknowledgements}\label{acknowledgements-1}}

We would like to acknowledge the
\href{https://www.researchobject.org/ro-crate/community.html}{RO-Crate
community} and the
\href{https://about.workflowhub.eu/project/acknowledgements/}{WorkflowHub
Club}.

\hypertarget{funding-1}{%
\subsection*{Funding}\label{funding-1}}

European Commission Horizon 2020 (BioExcel-2
\href{https://cordis.europa.eu/project/id/823830}{823830}, EOSC-Life
\href{https://cordis.europa.eu/project/id/824087}{824087}), Horizon
Europe (BY-COVID
\href{https://cordis.europa.eu/project/id/101046203}{101046203},
FAIR-IMPACT
\href{https://cordis.europa.eu/project/id/101057344}{101057344}).

Daniel Garijo is supported by the Madrid Government (Comunidad de
Madrid-Spain) under the Multiannual Agreement with Universidad
Politécnica de Madrid in the line Support for R\&D projects for Beatriz
Galindo researchers, in the context of the V PRICIT (Regional Programme
of Research and Technological Innovation).

Leyla Jael Castro is supported by a German Research Foundation DFG grant
for NFDI4DataScience.

Frederik Coppens is supported by Research Foundation - Flanders (FWO)
for ELIXIR Belgium (I002819N).


%% chapter 5

\section{Acknowledgements Chapter 5}

This chapter was published in the journal \emph{Data Science}.

\subsection*{Cite as}

Stian Soiland-Reyes, Peter Sefton, Mercè Crosas, Leyla Jael Castro,
Frederik Coppens, José M. Fernández, Daniel Garijo, Björn Grüning, Marco
La Rosa, Simone Leo, Eoghan Ó Carragáin, Marc Portier, Ana Trisovic,
RO-Crate Community, Paul Groth, Carole Goble (2022):\\
\textbf{Packaging research artefacts with RO-Crate}.\\
\emph{Data Science} \textbf{5}(2)\\
\url{https://doi.org/10.3233/DS-210053}

An \href{https://w3id.org/ro/doi/10.5281/zenodo.5146227}{RO-Crate for
this article} is archived at
\url{https://doi.org/10.5281/zenodo.5146227}

\subsection*{License and modifications}

\begin{itemize}
\tightlist
\item
  \textbf{License}: Creative Commons Attribution License
  (\href{https://spdx.org/licenses/CC-BY-4.0}{CC BY 4.0}).
\item
  \textbf{Modifications}: Formatting as Markdown; figure caption
  formatting; reference in s11 house style; added identifiers, authors
  and years clarified where missing in citations; inline citation
  hyperlinks to open access version where available; citations merged and renumbered; 
  acknowledgements and references moved to separate chapters.
\end{itemize}


\subsection*{Funding}

This work has received funding from the European Commission's Horizon
2020 research and innovation programme for projects
\href{https://cordis.europa.eu/project/id/823830}{BioExcel-2}
(H2020-INFRAEDI-2018-1 823830),
\href{https://cordis.europa.eu/project/id/730976}{IBISBA 1.0}
(H2020-INFRAIA-2017-1-two-stage 730976),
\href{https://cordis.europa.eu/project/id/871118}{PREP-IBISBA}
(H2020-INFRADEV-2019-2 871118),
\href{https://cordis.europa.eu/project/id/824087}{EOSC-Life}
(H2020-INFRAEOSC-2018-2 824087),
\href{https://cordis.europa.eu/project/id/823827}{SyntheSys+}
(H2020-INFRAIA-2018-1 823827). From the Horizon Europe Framework
Programme this work has received funding for
\href{https://cordis.europa.eu/project/id/101046203}{BY-COVID}
(HORIZON-INFRA-2021-EMERGENCY-01 101046203).

Björn Grüning is supported by DataPLANT
(\href{https://gepris.dfg.de/gepris/projekt/442077441}{NFDI 7/1 --
42077441}), part of the German National Research Data Infrastructure
(NFDI), funded by the Deutsche Forschungsgemeinschaft (DFG).

Ana Trisovic is funded by the Alfred P. Sloan Foundation
\href{https://sloan.org/grant-detail/9555}{(grant number P-2020-13988)}.
Harvard Data Commons is supported by an award from Harvard University
Information Technology (HUIT).


\subsection*{Contributions}

We would also like to acknowledge contributions from:

\begin{description}
\tightlist
\item[Finn Bacall]
Software, Methodology
\item[Herbert Van de Sompel]
Writing -- review \& editing
\item[Ignacio Eguinoa]
Software, Methodology
\item[Nick Juty]
Writing -- review \& editing
\item[Oscar Corcho]
Writing -- review \& editing
\item[Stuart Owen]
Writing -- review \& editing
\item[Laura Rodríguez-Navas]
Software, Visualization, Writing -- review \& editing
\item[Alan R. Williams]
Writing -- review \& editing
\end{description}

\hypertarget{communitylist}{%
\subsection*{ RO-Crate Community}\label{communitylist}}

As of 2021-10-04, the \emph{RO-Crate} Community members are:

\begin{itemize}
\tightlist
\item
  Peter Sefton \url{https://orcid.org/0000-0002-3545-944X} (co-chair)
\item
  Stian Soiland-Reyes \url{https://orcid.org/0000-0001-9842-9718}
  (co-chair)
\item
  Eoghan Ó Carragáin \url{https://orcid.org/0000-0001-8131-2150}
  (emeritus chair)
\item
  Oscar Corcho \url{https://orcid.org/0000-0002-9260-0753}
\item
  Daniel Garijo \url{https://orcid.org/0000-0003-0454-7145}
\item
  Raul Palma \url{https://orcid.org/0000-0003-4289-4922}
\item
  Frederik Coppens \url{https://orcid.org/0000-0001-6565-5145}
\item
  Carole Goble \url{https://orcid.org/0000-0003-1219-2137}
\item
  José María Fernández \url{https://orcid.org/0000-0002-4806-5140}
\item
  Kyle Chard \url{https://orcid.org/0000-0002-7370-4805}
\item
  Jose Manuel Gomez-Perez \url{https://orcid.org/0000-0002-5491-6431}
\item
  Michael R Crusoe \url{https://orcid.org/0000-0002-2961-9670}
\item
  Ignacio Eguinoa \url{https://orcid.org/0000-0002-6190-122X}
\item
  Nick Juty \url{https://orcid.org/0000-0002-2036-8350}
\item
  Kristi Holmes \url{https://orcid.org/0000-0001-8420-5254}
\item
  Jason A. Clark \url{https://orcid.org/0000-0002-3588-6257}
\item
  Salvador Capella-Gutierrez \url{https://orcid.org/0000-0002-0309-604X}
\item
  Alasdair J. G. Gray \url{https://orcid.org/0000-0002-5711-4872}
\item
  Stuart Owen \url{https://orcid.org/0000-0003-2130-0865}
\item
  Alan R Williams \url{https://orcid.org/0000-0003-3156-2105}
\item
  Giacomo Tartari \url{https://orcid.org/0000-0003-1130-2154}
\item
  Finn Bacall \url{https://orcid.org/0000-0002-0048-3300}
\item
  Thomas Thelen \url{https://orcid.org/0000-0002-1756-2128}
\item
  Hervé Ménager \url{https://orcid.org/0000-0002-7552-1009}
\item
  Laura Rodríguez-Navas \url{https://orcid.org/0000-0003-4929-1219}
\item
  Paul Walk \url{https://orcid.org/0000-0003-1541-5631}
\item
  brandon whitehead \url{https://orcid.org/0000-0002-0337-8610}
\item
  Mark Wilkinson \url{https://orcid.org/0000-0001-6960-357X}
\item
  Paul Groth \url{https://orcid.org/0000-0003-0183-6910}
\item
  Erich Bremer \url{https://orcid.org/0000-0003-0223-1059}
\item
  LJ Garcia Castro \url{https://orcid.org/0000-0003-3986-0510}
\item
  Karl Sebby \url{https://orcid.org/0000-0001-6022-9825}
\item
  Alexander Kanitz \url{https://orcid.org/0000-0002-3468-0652}
\item
  Ana Trisovic \url{https://orcid.org/0000-0003-1991-0533}
\item
  Gavin Kennedy \url{https://orcid.org/0000-0003-3910-0474}
\item
  Mark Graves \url{https://orcid.org/0000-0003-3486-8193}
\item
  Jasper Koehorst \url{https://orcid.org/0000-0001-8172-8981}
\item
  Simone Leo \url{https://orcid.org/0000-0001-8271-5429}
\item
  Marc Portier \url{https://orcid.org/0000-0002-9648-6484}
\item
  Paul Brack \url{https://orcid.org/0000-0002-5432-2748}
\item
  Milan Ojsteršek \url{https://orcid.org/0000-0003-1743-8300}
\item
  Bert Droesbeke \url{https://orcid.org/0000-0003-0522-5674}
\item
  Chenxu Niu \url{https://github.com/UstcChenxu}
\item
  Kosuke Tanabe \url{https://orcid.org/0000-0002-9986-7223}
\item
  Tomasz Miksa \url{https://orcid.org/0000-0002-4929-7875}
\item
  Marco La Rosa \url{https://orcid.org/0000-0001-5383-6993}
\item
  Cedric Decruw \url{https://orcid.org/0000-0001-6387-5988}
\item
  Andreas Czerniak \url{https://orcid.org/0000-0003-3883-4169}
\item
  Jeremy Jay \url{https://orcid.org/0000-0002-5761-7533}
\item
  Sergio Serra \url{https://orcid.org/0000-0002-0792-8157}
\item
  Ronald Siebes \url{https://orcid.org/0000-0001-8772-7904}
\item
  Shaun de Witt \url{https://orcid.org/0000-0003-4196-3658}
\item
  Shady El Damaty \url{https://orcid.org/0000-0002-2318-4477}
\item
  Douglas Lowe \url{https://orcid.org/0000-0002-1248-3594}
\item
  Sergio Serra \url{https://orcid.org/0000-0002-0792-8157}
\item
  Xuanqi Li \url{https://orcid.org/0000-0003-1498-6205}
\end{itemize}


\section{Acknowledgements Chapter 6}

Chapter \vref{making-canonical-workflow-building-blocks-interoperable-across-workflow-languages} is adapted from journal article published in \emph{Data Intelligence}.

\subsection*{Cite as}

Stian Soiland-Reyes, Genís Bayarri, Pau Andrio, Robin Long, Douglas
Lowe, Ania Niewielska, Adam Hospital, Paul Groth (2022):\\
\textbf{Making Canonical Workflow Building Blocks interoperable across
workflow languages}.\\
\emph{Data Intelligence} \textbf{4}(2)\\
\url{https://doi.org/10.1162/dint_a_00135}

\hypertarget{acknowledgements-3}{%
\subsection*{Acknowledgements}}

This work has been done as part of the BioExcel CoE
(\url{https://www.bioexcel.eu/}), a project funded by the European Union
contracts
\href{https://cordis.europa.eu/project/id/823830}{H2020-INFRAEDI-02-2018
823830},
\href{https://cordis.europa.eu/project/id/675728}{H2020-EINFRA-2015-1
675728}. Additional work is funded through EOSC-Life
(\url{https://www.eosc-life.eu/}) contract
\href{https://cordis.europa.eu/project/id/824087}{H2020-INFRAEOSC-2018-2
824087}, and ELIXIR-CONVERGE (\url{https://elixir-europe.org/}) contract
\href{https://cordis.europa.eu/project/id/871075}{H2020-INFRADEV-2019-2
871075}.

The authors would also like to acknowledge contributions from: Felix
Amaladoss, Cibin Sadasivan Baby, Finn Bacall, Rosa M. Badia, Sarah
Butcher, Gerard Capes, Michael R. Crusoe, Alberto Eusebi, Carole Goble,
Josep Lluís Gelpí, Modesto Orozco, Geoff Williams, Felix Amaladoss


%% Chapter 7

\section{Acknowledgements Chapter 7}

This chapter was adapted from an abstract
presented as poster by Stian Soiland-Reyes at 
First International Conference on FAIR Digital Objects 
(FDO2022) on
2022-08-26/--28 in Leiden, The Netherlands. 
\url{https://www.fdo2022.org/}

\subsection*{Cite As}

Oliver Woolland, Paul Brack, Stian Soiland-Reyes, Ben Scott, Laurence
Livermore (2022):\\
\textbf{Incrementally building FAIR Digital Objects with Specimen Data
Refinery workflows}.\\
1st International Conference on FAIR Digital Objects
(\href{https://www.fdo2022.org/}{FDO 2022}) (poster)\\
\emph{Research Ideas and Outcomes} \textbf{8}:e94349\\
\url{https://doi.org/10.3897/rio.8.e94349}


\subsection*{Acknowledgements}

We acknowledge the \href{https://www.synthesys.info/}{SYNTHESYS+} and
\href{https://www.dissco.eu/}{DiSSCO} project members who have been
invaluable in early evaluation and feedback on the development of SDR.

\hypertarget{funding-2}{%
\subsection*{Funding}\label{funding-2}}

This work has received funding from the European Union's Horizon 2020
research and innovation programme under grant agreement numbers
\href{https://doi.org/10.3030/https://doi.org/10.3030/}{823827}
(SYNTHESYS Plus), \href{https://doi.org/10.3030/871043}{871043} (DiSSCo
Prepare), \href{https://doi.org/10.3030/823830}{823830} (BioExcel-2),
\href{https://doi.org/10.3030/824087}{824087} (EOSC-Life).



%% Chapter 8

\section{Acknowledgements Chapter 8}

Chapter \vref{the-specimen-data-refinery} is adapted from a journal article published in \emph{Data Intelligence}.

\subsection*{Cite As}

Alex Hardisty, Paul Brack, Carole Goble, Laurence Livermore, Ben Scott,
Quentin Groom, Stuart Owen, Stian Soiland-Reyes (2022):\\
\textbf{The Specimen Data Refinery: A canonical workflow framework and
FAIR Digital Object approach to speeding up digital mobilisation of
natural history collections.}\\
\emph{Data Intelligence} \textbf{4}(2)\\
\url{https://doi.org/10.1162/dint_a_00134}


\subsection*{Acknowledgements}

This work has received funding from the European Union's Horizon 2020
research and innovation programme under grant agreement numbers 823827
(SYNTHESYS Plus), 871043 (DiSSCo Prepare), 823830 (BioExcel-2), 824087
(EOSC-Life).


%% Chapter 9



%%%%%%%%%%%%%%%%%% REFERENCES %%%%%%%%%%%%%%%%%%
%%\printbibliography[title={References},heading=bibintoc] % a single list of references for the whole thesis

%% Combined references

\begin{thebibliography}{9}

\bibitem[Afgan 2018]{Afgan 2018}
Enis Afgan, Dannon Baker, Bérénice Batut, Marius van
den Beek, Dave Bouvier, Martin Čech, John Chilton, Dave Clements, Nate
Coraor, Björn A Grüning, Aysam Guerler, Jennifer Hillman-Jackson, Saskia
Hiltemann, Vahid Jalili, Helena Rasche, Nicola Soranzo, Jeremy Goecks,
James Taylor, Anton Nekrutenko, Daniel Blankenberg (2018):\\
\textbf{The Galaxy platform for accessible, reproducible and
collaborative biomedical analyses: 2018 update}.\\
\emph{Nucleic Acids Research} \textbf{46}(W1) W537--W544\\
\url{https://doi.org/10.1093/nar/gky379}

\bibitem[Agarwal 2021]{Agarwal 2021}
Deborah~Agarwal, Carole~Goble, Stian~Soiland-Reyes,
Ugis~Sarkans, Daniel~Noesgaard, Uwe~Schindler, Martin~Fenner,
Paolo~Manghi, Shelley~Stall, Caroline~Coward, Chris~Erdmann (2021):\\
\textbf{Data Citation Community of Practice -- 8 June 2021 Workshop}.\\
\emph{Zenodo/AGU}\\
\url{https://data.agu.org/DataCitationCoP/2nd-workshop-data-citation}\\
\url{https://doi.org/10.5281/zenodo.4916734}

\bibitem[Almeida 2019]{Almeida 2019}
Alexandre Almeida, Alex L. Mitchell, Miguel Boland, Samuel C.
Forster, Gregory B. Gloor, Aleksandra Tarkowska, Trevor D. Lawley,
Robert D. Finn (2019):\\
\textbf{A new genomic blueprint of the human gut microbiota}.\\
\emph{Nature} \textbf{568}(7753) 499--504.\\
\url{https://doi.org/10.1038/s41586-019-0965-1}

\bibitem[Alterovitz 2018]{Alterovitz 2018}
Gil Alterovitz, Dennis A Dean II, Carole Goble, Michael R
Crusoe, Stian Soiland-Reyes, Amanda Bell, Anais Hayes, Anita Suresh,
Charles Hadley S King IV, Dan Taylor, KanakaDurga Addepalli, Elaine
Johanson, Elaine E Thompson, Eric Donaldson, Hiroki Morizono, Hsinyi
Tsang, Jeet K Vora, Jeremy Goecks, Jianchao Yao, Jonas S Almeida,
Jonathon Keeney, KanakaDurga Addepalli, Konstantinos Krampis, Krista
Smith, Lydia Guo, Mark Walderhaug, Marco Schito, Matthew Ezewudo, Nuria
Guimera, Paul Walsh, Robel Kahsay, Srikanth Gottipati, Timothy C
Rodwell, Toby Bloom, Yuching Lai, Vahan Simonyan, Raja Mazumder
(2018):\\
\textbf{Enabling precision medicine via standard communication of HTS
provenance, analysis, and results}.\\
\emph{PLOS Biology} \textbf{16}(12):e3000099\\
\url{https://doi.org/10.1371/journal.pbio.3000099}

\bibitem[Alves 2022]{Alves 2022}
Renato Alves, Dimitrios Bampalikis, Leyla Jael Castro,
José María Fernández, Jennifer Harrow, Mateusz Kuzak, Eva Martin, Fotis
E. Psomopoulos, Allegra Via (2022):\\
\textbf{ELIXIR Software Management Plan for Life Sciences}.\\
\emph{BioHackrXiv}\\
\url{https://doi.org/10.37044/osf.io/k8znb}

\bibitem[Amorim 2016]{Amorim 2016}
Ricardo Carvalho Amorim, João Aguiar Castro, João Rocha da
Silva, Cristina Ribeiro (2016):\\
\textbf{A comparison of research data management platforms:
Architecture, flexible metadata and interoperability}.\\
\emph{Universal Access in the Information Society} \textbf{16} pp
851--862.\\
\url{https://doi.org/10.1007/s10209-016-0475-y}

\bibitem[Arfaoui 2020]{Arfaoui 2020}
Ghaith Arfaoui, Maroua Jaoua (2020):\\
\textbf{RO-Crate RDA maDMP Mapper}.\\
\emph{Zenodo}
\url{https://github.com/GhaithArf/ro-crate-rda-madmp-mapper}\\
\url{https://doi.org/10.5281/zenodo.3922136}

\bibitem[Bacall 2019]{Bacall 2019}
Finn Bacall, Stian~Soiland-Reyes, Marina~Soares e Silva
(2019):\\
\textbf{eScienceLab: RO-Composer}.\\
\url{https://esciencelab.org.uk/projects/ro-composer/}\\
\url{https://github.com/ResearchObject/research-object-composer}

\bibitem[Bacall 2022]{Bacall 2022}
Finn Bacall, Alan R. Williams, Stuart Owen, Stian
Soiland-Reyes (2022):\\
\textbf{Workflow RO-Crate Profile 1.0.}\\
\url{https://w3id.org/workflowhub/workflow-ro-crate/1.0}

\bibitem[Bacall 2022b]{Bacall 2022b}
Finn Bacall, Martyn~Whitwell (2022):\\
\textbf{GitHub -- ResearchObject/ro-crate-ruby: A Ruby gem for creating,
manipulating and reading RO-Crates}.\\
\url{https://github.com/ResearchObject/ro-crate-ruby}

\bibitem[Bayarri 2022]{Bayarri 2022}
Genís Bayarri, Adam Hospital (2022): \textbf{CWL GMX
Automatic Ligand Parameterization tutorial}.\\
\emph{Workflow Hub} (Common Workflow Language)\\
\url{https://doi.org/10.48546/workflowhub.workflow.255.1}

\bibitem[Baker 2020]{Baker 2020}
Dannon Baker, Marius van den Beek, Daniel Blankenberg, Dave
Bouvier, John Chilton, Nate Coraor, Frederik Coppens, Ignacio Eguinoa,
Simon Gladman, Björn Grüning, Nicholas Keener, Delphine Larivière,
Andrew Lonie, Sergei Kosakovsky Pond, Wolfgang Maier, Anton Nekrutenko,
James Taylor, Steven Weaver (2020):\\
\textbf{No more business as usual: Agile and effective responses to
emerging pathogen threats require open data and open analytics}.\\
\emph{PLOS Pathogens} \textbf{16}(8):e1008643.\\
\url{https://doi.org/10.1371/journal.ppat.1008643}

\bibitem[Barker 2019]{Barker 2019}
Michelle Barker, Ross Wilkinson, Andrew Treloar (2019):\\
\textbf{The Australian Research Data Commons}.\\
\emph{Data Science Journal} \textbf{18} (2019).\\
\url{https://doi.org/10.5334/dsj-2019-044}

\bibitem[Bechhofer 2013]{Bechhofer 2013}
Sean Bechhofer, Iain Buchan, David De Roure, Paolo Missier,
John Ainsworth, Jiten Bhagat, Phillip Couch, Don Cruickshank, Mark
Delderfield, Ian Dunlop, Matthew Gamble, Danius Michaelides, Stuart
Owen, David Newman, Shoaib Sufi, Carole Goble (2013):\\
\textbf{Why Linked Data is not enough for scientists}.\\
\emph{Future Generation Computer Systems} \textbf{29}(2),
pp.~599--611.\\
\url{https://doi.org/10.1016/j.future.2011.08.004}

\bibitem[Belchev 2021]{Belchev 2021}
Kostadin~Belchev (2021):\\
\textbf{KockataEPich/CheckMyCrate: A command line application for
validating a RO-Crate object against a JSON profile}.\\
\emph{GitHub}.\\
\url{https://github.com/KockataEPich/CheckMyCrate}

\bibitem[Belhajjame 2015]{Belhajjame 2015}
Khalid Belhajjame, Jun Zhao, Daniel Garijo, Matthew Gamble,
Kristina Hettne, Raul Palma, Eleni Mina, Oscar Corcho, José Manuel
Gómez-Pérez, Sean Bechhofer, Graham Klyne, Carole Goble (2015):\\
\textbf{Using a suite of ontologies for preserving workflow-centric
research objects}.\\
\emph{Web Semantics: Science, Services and Agents on the World Wide Web}
\textbf{32} pp.~16--42.\\
\url{https://doi.org/10.1016/j.websem.2015.01.003}

\bibitem[Benureau 2017]{Benureau 2017}
Fabien C. Y. Benureau, Nicolas P. Rougier (2017):\\
\textbf{Re-run, repeat, reproduce, reuse, replicate: Transforming code
into scientific contributions}.\\
\emph{Frontiers in Neuroinformatics} \textbf{11}:69.\\
\url{https://doi.org/10.3389/fninf.2017.00069}

\bibitem[Berman 2007]{Berman 2007}
Helen Berman, Kim Henrick, Haruki Nakamura, John L Markley
(2007):\\
\textbf{The worldwide Protein Data Bank (wwPDB): Ensuring a single,
uniform archive of PDB data}.\\
\emph{Nucleic Acids Research} \textbf{35}(Database issue) ,
D301--D303.\\
\url{https://doi.org/10.1093/nar/gkl971}

\bibitem[Bietrix 2021]{Bietrix 2021}
Florence Bietrix, José Maria Carazo, Salvador
Capella-Gutierrez, Frederik Coppens, Maria Luisa Chiusano, Romain David,
Jose Maria Fernandez, Maddalena Fratelli, Jean-Karim Heriche, Carole
Goble, Philip Gribbon, Petr Holub, Robbie Joosten, Simone Leo, Stuart
Owen, Helen Parkinson, Roland Pieruschka, Luca Pireddu, Luca Porcu,
Michael Raess, Laura Rodriguez-Navas, Andreas Scherer, Stian
Soiland-Reyes, Jing Tang (2021):\\
\textbf{EOSC-life methodology framework to enhance reproducibility
within EOSC-life}.\\
\emph{Zenodo}\\
\url{https://doi.org/10.5281/zenodo.4705078}

\bibitem[Bizer 2009]{Bizer 2009}
C Bizer, Tom Heath, Tim Berners-Lee (2009):\\
\textbf{Linked Data - The Story So Far}.\\
\emph{International Journal on Semantic Web and Information Systems}
\textbf{5}(3)\\
\url{https://doi.org/10.4018/jswis.2009081901}

\bibitem[Bizer 2011]{Bizer 2011}
Christian Bizer, Tom Heath, Tim Berners-Lee (2011):\\
\textbf{Linked data: The story so far}.\\
In \emph{Semantic Services, Interoperability and Web Applications:
Emerging Concepts}, Amit Sheth (ed.) ISBN 9781609605933\\
\url{https://doi.org/10.4018/978-1-60960-593-3.ch008}

\bibitem[Bonino 2019]{bonino2019}
Luiz Bonino, Peter Wittenburg, Bonnie Carroll, Alex
Hardisty, Mark Leggott, Carlo Zwölf (2019):\\
\textbf{FAIR digital object framework v1.02. \emph{FDOF technical
implementation guideline}}\\
\url{https://github.com/GEDE-RDA-Europe/GEDE/blob/master/FAIR\%20Digital\%20Objects/FDOF/FAIR\%20Digital\%20Object\%20Framework-v1-02.docx}

\bibitem[Bonino 2021]{bonino2021}
Luiz Olavo Bonino da Silva Santos (2021):\\
\textbf{FAIR Digital Object Framework Documentation}.\\
\emph{Working Draft}\\
\url{https://fairdigitalobjectframework.org/}

\bibitem[Brack 2022]{Brack 2022}
Paul Brack, Oliver Woolland, Laurence Livermore
(2022):\\
\textbf{De novo digitisation.} (Galaxy workflow) \emph{WorkflowHub}\\
\url{https://doi.org/10.48546/workflowhub.workflow.373.1}

\bibitem[Brand 2015]{Brand 2015}
Amy Brand, Liz Allen, Micah Altman, Marjorie Hlava, Jo Scott
(2015):\\
\textbf{Beyond authorship: Attribution, contribution, collaboration, and
credit}.\\
\emph{Learned Publishing} \textbf{28}(2) pp.~151--155.\\
\url{https://doi.org/10.1087/20150211}

\bibitem[Brenner 2020]{Brenner 2020}
Gabriel~Brenner (2020):\\
\textbf{BrennerG/Ro-Crate\_2\_ma-DMP: v1.0.0}.\\
\url{https://github.com/BrennerG/Ro-Crate_2_ma-DMP}\\
\url{https://doi.org/10.5281/zenodo.3903463}

\bibitem[Cardoso 2020a]{Cardoso 2020a}
João Cardoso, Diogo Proença, José Borbinha (2020):\\
\textbf{Machine-actionable data management plans: A knowledge retrieval
approach to automate the assessment of funders' requirements}.\\
\emph{ECIR 2020: Advances in Information Retrieval}\\
ISBN 978-3-030-45442-5.\\
\url{https://doi.org/10.1007/978-3-030-45442-5_15}

\bibitem[Cardoso 2020b]{Cardoso 2020b}
João Cardoso, Leyla Jael Garcia Castro, Fajar Ekaputra, Marie-Christine Jacquemot-Perbal, Tomasz Miksa, José Borbinha (2020):\\
\textbf{Towards Semantic Representation of Machine-Actionable Data
Management Plans}.\\
\emph{PUBLISSO}.\\
\url{https://repository.publisso.de/resource/frl:6423289}\\
\url{https://doi.org/10.4126/frl01-006423289}

\bibitem[Chard 2016]{Chard 2016}
Kyle Chard, Mike D' Arcy, Ben Heavner, Ian Foster, Carl
Kesselman, Ravi Madduri, Alexis Rodriguez, Stian Soiland-Reyes, Carole
Goble, Kristi Clark, Eric W. Deutsch, Ivo Dinov, Nathan Price, Arthur
Toga (2016):\\
\textbf{I'll take that to go: Big data bags and minimal identifiers for
exchange of large, complex datasets}.\\
\emph{2016 IEEE International Conference on Big Data (Big Data)}, IEEE,
pp.~319--328.\\
ISBN 978-1-4673-9005-7.\\
\url{https://static.aminer.org/pdf/fa/bigdata2016/BigD418.pdf}\\
\url{https://doi.org/10.1109/BigData.2016.7840618}

\bibitem[Chard 2019]{Chard 2019}
Kyle Chard, Niall Gaffney, Matthew B. Jones, Kacper Kowalik,
Bertram Ludascher, Timothy McPhillips, Jarek Nabrzyski, Victoria
Stodden, Ian Taylor, Thomas Thelen, Matthew J. Turk, Craig Willis
(2019):\\
\textbf{Application of BagIt-serialized research object bundles for
packaging and re-execution of computational analyses}.\\
\emph{15th International Conference on eScience (eScience 2019)}, IEEE,
pp.~514--521.\\
ISBN 978-1-7281-2451-3.\\
\url{https://zenodo.org/record/3381754}\\
\url{https://doi.org/10.1109/eScience.2019.00068}

\bibitem[Chard 2014]{Chard 2014}
Kyle~Chard, Steven~Tuecke and Ian~Foster (2014):\\
\textbf{Efficient and secure transfer, synchronization, and sharing of
big data}.\\
\emph{IEEE Cloud Computing} \textbf{1}(3) pp.~46--55.\\
\url{https://doi.org/10.1109/MCC.2014.52}

\bibitem[Ciccarese 2017]{Ciccarese 2017}
Paolo Ciccarese, Robert Sanderson, Benjamin Young (2017):\\
\textbf{Web Annotation Data Model}.\\
\emph{W3C Recommendation} 23 February 2017.
\url{https://www.w3.org/TR/2017/REC-annotation-model-20170223/}

\bibitem[Claerbout 1992]{Claerbout 1992}
Jon F. Claerbout, Martin Karrenbach (1992):\\
\textbf{Electronic documents give reproducible research a new
meaning}.\\
\emph{SEG Technical Program Expanded Abstracts 1992}, Society of
Exploration Geophysicists, pp.~601--604.\\
\url{http://sep.stanford.edu/oldsep/matt/join/redoc/web/seg92.html}\\
\url{https://doi.org/10.1190/1.1822162}

\bibitem[Cohen-Boulakia 2017]{Cohen-Boulakia 2017}
Sarah Cohen-Boulakia, Khalid Belhajjame, Olivier Collin, Jérôme
Chopard, Christine Froidevaux, Alban Gaignard, Konrad Hinsen, Pierre
Larmande, Yvan Le Bras, Frédéric Lemoine, Fabien Mareuil, Hervé Ménager,
Christophe Pradal, Christophe Blanchet (2017):\\
\textbf{Scientific workflows for computational reproducibility in the
life sciences: Status, challenges and opportunities}.\\
\emph{Future Generation Computer Systems} \textbf{75} pp.~284--298.\\
\url{https://hal.archives-ouvertes.fr/hal-01516082}\\
\url{https://doi.org/10.1016/j.future.2017.01.012}

\bibitem[Cossu 2018]{Cossu 2018}
Stefano Cossu, Esmé Cowles, Karen Estlund, Christina Harlow,
Tom Johnson, Mark Matienzo, Danny Lamb, Lynette Rayle, Rob Sanderson,
Jon Stroop, Andrew Woods (2018):\\
\textbf{Portland Common Data Model}.\\
\emph{GitHub duraspace/pcdm Wiki} (2018-06-15)
\url{https://github.com/duraspace/pcdm/wiki}

\bibitem[Crosas 2011]{Crosas 2011}
Mercè~Crosas (2011):\\
\textbf{The DataVerse Network: An open-source application for sharing,
discovering and preserving data}.\\
\emph{D-Lib Magazine} \textbf{17}(1/2)a\\
\url{https://doi.org/10.1045/january2011-crosas}

\bibitem[Crosas 2020]{Crosas 2020}
Mercè~Crosas (2020):\\
\textbf{Harvard Data Commons}.\\
\emph{European Dataverse Workshop 2020}, Tromsø, Norway. ISSN
2387-3086.\\
\url{https://doi.org/10.7557/5.5422}

\bibitem[Crosswell 2012]{Crosswell 2012}
Lindsey C Crosswell, Janet M Thornton (2012):\\
\textbf{ELIXIR: A distributed infrastructure for European biological
data}.\\
\emph{Trends in Biotechnology} \textbf{30}(5) pp.~241--242.\\
\url{https://doi.org/10.1016/j.tibtech.2012.02.002}

\bibitem[CRS4 2022]{CRS4 2022}
CRS4 (2022):\\
\textbf{LifeMonitor, a testing and monitoring service for scientific
workflows}.\\
\url{https://about.lifemonitor.eu/}

\bibitem[Crusoe 2022]{Crusoe 2022}
Michael R. Crusoe, Sanne Abeln, Alexandru Iosup, Peter
Amstutz, John Chilton, Nebojša Tijanić, Hervé Ménager, Stian
Soiland-Reyes, Bogdan Gavrilović, Carole Goble, The CWL Community
(2022):\\
\textbf{Methods Included: Standardizing
Computational Reuse and Portability with the Common Workflow
Language}.\\
\emph{Communications of the ACM} \textbf{65}(6)\\
\url{https://arxiv.org/abs/2105.07028}\\
\url{https://doi.org/10.1145/3486897}

\bibitem[DCAT2 2020]{DCAT2 2020}
Riccardo~Albertoni, David~Browning, Simon~Cox,
Alejandra~Gonzalez Beltran, Andrea~Perego, Peter~Winstanley, Dataset
Exchange Working Group (2020):\\
\textbf{Data Catalog Vocabulary (DCAT) -- Version 2}.\\
\emph{W3C Recommendation} (2020)\\
\url{https://www.w3.org/TR/2020/REC-vocab-dcat-2-20200204/}


\bibitem[De Geest 2022]{De Geest 2022}
Paul De Geest, Frederik Coppens, Stian
Soiland-Reyes, Ignacio Eguinoa, Simone Leo (2022):\\
\textbf{Enhancing RDM in Galaxy by integrating RO-Crate}.\\
1st International Conference on FAIR Digital Objects
(\href{https://www.fdo2022.org/}{FDO 2022}) (poster)\\
\emph{Research Ideas and Outcomes} (accepted)

\bibitem[de Mello 2022]{de Mello 2022}
Blanda Helena de Mello, Sandro José Rigo, Cristiano
André da Costa, Rodrigo da Rosa Righi, Bruna Donida, Marta Rosecler Bez,
Luana Carina Schunke (2022):\\
\textbf{Semantic interoperability in health records standards: a
systematic literature review}.\\
\emph{Health and Technology} \textbf{12}\\
\url{https://doi.org/10.1007/s12553-022-00639-w}

\bibitem[DONA 2018]{DONA 2018}
DONA Foundation (2018):\\
\textbf{Digital Object Interface Protocol specification, version 2.0}.\\
\url{https://hdl.handle.net/0.DOIP/DOIPV2.0}

\bibitem[De Smedt 2020]{De Smedt 2020}
Koenraad De Smedt, Dimitris Koureas, Peter
Wittenburg (2020):\\
\textbf{FAIR Digital Objects for Science: From Data Pieces to Actionable
Knowledge Units}.\\
\emph{Publications)} \textbf{8}(2):21\\
\url{https://doi.org/10.3390/publications8020021}


\bibitem[Dillen 2019]{Dillen 2019}
Mathias Dillen, Quentin Groom, Donat Agosti, Lars Nielsen
(2019):\\
\textbf{Zenodo, an archive and publishing repository: A tale of two
herbarium specimen pilot projects}.\\
\emph{Biodiversity Information Science and Standards} \textbf{3}:e37080
(2019).\\
\url{https://doi.org/10.3897/biss.3.37080}

\bibitem[Droesbeke 2022]{Droesbeke 2022}
Bert Droesbeke, Ignacio Eguinoa, Alban Gaignard, Leo Simone,
Luca Pireddu, Laura Rodríguez-Navas, Stian Soiland-Reyes (2022):\\
\textbf{GitHub -- ResearchObject/ro-crate-py: Python library for
RO-Crate}.\\
\url{https://github.com/researchobject/ro-crate-py}\\
\url{https://doi.org/10.5281/zenodo.3956493}

\bibitem[Dürst 2005]{Duerst 2005}
M. Duerst and M.~Suignard (2005):\\
\textbf{Internationalized resource identifiers (IRIs)}.\\
RFC 3987, \emph{Internet Requests for Comments}, RFC Editor, (2005).\\
\url{https://doi.org/10.17487/rfc3987}

\bibitem[EMBL-EBI 2019]{EMBL-EBI 2019}
EMBL-EBI Microbiome Informatics Team (2019):\\
\textbf{FTP index of /pub/databases/metagenomics/umgs\_analyses/}.\\
\url{http://ftp.ebi.ac.uk/pub/databases/metagenomics/umgs_analyses/}

\bibitem[EMBL-EBI 2020]{EMBL-EBI 2020}
EMBL-EBI Microbiome Informatics Team (2020):\\
\textbf{GitHub -- Finn-Lab/MGS-gut: Analysing Metagenomic Species
(MGS)}.\\
\url{https://github.com/Finn-Lab/MGS-gut}

\bibitem[Ewels 2020]{Ewels 2020}
Philip A Ewels, Alexander Peltzer, Sven Fillinger, Harshil
Patel, Johannes Alneberg, Andreas Wilm, Maxime Ulysse Garcia, Paolo Di
Tommaso, Sven Nahnsen (2020):\\
\textbf{The nf-core framework for community-curated bioinformatics
pipelines}.\\
\emph{Nature Biotechnology} \textbf{38}(3), 276--278.\\
\url{https://doi.org/10.1038/s41587-020-0439-x}

\bibitem[Farnel 2014]{Farnel 2014}
Sharon Farnel, Ali Shiri (2014):\\
\textbf{Metadata for research data: Current practices and trends}.\\
\emph{2014 Proceedings of the International Conference on Dublin Core
and Metadata Applications}, ISSN 1939-1366.\\
\url{https://dcpapers.dublincore.org/pubs/article/view/3714}.

\bibitem[Galaxy 2022]{Galaxy 2022}
The Galaxy Community (E. Afgan, A. Nekrutenko, B. A.
Grüning, D. Blankenberg, J. Goecks, M. C. Schatz, A. E. Ostrovsky, A.
Mahmoud, A. J. Lonie, A. Syme, A. Fouilloux, A. Bretaudeau, A.
Nekrutenko, A. Kumar, A. C. Eschenlauer, A. D. DeSanto, A. Guerler, B.
Serrano-Solano, B. Batut, B. A. Grüning, B. W. Langhorst, B. Carr, B. A.
Raubenolt, C. J. Hyde, C. J. Bromhead, C. B. Barnett, C. Royaux, C.
Gallardo, D. Blankenberg, D. J. Fornika, D. Baker, D. Bouvier, D.
Clements, D. A. de Lima Morais, D. L. Tabernero, D. Lariviere, E. Nasr,
E. Afgan, F. Zambelli, F. Heyl, F. Psomopoulos, F. Coppens, G. R. Price,
G. Cuccuru, G. L. Corguillé, G. Von Kuster, G. G. Akbulut, H. Rasche, H.
Hans-Rudolf, I. Eguinoa, I. Makunin, I. J. Ranawaka, J. P. Taylor, J.
Joshi, J. Hillman-Jackson, J. Goecks, J. M. Chilton, K. Kamali, K.
Suderman, K. Poterlowicz, L. B. Yvan, L. Lopez-Delisle, L. Sargent, M.
E. Bassetti, M. A. Tangaro, M. van den Beek, M. Čech, M. Bernt, M.
Fahrner, M. Tekman, M. C. Föll, M. C. Schatz, M. R. Crusoe, M.
Roncoroni, N. Kucher, N. Coraor, N. Stoler, N. Rhodes, N. Soranzo, N.
Pinter, N. A. Goonasekera, P. A. Moreno, P. Videm, P. Melanie, P.
Mandreoli, P. D. Jagtap, Q. Gu, R. J. M. Weber, R. Lazarus, R. H. P.
Vorderman, S. Hiltemann, S. Golitsynskiy, S. Garg, S. A. Bray, S. L.
Gladman, S. Leo, S. P. Mehta, T. J. Griffin, V. Jalili, V. Yves, V. Wen,
V. K. Nagampalli, W. A. Bacon, W. de Koning, W. Maier, P. J. Briggs)
(2022):\\
\textbf{The Galaxy platform for accessible, reproducible and
collaborative biomedical analyses: 2022 update}.\\
\emph{Nucleic Acids Research} \textbf{50}\\
\url{https://doi.org/10.1093/nar/gkac247}

\bibitem[Goble 2021]{Goble 2021}
Carole Goble, Stian Soiland-Reyes, Finn Bacall, Stuart
Owen, Alan Williams, Ignacio Eguinoa, Bert Droesbeke, Simone Leo, Luca
Pireddu, Laura Rodríguez-Navas, José Mª Fernández, Salvador
Capella-Gutierrez, Hervé Ménager, Björn Grüning, Beatriz Serrano-Solano,
Philip Ewels, Frederik Coppens (2021):\\
\textbf{Implementing FAIR Digital Objects in the EOSC-Life Workflow
Collaboratory}.\\
\emph{Zenodo}\\
\url{https://doi.org/10.5281/zenodo.4605654}

\bibitem[Guha 2016]{Guha 2016}
R. V. Guha, Dan Brickley, Steve Macbeth (2016):\\
\textbf{Schema.org: evolution of structured data on the web}.\\
\emph{Communications of the ACM} \textbf{59}(2)\\
\url{https://doi.org/10.1145/2844544}

\bibitem[Hardisty 2019]{Hardisty 2019}
Alex R Hardisty, Keping Ma, Gil Nelson, Jose Fortes
(2019):\\
\textbf{`openDS' -- A New Standard for Digital Specimens and Other
Natural Science Digital Object Types}.\\
\emph{Biodiversity Information Science and Standards}
\textbf{3}:e37033\\
\url{https://doi.org/10.3897/biss.3.37033}

\bibitem[Hardisty 2022]{Hardisty 2022}
Alex Hardisty, Paul Brack, Carole Goble, Laurence
Livermore, Ben Scott, Quentin Groom, Stuart Owen, Stian Soiland-Reyes
(2022):\\
\href{../specimen-data-refinery/}{\textbf{The Specimen Data Refinery: A
canonical workflow framework and FAIR Digital Object approach to
speeding up digital mobilisation of natural history collections}}.\\
\emph{Data Intelligence} \textbf{4}(2)\\
\url{https://doi.org/10.1162/dint_a_00134}

\bibitem[Hasnain 2018]{Hasnain 2018}
Ali Hasnain, Dietrich Rebholz-Schuhmann (2019):\\
\textbf{Assessing FAIR Data Principles Against the 5-Star Open Data
Principles}.\\
ESWC 2018: The Semantic Web: ESWC 2018 Satellite Events,\\
\emph{Lecture Notes in Computer Science} \textbf{11155}\\
\url{https://doi.org/10.1007/978-3-319-98192-5_60}

\bibitem[Jacobsen 2020]{Jacobsen 2020}
Annika Jacobsen, Ricardo de Miranda Azevedo, Nick
Juty, Dominique Batista, Simon Coles, Ronald Cornet, Mélanie Courtot,
Mercè Crosas, Michel Dumontier, Chris T. Evelo, Carole Goble, Giancarlo
Guizzardi, Karsten Kryger Hansen, Ali Hasnain, Kristina Hettne, Jaap
Heringa, Rob W.W. Hooft, Melanie Imming, Keith G. Jeffery, Rajaram
Kaliyaperumal, Martijn G. Kersloot, Christine R. Kirkpatrick, Tobias
Kuhn, Ignasi Labastida, Barbara Magagna, Peter McQuilton, Natalie
Meyers, Annalisa Montesanti, Mirjam van Reisen, Philippe Rocca-Serra,
Robert Pergl, Susanna-Assunta Sansone, Luiz Olavo Bonino da Silva
Santos, Juliane Schneider, George Strawn, Mark Thompson, Andra
Waagmeester, Tobias Weigel, Mark D. Wilkinson, Egon L. Willighagen,
Peter Wittenburg, Marco Roos, Barend Mons, Erik Schultes (2020):\\
\textbf{FAIR Principles: Interpretations and Implementation
Considerations}.\\
\emph{Data Intelligence} \textbf{2}(1):10--29\\
\url{https://doi.org/10.1162/dint_r_00024}

\bibitem[Kuhn 2021]{Kuhn 2021}
Tobias Kuhn, Vincent Emonet, Haris Antonatos, Stian
Soiland-Reyes, Michel Dumontier (2021):\\
\href{../../../2021/phd/nanopub/}{\textbf{Semantic micro-contributions
with decentralized nanopublication services}}.\\
\emph{PeerJ Computer Science} \textbf{7}:e387\\
\url{https://doi.org/10.7717/peerj-cs.387}

\bibitem[Livermore 2022a]{Livermore 2022a}
Laurence Livermore, Oliver Woolland (2022):\\
\textbf{DLA-Collections-test.} (Galaxy workflow)\\
\emph{WorkflowHub}
\url{https://doi.org/10.48546/workflowhub.workflow.374.1}

\bibitem[Livermore 2022b]{Livermore 2022b}
Laurence Livermore, Oliver Woolland (2022):\\
\textbf{HTR-Collections-test}. (Galaxy workflow)\\
\emph{WorkflowHub}\\
\url{https://doi.org/10.48546/workflowhub.workflow.375.1}

\bibitem[Mai Chan 1995]{Mai Chan 1995}
Lois Mai~Chan (1995):\\
\textbf{Library of Congress Subject Headings: Principles and
Application}, 3rd edn, p.~556.\\
\href{https://identifiers.org/isbn/9781563081910}{ISBN 9781563081910}.

\bibitem[McMurry 2017]{McMurry 2017}
Julie A McMurry, Nick Juty, Niklas Blomberg, Tony
Burdett, Tom Conlin, Nathalie Conte, Mélanie Courtot, John Deck, Michel
Dumontier, Donal K Fellows, Alejandra Gonzalez-Beltran, Philipp
Gormanns, Jeffrey Grethe, Janna Hastings, Jean-Karim Hériché, Henning
Hermjakob, Jon C Ison, Rafael C Jimenez, Simon Jupp, John Kunze, Camille
Laibe, Nicolas Le Novère, James Malone, Maria Jesus Martin, Johanna R
McEntyre, Chris Morris, Juha Muilu, Wolfgang Müller, Philippe
Rocca-Serra, Susanna-Assunta Sansone, Murat Sariyar, Jacky L Snoep,
Stian Soiland-Reyes, Natalie J Stanford, Neil Swainston, Nicole
Washington, Alan R Williams, Sarala M Wimalaratne, Lilly M Winfree,
Katherine Wolstencroft, Carole Goble, Cristopher J Mungall, Melissa A
Haendel, Helen Parkinson (2017):\\
\textbf{Identifiers for the 21st century: How to design, provision, and
reuse identifiers to maximize utility and impact of life science
data.}\\
\emph{PLOS Biology} \textbf{15}(6):e2001414
\url{https://doi.org/10.1371/journal.pbio.2001414}

\bibitem[Miksa 2020]{Miksa 2020}
Tomasz Miksa, Maroua Jaoua, Ghaith Arfaoui (2020):\\
\textbf{Research Object Crates and Machine-actionable Data Management
Plans}.\\
\emph{First Workshop on Data and Research Objects Management for Linked
Open Science (DaMaLOS)} at The 19th International Semantic Web
Conference (ISWC 2020).\\
\url{https://doi.org/10.4126/frl01-006423291}

\bibitem[Mons 2017]{Mons 2017}
Barend Mons, Cameron Neylon, Jan Velterop, Michel
Dumontier, Luiz Olavo Bonino da Silva Santos, Mark D. Wilkinson
(2017):\\
\textbf{Cloudy, increasingly FAIR; revisiting the FAIR Data guiding
principles for the European Open Science Cloud}.\\
\emph{Information Services \& Use} \textbf{37}(1)\\
\url{https://doi.org/10.3233/ISU-170824}

\bibitem[Ó Carragáin 2019]{Ó Carragáin 2019}
Eoghan~Ó~Carragáin, Carole~Goble, Peter Sefton,
Stian~Soiland-Reyes (2019):\\
\textbf{A lightweight approach to research object data packaging}.\\
\emph{Bioinformatics Open Source Conference (BOSC2019)},
2019-07-24/2019-07-25, Basel, Switzerland.\\
\emph{Zenodo}.\\
\url{https://doi.org/10.5281/zenodo.3250687}

\bibitem[Pergl 2019]{Pergl 2019}
Robert Pergl, Rob Hooft, Marek Suchánek, Vojtěch
Knaisl, Jan Slifka (2019):\\
\textbf{``Data Stewardship Wizard'': A Tool Bringing Together
Researchers, Data Stewards, and Data Experts around Data Management
Planning}.\\
\emph{Data Science Journal} \textbf{18}(1)\\
\url{https://doi.org/10.5334/dsj-2019-059}

\bibitem[RDF 1.1 2014]{RDF 1.1 2014}
RDF Working Group (2014):\\
\textbf{RDF 1.1 Concepts and Abstract Syntax}.\\
\emph{W3C Recommendation} 25 Feb 2014.
\url{https://www.w3.org/TR/2014/REC-rdf11-concepts-20140225/}.

\bibitem[Schröder 2022]{Schröder 2022}
Max Schröder, Susanne Staehlke, Paul Groth, J.
Barbara Nebe, Sascha Spors, Frank Krüger (2022):\\
\textbf{Structure-based knowledge acquisition from electronic lab
notebooks for research data provenance documentation}.\\
\emph{Journal of Biomedical Semantics} \textbf{13}:4\\
\url{https://doi.org/10.1186/s13326-021-00257-x}

\bibitem[Schultes 2019]{Schultes 2019}
Erik Schultes, Peter Wittenburg (2019):\\
\textbf{FAIR principles and digital objects: Accelerating convergence on
a data infrastructure}.\\
\emph{Data analytics and management in data intensive domains: 20th
international conference} (DAMDID/RCDL 2018)\\
\url{https://doi.org/10.1007/978-3-030-23584-0_1}

\bibitem[Sefton 2021]{Sefton 2021}
Peter Sefton (2021):\\
\textbf{FAIR Data Management; It's a lifestyle not a lifecycle}.\\
\emph{ptsefton.com}. \url{http://ptsefton.com/2021/04/07/rdmpic/}

\bibitem[Soiland-Reyes 2022]{Soiland-Reyes 2022}
Stian Soiland-Reyes, Peter Sefton, Mercè
Crosas, Leyla Jael Castro, Frederik Coppens, José M. Fernández, Daniel
Garijo, Björn Grüning, Marco La Rosa, Simone Leo, Eoghan Ó Carragáin,
Marc Portier, Ana Trisovic, RO-Crate Community, Paul Groth, Carole Goble
(2022):\\
\textbf{Packaging research artefacts with RO-Crate}.\\
\emph{Data Science} \textbf{5}(2)\\
\url{https://doi.org/10.3233/DS-210053}

\bibitem[Soiland-Reyes 2022a]{Soiland-Reyes 2022a}
Stian Soiland-Reyes, Genís Bayarri, Pau
Andrio, Robin Long, Douglas Lowe, Ania Niewielska, Adam Hospital, Paul
Groth (2022):\\
\href{../canonical-workflow-building-blocks/}{\textbf{Making Canonical
Workflow Building Blocks interoperable across workflow languages}}.\\
\emph{Data Intelligence} \textbf{4}(2)\\
\url{https://doi.org/10.1162/dint_a_00135}

\bibitem[Van de Sompel 2022]{Van de Sompel 2022}
Herbert Van de Sompel, Martin Klein, Shawn
Jones, Michael L. Nelson, Simeon Warner, Anusuriya Devaraju, Robert
Huber, Wilko Steinhoff, Vyacheslav Tykhonov, Luc Boruta, Enno Meijers,
Stian Soiland-Reyes, Mark Wilkinson (2022):\\
\textbf{FAIR Signposting Profile}. (version 20220727).\\
\url{https://signposting.org/FAIR/}

\bibitem[Walton 2020]{Walton 2020}
Stephanie Walton, Laurence Livermore, Olaf Bánki,
Robert Cubey, Robyn Drinkwater, Markus Englund, Carole Goble, Quentin
Groom, Christopher Kermorvant, Isabel Rey, Celia Santos, Ben Scott, Alan
Williams, Zhengzhe Wu (2020):\\
\textbf{Landscape Analysis for the Specimen Data Refinery}.\\
\emph{Research Ideas and Outcomes} \textbf{6}:e57602\\
\url{https://doi.org/10.3897/rio.6.e57602}

\bibitem[Wilkinson 2016]{Wilkinson 2016}
Mark D. Wilkinson, Michel Dumontier, IJsbrand Jan
Aalbersberg, Gabrielle Appleton, Myles Axton, Arie Baak, Niklas
Blomberg, Jan-Willem Boiten, Luiz Bonino da Silva Santos, Philip E.
Bourne, Jildau Bouwman, Anthony J. Brookes, Tim Clark, Mercè Crosas,
Ingrid Dillo, Olivier Dumon, Scott Edmunds, Chris T. Evelo, Richard
Finkers, Alejandra Gonzalez-Beltran, Alasdair J.G. Gray, Paul Groth,
Carole Goble, Jeffrey S. Grethe, Jaap Heringa, Peter A.C 't Hoen, Rob
Hooft, Tobias Kuhn, Ruben Kok, Joost Kok, Scott J. Lusher, Maryann E.
Martone, Albert Mons, Abel L. Packer, Bengt Persson, Philippe
Rocca-Serra, Marco Roos, Rene van Schaik, Susanna-Assunta Sansone, Erik
Schultes, Thierry Sengstag, Ted Slater, George Strawn, Morris A. Swertz,
Mark Thompson, Johan van der Lei, Erik van Mulligen, Jan Velterop, Andra
Waagmeester, Peter Wittenburg, Katherine Wolstencroft, Jun Zhao, Barend
Mons (2016):\\
\textbf{The FAIR Guiding Principles for scientific data management and
stewardship}.\\
\emph{Nature Scientific Data} \textbf{3}:160018\\
\url{https://doi.org/10.1038/sdata.2016.18}

%%%%
%%% Below is NOT MERGED
%%% 

%% Chapter 02 references
\section{References from Chapter 2.2}

\bibitem{InfoURIRegistry}
\emph{”info” URI Registry (Frozen)} (2023).  \url
{https://oclc-research.github.io/infoURI-Frozen/} (visited on 2023-01-24).
{}
\bibitem{allcockGlobusStripedGridFTP}
Allcock, W. et al. (n.d.). “The ––Globus Striped GridFTP Framework ̋ ̋ and ––Server ̋ ̋”. In:
\emph{––ACM ̋ ̋/––IEEE SC ̋ ̋ 2005 ––Conference ̋ ̋ (––SC ̋ ̋’05)}. IEEE. \url
{https://doi.org/10.1109/sc.2005.72}.
{}
\bibitem{fdo-Overview}
Anders, I. et al. (2022-11-19). \emph{––FAIR ̋ ̋ Digital Object Specifications}. Proposed
Recommendation Full FDO SpecDoc-PR-1.2. HDL: \href
{http://hdl.handle.net/20.500.14132/fdo-spec-docs} {\nolinkurl {20.500.14132/fdo-spec-docs}}.
 \url {https://drive.google.com/file/d/1NIrPJPv1T0WvExvF83ac2UD6fWqQ3q41}
(visited on 2023-02-02).
{}
\bibitem{fdo-PIDProfileAttributes}
Anders, Ivonne et al. (2022-10-17). \emph{––FDO PID ̋ ̋ Profiles “\& Attributes}. Proposed
Recommendation PR-PIDProfileAttributes-2.1-20221017. HDL: \href
{http://hdl.handle.net/20.500.14132/fdo-spec-docs} {\nolinkurl {20.500.14132/fdo-spec-docs}}.
 \url
{https://drive.google.com/file/d/12HSPXUSqtXtM9AiXMJbTDLCQbF5H0Uoy} (visited on
2022-11-30).
{}
\bibitem{bahimFAIRDataMaturity2020a}
Bahim, Christophe et al. (2020-10-27). “The ––FAIR ̋ ̋ Data Maturity Model: ––An ̋ ̋ Approach to
Harmonise ––FAIR ̋ ̋ Assessments”. In: \emph{Data Science Journal} 19. \textsc{issn}:
1683-1470. \url {https://doi.org/10.5334/dsj-2020-041}.
{}
\bibitem{rfc7540}
Belshe, M., R. Peon, and M. Thomson (2015-05). \emph{Hypertext ––Transfer Protocol Version ̋ ̋
2 (––HTTP ̋ ̋/2)}. \url {https://doi.org/10.17487/rfc7540}.
{}
\bibitem{rfc3986}
Berners-Lee, T., R. Fielding, and L. Masinter (2005-01). \emph{Uniform ––Resource Identifier ̋ ̋
(––URI ̋ ̋): ––Generic Syntax ̋ ̋}. \url {https://doi.org/10.17487/rfc3986}.
{}
\bibitem{berners-lee-cool-uris}
Berners-Lee, Tim (1998). \emph{Cool ––URIs ̋ ̋ don’t change}.  \url
{https://www.w3.org/Provider/Style/URI} (visited on 2023-02-02).
{}
\bibitem{LinkedDataDesign}
— (2006-07-27). \emph{Linked ––Data ̋ ̋ - ––Design Issues ̋ ̋}. W3C Design Issues.
 \url {https://www.w3.org/DesignIssues/LinkedData.html} (visited on 2022-05-26).
{}
\bibitem{SemanticWebXML2000}
— (2023). \emph{Semantic ––Web ̋ ̋ - ––XML2000 ̋ ̋ - Slide ”––Architecture ̋ ̋”}. 
\url {https://www.w3.org/2000/Talks/1206-xml2k-tbl/slide10-0.html} (visited on 2023-01-24).
{}
\bibitem{berners-leeWeavingWebOriginal1999}
Berners-Lee, Tim and Mark Fischetti (1999). \emph{Weaving the ––Web ̋ ̋: The Original Design
and Ultimate Destiny of the ––World Wide Web ̋ ̋ by Its Inventor}. 1st ed. San Francisco:
HarperSanFrancisco. 226 pp. \textsc{isbn}: 978-0-06-251586-5.
{}
\bibitem{bernsteinNewLookSemantic2016a}
Bernstein, Abraham, James Hendler, and Natalya Noy (2016-08-24). “A New Look at the Semantic
Web”. In: \emph{Communications of the ACM} 59.9, pp. 35–37. \textsc{issn}: 00010782. \url
{https://doi.org/10.1145/2890489}.
{}
\bibitem{rfc9114}
\emph{HTTP/3} (2022-06). Tech. rep. \url {https://doi.org/10.17487/rfc9114}.
{}
\bibitem{bizerLinkedDataStory2009a}
Bizer, Christian, Tom Heath, and Tim Berners-Lee (2009-07). “Linked Data - the Story so Far”. In:
\emph{International journal on Semantic Web and information systems} 5.3, pp. 1–22.
\textsc{issn}: 1552-6283. \url {https://doi.org/10.4018/jswis.2009081901}.
{}
\bibitem{fdo-ImplAttributesTypesProfiles}
Blanchi, C. et al. (2022-11-27). \emph{Implementation of Attributes, Types, Profiles and
Registries}. Working Draft WD-ImplAttributesTypesProfiles-0.2-20221127.  \url
{https://drive.google.com/file/d/1RrOiwMhkl-hRzWmlluA2iXCzHK-bj7 ̇80LlMXgWx4w4}
(visited on 2022-11-30).
{}
\bibitem{fdo-FDO-Upload}
Blanchi, Christophe et al. (2022-10-17). \emph{––FDO ̋ ̋ – Upload of ––FDO ̋ ̋}. Proposed
Endorsement Note PEN-FDO-Upload-1.1-20221017. HDL: \href
{http://hdl.handle.net/20.500.14132/fdo-spec-docs} {\nolinkurl {20.500.14132/fdo-spec-docs}}.
 \url {https://drive.google.com/file/d/1b7Ulpo6w1oqTInZ1j-DJ8HghCCzc2zk7}
(visited on 2023-02-02).
{}
\bibitem{boninoFAIRDigitalObject}
Bonino, Luiz et al. (2019-11-22). \emph{––FAIR ̋ ̋ Digital Object Framework}. FDOF Technical
Implementation Guideline.  \url
{https://github.com/GEDE-RDA-Europe/GEDE/blob/master/FAIR\%20Digital\%20Objects/FDOF/FAIR\%20Digita}
%%% TIXMS
{}
\bibitem{FDOFramework}
Bonino da Silva Santos, Luiz Olavo (2022-10-27). \emph{––FAIR Digital Object Framework
Documentation ̋ ̋}. Ed. by Luiz Olavo Bonino da Silva Santos.  \url
{https://fairdigitalobjectframework.org/} (visited on 2022-05-26).
{}
\bibitem{boninodasilvasantosFAIRDataPoints2016a}
Bonino Da Silva Santos, Luiz Olavo et al. (2016). “––FAIR ̋ ̋ Data Points Supporting Big Data
Interoperability”. In: \emph{Enterprise Interoperability in the Digitized and Networked Factory of
the Future}. Ed. by Martin Zelm, Guy Doumeingts, and Joao Pedro Mendonc ̧a. iSTE Press,
pp. 270–279. \textsc{isbn}: 978-1-84704-044-2.  \url
{https://www.researchgate.net/publication/309468587 ̇FAIR ̇Data ̇Points ̇Supporting ̇Big ̇Data ̇Interoperability}
(visited on 2022-11-30).
{}
\bibitem{FOAFVocabularySpecification}
Brickley, Dan and Libby Miller (2014-01-14). \emph{––FOAF Vocabulary Specification ̋ ̋}.
 \url {http://xmlns.com/foaf/spec/} (visited on 2022-05-26).
{}
\bibitem{fdo-Glossary}
Broeder, Daan and Peter Wittenburg (2022-11-19). \emph{–FDO ̋ Glossary November 2022}.
Spreadsheet FDO Glossary Nov 2022. HDL: \href
{http://hdl.handle.net/20.500.14132/fdo-spec-docs} {\nolinkurl {20.500.14132/fdo-spec-docs}}.
 \url {https://drive.google.com/file/d/1KJ9l0p96naKi ̇2HPJ ̇MPqPTwS ̇zlP92G}
(visited on 2023-02-02).
{}
\bibitem{fdo-KernelAttributes}
Broeder, Daan, Peter Wittenburg, et al. (2022-10-17). \emph{––FDO ̋ ̋ – Kernel Attributes “\&
Metadata}. Proposed Recommendation PR-FDO-KernelAttributesAndMetadata-2.0-20221017.
HDL: \href {http://hdl.handle.net/20.500.14132/fdo-spec-docs} {\nolinkurl
{20.500.14132/fdo-spec-docs}}.  \url
{https://drive.google.com/file/d/12ceEFiG2RxHZX4isMI5HdoGAT6fyl29a} (visited on
2023-02-02).
{}
\bibitem{carrieroLandscapeOntologyReuse2020a}
Carriero, Valentina Anita et al. (2020-11-12). “The Landscape of Ontology Reuse Approaches”. In:
\emph{Applications and Practices in Ontology Design, Extraction, and Reasoning}. Ed. by
Giuseppe Cota, Marilena Daquino, and Gian Luca Pozzato. Studies on the Semantic Web. IOS
Press. \textsc{isbn}: 978-1-64368-142-9. \url {https://doi.org/10.3233/ssw200033}.
{}
\bibitem{ciccaresePAVOntologyProvenance2013e}
Ciccarese, Paolo et al. (2013). “––PAV ̋ ̋ Ontology: Provenance, Authoring and Versioning”. In:
\emph{Journal of Biomedical Semantics} 4.1, p. 37. \url
{https://doi.org/10.1186/2041-1480-4-37}.
{}
\bibitem{rfc3253}
Clemm, G. et al. (2002-03). \emph{Versioning ––Extensions ̋ ̋ to ––WebDAV ̋ ̋ (––Web
Distributed Authoring ̋ ̋ and ––Versioning ̋ ̋)}. \url {https://doi.org/10.17487/rfc3253}.
{}
\bibitem{ContentNegotiationHTTP}
\emph{Content Negotiation - HTTP — MDN} (2022).  \url
{https://developer.mozilla.org/en-US/docs/Web/HTTP/Content ̇negotiation} (visited on
2022-05-26).
{}
\bibitem{corchoEOSCInteroperabilityFramework2021b}
Corcho, Oscar et al. (2021-02-05). \emph{––EOSC ̋ ̋ Interoperability Framework}. \url
{https://doi.org/10.2777/620649}.
{}
\bibitem{DataW3C}
\emph{Data - W3C} (2022).  \url {https://www.w3.org/standards/semanticweb/data}
(visited on 2022-05-26).
{}
\bibitem{w3-vocab-dcat-2}
\emph{Data Catalog Vocabulary (DCAT) - Version 2} (2020-02-04).  \url
{https://www.w3.org/TR/2020/REC-vocab-dcat-2-20200204/}.
{}
\bibitem{DataInformationView}
\emph{Data Information View} (2023). HDL: \href
{http://hdl.handle.net/21.14100/2fcf49d3-0608-3373-a47f-0e721b7eaa87} {\nolinkurl
{21.14100/2fcf49d3-0608-3373-a47f-0e721b7eaa87}}.  \url
{https://hdl.handle.net/21.14100/2fcf49d3-0608-3373-a47f-0e721b7eaa87} (visited on
2023-01-24).
{}
\bibitem{w3-vocab-dcat-3}
Dataset Exchange Working Group (2022-05-10). \emph{Data ––Catalog Vocabulary ̋ ̋
(––DCAT ̋ ̋) - ––Version ̋ ̋ 3}. W3C Working Draft.  \url
{https://www.w3.org/TR/2022/WD-vocab-dcat-3-20220510/}.
{}
\bibitem{DCMIMetadataTerms}
DCMI Usage Board (2020-01-20). \emph{––DCMI Metadata Terms ̋ ̋}. DCMI Recommendation.
 \url
{https://www.dublincore.org/specifications/dublin-core/dcmi-terms/2020-01-20/} (visited on
2022-05-26).
{}
\bibitem{delgadoInteroperabilityFrameworkDistributed2016a}
Delgado, Jos ́e Carlos Martins (2016). “An ––Interoperability Framework ̋ ̋ and ––Distributed
Platform ̋ ̋ for ––Fast Data Applications ̋ ̋”. In: \emph{Data ––Science ̋ ̋ and ––Big Data
Computing ̋ ̋}. Springer International Publishing, pp. 3–39.
{}
\bibitem{foundationDigitalObjectInterface}
\emph{Digital Object Interface Protocol Specification, Version 2.0} (n.d.). HDL: \href
{http://hdl.handle.net/0.DOIP/DOIPV2.0} {\nolinkurl {0.DOIP/DOIPV2.0}}.  \url
{https://hdl.handle.net/0.DOIP/DOIPV2.0}.
{}
\bibitem{DOIHandbookResolution}
“DOI Handbook - Resolution” (2017-03-21). In: \emph{–DOI Handbook ̋}. \url
{https://doi.org/10.1000/182}.  \url
{https://www.doi.org/doi ̇handbook/3 ̇Resolution.html} (visited on 2023-01-24).
{}
\bibitem{DOIResolutionDocumentation}
\emph{DOI Resolution Documentation} (2020-07-04). International DOI Foundation.
 \url {https://www.doi.org/factsheets/DOIProxy.html} (visited on 2023-01-24).
{}
\bibitem{DOIPExamplesCordraa}
\emph{DOIP and Examples — Cordra Documentation} (2022).  \url
{https://www.cordra.org/documentation/api/doip.html} (visited on 2022-05-26).
{}
\bibitem{DOIPAPIHTTPa}
\emph{DOIP API for HTTP Clients — Cordra Documentation} (2023).  \url
{https://www.cordra.org/documentation/api/doip-api-for-http-clients.html} (visited on
2023-01-24).
{}
\bibitem{Draftbhuttonjsonschema00}
\emph{Draft-Bhutton-Json-Schema-00} (2022).  \url
{https://datatracker.ietf.org/doc/html/draft-bhutton-json-schema-00} (visited on 2022-05-26).
{}
\bibitem{Draftietfmediamansuffixes00MediaTypes}
\emph{Draft-Ietf-Mediaman-Suffixes-00 - Media Types with Multiple Suffixes} (2022).
 \url {https://datatracker.ietf.org/doc/draft-ietf-mediaman-suffixes/00/} (visited on
2022-05-26).
{}
\bibitem{Draftkellyjsonhal08}
\emph{Draft-Kelly-Json-Hal-08} (2022).  \url
{https://datatracker.ietf.org/doc/html/draft-kelly-json-hal-08} (visited on 2022-05-26).
{}
\bibitem{rfc3987}
Duerst, M. and M. Suignard (2005-01). \emph{Internationalized ––Resource Identifiers ̋ ̋
(––IRIs ̋ ̋)}. \url {https://doi.org/10.17487/rfc3987}.
{}
\bibitem{rfc4918}
Dusseault, L. (2007-06). \emph{––HTTP Extensions ̋ ̋ for ––Web Distributed Authoring ̋ ̋ and
––Versioning ̋ ̋ (––WebDAV ̋ ̋)}. \url {https://doi.org/10.17487/rfc4918}.
{}
\bibitem{groupFAIRDataMaturity2020}
FAIR Data Maturity Model Working Group (2020-06-25). “––FAIR ̋ ̋ Data Maturity Model:
––Specification ̋ ̋ and Guidelines”. In: \url {https://doi.org/10.15497/rda00050}.
{}
\bibitem{FAIRDigitalObjects}
\emph{FAIR Digital Objects Forum —} (2022).  \url {https://fairdo.org/} (visited on
2022-05-26).
{}
\bibitem{fdo-Specs}
\emph{FDO Specification Documents - November 2022} (2022-11-19). HDL: \href
{http://hdl.handle.net/20.500.14132/fdo-spec-docs} {\nolinkurl {20.500.14132/fdo-spec-docs}}.
 \url {https://fairdo.org/specifications/} (visited on 2023-02-02).
{}
\bibitem{fenselSemanticWebServices2011}
Fensel, Dieter et al. (2011). \emph{Semantic ––Web Services ̋ ̋}. Springer Berlin Heidelberg.
\textsc{isbn}: 978-3-642-19193-0. \url {https://doi.org/10.1007/978-3-642-19193-0}.
{}
\bibitem{rfc9110}
Fielding, R., M. Nottingham, and J. Reschke (2022-06). \emph{––HTTP Semantics ̋ ̋}. \url
{https://doi.org/10.17487/rfc9110}.
{}
\bibitem{rfc7230}
Fielding, R. and J. Reschke (2014-06). \emph{Hypertext ––Transfer Protocol ̋ ̋ (––HTTP ̋ ̋/1.1):
––Message Syntax ̋ ̋ and ––Routing ̋ ̋}. \url {https://doi.org/10.17487/rfc7230}.
{}
\bibitem{fieldingReflectionsRESTArchitectural2017a}
Fielding, Roy T. et al. (2017-09-04). “Reflections on the ––REST ̋ ̋ Architectural Style and
”Principled Design of the Modern Web Architecture” (Impact Paper Award)”. In:
\emph{Proceedings of the 2017 11th Joint Meeting on Foundations of Software Engineering -
––ESEC ̋ ̋/––FSE ̋ ̋ 2017}. New York, New York, USA: ACM Press, pp. 4–14. \textsc{isbn}:
978-1-4503-5105-8. \url {https://doi.org/10.1145/3106237.3121282}.
{}
\bibitem{fieldingArchitecturalStylesDesign2000a}
Fielding, Roy Thomas (2000). “Architectural Styles and the Design of Network-Based Software
Architectures”. Doctoral Thesis.  \url
{https://www.ics.uci.edu/ ̃fielding/pubs/dissertation/top.htm} (visited on 2022-06-28).
{}
\bibitem{gayoValidatingRDFData2017a}
Gayo, Jose Emilio Labra et al. (2017-09-28). “Validating ––RDF Data ̋ ̋”. In: \emph{Synthesis
Lectures on the Semantic Web: Theory and Technology} 7.1, pp. 1–328. \url
{https://doi.org/10.2200/s00786ed1v01y201707wbe016}.
{}
\bibitem{gobleStateNationData2008c}
Goble, Carole and Robert Stevens (2008-10-01). “State of the Nation in Data Integration for
Bioinformatics.” In: \emph{Journal of Biomedical Informatics} 41.5, pp. 687–693. \url
{https://doi.org/10.1016/j.jbi.2008.01.008}.
{}
\bibitem{rfc5023}
Gregorio, J. and B. de hOra (2007-10). \emph{The ––Atom Publishing Protocol ̋ ̋}. \url
{https://doi.org/10.17487/rfc5023}.
{}
\bibitem{rfc6570}
Gregorio, J., R. Fielding, et al. (2012-03). \emph{––URI Template ̋ ̋}. \url
{https://doi.org/10.17487/rfc6570}.
{}
\bibitem{grothAPIcentricLinkedData2014b}
Groth, Paul et al. (2014-12). “––API-centric Linked Data ̋ ̋ Integration: ––The Open PHACTS
Discovery Platform ̋ ̋ Case Study”. In: \emph{Journal of Web Semantics} 29, pp. 12–18. \url
{https://doi.org/10.1016/j.websem.2014.03.003}.
{}
\bibitem{HandleNetRegistry}
\emph{Handle.Net Registry} (2023).  \url
{https://www.handle.net/download ̇hnr.html} (visited on 2023-01-24).
{}
\bibitem{hardistySpecimenDataRefinery2022a}
Hardisty, Alex et al. (2022). “The ––Specimen Data Refinery ̋ ̋: ––A Canonical Workflow
Framework ̋ ̋ and ––FAIR ̋ ̋ ––Digital Object Approach ̋ ̋ to ––Speeding ̋ ̋ up ––Digital
Mobilisation ̋ ̋ of ––Natural History ̋ ̋ ––Collections ̋ ̋”. In: \emph{Data Intelligence} 4.2,
pp. 320–341. \url {https://doi.org/10.1162/dint ̇a ̇00134}.
{}
\bibitem{hasnainAssessingFAIRData2018a}
Hasnain, Ali and Dietrich Rebholz-Schuhmann (2018). “Assessing ––FAIR ̋ ̋ Data Principles
against the 5-Star Open Data Principles”. In: \emph{The Semantic Web: ––ESWC ̋ ̋ 2018 Satellite
Events: ––ESWC ̋ ̋ 2018 Satellite Events, Heraklion, Crete, Greece, June 3-7, 2018, Revised
Selected Papers}. Ed. by Aldo Gangemi et al. Vol. 11155. Lecture Notes in Computer Science.
Cham: Springer International Publishing, pp. 469–477. \textsc{isbn}: 978-3-319-98191-8. \url
{https://doi.org/10.1007/978-3-319-98192-5 ̇60}.
{}
\bibitem{fdo-Granularity}
Hellstr ̈om, Maggie, Carlo Zw ̈olf, and Peter Wittenburg (2022-10-17). \emph{––FDO ̋ ̋ –
Granularity, Versioning, Mutability}. Proposed Recommendation PR-Granularity-2.2-20221017.
HDL: \href {http://hdl.handle.net/20.500.14132/fdo-spec-docs} {\nolinkurl
{20.500.14132/fdo-spec-docs}}.  \url
{https://drive.google.com/file/d/1S8KlDSNmBsesovVVczAa9mi1CvpCq ̇pN} (visited on
2023-02-02).
{}
\bibitem{hollandIntroducingRole2014}
Holland, Vicki Tardif and Jason Johnson (2014-06-14). \emph{Introducing ’––Role ̋ ̋’}.
 \url {http://blog.schema.org/2014/06/introducing-role.html} (visited on 2023-01-24).
{}
\bibitem{horrocksSemanticWebISWC2002}
Horrocks, Ian and James Hendler, eds. (2002). \emph{The ––Semantic Web ̋ ̋ — ––ISWC ̋ ̋ 2002}.
Springer Berlin Heidelberg. \url {https://doi.org/10.1007/3-540-48005-6}.
{}
\bibitem{huHowMatchableAre2011a}
Hu, Wei et al. (2011). “How Matchable Are Four Thousand Ontologies on the Semantic Web”. In:
\emph{The Semantic Web: ––Research ̋ ̋ and Applications}. Ed. by Grigoris Antoniou et al.
Vol. 6643. Lecture Notes in Computer Science. Berlin, Heidelberg: Springer Berlin Heidelberg,
pp. 290–304. \textsc{isbn}: 978-3-642-21033-4.
{}
\bibitem{HydraW3CCommunity}
\emph{Hydra W3C Community Group} (2023).  \url {https://www.hydra-cg.com/}
(visited on 2023-01-24).
{}
\bibitem{iso16684}
\emph{ISO 16684-1} (2019-04). \emph{––ISO ̋ ̋ 16684-1:2019 — Graphic technology —
Extensible metadata platform (XMP) — Part 1: Data model, serialization and core properties}.
 \url {https://www.iso.org/standard/75163.html}.
{}
\bibitem{iso23009}
\emph{ISO/IEC 23009-1} (2022-08). \emph{––ISO ̋ ̋/––IEC ̋ ̋ 23009-1:2022 — Information
technology — Dynamic adaptive streaming over HTTP (DASH) — Part 1: Media presentation
description and segment formats}.  \url {https://www.iso.org/standard/83314.html}.
{}
\bibitem{rfc9000}
Iyengar, J. and M. Thomson (2021-05). \emph{––QUIC ̋ ̋: ––A UDP-Based Multiplexed ̋ ̋ and
––Secure Transport ̋ ̋}. \url {https://doi.org/10.17487/rfc9000}.
{}
\bibitem{joras2020}
Joras, Matt and Yang Chi (2020-10-21). \emph{How ––Facebook ̋ ̋ is bringing ––QUIC ̋ ̋ to
billions}.  \url
{https://engineering.fb.com/2020/10/21/networking-traffic/how-facebook-is-bringing-quic-to-billions}.
{}
\bibitem{jutyIdentifiersOrgMIRIAM2011}
Juty, N., N. Le Novere, and C. Laibe (2011-12-02). “Identifiers.Org and ––MIRIAM Registry ̋ ̋:
Community Resources to Provide Persistent Identification”. In: \emph{Nucleic Acids Research}
40.D1, pp. D580–D586. \url {https://doi.org/10.1093/nar/gkr1097}.
{}
\bibitem{kahnFrameworkDistributedDigital1995a}
Kahn, Robert and Robert Wilensky (1995-05-13). \emph{A Framework for Distributed Digital
Object Services}.  \url {http://www.cnri.reston.va.us/k-w.html} (visited on
2022-05-09).
{}
\bibitem{kahnFrameworkDistributedDigital2006b}
— (2006-04). “A Framework for Distributed Digital Object Services”. In: \emph{International
Journal on Digital Libraries} 6.2, pp. 115–123. \textsc{issn}: 1432-5012. \url
{https://doi.org/10.1007/s00799-005-0128-x}.  \url
{https://www.doi.org/topics/2006 ̇05 ̇02 ̇Kahn ̇Framework.pdf} (visited on 2022-05-05).
{}
\bibitem{kamdarSystematicAnalysisTerm2017a}
Kamdar, Maulik R., Tania Tudorache, and Mark A. Musen (2017-08-07). “A Systematic Analysis
of Term Reuse and Term Overlap across Biomedical Ontologies”. In: \emph{Semantic Web} 8.6.
Ed. by Guo-Qiang Zhang, pp. 853–871. \url {https://doi.org/10.3233/sw-160238}.
{}
\bibitem{rfc2817}
Khare, R. and S. Lawrence (2000-05). \emph{Upgrading to ––TLS Within HTTP ̋ ̋/1.1}. \url
{https://doi.org/10.17487/rfc2817}.
{}
\bibitem{kleinScholarlyContextNot2014a}
Klein, Martin et al. (2014-12-26). “Scholarly ––Context Not Found ̋ ̋: ––One ̋ ̋ in ––Five Articles
Suffers ̋ ̋ from ––Reference Rot ̋ ̋”. In: \emph{PLoS ONE} 9.12. Ed. by Judit Bar-Ilan, e115253.
\url {https://doi.org/10.1371/journal.pone.0115253}.
{}
\bibitem{klimekSurveyToolsLinked2019a}
Kl ́ımek, Jakub, Petr ˇSkoda, and Martin Neˇcask ́y (2019-05-23). “Survey of Tools for Linked Data
Consumption”. In: \emph{Satya Widya} 10.4, pp. 665–720. \textsc{issn}: 22104968. \url
{https://doi.org/10.3233/SW-180316}.
{}
\bibitem{lamprechtPerspectivesAutomatedComposition2021b}
May, W. and R. Pantos (2017-08). \emph{––HTTP Live Streaming ̋ ̋}. \url
{https://doi.org/10.17487/rfc8216}.
{}
\bibitem{merono-penuelaConclusionFutureChallenges2021a}
Mero ̃no-Pe ̃nuela, Albert, Pasquale Lisena, and Carlos Mart ́ınez-Ortiz (2021a). “Conclusion and
Future Challenges”. In: \emph{Web Data Apis for Knowledge Graphs: ––Easing ̋ ̋ Access to
Semantic Data for Application Developers}. Synthesis Lectures on Data, Semantics, and
Knowledge. Cham: Springer International Publishing, pp. 77–80. \textsc{isbn}:
978-3-031-00789-7. \url {https://doi.org/10.1007/978-3-031-01917-3 ̇7}.
{}
\bibitem{merono-penuelaWebDataApis2021b}
— (2021b). “Web Data Apis over ––SPARQL ̋ ̋”. In: \emph{Web Data Apis for Knowledge
Graphs: ––Easing ̋ ̋ Access to Semantic Data for Application Developers}. Synthesis Lectures on
Data, Semantics, and Knowledge. Cham: Springer International Publishing, pp. 27–38.
\textsc{isbn}: 978-3-031-00789-7. \url {https://doi.org/10.1007/978-3-031-01917-3 ̇3}.
{}
\bibitem{OpenData}
Michael Hausenblas et al. (2012-01-22). \emph{-star ––Open Data ̋ ̋}.  \url
{http://5stardata.info/} (visited on 2023-01-24).
{}
\bibitem{HTMLStandard}
\emph{Microdata} (2023-01-28). In: \emph{–HTML Living Standard ̋}. WHATWG.
 \url {https://html.spec.whatwg.org/multipage/microdata.html} (visited on
2022-05-26).
{}
\bibitem{monsCloudyIncreasinglyFAIR2017b}
Mons, Barend et al. (2017-03-07). “Cloudy, Increasingly ––FAIR ̋ ̋; Revisiting the ––FAIR ̋ ̋ Data
Guiding Principles for the European Open Science Cloud”. In: \emph{Information Services “\&
Use} 37.1, pp. 49–56. \textsc{issn}: 18758789. \url {https://doi.org/10.3233/ISU-170824}.
{}
\bibitem{NCBOBioPortal}
\emph{NCBO BioPortal} (2022).  \url {https://bioportal.bioontology.org/ontologies}
(visited on 2022-05-26).
{}
\bibitem{neumannAnalysisPublicREST2021a}
Neumann, Andy, Nuno Laranjeiro, and Jorge Bernardino (2021-07-01). “An Analysis of Public
––REST ̋ ̋ Web Service Apis”. In: \emph{IEEE Transactions on Services Computing} 14.4,
pp. 957–970. \textsc{issn}: 1939-1374. \url {https://doi.org/10.1109/TSC.2018.2847344}.
{}
\bibitem{rfc8288}
Nottingham, M. (2017-10). \emph{Web ––Linking ̋ ̋}. \url {https://doi.org/10.17487/rfc8288}.
{}
\bibitem{OpenAPISpecificationV3}
\emph{OpenAPI Specification v3.1.0 — Introduction, Definitions, \& More} (2022). 
\url {https://spec.openapis.org/oas/v3.1.0.html} (visited on 2022-05-26).
{}
\bibitem{ORESpecificationAbstract}
\emph{ORE Specification - Abstract Data Model} (2023).  \url
{http://www.openarchives.org/ore/1.0/datamodel#Proxies} (visited on 2023-01-24).
{}
\bibitem{w3-ldn}
Social Web Working Group (2017a-05-02). \emph{Linked ––Data Notifications ̋ ̋}. W3C
Recommendation.  \url {https://www.w3.org/TR/2017/REC-ldn-20170502/} (visited
on 2022-05-26).
{}
\bibitem{w3-micropub}
— (2017b-05-23). \emph{Micropub}. W3C Recommendation.  \url
{https://www.w3.org/TR/2017/REC-micropub-20170523/}.
{}
\bibitem{w3-sparql11-overview}
\emph{SPARQL 1.1 Overview} (2022).  \url
{http://www.w3.org/TR/sparql11-overview/} (visited on 2022-05-26).
{}
\bibitem{w3-json-ld}
Sporny, Manu et al. (2020-07-16). \emph{––JSON-LD ̋ ̋ 1.1}. W3C Recommendation. JSON-LD
Working Group.  \url {https://www.w3.org/TR/2020/REC-json-ld11-20200716/}.
{}
\bibitem{stallingsHandbookComputercommunicationsStandards1990}
Stallings, William (1990). \emph{Handbook of Computer-Communications Standards: ––The ̋ ̋
Open Systems (––OSI ̋ ̋) Model and ––OSI-related ̋ ̋ Standards}. 2nd ed. Carmel, IN, USA: Sams.
\textsc{isbn}: 978-0-672-22697-7.
{}
\bibitem{stanczykProcessModellingInformation1987}
Stanczyk, Stefan K (1987). “Process Modelling for Information System Description”. In:
\emph{The Open University}. \url {https://doi.org/10.21954/ou.ro.0000f821}.
{}
\bibitem{stefiDevelopersMakeUnbiased2015}
Stefi, Anisa (2015). \emph{Do ––Developers Make Unbiased Decisions ̋ ̋? - ––The Effect ̋ ̋ of
––Mindfulness ̋ ̋ and ––Not-Invented-Here Bias ̋ ̋ on the ––Adoption ̋ ̋ of ––Software
Components ̋ ̋}. \url {https://doi.org/10.18151/7217489}.
{}
\bibitem{stefiDevelopReuseTwo2015a}
Stefi, Anisa and Thomas Hess (2015). “To Develop or to Reuse? ––Two ̋ ̋ Perspectives on External
Reuse in Software Projects”. In: \emph{Software Business}. Ed. by Jo ̃ao M. Fernandes,
Ricardo J. Machado, and Krzysztof Wnuk. Vol. 210. Lecture Notes in Business Information
Processing. Cham: Springer International Publishing, pp. 192–206. \textsc{isbn}:
978-3-319-19592-6.
{}
\bibitem{fdo-RequirementSpec}
Strawn, G. et al. (2022-10-17). \emph{––FDO Forum FDO ̋ ̋ Requirement Specifications}.
Proposed Recommendation PR-RequirementSpec-2.1. HDL: \href
{http://hdl.handle.net/20.500.14132/fdo-spec-docs} {\nolinkurl {20.500.14132/fdo-spec-docs}}.
 \url {https://drive.google.com/file/d/1jpsdtN2Hn2FHKhVP294TtNfdGbGLumxs}
(visited on 2023-02-02).
{}
\bibitem{rfc3650}
Sun, S., L. Lannom, and B. Boesch (2003-11). \emph{Handle ––System Overview ̋ ̋}. \url
{https://doi.org/10.17487/rfc3650}.
{}
\bibitem{vandesompelFAIRSignpostingProfile2022}
Van de Sompel, Herbert et al. (2022-07-27). \emph{––FAIR Signposting Profile ̋ ̋}. 
\url {https://signposting.org/FAIR/} (visited on 2023-01-05).
{}
\bibitem{DesigningLinkedData2018}
Verborgh, Ruben (2018-12-28). \emph{Designing a ––Linked Data ̋ ̋ Developer Experience}.
 \url
{https://ruben.verborgh.org/blog/2018/12/28/designing-a-linked-data-developer-experience/}
(visited on 2022-05-26).
{}
\bibitem{verborghSemanticWebIdentity2020a}
Verborgh, Ruben and Miel Vander Sande (2020-01-31). “The Semantic Web Identity Crisis: ––In ̋ ̋
Search of the Trivialities That Never Were”. In: \emph{Satya Widya} 11.1, pp. 19–27.
\textsc{issn}: 22104968. \url {https://doi.org/10.3233/SW-190372}.
{}
\bibitem{w3-xmlschema11}
\emph{W3C XML Schema Definition Language (XSD) 1.1 Part 1: Structures} (2022).
 \url {http://www.w3.org/TR/xmlschema11-1/} (visited on 2022-05-26).
{}
\bibitem{w3-wsdl20-primer}
\emph{Web Services Description Language (WSDL) Version 2.0 Part 0: Primer} (2022).
 \url {http://www.w3.org/TR/wsdl20-primer/} (visited on 2022-05-26).
{}
\bibitem{weigelRDARecommendationPID2018}
Weigel, Tobias et al. (2018). “––RDA Recommendation ̋ ̋ on ––PID Kernel Information ̋ ̋”. In:
\emph{Research Data Alliance}. \url {https://doi.org/10.15497/rda00031}.
{}
\bibitem{fdo-DocProcessStd}
Weiland, C. et al. (2022-01-29). \emph{––FDO Forum Document Standards ̋ ̋}. Working Draft
WD-DocProcessStd-1.1-20220129. (internal draft).  \url
{https://drive.google.com/file/d/1lPNBBROjEoZ6fTfrtdqcMa3Q2G27PoC ̇} (visited on
2022-11-30).
{}
\bibitem{fdo-MachineActionDef}
Weiland, Claus et al. (2022-11-19). \emph{––FDO ̋ ̋ Machine Actionability}. Proposed
Recommendation PR-MachineActionDef-2.2-20221119. HDL: \href
{http://hdl.handle.net/20.500.14132/fdo-spec-docs} {\nolinkurl {20.500.14132/fdo-spec-docs}}.
 \url {https://drive.google.com/file/d/14Pg6qcrOnHRMu1SCN3jZ1wX9SLcIOnoo}
(visited on 2023-02-02).
{}
\bibitem{wieczorekDarwinCoreEvolving2012}
Wieczorek, John et al. (2012-01-06). “Darwin ––Core ̋ ̋: ––An Evolving Community-Developed
Biodiversity Data Standard ̋ ̋”. In: \emph{PLoS ONE} 7.1. Ed. by Indra Neil Sarkar, e29715. \url
{https://doi.org/10.1371/journal.pone.0029715}.
{}
\bibitem{rfc6906}
Wilde, E. (2013-03). \emph{The ’profile’ ––Link Relation Type ̋ ̋}. \url
{https://doi.org/10.17487/rfc6906}.
{}
\bibitem{wilkinsonFAIRGuidingPrinciples2016e}
Wilkinson, Mark D. et al. (2016-03-15). “The ––FAIR Guiding Principles ̋ ̋ for Scientific Data
Management and Stewardship”. In: \emph{Scientific Data} 3.1. \url
{https://doi.org/10.1038/sdata.2016.18}.
{}
\bibitem{wilkinsonWorkflowsWhenParts2022b}
Wilkinson, Sean R. et al. (2022). “F*** Workflows: When Parts of ––FAIR ̋ ̋ Are Missing”. In:
\emph{arXiv}. \url {https://doi.org/10.48550/arxiv.2209.09022}.
{}
\bibitem{williamsOpenPHACTSSemantic2012c}
Williams, Antony J. et al. (2012-11). “Open ––PHACTS ̋ ̋: Semantic Interoperability for Drug
Discovery”. In: \emph{Drug Discovery Today} 17.21-22, pp. 1188–1198. \url
{https://doi.org/10.1016/j.drudis.2012.05.016}.
{}
\bibitem{wittenburgFAIRDigitalObject2022b}
Wittenburg, Peter, Ivonne Anders, et al. (2022). “––FAIR ̋ ̋ Digital Object Demonstrators 2021”. In:
\emph{Zenodo}. \url {https://doi.org/10.5281/zenodo.5872645}.
{}
\bibitem{wittenburgDigitalObjectsDrivers2019a}
Wittenburg, Peter, George Strawn, et al. (2019-01-06). “Digital Objects as Drivers towards
Convergence in Data Infrastructures”. In: \emph{https://b2share.eudat.eu}. \url
{https://doi.org/10.23728/b2share.b605d85809ca45679b110719b6c6cb11}.
{}
\bibitem{wolstencroftRightFieldEmbeddingOntology2011b}
Wolstencroft, K. et al. (2011-05-26). “––RightField ̋ ̋: Embedding Ontology Annotation in
Spreadsheets”. In: \emph{Bioinformatics} 27.14, pp. 2021–2022. \url
{https://doi.org/10.1093/bioinformatics/btr312}.
{}
\bibitem{wolstencroftTavernaWorkflowSuite2013d}
Wolstencroft, Katherine et al. (2013-05-02). “The ––Taverna ̋ ̋ Workflow Suite: Designing and
Executing Workflows of ––Web Services ̋ ̋ on the Desktop, Web or in the Cloud”. In:
\emph{Nucleic Acids Research} 41.W1, W557–W561. \url {https://doi.org/10.1093/nar/gkt328}.
{}
\bibitem{x1255FrameworkDiscovery}
\emph{X.1255 : Framework for Discovery of Identity Management Information} (2022).
 \url {https://www.itu.int/rec/T-REC-X.1255-201309-I} (visited on 2022-05-26).
{}
\bibitem{zarrasComparisonFrameworkMiddleware2004a}
Zarras, Apostolos (2004). “A ––Comparison Framework ̋ ̋ for ––Middleware Infrastructures ̋ ̋.” In:
\emph{The Journal of Object Technology} 3.5, p. 103. \url
{https://doi.org/10.5381/jot.2004.3.5.a2}.



%% Chapter 05 references
\section{References from Chapter 5}


\bibitem[DONA 2018]{ch5-47}
DONA~Foundation (2018):\\
\textbf{Digital Object Interface Protocol Specification, version
2.0}.\\
Technical Report.\\
\url{https://www.dona.net/sites/default/files/2018-11/DOIPv2Spec_1.pdf}

\bibitem[Garcia-Silva 2019]{ch5-48}
Andres Garcia-Silva, Jose Manuel Gomez-Perez, Raul Palma,
Marcin Krystek, Simone Mantovani, Federica Foglini, Valentina Grande,
Francesco De Leo, Stefano Salvi, Elisa Trasatti, Vito Romaniello, Mirko
Albani, Cristiano Silvagni, Rosemarie Leone, Fulvio Marelli, Sergio
Albani, Michele Lazzarini, Hazel J. Napier, Helen M. Glaves, Timothy
Aldridge, Charles Meertens, Fran Boler, Henry W. Loescher, Christine
Laney, Melissa A. Genazzio, Daniel Crawl, Ilkay Altintas (2019):\\
\textbf{Enabling FAIR research in Earth science through research
objects}.\\
\emph{Future Generation Computer Systems} \textbf{98} pp.~550--564.\\
\url{https://arxiv.org/abs/1809.10617}\\
\url{https://doi.org/10.1016/j.future.2019.03.046}

\bibitem[Sefton 2021]{ch5-49}
Peter Sefton, Mike Lynch, Stian Soiland-Reyes (2021):\\
\textbf{GitHub -- UTS-eResearch/ro-crate-js}: Research Object Crate
(RO-Crate) utilities.\\
\url{https://github.com/UTS-eResearch/ro-crate-js}

\bibitem[Eguinoa 2020]{ch5-50}
Ignacio Eguinoa, Stian Soiland-Reyes, Bert Droesbeke, Michael
R. Crusoe (2020):\\
\textbf{GitHub workflowhub-eu/galaxy2cwl}: Standalone version tool to
get cwl descriptions (initially an abstract cwl interface) of galaxy
workflows and Galaxy workflows executions.\\
\url{https://github.com/workflowhub-eu/galaxy2cwl}

\bibitem[La Rosa 2021a]{ch5-51}
Marco La Rosa (2021):\\
\textbf{GitHub -- CoEDL/modpdsc},
\url{https://github.com/CoEDL/modpdsc/}

\bibitem[La Rosa 2021b]{ch5-52}
Marco La Rosa (2021):\\
\textbf{GitHub -- CoEDL/ocfl-tools: Tools to process and manipulate an OCFL tree}.\\
\url{https://github.com/CoEDL/ocfl-tools}

\bibitem[Nature 2019]{ch5-53}
Nature Editorial (2019):\\
\textbf{Giving software its due}.\\
\emph{Nature Methods} \textbf{16}(3) (2019), 207--207.\\
\url{https://doi.org/10.1038/s41592-019-0350-x}

\bibitem[Goble 2016]{ch5-54}
Carole~Goble (2016):\\
\textbf{What Is Reproducibility? The R* Brouhaha}.\\
\emph{SciRepro Workshop}, TPDL, Hannover, Germany, 2016.
\url{http://repscience2016.research-infrastructures.eu/img/CaroleGoble-ReproScience2016v2.pdf}

\bibitem[Goble 2019]{ch5-55}
Carole Goble, Sarah Cohen-Boulakia, Stian Soiland-Reyes, Daniel
Garijo, Yolanda Gil, Michael R. Crusoe, Kristian Peters, Daniel Schober
(2019):\\
\textbf{FAIR Computational Workflows}.\\
\emph{Data Intelligence} \textbf{2}(1--2) pp.~108--121.\\
\url{https://doi.org/10.1162/dint_a_00033}

\bibitem[Goble 2021]{ch5-56}
Carole Goble, Stian Soiland-Reyes, Finn Bacall, Stuart Owen,
Alan Williams, Ignacio Eguinoa, Bert Droesbeke, Simone Leo, Luca
Pireddu, Laura Rodriguez-Navas, José Mª Fernández, Salvador
Capella-Gutierrez, Hervé Ménager, Björn Grüning, Beatriz Serrano-Solano,
Philip Ewels, Frederik Coppens (2021):\\
\textbf{Implementing FAIR digital objects in the EOSC-life workflow
collaboratory}.\\
\emph{Zenodo}. \url{https://doi.org/10.5281/zenodo.4605654}

\bibitem[Goble 2010]{ch5-57}
Carole A Goble, Jiten Bhagat, Sergejs Aleksejevs, Don
Cruickshank, Danius Michaelides, David Newman, Mark Borkum, Sean
Bechhofer, Marco Roos, Peter Li, David De Roure (2010):\\
\textbf{myExperiment: A repository and social network for the sharing of
bioinformatics workflows}.\\
\emph{Nucleic Acids Research} \textbf{38}(Web Server issue)
W677--W682.\\
\url{https://doi.org/10.1093/nar/gkq429}

\bibitem[Gray 2017]{ch5-58}
Alasdair Gray, Carole Goble, Rafael Jimenez, Bioschemas
Community (2017):\\
\textbf{Bioschemas: From Potato Salad to Protein Annotation}.\\
\emph{ISWC}, Vienna, Austria.\\
\url{https://iswc2017.semanticweb.org/paper-579/}

\bibitem[Grossman 2016]{ch5-59}
Robert L Grossman, Allison Heath, Mark Murphy, Maria Patterson,
Walt Wells (2016):\\
\textbf{A case for data commons: Toward data science as a service}.\\
\emph{Computing in Science \& Engineering} \textbf{18}(5) pp.~10--20.\\
\url{https://doi.org/10.1109/MCSE.2016.92}

\bibitem[Grüning 2018a]{ch5-60}
Björn Grüning, John Chilton, Johannes Köster, Ryan Dale, Nicola
Soranzo, Marius van den Beek, Jeremy Goecks, Rolf Backofen, Anton
Nekrutenko, James Taylor (2018):\\
\textbf{Practical computational reproducibility in the life sciences}.\\
\emph{Cell Systems} \textbf{6}(6) pp.~631--635.\\
\url{https://doi.org/10.1016/j.cels.2018.03.014}

\bibitem[Grüning 2018b]{ch5-61}
Björn Grüning, Ryan Dale, Andreas Sjödin, Brad A Chapman,
Jillian Rowe, Christopher H Tomkins-Tinch, Renan Valieris, Johannes
Köster, Bioconda Team (2018):\\
\textbf{Bioconda: Sustainable and comprehensive software distribution
for the life sciences}.\\
\emph{Nature Methods} \textbf{15}(7) pp.~475--476.\\
\url{https://doi.org/10.1038/s41592-018-0046-7}

\bibitem[Guha 2015]{ch5-62}
Ramanathan V Guha, Dan Brickley, Steve Macbeth (2015):\\
\textbf{Schema.org: Evolution of Structured Data on the Web: Big data
makes common schemas even more necessary}.\\
\emph{Queue} \textbf{13}(9) pp.~10--37.\\
\url{https://doi.org/10.1145/2857274.2857276}

\bibitem[Heath 2011]{ch5-63}
Tom Heath, Christian Bizer (2011):\\
\textbf{Linked Data: Evolving the Web into a Global Data Space}.\\
\emph{Synthesis Lectures on the Semantic Web: Theory and Technology}
\textbf{1} pp.~1--136, ISSN 2160-4711. ISBN 9781608454310 / ISBN
9781608454303. \url{https://identifiers.org/isbn/9781608454303}\\
\url{https://doi.org/10.2200/S00334ED1V01Y201102WBE001}

\bibitem[IEEE 2791-2020]{ch5-64}
~\textbf{IEEE Standard for Bioinformatics Analyses Generated by
High-Throughput Sequencing (HTS) to Facilitate Communication} (2020).\\
\emph{IEEE Std} \textbf{2791-2020}.\\
ISBN 978-1-5044-6466-6.\\
\url{https://www.research.manchester.ac.uk/portal/en/publications/ieee-2791(936de52b-ac53-4f0e-9927-77fd7073e88d).html}\\
\url{https://doi.org/10.1109/ieeestd.2020.9094416}

\bibitem[Jensen 2017]{ch5-65}
Mark A Jensen, Vincent Ferretti, Robert L Grossman, Louis M
Staudt (2017):\\
\textbf{The NCI Genomic Data Commons as an engine for precision
medicine}.\\
\emph{Blood} \textbf{130}(4) pp.~453--459.\\
\url{https://doi.org/10.1182/blood-2017-03-735654}

\bibitem[Jones 2021]{ch5-66}
Matthew B. Jones, Stephen Richard, Dave Vieglais, Adam
Shepherd, Ruth Duerr, Dougl Fils, Lewis McGibbney (2021):\\
\textbf{Science-on-Schema.org v1.2.0}\\
\url{https://doi.org/10.5281/zenodo.4477164}

\bibitem[Katsumi 2016]{ch5-67}
Megan Katsumi, Michael Grüninger (2016):\\
\textbf{What is ontology reuse?}.\\
In: \emph{Formal Ontology in Information Systems}, R.~Ferrario and
W.~Kuhn, eds,\\
\emph{Frontiers in Artificial Intelligence and Applications}
\textbf{283}\\
ISBN 978-1-61499-660-6.\\
\url{https://doi.org/10.3233/978-1-61499-660-6-9}

\bibitem[Khan 2019]{ch5-68}
Farah Zaib Khan, Stian Soiland-Reyes, Richard O. Sinnott,
Andrew Lonie, Carole Goble, Michael R. Crusoe (2019):\\
\textbf{Sharing interoperable workflow provenance: A review of best
practices and their practical application in CWLProv}.\\
\emph{GigaScience} \textbf{8}(11).\\
\url{https://doi.org/10.1093/gigascience/giz095}

\bibitem[Kim 2008]{ch5-69}
Jihie Kim, Ewa Deelman, Yolanda Gil, Gaurang Mehta, Varun
Ratnakar (2008):\\
\textbf{Provenance trails in the Wings/Pegasus system}.\\
\emph{Concurrency and Computation: Practice and Experience}
\textbf{20}(5) pp.~587--597.\\
\url{https://doi.org/10.1002/cpe.1228}

\bibitem[Kluyver 2016]{ch5-70}
Thomas Kluyver, Benjamin Ragan-Kelley, Fernando Pérez, Brian
Granger, Matthias Bussonnier, Jonathan Frederic, Kyle Kelley, Jessica
Hamrick, Jason Grout, Sylvain Corlay, Paul Ivanov, Damián Avila, Safia
Abdalla, Carol Willing, Jupyter Development Team (2016):\\
\textbf{Jupyter Notebooks -- a publishing format for reproducible
computational workflows}.\\
in: \emph{Positioning and Power in Academic Publishing: Players, Agents
and Agendas},\\
\emph{Proceedings of the 20th International Conference on Electronic
Publishing}, pp.~87--90, ISBN 978-1-61499-649-1.\\
\url{https://doi.org/10.3233/978-1-61499-649-1-87}

\bibitem[Koesten 2021]{ch5-71}
Laura Koesten, Kathleen Gregory, Paul Groth, Elena Simperl
(2021):\\
\textbf{Talking datasets -- understanding data sensemaking
behaviours}.\\
\emph{International journal of human-computer studies}
\textbf{146}:102562.\\
\url{https://doi.org/10.1016/j.ijhcs.2020.102562}

\bibitem[Koesten 2020]{ch5-72}
Laura Koesten, Pavlos Vougiouklis, Elena Simperl, Paul Groth
(2020):\\
\textbf{Dataset reuse: Toward translating principles to practice}.\\
\emph{Patterns} \textbf{1}(8):100136.\\
\url{https://doi.org/10.1016/j.patter.2020.100136}

\bibitem[Köster 2012]{ch5-73}
Johannes Köster, Sven Rahmann (2012):\\
\textbf{Snakemake -- a scalable bioinformatics workflow engine}.\\
\emph{Bioinformatics} \textbf{28}(19) pp.~2520--2522.\\
\url{https://doi.org/10.1093/bioinformatics/bts480}

\bibitem[Kunze 2018]{ch5-74}
J.~Kunze, J.~Littman, E.~Madden, J.~Scancella, C.~Adams
(2018):\\
\textbf{The BagIt File Packaging Format}, (V1.0), RFC 8493,
\emph{Internet Requests for Comments}, RFC Editor.\\
\url{https://doi.org/10.17487/RFC8493}

\bibitem[Kurowski 2021]{ch5-75}
Krzysztof Kurowski, Oscar Corcho, Christine Choirat, Magnus
Eriksson, Frederik Coppens, Mark van de Sanden, Milan Ojsteršek
(2021):\\
\textbf{EOSC Interoperability Framework}.\\
\emph{Publications Office of the EU}, Technical Report, 2021.\\
\url{https://doi.org/10.2777/620649}

\bibitem[Chard 2020]{ch5-76}
Kyle Chard, Niall Gaffney, Mihael Hategan, Kacper Kowalik,
Bertram Ludäscher, Timothy McPhillips, Jarek Nabrzyski, Victoria
Stodden, Ian Taylor, Thomas Thelen, Matthew J. Turk, Craig Willis
(2020):\\
\textbf{Toward enabling reproducibility for data-intensive research
using the Whole Tale platform}.\\
\emph{Advances in Parallel Computing} \textbf{36} pp 766--778.\\
\url{https://doi.org/10.3233/APC200107}

\bibitem[La Rosa 2021c]{ch5-77}
M.~La Rosa (2021):\\
\textbf{Arkisto Platform: Describo Online}.\\
\url{https://arkisto-platform.github.io/describo-online/}

\bibitem[La Rosa 2021d]{ch5-78}
M.~La Rosa and Peter Sefton (2021):\\
\textbf{Arkisto Platform: Describo}.\\
\url{https://arkisto-platform.github.io/describo/}

\bibitem[Lammey 2020]{ch5-79}
R.~Lammey (2020):\\
\textbf{Solutions for identification problems: A look at the research
organization registry}.\\
\emph{Science Editing} \textbf{7}(1) pp.~65--69.\\
\url{https://doi.org/10.6087/kcse.192}

\bibitem[Lamprecht 2019]{ch5-80}
Anna-Lena Lamprecht, Leyla Garcia, Mateusz Kuzak, Carlos
Martinez, Ricardo Arcila, Eva Martin Del Pico, Victoria Dominguez Del
Angel, Stephanie Van De Sandt, Jon Ison, Paula Andrea Martinez, Peter
Mcquilton, Alfonso Valencia, Jennifer Harrow, Fotis Psomopoulos, Josep
Ll. Gelpi, Neil Chue Hong, Carole Goble, Salvador Capella-Gutierrez
(2019):\\
\textbf{Towards FAIR principles for research software}.\\
\emph{Data Science} \textbf{3}(1) pp.~1--23.\\
\url{https://doi.org/10.3233/DS-190026}

\bibitem[Lebo 2013]{ch5-81}
T.~Lebo, S.~Sahoo, D.~McGuinness, K.~Belhajjame, J.~Cheney,
D.~Corsar, D.~Garijo, Stian~Soiland-Reyes, S.~Zednik and J.~Zhao
(2013):\\
\textbf{PROV-O: The PROV Ontology}.\\
\emph{W3C Recommendation} 30 April 2013.
\url{https://www.w3.org/TR/2013/REC-prov-o-20130430/}

\bibitem[Leipzig 2021]{ch5-82}
J.~Leipzig, D.~Nüst, C.T.~Hoyt, K.~Ram and J.~Greenberg
(2021):\\
\textbf{The role of metadata in reproducible computational research}.\\
\emph{Patterns} \textbf{2}(9):100322.\\
\url{https://doi.org/10.1016/j.patter.2021.100322}

\bibitem[Lowe 2021]{ch5-83}
D.~Lowe and G.~Bayarri (2021):\\
\textbf{Protein Ligand Complex MD Setup tutorial using BioExcel Building
Blocks (biobb) (jupyter notebook)}.\\
\url{https://doi.org/10.48546/workflowhub.workflow.56.1}

\bibitem[Lynch 2022]{ch5-84}
M.~Lynch and Peter Sefton (2022):\\
\textbf{npm: ro-crate-excel}.\\
\emph{npm} \url{https://www.npmjs.com/package/ro-crate-excel}

\bibitem[GitHub 2021]{ch5-85}
GitHub (2021):\\
\textbf{Managing large files -- GitHub Docs}.\\
\url{https://docs.github.com/en/repositories/working-with-files/managing-large-files}

\bibitem[McMurry 2017]{ch5-86}
Julie A McMurry, Nick Juty, Niklas Blomberg, Tony Burdett, Tom
Conlin, Nathalie Conte, Mélanie Courtot, John Deck, Michel Dumontier,
Donal K Fellows, Alejandra Gonzalez-Beltran, Philipp Gormanns, Jeffrey
Grethe, Janna Hastings, Jean-Karim Hériché, Henning Hermjakob, Jon C
Ison, Rafael C Jimenez, Simon Jupp, John Kunze, Camille Laibe, Nicolas
Le Novère, James Malone, Maria Jesus Martin, Johanna R McEntyre, Chris
Morris, Juha Muilu, Wolfgang Müller, Philippe Rocca-Serra,
Susanna-Assunta Sansone, Murat Sariyar, Jacky L Snoep, Stian
Soiland-Reyes, Natalie J Stanford, Neil Swainston, Nicole Washington,
Alan R Williams, Sarala M Wimalaratne, Lilly M Winfree, Katherine
Wolstencroft, Carole Goble, Cristopher J Mungall, Melissa A Haendel,
Helen Parkinson (2017):\\
\textbf{Identifiers for the 21st century: How to design, provision, and
reuse persistent identifiers to maximize utility and impact of life
science data}.\\
\emph{PLOS Biology} \textbf{15}(6):e2001414.\\
\url{https://doi.org/10.1371/journal.pbio.2001414}

\bibitem[Miksa 2020]{ch5-87}
T.~Miksa, M.~Jaoua and G.~Arfaoui (2020):\\
\textbf{Research object crates and machine-actionable data management
plans}.\\
\emph{1st Workshop on Research Data Management for Linked Open
Science}.\\
\url{https://doi.org/10.4126/frl01-006423291}

\bibitem[Miksa 2019]{ch5-88}
T.~Miksa, S.~Simms, D.~Mietchen and S.~Jones (2019):\\
\textbf{Ten principles for machine-actionable data management plans}.\\
\emph{PLOS Computational Biology} \textbf{15}(3): e1006750.\\
\url{https://doi.org/10.1371/journal.pcbi.1006750}

\bibitem[Möller 2010]{ch5-89}
Steffen Möller, Hajo Nils Krabbenhöft, Andreas Tille, David
Paleino, Alan Williams, Katy Wolstencroft, Carole Goble, Richard
Holland, Dominique Belhachemi, Charles Plessy (2010):\\
\textbf{Community-driven computational biology with Debian Linux}.\\
\emph{BMC Bioinformatics} \textbf{11}(Suppl 12):S5.\\
\url{https://doi.org/10.1186/1471-2105-11-S12-S5}

\bibitem[Möller 2017]{ch5-90}
Steffen Möller, Stuart W. Prescott, Lars Wirzenius; Petter
Reinholdtsen, Brad Chapman, Pjotr Prins, Stian Soiland-Reyes, Fabian
Klötzl, Andrea Bagnacani, Matúš Kalaš, Andreas Tille, Michael R. Crusoe
(2017):\\
\textbf{Robust cross-platform workflows: How technical and scientific
communities collaborate to develop, test and share best practices for
data analysis}.\\
\emph{Data Science and Engineering} \textbf{2}(3) pp.~232--244.\\
\url{https://doi.org/10.1007/s41019-017-0050-4}

\bibitem[Mons 2018]{ch5-91}
Barend~Mons (2018):\\
\textbf{Data Stewardship for Open Science}, 1st edn. Taylor \& Francis,
p.~240. \href{https://identifiers.org/isbn/9781315351148}{ISBN
9781315351148}.

\bibitem[myExperiment 2009]{ch5-92}
myExperiment (2009):\\
\textbf{myExperiment Ontology Modules}.\\
\emph{myExperiment} / \emph{Internet Archive}\\
\url{https://web.archive.org/web/20091115080336/http\%3a\%2f\%2frdf.myexperiment.org/ontologies}

\bibitem[Newman 2009]{ch5-93}
D.~Newman, S.~Bechhofer and D.~De Roure (2009):\\
\textbf{myExperiment: An ontology for e-Research}.\\
in: \emph{Proceedings of the Workshop on Semantic Web Applications in
Scientific Discourse (SWASD 2009)}, T.~Clark, J.S.~Luciano,
M.S.~Marshall, E.~Prud'Hommeaux and S.~Stephens, eds,\\
\emph{CEUR Workshop Proceedings} \textbf{523}. ISSN 1613-0073.\\
\url{http://ceur-ws.org/Vol-523/Newman.pdf}

\bibitem[Neylon 2017]{ch5-94}
Cameron~Neylon (2017):\\
\textbf{As a researcher \ldots{} I'm a bit bloody fed up with Data
Management}.\\
\emph{Science in the Open} (blog)
\url{https://cameronneylon.net/blog/as-a-researcher-im-a-bit-bloody-fed-up-with-data-management/}.

\bibitem[ro-crate-html-js]{ch5-95}
npm:\\
\textbf{ro-crate-html-js}\\
\url{https://www.npmjs.com/package/ro-crate-html-js}

\bibitem[OCFL 2020]{ch5-96}
~\textbf{OCFL, Oxford Common File Layout Specification},
Recommendation, 2020.\\
\url{https://ocfl.io/1.0/spec/}

\bibitem[Piper 2020]{ch5-97}
A.~Piper (2020):\\
\textbf{Digital crowdsourcing and public understandings of the past:
Citizen historians meet criminal characters}.\\
\emph{History Australia} \textbf{17}(3) pp.~525--541.\\
\url{https://doi.org/10.1080/14490854.2020.1796500}


\bibitem[Rehm 2021]{ch5-99}
Heidi L. Rehm, Angela J.H. Page, Lindsay Smith, Jeremy B.
Adams, Gil Alterovitz, Lawrence J. Babb, Maxmillian P. Barkley, Michael
Baudis, Michael J.S. Beauvais, Tim Beck, Jacques S. Beckmann, Sergi
Beltran, David Bernick, Alexander Bernier, James K. Bonfield, Tiffany F.
Boughtwood, Guillaume Bourque, Sarion R. Bowers, Anthony J. Brookes,
Michael Brudno, Matthew H. Brush, David Bujold, Tony Burdett, Orion J.
Buske, Moran N. Cabili, Daniel L. Cameron, Robert J. Carroll, Esmeralda
Casas-Silva, Debyani Chakravarty, Bimal P. Chaudhari, Shu Hui Chen, J.
Michael Cherry, Justina Chung, Melissa Cline, Hayley L. Clissold, Robert
M. Cook-Deegan, Mélanie Courtot, Fiona Cunningham, Miro Cupak, Robert M.
Davies, Danielle Denisko, Megan J. Doerr, Lena I. Dolman, Edward S.
Dove, L. Jonathan Dursi, Stephanie O.M. Dyke, James A. Eddy, Karen
Eilbeck, Kyle P. Ellrott, Susan Fairley, Khalid A. Fakhro, Helen V.
Firth, Michael S. Fitzsimons, Marc Fiume, Paul Flicek, Ian M. Fore,
Mallory A. Freeberg, Robert R. Freimuth, Lauren A. Fromont, Jonathan
Fuerth, Clara L. Gaff, Weiniu Gan, Elena M. Ghanaim, David Glazer,
Robert C. Green, Malachi Griffith, Obi L. Griffith, Robert L. Grossman,
Tudor Groza, Jaime M. Guidry Auvil, Roderic Guigó, Dipayan Gupta,
Melissa A. Haendel, Ada Hamosh, David P. Hansen, Reece K. Hart, Dean
Mitchell Hartley, David Haussler, Rachele M. Hendricks-Sturrup, Calvin
W.L. Ho, Ashley E. Hobb, Michael M. Hoffman, Oliver M. Hofmann, Petr
Holub, Jacob Shujui Hsu, Jean-Pierre Hubaux, Sarah E. Hunt, Ammar
Husami, Julius O. Jacobsen, Saumya S. Jamuar, Elizabeth L. Janes,
Francis Jeanson, Aina Jené, Amber L. Johns, Yann Joly, Steven J.M.
Jones, Alexander Kanitz, Kazuto Kato, Thomas M. Keane, Kristina
Kekesi-Lafrance, Jerome Kelleher, Giselle Kerry, Seik-Soon Khor, Bartha
M. Knoppers, Melissa A. Konopko, Kenjiro Kosaki, Martin Kuba, Jonathan
Lawson, Rasko Leinonen, Stephanie Li, Michael F. Lin, Mikael Linden,
Xianglin Liu, Isuru Udara Liyanage, Javier Lopez, Anneke M. Lucassen,
Michael Lukowski, Alice L. Mann, John Marshall, Michele Mattioni,
Alejandro Metke-Jimenez, Anna Middleton, Richard J. Milne, Fruzsina
Molnár-Gábor, Nicola Mulder, Monica C. Munoz-Torres, Rishi Nag, Hidewaki
Nakagawa, Jamal Nasir, Arcadi Navarro, Tristan H. Nelson, Ania
Niewielska, Amy Nisselle, Jeffrey Niu, Tommi H. Nyrönen, Brian D.
O'Connor, Sabine Oesterle, Soichi Ogishima, Vivian Ota Wang, Laura A.D.
Paglione, Emilio Palumbo, Helen E. Parkinson, Anthony A. Philippakis,
Angel D. Pizarro, Andreas Prlic, Jordi Rambla, Augusto Rendon, Renee A.
Rider, Peter N. Robinson, Kurt W. Rodarmer, Laura Lyman Rodriguez, Alan
F. Rubin, Manuel Rueda, Gregory A. Rushton, Rosalyn S. Ryan, Gary I.
Saunders, Helen Schuilenburg, Torsten Schwede, Serena Scollen, Alexander
Senf, Nathan C. Sheffield, Neerjah Skantharajah, Albert V. Smith, Heidi
J. Sofia, Dylan Spalding, Amanda B. Spurdle, Zornitza Stark, Lincoln D.
Stein, Makoto Suematsu, Patrick Tan, Jonathan A. Tedds, Alastair A.
Thomson, Adrian Thorogood, Timothy L. Tickle, Katsushi Tokunaga, Juha
Törnroos, David Torrents, Sean Upchurch, Alfonso Valencia, Roman Valls
Guimera, Jessica Vamathevan, Susheel Varma, Danya F. Vears, Coby Viner,
Craig Voisin, Alex H. Wagner, Susan E. Wallace, Brian P. Walsh, Marc S.
Williams, Eva C. Winkler, Barbara J. Wold, Grant M. Wood, J. Patrick
Woolley, Chisato Yamasaki, Andrew D. Yates, Christina K. Yung, Lyndon J.
Zass, Ksenia Zaytseva, Junjun Zhang, Peter Goodhand, Kathryn North, Ewan
Birney (2021):\\
\textbf{GA4GH: International policies and standards for data sharing
across genomic research and healthcare}.\\
\emph{Cell Genomics} \textbf{1}(2):100029.\\
\url{https://doi.org/10.1016/j.xgen.2021.100029}

\bibitem[Rettberg 2015]{ch5-100}
N.~Rettberg and B.~Schmidt (2015):\\
\textbf{OpenAIRE: Supporting a European open access mandate}.\\
\emph{College \& Research Libraries News} \textbf{76}(6) pp.~306--310.
\url{http://resolver.sub.uni-goettingen.de/purl?gs-1/11942}\\
\url{https://doi.org/10.5860/crln.76.6.9326}

\bibitem[Sandve 2013]{ch5-101}
G.K.~Sandve, A.~Nekrutenko, J.~Taylor and E.~Hovig (2013):\\
\textbf{Ten simple rules for reproducible computational research}.\\
\emph{PLOS Computational Biology} \textbf{9}(10):e1003285.\\
\url{https://doi.org/10.1371/journal.pcbi.1003285}

\bibitem[Schriml 2020]{ch5-102}
Lynn M. Schriml, Maria Chuvochina, Neil Davies, Emiley A.
Eloe-Fadrosh, Robert D. Finn, Philip Hugenholtz, Christopher I. Hunter,
Bonnie L. Hurwitz, Nikos C. Kyrpides, Folker Meyer, Ilene Karsch
Mizrachi, Susanna-Assunta Sansone, Granger Sutton, Scott Tighe, Ramona
Walls (2020):\\
\textbf{COVID-19 pandemic reveals the peril of ignoring metadata
standards}.\\
\emph{Scientific Data} \textbf{7}(1):188.\\
\url{https://doi.org/10.1038/s41597-020-0524-5}

\bibitem[Sefton 2018]{ch5-103}
Peter Sefton, G.~Devine, C.~Evenhuis, M.~Lynch, S.~Wise,
M.~Lake and D.~Loxton (2018):\\
\textbf{DataCrate: a method of packaging, distributing, displaying and
archiving Research Objects}.\\
in: \emph{Workshop on Research Objects (RO 2018)}, 29 Oct 2018 at IEEE
eScience 2018, Amsterdam, Netherland. \emph{Zenodo}\\
\url{https://doi.org/10.5281/zenodo.1445817}


\bibitem[RO-Crate 1.0]{ch5-105}
Peter Sefton, Eoghan Ó Carragáin, Stian Soiland-Reyes, Oscar
Corcho, Daniel Garijo, Raul Palma, Frederik Coppens, Carole Goble, José
María Fernández, Kyle Chard, Jose Manuel Gomez-Perez, Michael R Crusoe,
Ignacio Eguinoa, Nick Juty, Kristi Holmes, Jason A. Clark, Salvador
Capella-Gutierrez, Alasdair J. G. Gray, Stuart Owen, Alan R. Williams,
Giacomo Tartari, Finn Bacall, Thomas Thelen (2019):\\
\textbf{RO-Crate Metadata Specification 1.0}.\\
\url{https://doi.org/10.5281/zenodo.3541888}

\bibitem[RO-Crate 1.1.1]{ch5-106}
Peter Sefton, Eoghan Ó Carragáin, Stian Soiland-Reyes, Oscar
Corcho, Daniel Garijo, Raul Palma, Frederik Coppens, Carole Goble, José
María Fernández, Kyle Chard, Jose Manuel Gomez-Perez, Michael R Crusoe,
Ignacio Eguinoa, Nick Juty, Kristi Holmes, Jason A. Clark, Salvador
Capella-Gutierrez, Alasdair J. G. Gray, Stuart Owen, Alan R. Williams,
Giacomo Tartari, Finn Bacall, Thomas Thelen, Hervé Ménager, Laura
Rodríguez-Navas, Paul Walk, brandon whitehead, Mark Wilkinson, Paul
Groth, Erich Bremer, LJ Garcia Castro, Karl Sebby, Alexander Kanitz, Ana
Trisovic, Gavin Kennedy, Mark Graves, Jasper Koehorst, Simone Leo, Marc
Portier (2021):\\
\textbf{RO-Crate Metadata Specification 1.1.1}.\\
\url{https://doi.org/10.5281/zenodo.4541002}

\bibitem[RO-Crate 1.1]{ch5-107}
Peter Sefton, Eoghan Ó Carragáin, Stian Soiland-Reyes, Oscar
Corcho, Daniel Garijo, Raul Palma, Frederik Coppens, Carole Goble, José
María Fernández, Kyle Chard, Jose Manuel Gomez-Perez, Michael R Crusoe,
Ignacio Eguinoa, Nick Juty, Kristi Holmes, Jason A. Clark, Salvador
Capella-Gutierrez, Alasdair J. G. Gray, Stuart Owen, Alan R. Williams,
Giacomo Tartari, Finn Bacall, Thomas Thelen, Hervé Ménager, Laura
Rodríguez-Navas, Paul Walk, brandon whitehead, Mark Wilkinson, Paul
Groth, Erich Bremer, LJ Garcia Castro, Karl Sebby, Alexander Kanitz, Ana
Trisovic, Gavin Kennedy, Mark Graves, Jasper Koehorst, Simone Leo
(2020):\\
\textbf{RO-Crate Metadata Specification 1.1}.\\
\url{https://doi.org/10.5281/zenodo.4031327}

\bibitem[Soiland-Reyes 2020a]{ch5-108}
Stian~Soiland-Reyes (2020):\\
\textbf{I am looking for which bioinformatics journals encourage authors
to submit their code/pipeline/workflow supporting data analysis},
\emph{Twitter}
\url{https://twitter.com/soilandreyes/status/1250721245622079488}

\bibitem[Soiland-Reyes 2021a]{ch5-109}
Stian~Soiland-Reyes (2021):\\
\textbf{Describing and packaging workflows using RO-Crate and BioCompute
Objects} \emph{Zenodo}, Webinar for U.S. Food and Drug Administration
(FDA), 2021-05-12.\\
\url{https://doi.org/10.5281/zenodo.4633732}

\bibitem[Soiland-Reyes 2016]{ch5-110}
Stian~Soiland-Reyes, Pinar Alper, Carole~Goble (2016):\\
\textbf{Tracking Workflow Execution With TavernaPROV},
\emph{ProvenanceWeek 2016}, session ``PROV: Three Years Later''.
\url{https://s11.no/2016/provweek-tavernaprov/}\\
\url{https://doi.org/10.5281/zenodo.51314}

\bibitem[Soiland-Reyes 2014]{ch5-111}
Stian~Soiland-Reyes, Matthew Gamble, Robert Haines (2014):\\
\textbf{Research Object Bundle 1.0}.\\
\url{https://w3id.org/bundle/2014-11-05/}
\url{https://doi.org/10.5281/zenodo.12586}.

\bibitem[Sporny 2014]{ch5-112}
M.~Sporny, D.~Longley, G.~Kellogg, M.~Lanthaler and
N.~Lindström (2014):\\
\textbf{JSON-LD 1.0}, W3C Recommendation.
\url{https://www.w3.org/TR/2014/REC-json-ld-20140116/}

%\bibitem[113] V.~Stodden, M.~McNutt, D.H.~Bailey, E.~Deelman, Y.~Gil,
%B.~Hanson, M.A.~Heroux, J.P.A.~Ioannidis and M.~Taufer (2016):\\
%\textbf{Enhancing reproducibility for computational methods}.\\
%\emph{Science} \textbf{354}(6317) pp.~1240--1241.\\
%\url{https://doi.org/10.1126/science.aah6168}

\bibitem[Thieberger 2012]{ch5-114}
N.~Thieberger, L.~Barwick (2012):\\
\textbf{Keeping records of language diversity in melanesia: The Pacific
and regional archive for digital sources in endangered cultures
(PARADISEC)}, in: \emph{Melanesian Languages on the Edge of Asia:
Challenges for the 21st Century}, N.~Evans and M.~Klamer, eds,\\
\emph{Language Documentation \& Conservation Special Publication}
\textbf{SP05} University of Hawai'i Press, pp.~239--253. ISBN
978-0-9856211-2-4\\
\url{http://hdl.handle.net/10125/4567}

\bibitem[Arkisto 2022]
Arkisto (2022):\\
~\textbf{Tools: Data Portal \& Discovery}.\\
\url{https://arkisto-platform.github.io/tools/portal/}

\bibitem[Troncy 2010]{ch5-116}
R.~Troncy, W.~Bailer, M.~Höffernig and M.~Hausenblas (2010):\\
\textbf{VAMP: A service for validating MPEG-7 descriptions w.r.t. to
formal profile definitions}.\\
\emph{Multimedia tools and applications} \textbf{46}(2--3)
pp.~307--329.\\
\url{https://www.persistent-identifier.nl/urn:nbn:nl:ui:18-14511}\\
\url{https://doi.org/10.1007/s11042-009-0397-2}

\bibitem[Van de Sompel 2007]{ch5-117}
Herbert~Van de Sompel, Carl~Lagoze (2007):\\
\textbf{Interoperability for the discovery, use, and re-use of units of
scholarly communication}.\\
\emph{CTWatch Quarterly} \textbf{3}(3).\\
\url{http://icl.utk.edu/ctwatch/quarterly/articles/2007/08/interoperability-for-the-discovery-use-and-re-use-of-units-of-scholarly-communication/}

\bibitem[Vergoulis 2021]{ch5-118}
T.~Vergoulis, K.~Zagganas, L.~Kavouras, M.~Reczko,
S.~Sartzetakis and T.~Dalamagas (2021):\\
\textbf{SCHeMa: Scheduling Scientific Containers on a Cluster of
Heterogeneous Machines}.\\
\url{https://arxiv.org/abs/2103.13138v1}

\bibitem[Volk 2014]{ch5-119}
C.J.~Volk, Y.~Lucero and K.~Barnas (2014):\\
\textbf{Why is data sharing in collaborative natural resource efforts so
hard and what can we do to improve it?}.\\
\emph{Environmental Management} \textbf{x53}(5) pp.~883--893.\\
\url{https://doi.org/10.1007/s00267-014-0258-2}

\bibitem[httpRange-14]{ch5-120}
W3C Technical Architecture Group (2007):\\
\textbf{Dereferencing HTTP URIs}.\\
Draft Tag Finding, 2007.
\url{https://www.w3.org/2001/tag/doc/httpRange-14/2007-08-31/HttpRange-14.html}

\bibitem[Walk 2019]{ch5-121}
P.~Walk, T.~Miksa, P.~Neish (2019):\\
\textbf{RDA DMP Common Standard for Machine-Actionable Data Management
Plans}.\\
\emph{Research Data Alliance}\\
\url{https://doi.org/10.15497/rda00039}

\bibitem[Walton 2020]{ch5-122}
Stephanie Walton, Laurence Livermore, Olaf Bánki, Robert W. N.
Cubey, Robyn Drinkwater, Markus Englund, Carole Goble, Quentin Groom,
Christopher Kermorvant, Isabel Rey, Celia M Santos, Ben Scott, Alan R.
Williams, Zhengzhe Wu (2020):\\
\textbf{Landscape analysis for the specimen data refinery}.\\
\emph{Research Ideas and Outcomes} \textbf{6}.\\
\url{https://doi.org/10.3897/rio.6.e57602}

\bibitem[Wilkinson 2016]{ch5-123}
Mark D. Wilkinson, Michel Dumontier, IJsbrand Jan Aalbersberg,
Gabrielle Appleton, Myles Axton, Arie Baak, Niklas Blomberg, Jan-Willem
Boiten, Luiz Bonino da Silva Santos, Philip E. Bourne, Jildau Bouwman,
Anthony J. Brookes, Tim Clark, Mercè Crosas, Ingrid Dillo, Olivier
Dumon, Scott Edmunds, Chris T. Evelo, Richard Finkers, Alejandra
Gonzalez-Beltran, Alasdair J.G. Gray, Paul Groth, Carole Goble, Jeffrey
S. Grethe, Jaap Heringa, Peter A.C 't Hoen, Rob Hooft, Tobias Kuhn,
Ruben Kok, Joost Kok, Scott J. Lusher, Maryann E. Martone, Albert Mons,
Abel L. Packer, Bengt Persson, Philippe Rocca-Serra, Marco Roos, Rene
van Schaik, Susanna-Assunta Sansone, Erik Schultes, Thierry Sengstag,
Ted Slater, George Strawn, Morris A. Swertz, Mark Thompson, Johan van
der Lei, Erik van Mulligen, Jan Velterop, Andra Waagmeester, Peter
Wittenburg, Katherine Wolstencroft, Jun Zhao, Barend Mons (2016):\\
\textbf{The FAIR guiding principles for scientific data management and
stewardship}.\\
\emph{Scientific Data} \textbf{3}:160018.\\
\url{https://doi.org/10.1038/sdata.2016.18}

\bibitem[WorkflowHub 2023]
WorkflowHub (2023):\\
~\textbf{WorkflowHub project: Project pages for developing and
running the WorkflowHub, a registry of scientific workflows}.\\
\url{https://w3id.org/workflowhub/}

\bibitem[Zhao 2012]{ch5-125}
Jun Zhao, Jose Manuel Gomez-Perezy, Khalid Belhajjame, Graham
Klyne, Esteban Garcia-Cuestay, Aleix Garridoy, Kristina Hettne, Marco
Roos, David De Roure, Carole Goble (2012):\\
\textbf{Why workflows break -- understanding and combating decay in
taverna workflows}.\\
\emph{2012 IEEE 8th International Conference on e-Science}, IEEE.
\url{https://www.research.manchester.ac.uk/portal/files/174861334/why_decay.pdf}
ISBN 978-1-4673-4466-1.\\
\url{https://doi.org/10.1109/eScience.2012.6404482}

\bibitem[Zoubek 2021]{ch5-126}
F.~Zoubek and M.~Winkler (2021):\\
\textbf{RO Crates and Excel}.\\
\url{https://github.com/e11938258/RO-Crates-and-Excel}
\url{https://doi.org/10.5281/zenodo.5068950}

\bibitem[Žumer 2009]{ch5-127}
M.~Žumer (2009):\\
\textbf{National Bibliographies in the Digital Age: Guidance and New
Directions}.\\
\emph{IFLA Series on Bibliographic Control}, IFLA Working Group on
Guidelines for National Bibliographies, Walter de Gruyter -- K. G. Saur,
2009, ISSN 1868-8438. ISBN 9783598441844.\\
\url{https://doi.org/10.1515/9783598441844}


%% Chapter 06 references
\section{References from Chapter 6}

\bibitem[Stodden 2016]{ch6-1}
Victoria Stodden, Marcia McNutt, David H. Bailey, Ewa Deelman,
Yolanda Gil, Brooks Hanson, Michael A. Heroux, John P.A. Ioannidis,
Michela Taufer (2016):\\
\textbf{Enhancing reproducibility for computational methods.}
\emph{Science} \textbf{354}(6317) pp 1240--1241\\
\url{https://doi.org/10.1126/science.aah6168}

\bibitem[Leipzig 2021]{ch6-2}
Jeremy Leipzig, Daniel Nüst, Charles Tapley Hoyt, Karthik Ram,
Jane Greenberg (2021):\\
\textbf{The role of metadata in reproducible computational research}.
\emph{Patterns} \textbf{2}(9):100322.\\
\url{https://doi.org/10.1016/j.patter.2021.100322}

\bibitem[Katz 2021]{ch6-3}
Daniel S. Katz, Morane Gruenpeter, Tom Honeyman, Lorraine Hwang,
Mark D. Wilkinson, Vanessa Sochat, Hartwig Anzt, Carole Goble, FAIR4RS
Subgroup 1 (2021):\\
\textbf{A Fresh Look at FAIR for Research Software}.\\
\href{https://arxiv.org/abs/2101.10883}{arXiv:2101.10883}

\bibitem[Möller 2017]{ch6-4}
Steffen Möller, Stuart W. Prescott, Lars Wirzenius, Petter
Reinholdtsen, Brad Chapman, Pjotr Prins, Stian Soiland-Reyes, Fabian
Klötzl, Andrea Bagnacani, Matúš Kalaš, Andreas Tille, Michael R. Crusoe
(2017):\\
\textbf{Robust cross-platform workflows: How technical and scientific
communities collaborate to develop, test and share best practices for
data analysis}.\\
\emph{Data Science and Engineering} \textbf{2}\\
\url{https://doi.org/10.1007/s41019-017-0050-4}

\bibitem[Cohen-Boulakia 2017]{ch6-5}
Sarah Cohen-Boulakia, Khalid Belhajjame, Olivier Collin, Jérôme
Chopard, Christine Froidevaux, Alban Gaignard, Konrad Hinsen, Pierre
Larmande, Yvan Le Bras, Frédéric Lemoine, Fabien Mareuil, Hervé Ménager,
Christophe Pradal, Christophe Blanchet (2017):\\
\textbf{Scientific Workflows for Computational Reproducibility in the
Life Sciences: Status, Challenges and Opportunities}. \emph{Future
Generation Computer Systems} \textbf{75} pp 284--298.\\
\url{https://doi.org/10.1016/j.future.2017.01.012}

\bibitem[Grüning 2018]{ch6-6}
Björn Grüning, John Chilton, Johannes Köster, Ryan Dale, Nicola
Soranzo, Marius van den Beek, Jeremy Goecks, Rolf Backofen, Anton
Nekrutenko, James Taylor (2018):\\
\textbf{Practical Computational Reproducibility in the Life Sciences}.
\emph{Cell Systems} \textbf{6}(6) pp 631--635.\\
\url{https://doi.org/10.1016/j.cels.2018.03.014}

\bibitem[Lamprecht]{ch6-7}
Anna-Lena Lamprecht, Leyla Garcia, Mateusz Kuzak, Carlos
Martinez, Ricardo Arcila, Eva Martin Del Pico, Victoria Dominguez Del
Angel, et al.~\textbf{Towards FAIR Principles for Research Software}.
\emph{Data Science} \textbf{3}(1) pp 37--59.\\
\url{https://doi.org/10.3233/ds-190026}

\bibitem[De Smedt 2020]{ch6-8}
Koenraad De Smedt, Dimitris Koureas, Peter Wittenburg (2020):\\
\textbf{FAIR Digital Objects for Science: From Data Pieces to Actionable
Knowledge Units}.\\
\emph{Publications} \textbf{8}(2):21\\
\url{https://doi.org/10.3390/publications8020021}

\bibitem[Goble 2020]{ch6-9}
Carole Goble, Sarah Cohen-Boulakia, Stian Soiland-Reyes, Daniel
Garijo, Yolanda Gil, Michael R. Crusoe, Kristian Peters, Daniel Schober
(2020):\\
\textbf{FAIR Computational Workflows}. \emph{Data Intelligence}
\textbf{2}(1) pp 108--121.\\
\url{https://doi.org/10.1162/dint_a_00033}

\bibitem[Andrio 2019]{ch6-10}
Pau Andrio, Adam Hospital, Javier Conejero, Luis Jordá, Marc
Del Pino, Laia Codo, Stian Soiland-Reyes, Carole Goble, Daniele Lezzi,
Rosa M. Badia, Modesto Orozco, Josep Ll. Gelpi (2019):\\
\textbf{BioExcel Building Blocks, a software library for interoperable
biomolecular simulation workflows}. \emph{Scientific Data}
\textbf{6}:169\\
\url{https://doi.org/10.1038/s41597-019-0177-4}

\bibitem[Ison 2013]{ch6-11}
Jon Ison, Matúš Kalaš, Inge Jonassen, Dan Bolser, Mahmut
Uludag, Hamish McWilliam, James Malone, Rodrigo Lopez, Steve Pettifer,
Peter Rice (2013):\\
\textbf{EDAM: an ontology of bioinformatics operations, types of data
and identifiers, topics and formats}. \emph{Bioinformatics}
\textbf{29}(10) pp 1325--1332.\\
\url{https://doi.org/10.1093/bioinformatics/btt113}

\bibitem[Hospital 2020]{ch6-12}
Adam Hospital, Genís Bayarri, Stian Soiland-Reyes, Jose Lluis
Gelpi, Pau Andrio, Daniele Lezzi, Sarah Butcher, Ania Niewielska, Yvonne
Westermaier, Rosa Maria Badia, Rodrigo Vargas, Alexandre Bonvin
(2020):\\
\textbf{BioExcel-2 Deliverable 2.3 -- First release of demonstration
workflows.} Project deliverable, \emph{Zenodo}.\\
\url{https://doi.org/10.5281/zenodo.4540432}

\bibitem[Kluyver 2016]{ch6-13}
Thomas Kluyver, Benjamin Ragan-Kelley, Fernando Pérez, Brian
Granger, Matthias Bussonnier, Jonathan Frederic, Kyle Kelley, Jessica
Hamrick, Jason Grout, Sylvain Corlay, Paul Ivanov, Damián Avila, Safia
Abdalla, Carol Willing, Jupyter Development Team (2016):\\
\textbf{Jupyter Notebooks -- a publishing format for reproducible
computational workflows}. \emph{Positioning and Power in Academic
Publishing: Players, Agents and Agendas} pp 87 - 90. Fernando Loizides,
Birgit Schmidt (eds), Proceedings of the 20th International Conference
on Electronic Publishing.\\
\url{https://doi.org/10.3233/978-1-61499-649-1-87}

\bibitem[Beg 2021]{ch6-14}
Marijan Beg, Juliette Taka, Thomas Kluyver, Alexander
Konovalov, Min Ragan-Kelley, Nicolas M. Thiery, Hans Fangohr (2021):\\
\textbf{Using Jupyter for Reproducible Scientific Workflows}.
\emph{Computing in Science \& Engineering} \textbf{23}(2) pp 36--46.\\
\url{https://doi.org/10.1109/mcse.2021.3052101}

\bibitem[Jupyter 2018]{ch6-15}
Jupyter Project, Matthias Bussonnier, Jessica Forde, Jeremy
Freeman, Brian Granger, Tim Head, Chris Holdgraf, et al.~(2018):\\
\textbf{Binder 2.0 - Reproducible, Interactive, Sharable Environments
for Science at Scale}. \emph{Proceedings of the 17th Python in Science
Conference}. SciPy, 2018.\\
\url{https://doi.org/10.25080/majora-4af1f417-011}

\bibitem[Grüning 2018]{ch6-16}
Björn Grüning, Ryan Dale, Andreas Sjödin, Brad A. Chapman,
Jillian Rowe, Christopher H. Tomkins-Tinch, Renan Valieris, the Bioconda
Team, Johannes Köster (2018):\\
\textbf{Bioconda: Sustainable and Comprehensive Software Distribution
for the Life Sciences}. \emph{Nature Methods} \textbf{15} pp 475--476.\\
\url{https://doi.org/10.1038/s41592-018-0046-7}.

\bibitem[Niewielska 2020]{ch6-17}
Ania Niewielska, Sarah Butcher, Yvonne Westermaier (2020):\\
\textbf{BioExcel-2 Deliverable 2.5 - Provision of a Workflow Environment
at BioExcel portal}. \emph{Zenodo}\\
\url{https://doi.org/10.5281/zenodo.4916060}

\bibitem[Afgan 2018]{ch6-18}
Enis Afgan, Dannon Baker, Bérénice Batut, Marius van den Beek,
Dave Bouvier, Martin Čech, John Chilton, Dave Clements, Nate Coraor,
Björn Grüning, Aysam Guerler, Jennifer Hillman-Jackson, Vahid Jalili,
Helena Rasche, Nicola Soranzo, Jeremy Goecks, James Taylor, Anton
Nekrutenko, and Daniel Blankenberg (2018):\\
\textbf{The Galaxy platform for accessible, reproducible and
collaborative biomedical analyses: 2018 update}. \emph{Nucleic Acids
Research} \textbf{46}\emph{(W1)} pp W537--W544.\\
\url{https://doi.org/10.1093/nar/gky379}

\bibitem[Crusoe 2022]{ch6-19}
Michael R. Crusoe, Sanne Abeln, Alexandru Iosup, Peter Amstutz,
John Chilton, Nebojša Tijanić, Hervé Ménager, Stian Soiland-Reyes,
Carole Goble, The CWL Community (2022):\\
\textbf{Methods Included: Standardizing Computational Reuse and
Portability with the Common Workflow Language}. \emph{Communication of
the ACM} (in press). \emph{arXiv}:2105.07028\\
\url{https://doi.org/10.1145/3486897}

\bibitem[Tejedor 2017]{ch6-20}
Enric Tejedor, Yolanda Becerra, Guillem Alomar, Anna Queralt,
Rosa M. Badia, Jordi Torres, Toni Cortes, Jesús Labarta (2017):\\
\textbf{PyCOMPSs: Parallel computational workflows in Python}. \emph{The
International Journal of High Performance Computing Applications}
\textbf{31}(1) pp 66--82.\\
\url{https://doi.org/10.1177/1094342015594678}

\bibitem[Fillbrunn 2017]{ch6-21}
Alexander Fillbrunn, Christian Dietz, Julianus Pfeuffer, René
Rahn, Gregory A. Landrum, Michael R. Berthold (2017):\\
\textbf{KNIME for reproducible cross-domain analysis of life science
data}. \emph{Journal of Biotechnology} \textbf{261} pp 149--156.\\
\url{https://doi.org/10.1016/j.jbiotec.2017.07.028}

\bibitem[Lowe 2021]{ch6-22}
Douglas Lowe (2021):\\
\textbf{Protein MD Setup tutorial using BioExcel Building Blocks (biobb)
in Galaxy}. \emph{WorkflowHub.} Workflow (Galaxy).\\
\url{https://doi.org/10.48546/workflowhub.workflow.194.1}

\bibitem[Hospital 2021]{ch6-23}
Adam Hospital (2021):\\
\textbf{Protein MD Setup tutorial using BioExcel Building Blocks (biobb)
in KNIME}. \emph{WorkflowHub}. Workflow (KNIME).\\
\url{https://doi.org/10.48546/workflowhub.workflow.201.1}

\bibitem[Bayarri 2021a]{ch6-24}
Genís Bayarri, Robin Long (2021):\\
\textbf{Protein MD Setup tutorial using BioExcel Building Blocks (biobb)
in CWL}. \emph{WorkflowHub}. Workflow (CWL).\\
\url{https://doi.org/10.48546/workflowhub.workflow.29.3}

\bibitem[Bayarri 2021b]{ch6-25}
Genís Bayarri, Adam Hospital, Douglas Lowe (2021):\\
\textbf{Protein MD Setup tutorial using BioExcel Building Blocks (biobb)
in Jupyter Notebook}. \emph{WorkflowHub}. Workflow (Jupyter Notebook).\\
\url{https://doi.org/10.48546/workflowhub.workflow.120.2}

\bibitem[Hospital 2021]{ch6-26}
Adam Hospital, Pau Andrio (2021):\\
\textbf{Protein MD Setup HPC tutorial using BioExcel Building Blocks
(biobb) in PyCOMPSs}. \emph{WorkflowHub}. Workflow (PyCOMPSs).\\
\url{https://doi.org/10.48546/workflowhub.workflow.200.1}

\bibitem[BioMoby 2008]{ch6-27}
The BioMoby Consortium (2008):\\
\textbf{Interoperability with Moby 1.0---It's better than sharing your
toothbrush!} \emph{Briefings in Bioinformatics} \textbf{9}(3) pp
220--231.\\
\url{https://doi.org/10.1093/bib/bbn003}

\bibitem[Saltz 2006]{ch6-28}
Joel Saltz, Scott Oster, Shannon Hastings, Stephen Langella,
Tahsin Kurc, William Sanchez, Manav Kher, Arumani Manisundaram,
Krishnakant Shanbhag, Peter Covitz (2006):\\
\textbf{caGrid: design and implementation of the core architecture of
the cancer biomedical informatics grid}. \emph{Bioinformatics}
\textbf{22}(15) pp 1910--1916.\\
\url{https://doi.org/10.1093/bioinformatics/btl272}

\bibitem[Garijo 2011]{ch6-29}
Daniel Garijo, Yolanda Gil (2011):\\
\textbf{A New Approach for Publishing Workflows}. \emph{Proceedings of
the 6th Workshop on Workflows in Support of Large-Scale Science - WORKS
'11}.\\
\url{https://doi.org/10.1145/2110497.2110504}

\bibitem[Garijo 2014]{ch6-30}
Daniel Garijo, Pinar Alper, Khalid Belhajjame, Oscar Corcho,
Yolanda Gil, Carole Goble (2014):\\
\textbf{Common Motifs in Scientific Workflows: An Empirical Analysis}.
\emph{Future Generation Computer Systems} \textbf{36} pp 338--351.\\
\url{https://doi.org/10.1016/j.future.2013.09.018}

\bibitem[De Giovanni 2016]{ch6-31}
Renato De Giovanni, Alan R. Williams, Vera Hernández Ernst,
Robert Kulawik, Francisco Quevedo Fernandez, Alex R. Hardisty (2016):\\
\textbf{ENM Components: a new set of web service‐based workflow
components for ecological niche modelling.} \emph{Ecography}
\textbf{39}(4) \textbf{pp} 376--383.\\
\url{https://doi.org/10.1111/ecog.01552}

\bibitem[Blankenberg 2014]{ch6-32}
Daniel Blankenberg, Gregory Von Kuster, Emil Bouvier, Dannon
Baker, Enis Afgan, Nicholas Stoler, James Taylor, Anton Nekrutenko, the
Galaxy Team (2014):\\
\textbf{Dissemination of scientific software with Galaxy ToolShed}.
\emph{Genome Biology} \textbf{15}:403\\
\url{https://doi.org/10.1186/gb4161}

\bibitem[Vivian 2017]{ch6-33}
John Vivian, Arjun Arkal Rao, Frank Austin Nothaft, Christopher
Ketchum, Joel Armstrong, Adam Novak, Jacob Pfeil, Jake Narkizian, Alden
D Deran, Audrey Musselman-Brown, Hannes Schmidt, Peter Amstutz, Brian
Craft, Mary Goldman, Kate Rosenbloom, Melissa Cline, Brian O'Connor,
Megan Hanna, Chet Birger, W James Kent, David A Patterson, Anthony D
Joseph, Jingchun Zhu, Sasha Zaranek, Gad Getz, David Haussler \&
Benedict Paten (2017):\\
\textbf{Toil enables reproducible, open source, big biomedical data
analyses}. \emph{Nature Biotechnology} \textbf{35} pp 314--316.\\
\url{https://doi.org/10.1038/nbt.3772}

\bibitem[Soiland-Reyes 2022]{ch6-34}
Stian Soiland-Reyes, Peter Sefton, Mercè Crosas, Leyla Jael
Castro, Frederik Coppens, José M. Fernández, Daniel Garijo, Björn
Grüning, Marco La Rosa, Simone Leo, Eoghan Ó Carragáin, Marc Portier,
Ana Trisovic, RO-Crate Community, Paul Groth, Carole Goble (2022):\\
\textbf{Packaging research artefacts with RO-Crate}. \emph{Data Science}
(pre-press).\\
\url{https://doi.org/10.3233/DS-210053}

\bibitem[Ison 2021]{ch6-35}
Jon Ison, Hans Ienasescu, Emil Rydza, Piotr Chmura, Kristoffer
Rapacki, Alban Gaignard, Veit Schwämmle, Jacques van Helden, Matúš
Kalaš, Hervé Ménager (2021):\\
\textbf{biotoolsSchema: a formalized schema for bioinformatics software
description.} \emph{GigaScience}, \textbf{10}(1):giaa157\\
\url{https://doi.org/10.1093/gigascience/giaa157}

\bibitem[CWFR 2021]{ch6-36}
The CWFR Group, Alex Hardisty (ed.), Peter Wittenburg (ed.)
(2021):\\
\textbf{CWFR Position Paper}. \emph{OSF}, January 6, 2021.\\
\url{https://osf.io/3rekv/}

\bibitem[Brack 2022]{ch6-37}
Paul Brack, Peter Crowther, Stian Soiland-Reyes, Stuart Owen,
Douglas Lowe, Alan R Williams, Quentin Groom, Mathias Dillen, Frederik
Coppens, Björn Grüning, Ignacio Eguinoa, Phil Ewels, Carole Goble
(2022):\\
\textbf{10 Simple Rules for making a software tool workflow-ready}.
\emph{PLOS Computational Biology} (accepted), preprint
PCOMPBIOL-D-21-01704R1 at \url{https://zenodo.org/record/5901220}

\bibitem[McMurry 2017]{ch6-38}
Julie A McMurry, Nick Juty, Niklas Blomberg, Tony Burdett, Tom
Conlin, Nathalie Conte, Mélanie Courtot, John Deck, Michel Dumontier,
Donal K Fellows, Alejandra Gonzalez-Beltran, Philipp Gormanns, Jeffrey
Grethe, Janna Hastings, Jean-Karim Hériché, Henning Hermjakob, Jon C
Ison, Rafael C Jimenez, Simon Jupp, John Kunze, Camille Laibe, Nicolas
Le Novère, James Malone, Maria Jesus Martin, Johanna R McEntyre, Chris
Morris, Juha Muilu, Wolfgang Müller, Philippe Rocca-Serra,
Susanna-Assunta Sansone, Murat Sariyar, Jacky L Snoep, Stian
Soiland-Reyes, Natalie J Stanford, Neil Swainston, Nicole Washington,
Alan R Williams, Sarala M Wimalaratne, Lilly M Winfree, Katherine
Wolstencroft, Carole Goble, Cristopher J Mungall, Melissa A Haendel,
Helen Parkinson (2017):\\
\textbf{Identifiers for the 21st century: How to design, provision, and
reuse identifiers to maximize utility and impact of life science data}.
\emph{PLOS Biology} \textbf{15}(6):e2001414\\
\url{https://doi.org/10.1371/journal.pbio.2001414}

\bibitem[Ferreira da Silva 2021]{ch6-39}
Rafael Ferreira da Silva, Henri Casanova, Kyle Chard, Ilkay
Altintas, Rosa M Badia, Bartosz Balis, Tainã Coleman, Frederik Coppens,
Frank Di Natale, Bjoern Enders, Thomas Fahringer, Rosa Filgueira,
Grigori Fursin, Daniel Garijo, Carole Goble, Dorran Howell, Shantenu
Jha, Daniel S. Katz, Daniel Laney, Ulf Leser, Maciej Malawski, Kshitij
Mehta, Loïc Pottier, Jonathan Ozik, J. Luc Peterson, Lavanya
Ramakrishnan, Stian Soiland-Reyes, Douglas Thain, Matthew Wolf (2021):\\
\textbf{A Community Roadmap for Scientific Workflows Research and
Development}.\\
\href{https://arxiv.org/abs/2110.02168}{arXiv:2110.02168} 

\bibitem[Garcia 2020]{ch6-40}
Leyla Garcia, Erick Antezana, Alexander Garcia, Evan Bolton,
Rafael Jimenez, Pjotr Prins, Juan M. Banda, Toshiaki Katayama (2020):\\
\textbf{Ten simple rules to run a successful BioHackathon}. \emph{PLOS
Computational Biology} \textbf{16}(5):e1007808.\\
\url{https://doi.org/10.1371/journal.pcbi.1007808}


%% Chapter 08 references
\section{References from Chapter 8}

\bibitem[Walton 2020]{ch8-1}
Walton S, Livermore L, Bánki O, Cubey RWN, Drinkwater R, Englund
M, Goble C, Groom Q, Kermorvant C, Rey I, Santos CM, Scott B, Williams
AR, Wu Z (2020):\\
Landscape Analysis for the Specimen Data Refinery.\\
Research Ideas and Outcomes 6: e57602.\\
\url{https://doi.org/10.3897/rio.6.e57602}

\bibitem[Thiers 2016]{ch8-2}
Thiers, B. M., Tulig, M. C., \& Watson, K. A. (2016):\\
Digitization of the New York Botanical Garden herbarium.\\
Brittonia, 68(3), 324-333.\\
\url{https://doi.org/10.1007/s12228-016-9423-7}

\bibitem[Nelson 2019a]{ch8-3}
Nelson, G., Ellis, S., 2019:\\
The history and impact of digitization and digital data mobilization on
biodiversity research.\\
Philosophical Transactions of the Royal Society B: Biological Sciences
374, 20170391.\\
\url{https://doi.org/10.1098/rstb.2017.0391}

\bibitem[Nelson 2019b]{ch8-4}
Nelson, G., Paul, D.L., (2019):\\
DiSSCo, iDigBio and the Future of Global Collaboration.\\
\emph{Biodiversity Information Science and Standards}
\textbf{3}:e37896.\\
\url{https://doi.org/10.3897/biss.3.37896}

\bibitem[Addink 2019]{ch8-5}
Addink W, Koureas D, Rubio A. (2019):\\
DiSSCo as a New Regional Model for Scientific Collections in Europe.
\emph{Biodiversity Information Science and Standards}
\textbf{3}:e37502.\\
\url{https://doi.org/10.3897/biss.3.37502}

\bibitem[Lannom 2020]{ch8-6}
Lannom L, Koureas D, Hardisty AR (2020):\\
\textbf{FAIR Data and Services in Biodiversity Science and
Geoscience}.\\
\emph{Data Intelligence} \textbf{2}(1--2):122--130.\\
\url{https://doi.org/10.1162/dint_a_00034}

\bibitem[GBIF 2021]{ch8-7}
GBIF Secretariat. (2021). GBIF Science Review 2020.\\
\url{https://doi.org/10.35035/bezp-jj23}

\bibitem[Heberling 2021]{ch8-8}
Heberling, J. M., Miller, J. T., Noesgaard, D., Weingart, S. B.,
\& Schigel, D. (2021). Data integration enables global biodiversity
synthesis. Proceedings of the National Academy of Sciences, 118(6).\\
\url{https://doi.org/10.1073/pnas.2018093118}

\bibitem[Sweeney 2018]{ch8-9}
Sweeney, P.W., Starly, B., Morris, P.J., Xu, Y., Jones, A.,
Radhakrishnan, S., Grassa, C.J. and Davis, C.C. (2018), Large--scale
digitization of herbarium specimens: Development and usage of an
automated, high--throughput conveyor system. Taxon, 67: 165-178.\\
\url{https://doi.org/10.12705/671.10}

\bibitem[Allan 2019]{ch8-10}
Allan, E.L., Livermore, L., Price, B., Shchedrina, O., Smith,
V., (2019). A Novel Automated Mass Digitisation Workflow for Natural
History Microscope Slides. Biodiversity Data Journal 7, e32342.\\
\url{https://doi.org/10.3897/BDJ.7.e32342}

\bibitem[Hereld 2019]{ch8-11}
Hereld, M., Ferrier, N., (2019). LightningBug ONE: An
experiment in high-throughput digitization of pinned insects.
Biodiversity Information Science and Standards 3, e37228.\\
\url{https://doi.org/10.3897/biss.3.37228}

\bibitem[Price 2018]{ch8-12}
Price, B.W., Dupont, S., Allan, E.L., Blagoderov, V., Butcher,
A.J., Durrant, J., Holtzhausen, P., Kokkini, P., Livermore, L., Hardy,
H., Smith, V., (2018). ALICE: Angled Label Image Capture and Extraction
for high throughput insect specimen digitisation.\\
\url{https://doi.org/10.31219/osf.io/s2p73}

\bibitem[Tegelberg 2017]{ch8-13}
Tegelberg, R., Kahanpää, J., Karppinen, J., Mononen, T., Wu,
Z., Saarenmaa, H., (2017). Mass Digitization of Individual Pinned
Insects Using Conveyor-Driven Imaging, in: 2017 IEEE 13th International
Conference on E-Science (e-Science). Presented at the 2017 IEEE 13th
International Conference on e-Science (e-Science), pp.~523--527.\\
\url{https://doi.org/10.1109/eScience.2017.85}

\bibitem[Heberling 2019]{ch8-14}
Heberling, J.M., Prather, L.A., Tonsor, S.J. (2019) The
Changing Uses of Herbarium Data in an Era of Global Change: An Overview
Using Automated Content Analysis, BioScience, Volume 69, Issue 10,
October 2019, Pages 812--822,\\
\url{https://doi.org/10.1093/biosci/biz094}

\bibitem[Kharouba 2019]{ch8-15}
Kharouba Heather M., Lewthwaite Jayme M. M., Guralnick Rob,
Kerr Jeremy T. and Vellend Mark (2019) Using insect natural history
collections to study global change impacts: challenges and
opportunities. Phil. Trans. R. Soc. B3742017040520170405.\\
\url{https://doi.org/10.1098/rstb.2017.0405}

\bibitem[Watanabe 2019]{ch8-16}
Myrna E Watanabe, The Evolution of Natural History Collections:
New research tools move specimens, data to center stage., BioScience,
Volume 69, Issue 3, March 2019, Pages 163--169,\\
\url{https://doi.org/10.1093/biosci/biy163}

\bibitem[Lughadha 2019]{ch8-17}
Nic Lughadha, E.M., Graziele Staggemeier, V., Vasconcelos,
T.N.C., Walker, B.E., Canteiro, C. and Lucas, E.J. (2019), Harnessing
the potential of integrated systematics for conservation of
taxonomically complex, megadiverse plant groups. Conservation Biology,
33: 511-522.\\
\url{https://doi.org/10.1111/cobi.13289}

\bibitem[Livermore 2020]{ch8-18}
Owen D, Livermore L, Groom Q, Hardisty A, Leegwater T, van
Walsum M, et al.~(2020). Towards a scientific workflow featuring Natural
Language Processing for the digitisation of natural history collections.
Research Ideas and Outcomes 6: e58030.\\
\url{https://doi.org/10.3897/rio.6.e58030}

\bibitem[Harrow 2021]{ch8-19}
Jennifer Harrow, John Hancock, ELIXIR-EXCELERATE Community,
Niklas Blomberg (2021): ELIXIR-EXCELERATE: establishing Europe's data
infrastructure for the life science research of the future. \emph{EMBO
Journal} \textbf{40}(6):e107409\\
\url{https://doi.org/10.15252/embj.2020107409}

\bibitem[Afgan 2018]{ch8-20}
Afgan, E., Baker, D., Batut, B., van den Beek, M., Bouvier, D.,
Čech, M., et al.~(2018). The Galaxy platform for accessible,
reproducible and collaborative biomedical analyses: 2018 update. Nucleic
Acids Research 46, W537--W544.\\
\url{https://doi.org/10.1093/nar/gky379}

\bibitem[Crusoe 2021]{ch8-21}
Crusoe, M.R., Abeln, S., Iosup, A., Amstutz, P., Chilton, J.,
Tijanić, N., et al., (2021). Methods Included: Standardizing
Computational Reuse and Portability with the Common Workflow Language.\\
arXiv:2105.07028

\bibitem[Ó Carragáin 2019]{ch8-22}
Carragáin, E.Ó., Goble, C., Sefton, P., Soiland-Reyes, S.,
(2019). A lightweight approach to research object data packaging.
\emph{Bioinformatics Open Source Conference} (BOSC), ISMB/ECCB 2019,
Basel, Switzerland, 24-25 July 2019 (Session 1, Part 1075).\\
\url{https://doi.org/10.5281/zenodo.3250687}

\bibitem[Soiland-Reyes 2021]{ch8-23}
Soiland-Reyes, S., Sefton, P., Crosas, M., Castro, L.J.,
Coppens, F., Fernández, J.M., et al.~(2021). Packaging research
artefacts with RO-Crate. \emph{Data Science} (accepted).\\
arXiv:2108.06503.

\bibitem[Goble 2021]{ch8-24}
Goble, C., Soiland-Reyes, S., Bacall, F., Owen, S., Williams,
A., Eguinoa, I., et al.~(2021). Implementing FAIR Digital Objects in the
EOSC-Life Workflow Collaboratory. Zenodo.\\
\url{https://doi.org/10.5281/zenodo.4605654}

\bibitem[Addink 2019]{ch8-25}
Addink, W., Koureas, D., and Rubio, A. (2019). DiSSCo as a New
Regional Model for Scientific Collections in Europe. Biodiversity
Information Science and Standards, 3, e37502.\\
\url{https://doi.org/10.3897/biss.3.37502}

\bibitem[Wilkinson 2016]{ch8-26}
Wilkinson, M.D., Dumontier, M., Aalbersberg, Ij.J., Appleton,
G., Axton, M., Baak, et al.~(2016). The FAIR Guiding Principles for
scientific data management and stewardship. Scientific Data 3, 160018.\\
\url{https://doi.org/10.1038/sdata.2016.18}

\bibitem[Wittenburg 2021]{ch8-27}
Wittenburg, P., Hardisty, A., Peer, L., and LeFranc, Y. (2021).
Canonical Workflows to Make Data FAIR. \emph{Data Intelligence} - this
special issue.

\bibitem[Hardisty 2019]{ch8-28}
Hardisty, A. (2019). Provisional Data Management Plan for
DiSSCo infrastructure. Deliverable D6.6. ICEDIG.\\
\url{https://doi.org/10.5281/zenodo.3532937}

\bibitem[De Smedt 2020]{ch8-29}
Koenraad De Smedt, Dimitris Koureas, Peter Wittenburg (2020):\\
\textbf{FAIR Digital Objects for Science: From Data Pieces to Actionable
Knowledge Units}.\\
\emph{Publications} \textbf{8}(2):21\\
\url{https://doi.org/10.3390/publications8020021}

\bibitem[Hardisty 2020]{ch8-30}
Hardisty, A., Saarenmaa, H., Casino, A., Dillen, M., Gödderz,
K., Groom, Q., et al.~(2020). Conceptual design blueprint for the DiSSCo
digitization infrastructure - DELIVERABLE D8.1. Research Ideas and
Outcomes 6: e54280.\\
\url{https://doi.org/10.3897/rio.6.e54280}

\bibitem[FDO 2020]{ch8-31}
FDO Coordination Group (2020) FDO Framework.\\
\url{https://github.com/GEDE-RDA-Europe/GEDE/tree/master/FAIR\%20Digital\%20Objects/FDOF}
(accessed August 10, 2021).

\bibitem[Triki 2020]{ch8-32}
Triki, A., Bouaziz, B., Mahdi, W., \& Gaikwad, J. (2020).
Objects Detection from Digitized Herbarium Specimen based on Improved
YOLO V3. In VISIGRAPP (4: VISAPP) (pp.~523-529).\\
\url{https://doi.org/10.5220/0009170005230529}

\bibitem[Nieva de la Hidalga 2021]{ch8-33}
Nieva de la Hidalga, A., Rosin, P.L., Sun, X., Livermore, L.,
Durran, J., Turner, J., Dillen, M., Musson, A., Phillips, S., Groom, Q.,
and Hardisty, A.R. (2021) Cross-validation of a semantic segmentation
network for natural history collection specimens. (\emph{in second round
of review at present}).

\bibitem[Livermore 2020]{ch8-34}
Walton S, Livermore L, Dillen M, De Smedt S, Groom Q, Koivunen
A, Phillips S (2020) A cost analysis of transcription systems. Research
Ideas and Outcomes 6: e56211.\\
\url{https://doi.org/10.3897/rio.6.e56211}

\bibitem[Groom 2020]{ch8-35}
Groom, Q., Güntsch, A., Huybrechts, P., Kearney, N., Leachman,
S., Nicolson, N., et al.~(2020). People are essential to linking
biodiversity data, Database 2020.\\
\url{https://doi.org/10.1093/database/baaa072}

\bibitem[Knyshov 2021]{ch8-36}
Knyshov, A., Hoang, S., Weirauch, C., (2021). Pretrained
Convolutional Neural Networks Perform Well in a Challenging Test Case:
Identification of Plant Bugs (Hemiptera: Miridae) Using a Small Number
of Training Images. Insect Systematics and Diversity, Volume 5, Issue 2,
March 2021, 3.\\
\url{https://doi.org/10.1093/isd/ixab004}

\bibitem[Hussein 2021]{ch8-37}
Hussein, B.R., Malik, O.A., Ong, W.H., Slik, J.W.F.. (2021).
Application of Computer Vision and Machine Learning for Digitized
Herbarium Specimens: A Systematic Literature Review.\\
\url{https://arxiv.org/abs/2104.08732v1}

\bibitem[Carranza-Rojas 2017]{ch8-38}
Carranza-Rojas, J., Goeau, H., Bonnet, P. et al.~Going deeper
in the automated identification of Herbarium specimens. BMC Evol Biol
17, 181 (2017).\\
\url{https://doi.org/10.1186/s12862-017-1014-z}

\bibitem[Little 2020]{ch8-39}
Little, D. P., Tulig, M., Tan, K. C., Liu, Y., Belongie, S.,
Kaeser‐Chen, C., Michelangeli, F. A., et al.~2020. An algorithm
competition for automatic species identification from herbarium
specimens. Applications in Plant Sciences 8( 6): e11365.\\
\url{https://doi.org/10.1002/aps3.11365}

\bibitem[Pryer 2022]{ch8-40}
Pryer, K. M., Tomasi, C., Wang, X., Meineke, E. K., and
Windham, M. D.. 2020. Using computer vision on herbarium specimen images
to discriminate among closely related horsetails (Equisetum).
Applications in Plant Sciences 8( 6): e11372.

\bibitem[Unger 2016]{ch8-41}
Unger, J., Merhof, D. \& Renner, S. Computer vision applied to
herbarium specimens of German trees: testing the future utility of the
millions of herbarium specimen images for automated identification. BMC
Evol Biol 16, 248 (2016).\\
\url{https://doi.org/10.1186/s12862-016-0827-5}

\bibitem[Atkinson 2017]{ch8-42}
Atkinson, M., Gesing, S., Montagnat, J., Taylor, I., 2017.
Scientific workflows: Past, present and future. Future Generation
Computer Systems 75, 216--227.\\
\url{https://doi.org/10.1016/j.future.2017.05.041}

\bibitem[Amstutz 2021]{ch8-43}
Peter Amstutz, Maxim Mikheev, Michael R. Crusoe, Nebojša
Tijanić, Samuel Lampa, et al.~(2021): Existing Workflow systems.
\emph{Common Workflow Language wiki}, GitHub.\\
\url{https://s.apache.org/existing-workflow-systems} updated 2021-06-29.
(accessed September 9th, 2021)

\bibitem[Hui 2012]{ch8-44}
Hui, Y., (2012). What is a Digital Object? Metaphilosophy 43,
380--395.\\
\url{https://doi.org/10.1111/j.1467-9973.2012.01761.x}

\bibitem[Kallinikos 2013]{ch8-45}
Kallinikos, J., Aaltonen, A., Marton, A., (2013). The
Ambivalent Ontology of Digital Artifacts. MIS Quarterly 37, 357--370.
url:\\
\url{https://www.jstor.org/stable/43825913}

\bibitem[Kahn 2006]{ch8-46}
Kahn, R., Wilensky, R., (2006). A framework for distributed
digital object services. Int J Digit Libr 6, 115--123.\\
\url{https://doi.org/10.1007/s00799-005-0128-x}

\bibitem[openDS 2021]{ch8-47}
openDS. (2021). Draft specification for open Digital Specimens
(openDS). url:\\
\url{https://github.com/DiSSCo/openDS}. (accessed August 10, 2021).

\bibitem[Bray 2017]{ch8-48}
Bray, T., (2017). The JavaScript Object Notation (JSON) Data
Interchange Format (Request for Comments No.~RFC 8259). Internet
Engineering Task Force.\\
\url{https://doi.org/10.17487/RFC8259}

\bibitem[Bechhofer 2013]{ch8-49}
Bechhofer, S., Buchan, I., De Roure, D., Missier, P.,
Ainsworth, J., Bhagat, et al.~(2013). Why linked data is not enough for
scientists. Future Generation Computer Systems, Special section: Recent
advances in e-Science 29, 599--611.\\
\url{https://doi.org/10.1016/j.future.2011.08.004}

\bibitem[Kellogg 2020]{ch8-50}
Kellogg, G., Champin, P.A., Longley, D. (2020). JSON-LD 1.1 A
JSON-based Serialization for Linked Data. W3C Recommendation. Latest
published version:\\
\url{https://www.w3.org/TR/json-ld11/}

\bibitem[schema.org]{ch8-51}
Schema.org - Schema.org {[}WWW Document{]}, n.d. URL\\
\url{https://schema.org/} (accessed August 10, 2021).

\bibitem[Corcho 2021]{ch8-52}
Corcho, O., González, E., Garijo, D., Palma, R. (2021). D5.1 RO
Model Adapted to EOSC. RELIANCE deliverable, Zenodo.\\
\url{https://doi.org/10.5281/zenodo.4913285}

\bibitem[Goble 2021]{ch8-53}
Goble, C., Soiland-Reyes, S., Bacall, F., Owen, S., Williams,
A., Eguinoa, I., et al.~(2021). Implementing FAIR Digital Objects in the
EOSC-Life Workflow Collaboratory. Zenodo. Preprint:\\
\url{https://doi.org/10.5281/zenodo.4605654}

\bibitem[Bacall 2021]{ch8-54}
Finn Bacall, Alan R Williams, Stian Soiland-Reyes (2021):
Workflow RO-Crate profile 1.0. \emph{WorkflowHub community}\\
\url{https://w3id.org/workflowhub/workflow-ro-crate/1.0} (accessed
November 17th, 2021)

\bibitem[Van de Sompel 2021]{ch8-55}
Van de Sompel, H., Klein, M., Jones, S., Nelson, M.L., Warner,
S., Devaraju, A., et al.~(2021). FAIR Signposting Profile. 2021-04-20.\\
\url{https://www.signposting.org/FAIR/} (accessed August 10, 2021).

\bibitem[Lohonya 2020]{ch8-56}
Lohonya, K., Livermore, L., Penn, M.G. (2020). Georeferencing
the Natural History Museum's Chinese type collection: of plateaus,
pagodas and plants. Biodiversity Data Journal 8: e50503.\\
\url{https://doi.org/10.3897/BDJ.8.e50503}

\bibitem[De Roure 2010]{ch8-57}
De Roure, D., and Goble, C. (2010). Anchors in shifting sand:
the primacy of method in the web of data. Web Science Conference 2010,
United States. 26 - 27 Apr 2010. University of Southampton Institutional
Repository\\
\url{https://eprints.soton.ac.uk/270817/} (accessed August 10, 2021).

\bibitem[Hardisty 2016]{ch8-58}
Hardisty, A.R., Bacall, F., Beard, N., Balcázar-Vargas, M.-P.,
Balech, B., Barcza, Z., et al.~(2016). BioVeL: a virtual laboratory for
data analysis and modelling in biodiversity science and ecology. BMC
Ecology 16, 49.\\
\url{https://doi.org/10.1186/s12898-016-0103-y}

\bibitem[Dillen 2019]{ch8-59}
Dillen, M., Groom, Q., Chagnoux, S., Güntsch, A., Hardisty, A.,
Haston, E., et al.~(2019). A benchmark dataset of herbarium specimen
images with label data. Biodiversity Data Journal 7: e31817.\\
\url{https://doi.org/10.3897/BDJ.7.e31817}

\bibitem[Gössner 2021]{ch8-60}
Stefan Gössner, Glyn Normington, and Carsten Bormann. 2021.
``JSONPath: Query Expressions for JSON.'' Internet-Draft
draft-ietf-jsonpath-base-02. Internet Engineering Task Force.\\
\url{https://datatracker.ietf.org/doc/html/draft-ietf-jsonpath-base-02}

\bibitem[De Roure 2010]{ch8-61}
De Roure, D., \& Goble, C. (2010, March). Anchors in Shifting
Sand: The Primacy of Method in the Web of Data.\\
\url{https://eprints.soton.ac.uk/270817/}

\bibitem[DONA DOA]{ch8-62}
DONA Foundation. Digital Object Architecture.\\
\url{https://www.dona.net/digitalobjectarchitecture} (accessed August
10, 2021)

\bibitem[DONA 2018]{ch8-63}
Digital Object Interface Protocol Specification, version 2.0,
November 2018.\\
\url{https://www.dona.net/sites/default/files/2018-11/DOIPv2Spec_1.pdf}

\bibitem[Sun 2003]{ch8-64}
Sun S., Reilly, S., Lannom L., Petrone J. (2003) RFC 3652
Handle System Protocol (ver 2.1) Specification. RFC Editor, USA.\\
\url{https://doi.org/10.17487/RFC3652}

\bibitem[Islam 2020]{ch8-65}
Islam, S, Hardisty, A, Addink, W, Weiland, C., and Glöckler, F.
(2020). Incorporating RDA Outputs in the Design of a European Research
Infrastructure for natural history Collections. Data Science Journal,
19: 50, pp.~1--14.\\
\url{https://doi.org/10.5334/dsj-2020-050}

\bibitem[Speicher 2015]{ch8-66}
Speicher, S., Arwe, J., and Malhotra, A. (2015). Linked Data
Platform 1.0. W3C Recommendation. Latest published version:\\
\url{https://www.w3.org/TR/ldp/}


\end{thebibliography}




%%%%%%%%%%%%%%%%%% APPENDICES %%%%%%%%%%%%%%%%%%
\begin{uomappendix} 
  \chapter{First appendix}
    \section{Section in Appendix}
    \lipsum[1-6]
\end{uomappendix}


%%%%%%%%%%%%%%%%%% END MATTER %%%%%%%%%%%%%%%%%%
\end{document}