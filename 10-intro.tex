In this thesis I investigate Linked Data approaches to implementing FAIR Research Objects and sharing reproducible Computational Workflows.

\section{Research Outline and Questions}

I explore three research questions on this topic:

\paragraph{RQ1:} Can the FAIR Digital Object concept be realized using Web technologies?

\paragraph{RQ2:} Can pragmatic use of Linked Data practices implement FAIR Research Objects for scholarly communication?

\paragraph{RQ3:} How can we use improve reproducibility and shareability of Computational Workflows?


\subsection{Aims of FAIR Digital Objects on the Web (RQ1)}

FAIR Digital Object (FDO) has been proposed as a machine-actionable ecosystem of scholarly outputs \cite{Schultes 2019}, in theory realizing the FAIR principles \cite{Wilkinson 2016} for a programmable mesh of strongly typed objects that go beyond the open data publication practices that the FAIR guidelines have popularized.

FDO specifications \cite{fdo-Overview} are conceptual in nature, however most existing Digital Object implementation \cite{Kahn 2006} rely on the protocols DOIP \cite{DigitalObjectInterface} and Handle system \cite{rfc3650}, neither of which are particularly familiar to software developers.

The Web, on the other side, is ubiquitous in modern software engineering, used for everything from user interfaces, mobile applications, cross-platform integrations and microservices hosted by cloud computing. 

A research question therefore is if the promising FDO concept can be realised using existing Web technology, taking into account the lessons learnt from the early Semantic Web developments and more recent Linked Data practices. 


\subsection{Aims of FAIR Research Objects (RQ2)}

Research Objects (RO) \cite{Bechhofer 2013} were proposed as a mechanism to capture a range of diverse scholarly outputs in a single archivable item, with detailed metadata to. The RO concept was realized using Semantic Web ontologies \cite{Belhajjame 2015} and primarily implemented for the purpose of long-term preservation of computational workflows and their outputs.

However the principles of Research Objects go much beyond workflows, which were utillised as a mechanism to capture computational methods. Following the lessons learnt on RO implementations and the FAIR principles, the updated aims of FAIR Research Objects can be summarised as:

\begin{itemize}
    \item Describe and package data collections, datasets, software etc. with their metadata
    \item Platform-independent object exchange between repositories and services
    \item Support reproducibility and analysis: link data with codes and workflows
    \item Transfer of sensitive/large distributed datasets with persistent identifiers
    \item Aggregate citations and persistent identifiers
    \item Propagate provenance and existing metadata
    \item Publish and archive mixed objects and references
    \item Reuse existing standards, but hide their complexity
\end{itemize}

A research question is then if a more pragmatic use of Linked Data practices can better implement Research Objects for a wider developer audience by using more familiar Web technologies and give lightweight recommendations rather than stringent ontologies.


\subsection{Aims of FAIR Computational Workflows (RQ3)}

(FAIR CW)



\section{Main Contributions}

FDO and evaluation
RO-Crate
Workflow Run Crate profiles
Broad adaptation of RO-Crate 



\section{Thesis Overview}

Chapter \vref{chapter:background} gives the background of concepts \emph{FAIR Digital Object} (FDO) and \emph{Linked Data}, including a brief history of the \emph{Semantic Web} and a critical analysis of these technologies and their use. 

Chapter \vref{chapter:fdo} consider RQ1 and contributes a framework-based evaluation of Linked Data and FDO as possible architectures for implementing a distributed object system for the purpose of FAIR data publishing, and discusses how the two approaches can benefit from each others strengths. 

Chapter \vref{chapter:ro-crate} RQ2 by introduces the contribution of \emph{RO-Crate}, a pragmatic data packaging mechanism using Linked Data standards to implement FDO and be extensible for domain-specific metadata.  

Chapter \vref{chapter:workflows} address RQ3 by exploring the relationship between Computational Workflows and FAIR practices using RO-Crate and FDO, with use cases from molecular dynamics and specimen digitization. The contribution of the \emph{Workflow Run Crate profiles} is presented as an interoperable way to capture and publish workflow execution provenance. 

Chapter \vref{chapter:conclusions} summarizes and concludes the contributions from this thesis, and reflects on later developments and future work.


\section{Origins}

Chapter \ref{ch3:background} and section \ref{ch3:evaluating-fdo-ld} is based on the preprint \emph{Evaluating FAIR Digital Object and Linked Data as distributed object systems} \cite{soilandreyes2023c}  (see appendices \vref{ch11:fdo} and \vref{ch10:fdo}). I am the main author of this manuscript.

Section \ref{ch2:updating-linked-data-practices-for-fair-digital-object-principles} is based on \emph{Updating Linked Data practices for FAIR Digital Object principles} \cite{10.3897/rio.8.e94501} (see appendices \ref{ch11:updating-ld} and \ref{ch10:updating-ld}). I am the main author of this manuscript.

Sections \ref{ch5:packaging-research-artefacts-with-ro-crate} and \vref{ch5:formaldefinition} are based on the publication \emph{Packaging research artefacts with RO-Crate} \cite{Soiland-Reyes 2022} (see appendices \ref{ch11:packagingrocrate}, \ref{ch10:packagingrocrate} and \ref{ch10:formalizing}). I am the main author of this manuscript.

Section \ref{ch4:lightweight-fdo} is based on the publication \emph{Creating lightweight FAIR digital objects with RO-Crate} \cite{10.3897/rio.8.e93937} (see appendices \ref{ch11:lightweight} and \ref{ch10:lightweight}). I am the main author of this manuscript.

Section \ref{ch6:making-canonical-workflow-building-blocks-interoperable-across-workflow-languages} is based on the publication \emph{Making Canonical Workflow Building Blocks interoperable across workflow languages} \cite{Soiland-Reyes 2022b} (see appendices \ref{ch11:canonical} and \ref{ch10:canonical}). I am the main author of this manuscript.

Section \ref{ch8:the-specimen-data-refinery} is based on the publication \emph{The Specimen Data Refinery: A
canonical workflow framework and FAIR Digital Object approach to speeding up digital mobilisation of natural history collections} \cite{Hardisty 2022} (see appendices \ref{ch11:refinery} and \ref{ch10:refinery}). I mainly contributed to sections \ref{ch8:workflow-management-systems-and-canonical-workflows-for-research}, \ref{ch8:fair-packaging-of-researchworkflow-objects-with-ro-crate}, \ref{ch8:fdo-types}, \ref{ch8:discussion} in this manuscript.

Section \ref{ch7:incrementally-building-fair-digital-objects-with-specimen-data-refinery-workflows} is based on the publication \emph{Incrementally building FAIR Digital Objects with Specimen Data
Refinery workflows} \cite{Woolland 2022} (see appendices \ref{ch11:incrementally-fdo} and \ref{ch10:incrementally-fdo}). I am the main author of this manuscript.

Section \ref{ch54:wrroc} is based on the preprint \emph{
Recording provenance of workflow runs with RO-Crate} \cite{workflow-run-crate} (see appendices \ref{ch11:wrroc} and \ref{ch10:wrroc}). I am the last author of this manuscript, and have mainly contributed to sections \ref{ch54:introduction}, \ref{ch54:discussion}, \ref{ch54:trusted-workflow-run-crate}, \ref{ch54:bco-crate}.
where