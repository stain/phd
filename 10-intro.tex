\emph{This thesis chapter will introduce the purpose of this PhD
study and give an overview of the
thesis}

\section{Research Outline and Questions}

In this thesis I explore three research questions:


\paragraph{RQ1:} Can the FAIR Digital Object concept be realized using Web technologies?

\paragraph{RQ2:} How can we use Linked Data practices to implement FAIR Research Objects for scholarly communication?

\paragraph{RQ3:} How can we use improve reproducibility and shareability of Computational Workflows?


\subsection{Aims of FAIR Digital Objects on the Web}

(FDO)


\subsection{Aims of FAIR Research Objects}

(RO)

\begin{itemize}
    \item Describe and package data collections, datasets, software etc. with their metadata
    \item Platform-independent object exchange between repositories and services
    \item Support reproducibility and analysis: link data with codes and workflows
    \item Transfer of sensitive/large distributed datasets with persistent identifiers
    \item Aggregate citations and persistent identifiers
    \item Propagate provenance and existing metadata
    \item Publish and archive mixed objects and references
    \item Reuse existing standards, but hide their complexity
\end{itemize}

Implementations of FAIR Research Objects should be infrastructure independent – a solution should avoid dependence of silos such as particular repositories or platforms. 

For wide adoption it needs to be familiar, developer friendly, web native, machine- and human-readable, search-engine accessible. 

Embrace diversity, legacy, unknowns, open-ended, multi-interpretation, self-describing, interlingua.


\subsection{Aims of FAIR Computational Workflows}

(FAIR CW)


\section{Main Contributions}

FDO evaluation
RO-Crate
Workflow Run Crate profiles
Broad use 



\section{Thesis Overview}

Chapter \vref{chapter:background} gives the background of concepts \emph{FAIR Digital Object} (FDO) and \emph{Linked Data}, including a brief history of the \emph{Semantic Web} and a critical analysis of these technologies and their use. 

Chapter \vref{chapter:fdo} consider RQ1 and contributes a framework-based evaluation of Linked Data and FDO as possible architectures for implementing a distributed object system for the purpose of FAIR data publishing, and discusses how the two approaches can benefit from each others strengths. 

Chapter \vref{chapter:ro-crate} RQ2 by introduces the contribution of \emph{RO-Crate}, a pragmatic data packaging mechanism using Linked Data standards to implement FDO and be extensible for domain-specific metadata.  

Chapter \vref{chapter:workflows} address RQ3 by exploring the relationship between Computational Workflows and FAIR practices using RO-Crate and FDO, with use cases from molecular dynamics and specimen digitization. The contribution of the \emph{Workflow Run Crate profiles} is presented as an interoperable way to capture and publish workflow execution provenance. 

Chapter \vref{chapter:conclusions} summarizes and concludes the contributions from this thesis, and reflects on later developments and future work.


\section{Origins}

Chapter \ref{ch3:background} and section \ref{ch3:evaluating-fdo-ld} is based on the preprint \emph{Evaluating FAIR Digital Object and Linked Data as distributed object systems} \cite{soilandreyes2023c}  (see appendices \vref{ch11:fdo} and \vref{ch10:fdo}). I am the main author of this manuscript.

Section \ref{ch2:updating-linked-data-practices-for-fair-digital-object-principles} is based on \emph{Updating Linked Data practices for FAIR Digital Object principles} \cite{10.3897/rio.8.e94501} (see appendices \ref{ch11:updating-ld} and \ref{ch10:updating-ld}). I am the main author of this manuscript.

Sections \ref{ch5:packaging-research-artefacts-with-ro-crate} and \vref{ch5:formaldefinition} are based on the publication \emph{Packaging research artefacts with RO-Crate} \cite{Soiland-Reyes 2022} (see appendices \ref{ch11:packagingrocrate}, \ref{ch10:packagingrocrate} and \ref{ch10:formalizing}). I am the main author of this manuscript.

Section \ref{ch4:lightweight-fdo} is based on the publication \emph{Creating lightweight FAIR digital objects with RO-Crate} \cite{10.3897/rio.8.e93937} (see appendices \ref{ch11:lightweight} and \ref{ch10:lightweight}). I am the main author of this manuscript.

Section \ref{ch6:making-canonical-workflow-building-blocks-interoperable-across-workflow-languages} is based on the publication \emph{Making Canonical Workflow Building Blocks interoperable across workflow languages} \cite{Soiland-Reyes 2022b} (see appendices \ref{ch11:canonical} and \ref{ch10:canonical}). I am the main author of this manuscript.

Section \ref{ch8:the-specimen-data-refinery} is based on the publication \emph{The Specimen Data Refinery: A
canonical workflow framework and FAIR Digital Object approach to speeding up digital mobilisation of natural history collections} \cite{Hardisty 2022} (see appendices \ref{ch11:refinery} and \ref{ch10:refinery}). I mainly contributed to sections \ref{ch8:workflow-management-systems-and-canonical-workflows-for-research}, \ref{ch8:fair-packaging-of-researchworkflow-objects-with-ro-crate}, \ref{ch8:fdo-types}, \ref{ch8:discussion} in this manuscript.

Section \ref{ch7:incrementally-building-fair-digital-objects-with-specimen-data-refinery-workflows} is based on the publication \emph{Incrementally building FAIR Digital Objects with Specimen Data
Refinery workflows} \cite{Woolland 2022} (see appendices \ref{ch11:incrementally-fdo} and \ref{ch10:incrementally-fdo}). I am the main author of this manuscript.

Section \ref{ch54:wrroc} is based on the preprint \emph{
Recording provenance of workflow runs with RO-Crate} \cite{workflow-run-crate} . I am the last author and co-editor of this manuscript, and have mainly contributed to sections \ref{ch54:introduction}, \ref{ch54:discussion}, \ref{ch54:trusted-workflow-run-crate}, \ref{ch54:bco-crate} while supervising the Workflow Run Crate task force together with its chairs Simone Leo and Laura Rodríguez-Navas.
