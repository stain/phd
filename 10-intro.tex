\emph{This thesis chapter will introduce the purpose of this PhD
study and give an overview of the
thesis}

\section{Research Outline and Questions}
\subsection{?1}
\subsection{?1}
\section{Main Contributions}

\section{Thesis Overview}

Chapter \vref{chapter:background} gives the background of concepts \emph{FAIR Digital Object} (FDO) and \emph{Linked Data}, including a brief history of the \emph{Semantic Web} and a critical analysis of these technologies and their use. 

Chapter \vref{chapter:fdo} contributes a framework-based evaluation of Linked Data and FDO as possible architectures for implementing a distributed object system for the purpose of FAIR data publishing, and discusses how the two approaches can benefit from each others strengths. 

Chapter \vref{chapter:ro-crate} introduces the contribution of \emph{RO-Crate}, a pragmatic data packaging mechanism using Linked Data standards to implement FDO and be extensible for domain-specific metadata.  

Chapter \vref{chapter:workflows} explores the relationship between Computational Workflows and FAIR practices using RO-Crate and FDO, with use cases from molecular dynamics and specimen digitization. 

Chapter \vref{chapter:conclusions} summarizes and concludes the contributions from this thesis, and reflects on later developments and future work.


\section{Origins}

Chapter \ref{ch3:background} and section \ref{ch3:evaluating-fdo-ld} is based on the preprint \emph{Evaluating FAIR Digital Object and Linked Data as distributed object systems} \cite{soilandreyes2023c}  (see appendices \vref{ch11:fdo} and \vref{ch10:fdo}). I am the main author of this manuscript.

Section \ref{ch2:updating-linked-data-practices-for-fair-digital-object-principles} is based on \emph{Updating Linked Data practices for FAIR Digital Object principles} \cite{10.3897/rio.8.e94501} (see appendices \ref{ch11:updating-ld} and \ref{ch10:updating-ld}). I am the main author of this manuscript.

Sections \ref{ch5:packaging-research-artefacts-with-ro-crate} and \vref{ch5:formaldefinition} are based on the publication \emph{Packaging research artefacts with RO-Crate} \cite{Soiland-Reyes 2022} (see appendices \ref{ch11:packagingrocrate}, \ref{ch10:packagingrocrate} and \ref{ch10:formalizing}). I am the main author of this manuscript.

Section \ref{ch4:lightweight-fdo} is based on the publication \emph{Creating lightweight FAIR digital objects with RO-Crate} \cite{10.3897/rio.8.e93937} (see appendices \ref{ch11:lightweight} and \ref{ch10:lightweight}). I am the main author of this manuscript.

Section \ref{ch6:making-canonical-workflow-building-blocks-interoperable-across-workflow-languages} is based on the publication \emph{Making Canonical Workflow Building Blocks interoperable across workflow languages} \cite{Soiland-Reyes 2022b} (see appendices \ref{ch11:canonical} and \ref{ch10:canonical}). I am the main author of this manuscript.

Section \ref{ch8:the-specimen-data-refinery} is based on the publication \emph{The Specimen Data Refinery: A
canonical workflow framework and FAIR Digital Object approach to speeding up digital mobilisation of natural history collections} \cite{Hardisty 2022} (see appendices \ref{ch11:refinery} and \ref{ch10:refinery}). I mainly contributed to sections \ref{ch8:workflow-management-systems-and-canonical-workflows-for-research}, \ref{ch8:fair-packaging-of-researchworkflow-objects-with-ro-crate}, \ref{ch8:fdo-types}, \ref{ch8:discussion} in this manuscript.

Section \ref{ch7:incrementally-building-fair-digital-objects-with-specimen-data-refinery-workflows} is based on the publication \emph{Incrementally building FAIR Digital Objects with Specimen Data
Refinery workflows} \cite{Woolland 2022} (see appendices \ref{ch11:incrementally-fdo} and \ref{ch10:incrementally-fdo}). I am the main author of this manuscript.
