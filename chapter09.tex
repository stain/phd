(This thesis chapter will summarize this PhD study and give further
considerations)

\section{Main Findings}

\subsection{FAIR Digital Objects can benefit from Linked Data practices}

The main advantage of scholarly researchers publishing \emph{FAIR data} is to enable machine actionability \cite{Wilkinson 2016}, which again will accelerate further research. 
In practice, this is largely approached either by depositions in general and institutional repositories for Open Data such as Figshare and Zenodo \cite{Dillen 2019}, or to specialized domain-specific repositories such as GBIF in biodiversity \cite{ch8-7}. 

European research infrastructures supporting Open Science practices are coalescing their services to form the Europen Open Science Cloud (EOSC) \cite{10.2777/940154}, which are embracing FAIR principles \cite{Mons 2017} and building a common framework for interoperability \cite{eosc-interop-framework}. 

While existing practices for implementing FAIR has relied on the Linked Data stack, that is just one possible technology to achieve the benefits of interoperable machine actionability \cite{Mons 2017}. 

Chapter \vref{chapter:fdo} explored the concept of \emph{FAIR Digital Objects} (FDO) \cite{Schultes 2019} as a potential distributed object system for FAIR data, comparing its proposed principles and current practices with the established Linked Data approach. 
As detailed in section \vref{ch3:fdo}, FDO defines a handful of constraints and guides for a predictable way to organise complex machine actionable digital entities. 
Conceptually FDO can clearly be useful for realizing FAIR principles with more active digital objects that can form a consistent ecosystem, but this opens many questions on actual FDO implementations with regards to protocols and standards.

\subsubsection{Linked Data need more constraints and consistency to be FAIR}

Examined in section \vref{ch3:ld}, the principles of \emph{Linked Data} emerged from the Semantic Web as a more data-centric view with a focus on navigation and cross-site interoperability, rather than elaborate logical inferencing systems.  
Yet the bewildering landscape of technology choices for using RDF in data platforms means that the developers suffer and still face a steep learning curve. 
For clients consuming Linked Data, the situation is still baffling in that relatively small differences in identifiers, vocabularies and usage patterns across deployment result in incompatibilities that may require platform-specific workarounds and mappings \cite{?}. 

The ecosystem of FAIR tooling is not currently mature enough to support Linked Data consumption in a user-friendly and efficient way \cite{Thompson 2020}, although recent metrics and tools for assessing \emph{FAIRness} \cite{Wilkinson 2018} can assist both data providers and consumers. 
Evaluations by EOSC has since found that FAIRness metrics can vary widely across the different assessment tools for the same data resource \cite{Wilkinson 2023}, showing that further definitions of conventions and practices are needed for consistent Linked Data publishing and consumption. 

Making the FAIR principles achieve practical benefits for researchers and platform developers thus requires more specific constraints and broader consistency.

\subsubsection{FDOs can be a distributed object system on the Web}

The detailed comparisons in section \vref{ch3:results} went into the implementation details that affect both FDO and Linked Data, and evaluated to what extent a distributed object system 



\subsection{?1}
\subsection{?2}
\section{Future Directions}


\section{Conclusions from each chapter}


From \vref{ch3:evaluating-fdo-ld}:

In this work, we have considered FAIR Digital Objects (FDO) as a potential distributed object system for FAIR data and compared it with established Web approaches focusing on Linked Data. We have described the background of the Semantic Web and FAIR Digital Objects, and evaluated both using multiple conceptual frameworks.

We find that both FDO and Linked Data approaches can significantly benefit from each-other and should be aligned further. Namely, Linked Data proponents need to make their technologies more approachable, agreeing on predictable and consistent implementations of FAIR principles. 

The FDO recommendations show that FAIR thinking in this regard need to move beyond data publishing and into machine actionability across digital objects, and with broader community consensus. 
As flexibility for extensions is a necessary ingredient alongside rigidity for core concepts, the FDO community likewise need to settle on directly implementable specifications rather than just guidelines, and avoid making similar mistakes learnt by the early Semantic Web adopters. 

By implementing the goals of FAIR Digital Objects with the mature technology stack developed for Linked Data, EOSC research infrastructures and researchers in general can create and use FAIR machine-actionable research outputs for decades to come.




From \vref{ch2:meeting-fdo-principles-using-linked-data-standards}:

The Linked Data stack provides a set of specifications, tools and
guidelines in order to help the FDO principles become a reality. This
mature approach can accelerate uptake of FDO by scholars and existing
research infrastructures such as the European Open Science Cloud (EOSC).

However, the amount of standards and existing metadata vocabularies
poses a potential threat for adoption and interoperability. Yet, the
challenges for agreeing on usage profiles apply equally to DOIP as LD
approaches.

We have worked with different scientific communities to define 
RO-Crate \cite{Soiland-Reyes 2022}, a
lightweight method to package research outputs along with their
metadata. While RO-Crate's use of schema.org shows just one possible
metadata model, it's powerful enough to be able to express FDOs, and
familiar to web developers.

We have also used FAIR Signposting \cite{Van de Sompel 2022} with HTTP
\texttt{Link:} headers as a way to support navigation to the individual
core properties of an FDO (PID, type, metadata, licence, bytestream)
that does not require heuristics of content-negotiation and is agnostic
to particular metadata vocabularies and serialisations.

We believe that by adopting Linked Data principles, we can accelerate
FDO today -- and even start building practical ways to assist scientists
in efficiently answering topical questions based on knowledge graphs.





From \vref{ch5:packaging-research-artefacts-with-ro-crate}

RO-Crate has been established as an approach to packaging digital
research artefacts with structured metadata. This approach assists
developers and researchers to produce and consume FAIR archives of their
research.

RO-Crate is formed by a set of best practice recommendations, developed
by an open and broad community. These guidelines show how to use ``just
enough'' standards in a consistent way. The use of structured metadata
with a rich base vocabulary can cover general-purpose contextual
relations, with a Linked Data foundation that ensures extensibility to
domain- and application-specific uses. We can therefore consider an
RO-Crate not just as a structured data archive, but as a multimodal
scholarly knowledge graph that can help ``FAIRify'' and combine metadata
of existing resources.

The adoption of simple Web technologies in the RO-Crate specification
has helped a rapid development of a wide variety of supporting open
source tools and libraries. RO-Crate fits into the larger landscape of
open scholarly communication and FAIR Digital Object infrastructure, and
can be integrated into data repository platforms. RO-Crate can be
applied as a data/metadata exchange mechanism, assist in long-term
archival preservation of metadata and data, or simply used at a small
scale by individual researchers. Thanks to its strong community support,
new and improved profiles and tools are being continuously added to the
RO-Crate landscape, making it easier for adopters to find examples and
support for their own use case.

\subsubsection{Strictness vs
flexibility}\label{ch5:strictness-vs-flexibility}

There is always a tradeoff between flexibility and strictness \cite{ch5-116}
when deciding on semantics of metadata models. Strict requirements make
it easier for users and code to consume and populate a model, by
reducing choices and having mandated ``slots'' to fill in. But such
rigidity can also restrict richness and applicability of the model, as
it in turn enforce the initial assumptions about what can be described.

RO-Crate attempts to strike a balance between these tensions, and
provides a common metadata framework that encourages extensions.
However, just like the RO-Crate specification can be thought of as a
\emph{core profile} of Schema.org in JSON-LD, we cannot stress the
importance of also establishing domain-specific RO-Crate profiles and
conventions, as explored in Sections~\vref{ch5:profiles} and \vref
{ch5:inuse}. Specialization comes
hand-in-hand with the principle of \emph{graceful degradation}; RO-Crate
applications and users are free to choose the semantic detail level they
participate at, as long as they follow the common syntactic
requirements.

\subsection{Future Work}\label{ch5:futurework}

The direction of future RO-Crate work is determined by the community
around it as a collaborative effort. We currently plan on further
outreach, building training material (including a comprehensive
entry-level tutorial) and maturing the reference implementation
libraries. We will also collect and build examples of RO-Crate
\emph{consumption}, e.g.~Jupyter Notebooks that query multiple crates
using knowledge graphs. In addition, we are exploring ways to support
some entity types requested by users, e.g.~detailed workflow runs or
container provenance, which do not have a good match in Schema.org. Such
support could be added, for instance, by integrating other vocabularies
or by having separated (but linked) metadata files.

Furthermore, we want to better understand how the community uses
RO-Crate in practice and how it contrasts with other related efforts;
this will help us to improve our specification and tools. By discovering
commonalities in emerging usage (e.g.~additional Schema.org types), the
community helps to reduce divergence that could otherwise occur with
proliferation of further RO-Crate profiles. We plan to gather feedback
via user studies, with the Linked Open Data community or as part of EOSC
Bring-your-own-Data training events.

We operate in an open community where future and potential users of
RO-Crate are actively welcomed to participate and contribute feedback
and requirements. In addition, we are targeting a wider audience through
extensive
outreach
activities\footnote{\url{https://www.researchobject.org/ro-crate/outreach.html}} and by initiating new connections. Recent contacts include
American Geophysical Union (AGU) on Data Citation Reliquary
\cite{Agarwal 2021}, National
Institute of Standards and Technology (NIST) on material science, and
InvenioRDM\footnote{\url{https://inveniosoftware.org/products/rdm/}} used by the
Zenodo data repository. New Horizon Europe projects adapting RO-Crate
include BY-COVID,\footnote{\url{https://by-covid.org/}} which aims to improve FAIR
access to data on COVID-19 and other infectious diseases.

The main addition in the upcoming 1.2 release of the RO-Crate
specifications will be the formalization of
profiles\footnote{\url{https://www.researchobject.org/ro-crate/1.2-DRAFT/profiles}}
for different categories of crates. Additional entity types have been
requested by users, e.g. workflow runs, business workflows, containers
and software packages, tabular data structures; these are not always
matched well with existing Schema.org types, but may benefit from other
vocabularies or even separate metadata files, e.g. from Frictionless
Data.\footnote{\url{https://frictionlessdata.io/}} We will be further aligning
and collaborating with related research artefact description efforts
like CodeMeta\footnote{\url{https://codemeta.github.io/}} for software metadata,
Science-on-Schema.org\footnote{\url{https://science-on-schema.org/}}
\cite{ch5-66} for datasets, FAIR Digital
Objects\footnote{\url{https://fairdo.org/}} \cite{De Smedt 2020} and
activities in EOSC task forces\footnote{\url{https://www.eosc.eu/task-force-faq}}
including the EOSC Interoperability Framework \cite{eosc-interop-framework}.





From \vref{ch4:lightweight-fdo}

Further work on RO-Crate profiles include to formalise links to the API
operations and repositories (FDOF5,FDOF7), to include PIDs of
profiles and types in the FAIR Signposting, and HTTP navigation to
individual resources within the RO-Crate.

RO-Crate has shown a broad adoption by communities across many
scientific disciplines, providing a lightweight, and therefore easy to
adopt, approach to generating FAIR Digital Objects. It is rapidly
becoming an integral part of the interoperability fabric between the
different components as demonstrated here for WorkflowHub, contributing
to building the European Open Science Cloud.


From \vref{ch6:making-canonical-workflow-building-blocks-interoperable-across-workflow-languages}


The proposed concept of Canonical Workflow Building Blocks can bridge
the gap between FAIR Computational Workflows, interoperable
reproducibility and for building canonical workflow descriptions to be
used and described FAIRly across WfMSs.

The realisation of CWBBs can be achieved in many ways, not necessarily
using the Python programming language together with RO-Crate as explored
here. In particular if the envisioned Canonical Workflow Frameworks for
Research become established in multiple WfMSs with the use of FAIR
Digital Objects, the different implementations will need to agree on
object types, software packaging and metadata formats in order to reuse
tools and provide interoperable reproducibility for canonical workflows.

Likewise, to build a meaningful collection of building blocks for a
given research domain, a directed collaborative effort is needed to
consistently wrap tools for a related set of WfMSs, chosen to target
particular use cases (a family of canonical workflows).

For individual users, a library of Canonical Workflow Building Blocks
simplifies many aspects of building pipelines, beyond the FAIR aspects
and data compatibility across blocks. For instance, they can benefit
from training of a CWBB family using Jupyter Notebooks, and then use
this knowledge to utilise the same building blocks in a scalable HPC
workflow with a CWL engine like Toil, knowing they will perform
consistently thanks to the use of containers.

While we have demonstrated CWBB in the biomedical domain, this approach
is generally applicable to a wide range of sciences that execute
pipelines of multiple file-based command line tools, however it may be
harder to achieve with more algebraic ``in memory'' types of
computational workflows, where steps could be challenging to
containerize and distinguish as separate block.

We admit that biomolecular research is quite a homogenous field with
respect to computational analyses and now becoming relatively mature in
terms of tool composability in workflows, building on the experiences of
the ``FAIR pioneers'' in the field of bioinformatics. Other fields, such
as social sciences or ecology, can have a wider variety of methods and
computational tools, often with human interactions, and may have to
adapt the software to be workflow-ready \cite{ch6-37} before using them as
Canonical Workflow Building Blocks. Domains adapting CWBB approach (or
workflow systems in general) should take note of the great benefits of
hosting collaborative events where developers meet each other and their
potential users, demonstrated in our field with events such WorkflowsRI
\cite{ch6-39} and Biohackathons \cite{ch6-40}.

The Common Workflow Language shows promise as a general canonical
workflow building blocks mechanism: gathering execution details of tools
along with their metadata and references, augmented with
\footurl{https://docs.bioexcel.eu/cwl-best-practice-guide/devpractice/partial.html\#using-abstract-operations-as-placeholders}{abstract
workflows} to represent canonical workflows. However, this would need
further work to implement our CWBB recommendations in full. Future work
for the Canonical Workflow Building Blocks concept includes formalising
and automating publication practises, to make individual blocks
available as FAIR Digital Objects on their own or as part of an
aggregate collection like RO-Crate.

From \vref{ch8:the-specimen-data-refinery}



Both kinds of FDO are essential. They complement one another to support
implementation of the FAIR principles, especially the interoperable and
reusable principles by making workflows self-documenting. This renders
automated whole processes (or fragments thereof) for digitizing and
extending natural history specimens' data as FAIR without adding
additional load to the researchers that stand to benefit most from that
\cite{ch8-27}. Each FDO type originates from different Research
Infrastructures (ELIXIR, DiSSCo) with different implementation
frameworks. Yet, they interoperate effectively due to their clear roles,
common conceptual model and separation of concerns.


openDS FDOs have their heritage in distributed digital object services
\cite{ch8-46} and are implemented through Digital Object Architecture (DOA)
\cite{ch8-62} with Digital Object Interface Protocol (DOIP) \cite{DONA 2018}, Digital
Object Identifier Resolution Protocol (DO-IRP) \cite{rfc3652}, and
recommendations of the Research Data Alliance \cite{ch8-65}. Serialized as
JSON, they are machine-actionable and compatible with established
protocols of the World Wide Web.

RO-Crates are native to the World Wide Web, based on established web
protocols, machine-readable metadata using Linked Data Platform methods
\cite{ch8-66}, JSON-LD and Schema.org \cite{Bechhofer 2013}, and community-accepted
packaging mechanisms such as BagIt. This makes RO-Crates straightforward
to incorporate into pre-existing platforms such as Galaxy and data
repositories such as Zenodo and DataVerse.

Both kinds of FDO use Persistent identifiers (PID), allowing instances
to be both uniquely identified and their location to be determined;
RO-Crates, as web natives, use URIs whereas openDS, as DOA objects, use
Handle PIDs. Instances of both kinds are described by metadata and
contain or reference data.

RO-Crates are self-describing using a metadata file and use
openly-extensible profiles to type the Crates (profile-typing) to set
out expectations for their metadata and content. openDS uses an
object-oriented object typing and instance approach to define the
structure and content of data/metadata. Complex object types are
constructed from basic types, an extension-section basic type. Both
approaches seek to avoid locking objects into repository silos, ensuring
that FDO instances can be interpreted outside of the contexts in which
they were originally created/stored.

Structurally and semantically openDS FDOs and RO-Crate FDOs are
potentially isomorphic, although at different granularity levels. Their
main difference is in method calling. As a DOA object, openDS would
expect to respond to type-specific method calls if these were
implemented. RO-Crates delegate actionability to applications that
interpret their self-describing profile.

Within the SDR the two kinds of FDO fulfill distinct and interlocking
roles for data (openDS) and self-documented method (RO-Crate) so their
different forms is not an issue. In future there may be a need to map
and convert between the approaches (e.g., for reconstructing past
processing), which would be assisted by the common FDO conceptual model
\cite{bonino2019}.



From \vref{ch7:incrementally-building-fair-digital-objects-with-specimen-data-refinery-workflows}


SDR is an example of machine-assisted construction of FDOs, which
highlight the needs for intermediate digital objects that are not yet
FDO compliant. The passing of such ``local FDOs'' is beneficial not just
for efficiency and visual inspection, but also to simplify workflow
composition of canonical workflow building blocks. At the same time we
see that it is insufficient to only pass FDOs as JSON objects, as they
also have references to other data such as images, which should not need
to be re-downloaded.

Further work will investigate the use of RO-Crate as a wrapper of
partial FDOs, but this needs to be coupled with more flexible FDO types
as profiles, in order to restrict ``impossible'' ordering of steps
depending on particular inner FDO fragments. A distinction needs to be
made between open digital specimens that are in ``draft'' state and
those that can be pushed to DiSSCo registries.

We are experimenting with changing the SDR components into Canonical
Workflow Building Blocks \cite{Soiland-Reyes 2022a}
(\vref{ch6:making-canonical-workflow-building-blocks-interoperable-across-workflow-languages}) 
using the Common Workflow Language \cite{Crusoe 2022}. This gives
flexibility to scalably execute SDR workflows on different compute
backends such as HPC or local cluster, without the additional setup of
Galaxy servers.

