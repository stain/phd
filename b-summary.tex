\emph{a summary of the whole thesis}

% ChatGPT!

The thesis begins with an exploration of RO-Crate, FAIR Digital Objects (FDOs), and computational workflows, driven by three primary research questions. The main contributions involve the evaluation and integration of these components to enhance the FAIR principles in research data. It underscores the importance of a strong community in driving semantic model development and outlines the thesis's structure.

The background section delves into the evolution of the Semantic Web, Linked Data, and FAIR Digital Objects, providing context for the subsequent discussions. The evaluation of FAIR Digital Objects and Linked Data as distributed object systems is presented, analyzing frameworks and implementations. The synthesis of Linked Data practices for FAIR Digital Objects principles is explored, emphasizing the need for constraints and consistency.

RO-Crate is introduced as a tool for packaging research artifacts, detailing its development methodology, conceptual definition, and tooling. Various profiles of RO-Crate in use across domains such as bioinformatics, regulatory sciences, and digital humanities are highlighted. The section also discusses creating lightweight FAIR Digital Objects with RO-Crate and presents a formalization of RO-Crate in First Order Logic.

Computational workflows, treated as FAIR Digital Objects, are examined in detail. The thesis addresses the interoperability of canonical workflow building blocks across different languages and introduces the Specimen Data Refinery, a case study illustrating the challenges and benefits of implementing FAIR Digital Objects within workflows. The incremental construction of FAIR Digital Objects within workflows is discussed, emphasizing the need for flexible profiles to ensure typing accuracy.

The thesis concludes with a comprehensive discussion covering various aspects, including making ecosystems of FAIR Digital Objects predictable, the developer-friendly approach of RO-Crate, and future directions for both RO-Crate and computational workflows. The importance of user applications, web-based FDOs, and the adaptability of workflow profiles is emphasized. The conclusion reinforces the main findings and highlights the significance of community-driven pragmatic solutions over strict semantic correctness.

