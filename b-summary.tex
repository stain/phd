\textbf{FAIR Research Objects and Computational Workflows – A Linked Data Approach}

This PhD thesis explores the topics of RO-Crate, FAIR Digital Objects (FDOs), and computational workflows, in order to examine research questions relating to how these can be implemented and integrated using Linked Data approaches -- forming ``FAIR Research Objects''.

The background covers the evolution of the Semantic Web, Linked Data, and FAIR Digital Objects, which are then evaluated against the FAIR principles and several frameworks, to consider these technologies as potential middleware for a global distributed object system that enable machine-actionable research outputs. 

This work introduces the community-developed method \emph{RO-Crate} for packaging research artifacts with their contextual information, relationships and metadata -- taking advantage of Linked Data standards that have been simplified and documented for pragmatic use by software developers. The tension between flexibility for implementations and rigidity of semantic constraints is explored, and demonstrated by various profiles of RO-Crate that have been implemented across research domains such as bioinformatics, regulatory sciences, biodiversity and digital humanities. 

Computational workflows, commonly used by scientists for reproducible data analysis across execution platforms, are then examined as potential FAIR Digital Objects, considering them both as shareable research outputs (capturing the computational method for later reuse) as well as a part of provenance of computational results, captured in a Workflow Run Crate, defined as an RO-Crate profile. 
Additionally the concept of \emph{Canonical Workflow Building Blocks} is introduced as a method for FAIR sharing of tools across workflow systems. A case study from natural history museums and biodiversity shows how the combination of workflows and RO-Crate can be used to incrementally annotate digitised specimens and gradually build reproducible domain-specific FDOs. 

The discussion part of this thesis explores how the emerging ecosystem of FAIR Digital Objects can learn from the community development of RO-Crate and carefully adapt ``just enough'' of Linked Data technologies, balancing flexibility and predictability. Future directions for RO-Crate are examined, including new adoptions and further alignments with FAIR and FDO principles. Lessons from computational workflows further inform directions of FDO and RO-Crate. The main findings of this thesis concludes on the significance of community-driven pragmatic solutions over strict semantic correctness, supporting advancement of the FAIR principles through practical and interoperable implementations of Web standards.

