This chapter introduces \emph{RO-Crate}, a pragmatic method of packaging data alongside structured metadata that is inline with the FAIR principles. This has been implemented to investigate \textbf{RQ2} (\vpageref*{rq2}). 

Section \vref{ch5:packaging-research-artefacts-with-ro-crate} describes the RO-Crate purpose,  community effort and tooling and demonstrates how RO-Crate has been applied.

Section \vref{ch4:lightweight-fdo} shows how RO-Crate can be used to achieve the FDO principles covered in chapter \ref{chapter:fdo}.

Section \vref{ch5:formaldefinition} contributes a formal definition of RO-Crate using first order logic.

Supplementary material that may assist readers of this chapter includes the motivation of  \footurl{https://s11.no/2019/phd/ro-crate/}{RO-Crate, a lightweight approach to Research Object data packaging} \cite{10.5281/zenodo.3337883}. 

RO-Crate builds on the long history of \emph{Research Objects}, which is covered by earlier works  \cite{Bechhofer 2013,Belhajjame 2015,goble-ro2018} and the \footurl{https://s11.no/2020/archive/wf4ever/}{Wf4Ever project}.
