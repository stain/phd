\textbf{FAIR Onderzoeksobject en computationele workflows – een Linked Data-aanpak}

\begin{small}

Dit proefschrift verkent hoe de onderwerpen RO-Crate, FAIR Digital Objects (FDO) en computationele workflows gecombineerd kunnen worden en te onderzoeken hoe deze kunnen worden geïmplementeerd en geïntegreerd met behulp van Linked Data-benaderingen, zodat we uitkomen bij "FAIR Onderzoeksobject".

De achtergrond behandelt eerst de evolutie van het Semantisch Web, Linked Data en FAIR Digital Objects, welke vervolgens worden geëvalueerd aan de hand van de FAIR-principes (F: vindbaar, A: toegankelijk, I: interoperabel, R: herbruikbaar) en verschillende andere raamwerken als potentiële middleware voor een wereldwijd gedistribueerd objectensysteem. Een positieve evaluatie geeft aan dat het mogelijk is onderzoeksresultaten machinaal te hergebruiken, het uiteindelijke doel.

Dit onderzoek introduceert de door een bredere gemeenschap ontwikkelde methode \emph{RO-Crate} voor het verpakken van onderzoeksartefacten met hun contextuele informatie, relaties en metadata, waarbij gebruik wordt gemaakt van Linked Data-standaarden die zijn vereenvoudigd en in detail gedocumenteerd voor gebruik door softwareontwikkelaars. De spanning tussen flexibiliteit voor implementaties en de rigiditeit van semantische beperkingen wordt onderzocht en gedemonstreerd door verschillende profielen van RO-Crate die zijn geïmplementeerd in onderzoeksgebieden zoals bioinformatica, regelgevende wetenschappen, biodiversiteit en digitale geesteswetenschappen.

Computational workflows, die vaak worden gebruikt door wetenschappers voor reproduceerbare gegevensanalyse over uitvoeringsplatforms, worden vervolgens onderzocht als potentiële FAIR Digital Objects (FDO). Hierbij worden ze beschouwd als deelbare onderzoeksresultaten (waarbij de rekenkundige methode voor later hergebruik wordt vastgelegd) en als een weergave van de herkomst van berekende resultaten vastgelegt in een RO-Crate-profiel. Daarnaast wordt het concept van \emph{Primaire workflow-bouwstenen} geïntroduceerd als een methode voor FAIR-delen van rekeninstrumenten over verschillende workflowsystemen. Een casestudy uit natuurhistorische musea en biodiversiteit laat zien hoe de combinatie van workflows en RO-Crate kan worden gebruikt om gedigitaliseerde specimens stap voor step te annoteren en reproduceerbare, domeinspecifieke FDO's op te bouwen.

De Discussie bespreekt hoe het opkomende ecosysteem van FAIR Digital Objects verder kan bouwen op de resultaten uit de gemeenschappelijke ontwikkeling van RO-Crate en zorgvuldig "net genoeg" van Linked Data-technologieën kan hergebruiken, waarbij flexibiliteit en voorspelbaarheid in evenwicht worden gebracht. Toekomstige richtingen voor RO-Crate worden besproken, waaronder nieuwe adaptaties en verdere afstemming met FAIR- en FDO-principes. Lessen uit computationele workflows informeren ons verder over de richtingen van FDO en RO-Crate kunnen nemen. 

De belangrijkste bevindingen van dit proefschrift concluderen dat webstandaarden de doelstellingen van FDO kunnen bereiken, door gebruik te maken van bestaande standaarden met voldoende beperkingen die ontwikkelaars voorspelbaarheid en de nodige flexibiliteit geven. De lichtgewicht Linked Data-aanbevelingen van RO-Crate blijken implementeerbaar te zijn voor een reeks toepassingen, waarbij de vooruitgang van de FAIR-principes wordt ondersteund door praktisch en interoperabel gebruik van webstandaarden.

\end{small}