\textbf{FAIR Onderzoeksobject en computationele workflows – een Linked Data-aanpak}

%% This is semi-auto-translated and needs proofing by Dutch speaker

Dit proefschrift verkent de onderwerpen RO-Crate, FAIR Digital Objects (FDO) en computationele workflows om onderzoeksvragen te onderzoeken met betrekking tot hoe deze kunnen worden geïmplementeerd en geïntegreerd met behulp van Linked Data-benaderingen, wat resulteert in "FAIR Onderzoeksobject".

De achtergrond behandelt de evolutie van het Semantisch Web, Linked Data en FAIR Digital Objects, die vervolgens worden geëvalueerd aan de hand van de FAIR-principes en verschillende raamwerken, om deze technologieën te overwegen als potentiële middleware voor een wereldwijd gedistribueerd objectensysteem dat mogelijk maakt machinaal bruikbare onderzoeksresultaten.

Dit werk introduceert de door de gemeenschap ontwikkelde methode \emph{RO-Crate} voor het verpakken van onderzoeksartefacten met hun contextuele informatie, relaties en metadata, waarbij gebruik wordt gemaakt van Linked Data-standaarden die zijn vereenvoudigd en gedocumenteerd voor pragmatisch gebruik door softwareontwikkelaars. De spanning tussen flexibiliteit voor implementaties en de rigiditeit van semantische beperkingen wordt onderzocht en gedemonstreerd door verschillende profielen van RO-Crate die zijn geïmplementeerd in onderzoeksgebieden zoals bio-informatica, regelgevende wetenschappen, biodiversiteit en digitale geesteswetenschappen. 

Computational workflows, die vaak worden gebruikt door wetenschappers voor reproduceerbare gegevensanalyse over uitvoeringsplatforms, worden vervolgens onderzocht als potentiële FAIR Digital Objects, waarbij ze worden beschouwd als deelbare onderzoeksresultaten (waarbij de rekenkundige methode voor later hergebruik wordt vastgelegd) evenals een deel van de herkomst van computationele resultaten, vastgelegd in een RO-Crate-profiel. Daarnaast wordt het concept van "Canonieke workflow-bouwstenen" geïntroduceerd als een methode voor FAIR-delen van rekeninstrumenten over verschillende workflowsystemen. Een casestudy uit natuurhistorische musea en biodiversiteit laat zien hoe de combinatie van workflows en RO-Crate kan worden gebruikt om gedigitaliseerde specimens geleidelijk te annoteren en geleidelijk reproduceerbare, domeinspecifieke FDO's op te bouwen.

Het discussiegedeelte van deze scriptie onderzoekt hoe het opkomende ecosysteem van FAIR Digital Objects kan leren van de gemeenschapsontwikkeling van RO-Crate en zorgvuldig "net genoeg" van Linked Data-technologieën kan aanpassen, waarbij flexibiliteit en voorspelbaarheid in evenwicht worden gebracht. Toekomstige richtingen voor RO-Crate worden onderzocht, waaronder nieuwe adaptaties en verdere afstemmingen met FAIR- en FDO-principes. Lessen uit computationele workflows informeren verder over de richtingen van FDO en RO-Crate. De belangrijkste bevindingen van deze scriptie concluderen over de betekenis van door de gemeenschap gedreven pragmatische oplossingen boven strikte semantische correctheid, ter ondersteuning van de vooruitgang van de FAIR-principes via praktische en interoperabele implementaties van webstandaarden.