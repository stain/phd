\emph{(deze samenvatting zal op de een of andere manier zelfs in
het Nederlands herhaald
worden.)}

% ChatGPT

computationele workflows, gedreven door drie hoofdonderzoeksvragen. De belangrijkste bijdragen omvatten de evaluatie en integratie van deze componenten om de FAIR-principes in onderzoeksdata te verbeteren. Het benadrukt het belang van een sterke gemeenschap bij de ontwikkeling van semantische modellen en geeft de structuur van de scriptie weer.

In de achtergrondsectie wordt ingegaan op de evolutie van het Semantisch Web, Linked Data en FAIR Digital Objects, als context voor de daaropvolgende discussies. De evaluatie van FAIR Digital Objects en Linked Data als gedistribueerde objectensystemen wordt gepresenteerd, waarbij frameworks en implementaties worden geanalyseerd. De synthese van Linked Data-praktijken voor FAIR Digital Objects-principes wordt verkend, met nadruk op de noodzaak van beperkingen en consistentie.

RO-Crate wordt geïntroduceerd als een instrument voor het verpakken van onderzoeksartefacten, met details over de ontwikkelingsmethodologie, conceptuele definitie en tooling. Diverse profielen van RO-Crate in gebruik binnen domeinen zoals bio-informatica, regelgevende wetenschappen en digitale geesteswetenschappen worden benadrukt. Het gedeelte behandelt ook het creëren van lichte FAIR Digital Objects met RO-Crate en presenteert een formalisering van RO-Crate in de Eerste Orde Logica.

Computationele workflows, behandeld als FAIR Digital Objects, worden in detail onderzocht. De scriptie behandelt de interoperabiliteit van canonieke bouwstenen van workflows in verschillende talen en introduceert de Specimen Data Refinery, een casestudy die de uitdagingen en voordelen van het implementeren van FAIR Digital Objects binnen workflows illustreert. De incrementele constructie van FAIR Digital Objects binnen workflows wordt besproken, waarbij de noodzaak van flexibele profielen wordt benadrukt om een juiste typering te waarborgen.

De scriptie eindigt met een uitgebreide discussie die verschillende aspecten behandelt, waaronder het voorspelbaar maken van ecosystemen van FAIR Digital Objects, de ontwikkelaarsvriendelijke aanpak van RO-Crate en toekomstige richtingen voor zowel RO-Crate als computationele workflows. Het belang van gebruikersapplicaties, op web gebaseerde FDO's en de aanpasbaarheid van workflow-profielen wordt benadrukt. De conclusie bevestigt de belangrijkste bevindingen en benadrukt het belang van op de gemeenschap gerichte pragmatische oplossingen boven strikte semantische correctheid.